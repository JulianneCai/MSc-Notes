\documentclass[a4paper]{report}
\usepackage[usenames,dvipsnames]{xcolor}
\usepackage{hyperref, lineno} % TODO REMOVE
\usepackage{mystyle}
%
% \graphicspath{{images/}}
% \SetWatermarkAngle{90} % TODO REMOVE
% \SetWatermarkScale{0.3} % TODO REMOVE
% \SetWatermarkHorCenter{1cm} % TODO REMOVE
\setcounter{tocdepth}{3} % SUBSECTIONS IN TOC
%
\author{
  Julianne Cai\\
} 
\title{}
\date{\today}

\usepackage[myheadings]{fullpage}
\usepackage{fancyhdr}
\usepackage{lastpage}
\usepackage{graphicx, wrapfig, subcaption, setspace, booktabs}
\usepackage[protrusion=true, expansion=true]{microtype}
\usepackage{sectsty}
\usepackage{titling}

\theoremstyle{definition}
\newtheorem{definition}{Definition}
\theoremstyle{remark}
\newtheorem{remark}{Remark}
\theoremstyle{proposition}
\newtheorem{proposition}{Proposition}
\theoremstyle{conjecture}
\newtheorem{conjecture}{Conjecture}
\theoremstyle{lemma}
\newtheorem{lemma}{Lemma}
\theoremstyle{corollary}
\newtheorem{corollary}{Corollary}
\theoremstyle{exercise}
\newtheorem{exercise}{Exercise}

\newtheorem{example}{Example}

\newcommand{\R}{\mathbb{R}}
\newcommand{\C}{\mathbb{C}}

\newcommand{\mcal}{\mathcal}
\newcommand{\diff}{\,\mathrm{d}}

\newcommand{\Ell}{\mcal{E}\ell\ell}

\newcommand{\on}{\operatorname}

\newcommand{\spec}{\on{Spec}}
\newcommand{\proj}{\on{Proj}}
\newcommand{\bspec}{\on{\mathbf{Spec}}}
\newcommand{\sym}{\on{Sym}}

\newcommand{\D}{\on{D}}
\newcommand{\bounded}{\on{b}}

\newcommand{\qcoh}{\on{\mathbf{QCoh}}}
\newcommand{\coh}{\on{\mathbf{Coh}}}
\newcommand{\Mod}{\on{\mathbf{Mod}}}
\newcommand{\Vect}{\on{\mathbf{Vect}}}
\newcommand{\fin}{\on{fin}}

\newcommand{\fuch}{\on{fuch}}
\newcommand{\DiffEq}{\on{\mathbf{DiffEq}}}
\newcommand{\DiffMod}{\on{\mathbf{DiffMod}}}

\newcommand{\supp}{\on{supp}}

\newcommand{\Hom}{\on{Hom}}
\newcommand{\RHom}{\on{\textit{R}Hom}}
\newcommand{\Ext}{\on{Ext}}

\newcommand{\HRule}[1]{\rule{\linewidth}{#1}}

\begin{document}

\title{ \normalsize \textsc{Live-TeXed Lecture Notes}
		\\ [2.0cm]
		\HRule{0.5pt} \\
		\LARGE \textbf{MAST90023 - Algebraic Topology}
		\HRule{2pt} \\ [0.5cm]
        }

\date{}
\author{
		Julianne Cai \\\\
        }
\maketitle
\newpage

\tableofcontents
\newpage

\chapter{Week One}

\paragraph{This week, we learned about\ldots} Something

\section{Lecture 1, 28/02/2024}

\subsection{Toplogy and Topological Spaces}

\begin{definition}
    Let $X$ be a set. Then, the \emph{power set} of $X$ -- denoted by 
    $\mcal{P}(X)$ -- is the set of all subsets of $X$.
    A subset $\tau \subset \mcal{P}(X)$ is a \emph{topology} if the following
    holds:
    \begin{itemize}
        \item[(i)] $\emptyset, X \in \tau$,
        \item[(ii)] (Closure under Arbitrary Union) $\bigcup_{\alpha \in \Lambda} U_\alpha \in \tau$ if 
            $U_\alpha \in \tau$ for each $\alpha \in \Lambda$,
        \item[(iii)] (Closure under Finite Intersections) 
            $\bigcup_{i=1}^n U_i \in \tau$ if each $U_i \in \tau$.
    \end{itemize}
    The pair $(X,\tau)$ is called a \emph{topological space}.
\end{definition}

\begin{remark}
    Often, we will just drop $\tau$ and say that $X$ is a toplogical space, where
    there is no confusion.
\end{remark}

\begin{example}
    The \emph{one-point space} is given by 
    $\tau_{\on{in}} = \lbrace X,\emptyset\rbrace$.
    The \emph{discrete topology} $\tau_{\on{discrete}} = \mcal{P}(X)$ is the smallest topology
    that can exist.
\end{example}

\begin{example}[Metric Spaces]
    A metric space $(X,d)$ defines a topological space via open balls
    $$B_r(x) := \lbrace y \in X : d(x,y) < r\rbrace.$$
    In particular, we say that $U \in \tau_d$ if $U = \bigcup_{x \in U}B_r(x)$.
    This defines a topology on $(X,d)$, and is sometimes called the 
    \emph{induced topology}.
\end{example}

\begin{example}[Standard Euclidean Space]
    Let $X = \mathbb{R}^n$, and define 
    $$d(x,y) = \sqrt{\sum_{i=1}^n (x_i-y_i)^2},$$
    to be the \emph{Euclidean metric}. The metric space $(\mathbb{R}^n,d)$, 
    together with the induced topology
    $(\mathbb{R}^n,\tau_{d})$ is called the \emph{standard Euclidean space}.
\end{example}

\begin{example}[The $n$-Sphere]
    Let $$\mathbb{S}^n = \lbrace x \in \mathbb{R}^{n+1} : d(x,0) = 1\rbrace,$$
    be the \emph{$n$-sphere}. One can equip $\mathbb{S}^n$ with the structure
    of a topological space via the usual Euclidean metric, or we can take the one
    induced from the topological structure from the one on $\mathbb{R}^{n+1}$.
\end{example}

\subsection{Continuous Maps}

\begin{remark}[Diarmuid Wisdom]
    The point of topology is that it is the study of maps. Spaces are 
    interesting, but maps are the stars of the show. In particular, they are at 
    least as important as spaces (Diarmuid, 2024).
\end{remark}

\begin{definition}
    Given a map $f:  X \to Y$ of topological spaces, then $f$ is 
    \emph{continuous} if given any open set $U$ of $Y$, the 
    pre-image $f^{-1}(U)$ is open in $X$.
\end{definition}

\begin{exercise}
    Prove that a map $f : (X, d_X) \to (Y,d_Y)$ is continuous (in the 
    metric space sense) if and only if it is continuous (in the topology space
    sense).
\end{exercise}

Recall:

\begin{definition}[Metric Space Continuity]
    A map of metric spaces $f : (X, d_X) \to (Y,d_Y)$ is \emph{continuous} if 
    for all $\varepsilon > 0$, there exists $\delta > 0$ such that if
    $d_X(x,y) < \delta$, then $d_Y(f(x),f(y)) < \varepsilon$.
\end{definition}

\begin{example}
    The map 
    $$1_X : (X,\tau) \longrightarrow (X,\tau),$$
    is continuous. 
    The map 
    $$1_X : (X,\tau_{\on{in}}) \longrightarrow (X,\tau_{\on{discrete}}),$$
    is \emph{not} continuous in general.
\end{example}


\begin{lemma}
    If $f : X \to Y$ and $g : Y \to Z$ are both continuous, then 
    $f \circ g : X \to Z$ is continuous.
\end{lemma}

\begin{proof}
    Let $U\subseteq Z$ be open. Then, $g^{-1}(U)$ is open in $Y$ by continuity
    of $g$. By continuity of $f$, the set $(f\circ g)^{-1}(U)$ is open in $X$.
\end{proof}

\begin{remark}
    Given any function $f:  X \to Y$, and a subset $U \subseteq Y$,
    the \emph{pre-image} of $U$ is:
    $$f^{-1}(U) := \lbrace x \in X : f(x) \in U \rbrace.$$
\end{remark}

\begin{example}
    The addition function $+ : \mathbb{R}^2 \to \mathbb{R}$ and 
    the multiplication function $\cdot : \mathbb{R}^2 \to \mathbb{R}$ 
    are both continuous maps. 
    This is an example of a \emph{topological group} or \emph{toplogical ring}.
\end{example}

\begin{definition}
    A bijection $f : X \to Y$ is a \emph{homeomorphism} if $f$ and $f^{-1}$ 
    are continuous.
\end{definition}

\begin{remark}
    Homeomorphisms serve as a topological analogue for isomorphisms.
\end{remark}

\paragraph{Problem:} Given $X$ and $Y$, decide if they are homeomorphic or not. For instance, are $\mathbb{R}^2$ and $\mathbb{R}^3$ homeomorphic? The answer
is no, thankfully, but this is quite difficult to prove. But it follows from invariance of domain. We will prove this later.\\\\
This is quite a difficult problem in general. When proving theorems like this, 
we need to be careful of space-filling curves --- which are 
surjective maps $\gamma : I \to I^2$. This is quite shocking.

\subsection{Examples of Spaces}

\begin{definition}
    Let $(X, \tau_X)$ be a topological space, and $A\subseteq X$ a subset. 
    The \emph{subspace topology} $\tau_A$  is a topology on $A$ inherited from 
    the one on $X$ in the following way:
    $U \in \tau_A$ if and only if $U = A \cap V$ for some $V \in \tau_X$.
\end{definition}

\begin{exercise}
    Show that $\tau_A$ is a toplogy on $A$.
\end{exercise}

\begin{exercise}
    If $\tau_X = \tau_d$ for some metric $d$ on $X$, then the restriction
    $d_A := d\vert_{A\times A}$ defines a metric on $A$.
    Show that $(A,\tau_A) = (A,\tau_{d_A})$.
\end{exercise}

\begin{example}
    We can topologise $\mathbb{S}^n$ as a subspace of $\mathbb{R}^{n+1}$
    via the subspace topology.
\end{example}

\begin{example}[Cantor Set]
    Start with $C_0 = [0,1]$. Then, remove the middle third of the interval
    to get $$C_1 = [0,1/3]\cup [2/3,1],$$
    and punch out the middle thirds in each of the disjoint intervals to get 
    $C_2$. Then, repeat forever. It is a compact set.
\end{example}

\begin{definition}[Neighbourhoods]
    Given a point $x\in X$, then a subset $N\subset X$ is a 
    \emph{neighbourhood} of $x$ if there exists a set $U \in\tau$
    such that $x \in U \subseteq N$.
\end{definition}

\begin{remark}
    It is useful at times to think about neighbourhoods that are not open.
\end{remark}

\begin{definition}[Manifold]
    $M$ is an $n$-manifold if every $X\subseteq M$ has a neighbourhood 
    homeomorphic to $\mathbb{R}^n$, or an open subset of $\mathbb{R}^n$.
    In other words, $n$-manifolds are \emph{locally Euclidean of constant
    dimension}. 
\end{definition}

\begin{remark}
    We technically also require that $M$ be paracompact and Hausdorff.
    But, we will not define paracompactness. We will talk about 
    the Hausdorff property next time. They are also covered in the notes.
\end{remark}

\begin{example}
    $\mathbb{S}^1$ defines a $1$-manifold. $\mathbb{S}^n$ defines an
    $n$-manifold.
\end{example}

\section{Lecture 2, 29/02/2024}

\subsection{Examples of Spaces Continued}

\begin{definition}[Product Topology]
    Let $X$ and $Y$ be topological spaces. Then, define the 
    topology $\tau_{X\times Y}$ to be the topology whose open sets 
    are of the form $U = \bigcup_{\alpha,\beta \in \Lambda} U_\alpha \times V_\beta$,
    where $U_\alpha$ and $V_\beta$ are open in $X$ and $Y$, respectively.
\end{definition}

\begin{example}
    The $2$-torus is given by $T^2 = \mathbb{S}^1 \times \mathbb{S}^1$.
    One may also think of $T$ as a quotient of the space $I\times I$,
    where $I = [0,1]$ is the unit interval. \\\\
    Moreover, one may also take the $3$-torus by 
    $T^3 = \mathbb{S}^1 \times\mathbb{S}^1 \times \mathbb{S}^1$.
    More generally, the $n$-torus is given by 
    $T^n = \mathbb{S}^1 \times \cdots \times \mathbb{S}^1$. We may 
    topologise these spaces via the product topology, using the 
    topology on $\mathbb{S}^1$, which is induced from the one on 
    $\mathbb{R}^2$.\\\\
    Another interesting topological space that we can consider is 
    $M^4 := \mathbb{S}^2 \times \mathbb{S}^2$.
    Generally, $M = \mathbb{S}^{n_1}\times \mathbb{S}^{n_j}$,
    for $n_1,\cdots,n_j \in \mathbb{N}^+$. 
    We make the identification
    $\mathbb{S}^0 = \lbrace \pm 1\rbrace$ as the two-point space.
\end{example}

\begin{remark}
    The spaces $X\times (Y\times Z)$ and $(X\times Y)\times Z$ are canonically
    homeomorphic.
\end{remark}

\begin{lemma}\label{lem_prod_cts}
    Let $f : W \to Y$ and $g : W \to Z$, and $h : X \to Z$ be maps of topological
    spaces. Then, the following are continous:
    \begin{itemize}
        \item[(i)] $f\times g : W \to Y \times Z$ defined by 
            $w\mapsto (f(w),g(w))$,
        \item[(ii)] $f \times h : W \times X \to Y \times Z$ defined by 
            $(w,x) \mapsto (f(w),g(x))$.
    \end{itemize}
\end{lemma}

\begin{proof}
    \leavevmode
    \begin{itemize}
        \item[(i)] Let $U_\alpha \in \tau_Y$, and $V_\beta \in \tau_Z$. 
            Then, 
            $$(f\times g)^{-1}(U_\alpha \times V_\beta) = f^{-1}(U_\alpha) \cap g^{-1}(V_\beta),$$
            which is open by continuity of $f$ and $g$.
        \item[(ii)] Similarly, 
            $$(f \times h)^{-1}(U_\alpha \times V_\beta) = f^{-1}(U_\alpha) \times h^{-1}(V_\beta),$$
            which is open again by the continuity of $f$ and $h$.
    \end{itemize}
\end{proof}

\begin{remark}
    For any space $W$, we have $$\Delta : W \longrightarrow W \times W,\quad w \longmapsto (w,w),$$
    which is called the \emph{diagonal map}.
    Let us write 
    $$f \overline{\times} g := (f\times g)\circ \Delta.$$
\end{remark}

\begin{exercise}
    Show that $\Delta$ is continuous for all topological spaces.
\end{exercise}

\begin{lemma}\label{lem_proj_cts}
    Consider the map 
    $$\on{pr}_j : \prod_{i=1}^n X_i \longrightarrow X_j,$$
    which projects onto the $i$-th factor. 
    Then, $\on{pr}_j$ is continuous.
\end{lemma}

\begin{proof}
    Exercise.
\end{proof}

\begin{remark}
    $f : W \to Y \times Z$ is continuous if and only if 
    $\on{pr}_Y \circ f$ and $\on{pr}_Z \circ f$ are continuous.
\end{remark}

Recall from last time that we defined continuous maps 
$$ + : \mathbb{R}^2 \longrightarrow \mathbb{R}, \quad \times : \mathbb{R}^2 \longrightarrow \mathbb{R}.$$

\begin{lemma}
    The map
    $$\diff p : \mathbb{R}^n \times \mathbb{R}^n \longrightarrow \mathbb{R},\quad (\mathbf{x},\mathbf{y}) \longrightarrow \sum_{i=1}^n x_iy_i,$$
    is continuous.
\end{lemma}

\begin{proof}
    Take projections $\on{pr}_i \times \on{pr}_j : \mathbb{R}^n \times \mathbb{R}^n \to \mathbb{R} \times \mathbb{R}$,
    and compose this with $+ : \mathbb{R} \times \mathbb{R} \to \mathbb{R}$.
    The projection maps are continuous by Lemma \ref{lem_proj_cts}, and thus their products are continuous by Lemma \ref{lem_prod_cts}[(i)].
    The addition map is continuous, as aforementioned last lecture.
\end{proof}

\subsection{The Quotient Topology}

Let $X$ be a topological space, and $\sim$ an equivalence relation on $X$.
\begin{example}
    On $\mathbb{R}$, we can define an equivalence relation 
    stating that $x\sim y$ if and only if $x-y \in \mathbb{Z}$
\end{example}

\begin{example}
    If a group $G$ acts on $X$, then we set $x \sim y$ if and only if 
    there exists $g \in G$ such that $x = yg$. We can quotient by the 
    group action and obtain the space $X/G$.
\end{example}

Let $$\overline{X} := \lbrace [x] : x \in X\rbrace,$$
where $$[x] :=\lbrace y \in X : y \sim x\rbrace,$$
is the \emph{equivalence class} of $x$. 
There is a quotient map
$$q : X \longmapsto \overline{X},\quad x \longmapsto [x].$$
It is clear that $q$ is surjective.
We topologise $\overline{X}$ by defining open sets in $\overline{X}$
to be those sets $U$ for which $q^{-1}(U)$ is open under the topology on $X$.
Denote the resulting topology by $\overline{\tau}$.

\begin{example}
    Let $X = \mathbb{R}$, and define a $\mathbb{Z}$-action on 
    $\mathbb{R}$ by $n \cdot x := x + n$. This is equivalent to defining
    an equivalence $x \sim y$ on $\mathbb{R}$ defined by the condition that 
    $x-y \in \mathbb{Z}$. Then, the resulting quotient space is given by 
    $\overline{\mathbb{R}} = \mathbb{R}/\mathbb{Z}$, which is homeomorphic
    to $\mathbb{S}^1$, since $0\sim 1$.
\end{example}

\begin{lemma}
    Given a continuous map $f : X \to Y$, and 
    a quotient map $q : X \to \overline{X}$, there exists a 
    continuous map $\overline{f} : \overline{X} \to Y$ such that the diagram
    $$\begin{tikzcd}
X \arrow[rd, "q"'] \arrow[rr, "f"] &                                                  & Y \\
                                   & \overline{X} \arrow[ru, "\overline{f}"', dashed] &  
\end{tikzcd}$$
    commutes.
\end{lemma}

\begin{proof}
    Define $\overline{f}([x]) = f(x)$. Then, $\overline{f}$ is continuous
    and we are done.
\end{proof}

\begin{lemma}
    Any map $\overline{f} : \overline{X} \to Y$ is continuous
    if and only if $f := \overline{f} \circ q$ is continuous.
\end{lemma}

\begin{proof}
    Suppose $\overline{f}$ is continuous. Then, since $q$ is continuous,
    then $f = \overline{f} \circ q$ is also continuous. \\\\
    Suppose $f$ is continuous. Let $U \in \tau_Y$, and 
    consider the diagram 
    $$\begin{tikzcd}
X \arrow[rd, "q"'] \arrow[rr, "f"] &                                                  & Y \\
                                   & \overline{X} \arrow[ru, "\overline{f}"', dashed] &  
    \end{tikzcd}$$
    Observe that $f^{-1}(U) = q^{-1}(\overline{f}^{-1}(U))$. Since $f$ is continuous,
    it follows that $\overline{f}^{-1}(U)$ is also open.
\end{proof}

\begin{example}
    Consider the diagram
    $$\begin{tikzcd}
\mathbb{R} \arrow[rd, "q"'] \arrow[rr, "\exp"] &                                                           & \mathbb{S}^1 \\
                                               & \mathbb{R}/\mathbb{Z} \arrow[ru, "\overline{f}"', dashed] &             
    \end{tikzcd}$$
    where $\exp : t \mapsto e^{2\pi i t}$.
    Check that $\overline{f}$ is a continuous bijection.
    Since $\mathbb{R}/\mathbb{Z} = [0,1]/\sim$, it follows that 
    $\mathbb{R}/\mathbb{Z}$ is compact. Moreover, since $\mathbb{S}^1$
    is Hausdorff and compact, it follows then that $\overline{f}$
    is a homeomorphism.
\end{example}

\subsection{Topological Equivalence}

\begin{lemma}
    If $f : X \to Y$ and $g : Y \to Z$ are homeomorphisms, then 
    so are $g \circ f : Y \to Z$ and $f^{-1} : Y \to X$.
\end{lemma}

\begin{corollary}
    Homeomorphisms defines an equivalence relations on the set of all
    topological spaces.
\end{corollary}

\begin{exercise}
    Show that compactness is preserved under homeomorphisms.
\end{exercise}

\paragraph{Goal:} Define a weaker equivalence relation on topological spaces.\\\\
In particular, we wish to say that $X \cong Y$ if and only if 
there exists $f : X \to Y$ and $g : Y \to X$ such that $f\circ g = 1_Y$,
and $g\circ f = 1_X$.

\subsection{Homotopy}

Homotopy defines an equivalence relation on maps $X \to Y$. We say that 
$f,g : X \to Y$ are \emph{homotopic} if there exists a $H : I \times X \to Y$
such that $f = H\vert_{\lbrace 0 \rbrace \times X}$, and 
$g = H\vert_{\lbrace 1 \rbrace \times X}$.

\section{Lecture 3, 01/03/2024}

\begin{definition}
Let $f,g : X \to Y$ be continuous maps. Then, $f$ is \emph{homotopic} to $g$ --
denoted $f \sim g$ if there exists a map $H : X \times I \to Y$ such that 
$f = H \vert_{X \times \lbrace 0 \rbrace}$, and $g = H\vert_{X \times \lbrace 1\rbrace}$. 
\end{definition}
In effect, this allows us to ``interpolate'' between $f$ and $g$.

\begin{remark}
    One can topologise $\on{Map}(X,Y)$.
\end{remark}

Observe that we have a family of continuous maps 
$H_t := H\vert_{X \times \lbrace t\rbrace}$ 
for all $ t\in [0,1]$.

\begin{definition}
    A \emph{path} in a space $Z$ is a map $\gamma : I \to Z$. 
\end{definition}

With this definition, one can view homotopies as paths in the space of 
continuous maps $\on{Map}(X,Y)$. So, an alternate definition is to say that a 
homotopy is a path of continuous maps in $\on{Map}(X,Y)$.\\\\

\begin{example}
    Let $X = \mathbb{S}^1$, and $Y = \mathbb{R}^2 \supset \mathbb{S}^1$.
    Then, consider the maps $f,g : \mathbb{S}^1 \to \mathbb{R}^2$ by
    $f(x,y) = (x,y)$, and $g(x,y) = (2x,2y)$. 
    The first map is the identity map, and the second map scales 
    the radius of the circle by $2$.
    Define $$H : \mathbb{S}^1 \times I \longrightarrow \mathbb{R}^2,\quad ( (x,y),t) \longmapsto (1+t)(x,y),$$
    which defines a homotopy of $f$ into $g$.
    Observe in particular that $H$ lands in 
    $\mathbb{R}^2\setminus \lbrace 0\rbrace$.
    In fact, all maps into any $\mathbb{R}^n$ are homotopic. 
    Given any $f,g :  X \to \mathbb{R}^n$, we may define a homotopy 
    $$H : X \times I \longrightarrow \mathbb{R}^n,\quad (x,t) \longmapsto (1-t)f(x) + tg(x),$$
    called the \emph{straight line homotopy}.
\end{example}

\begin{example}
    Consider $f : \mathbb{S}^1 \to \mathbb{R}^2\setminus \lbrace0\rbrace$ and 
    $c : \mathbb{S}^1 \to \mathbb{R}^2\setminus \lbrace 0\rbrace$
    mapping $c : (x,y) \longmapsto x$. These maps are \emph{not} homotopic.
    We will see why this is the case later (in like, a month).
\end{example}

\subsection{Homotopy Equivalence of Spaces}

\begin{definition}
    A map $f : X \to Y$ is a \emph{homotopy equivalence} if and only if 
    there exists $g : Y \to X$ such that 
    $$g \circ f \sim \on{id}_X,\quad f \circ g \sim \on{id}_Y,$$
    are homotopic. If this is the case, then we say that $X$ and $Y$ 
    are \emph{homotopy equivalent} -- denoted by $X \simeq Y$.
\end{definition}

\begin{remark}
    Homeomorphisms are naturally homotopy equivalences. But this gives a much
    broader equivalence relation for topological spaces. 
\end{remark}
However, note that it remains to be shown that if 
$f : X \to Y$ and $h : Y \to Z$ are homotopy equivalences,
then so is $h \circ f$.\\\\
To show this, we know that $g\circ f \sim \on{id}_X$, $i \circ h \sim \on{id}_Y$,
where $i : Z \to Y$. Then, we wish to show that 
$(g\circ i ) \circ )(h\circ f) \sim \on{id}_X$.
Observe that this can be re-written as $g\circ (i\circ h)\circ f \sim \on{id}_Y$.
So, now we may construct a composition of maps
$$I \times X \stackrel{\on{id} \times f}{\longrightarrow} I \times Y \stackrel{H}{\longrightarrow} I \times Y \stackrel{\on{id} \times g}{\longrightarrow} I \times X.$$
It now remains to check that 
$(\on{id} \times f) \circ H \circ (\on{id} \times g)$ is a homotopy
from $g\circ i \circ h \circ f$ to $g \circ f$.
This follows as a result of the gluing lemma.

\begin{example}
    Let us show that $\mathbb{R}^n \simeq \on{pt}$, where $\on{pt}$ is the 
    one-point space.
\end{example}

\begin{proof}
    Define $c : \mathbb{R}^n \to \on{pt}$ and $i : \on{pt} \to \mathbb{R}^n$. 
    One checks that $c\circ i = \on{id}_{\on{pt}}$, and that 
    $i\circ c$ is the constant 
    map --- that is, every point 
    in $\mathbb{R}^n$ gets mapped to a point. Then just use the straight line homotopy.\\\\
    But all maps to $\mathbb{R}^n$ from any space are homotopic. Hence, $i \circ c$ is homotopic to $1_{\mathbb{R}^n}$. 
    We say that any space homotopy equivalen to $\on{pt}$ is \emph{contractible.}
    We say that a map homotopic to a constant map is \emph{null-homotopic}. Further, a non-null-homotopic map is \emph{essential.}
\end{proof}

\begin{example}
    Consider the identity map
    $1_{\mathbb{S}^1} : \mathbb{S}^1 \to \mathbb{S}^1$ 
    Is this essential or null-homotopic? The answer is that this map is essential.
\end{example}

\begin{example}
    $1_X : X \to X$ is null-homotopic if and only if $X$ is contractible.
\end{example}

Let us consider the map:
$$f_d : \mathbb{S}^1 \longrightarrow \mathbb{S}^1,\quad z\longmapsto z^d, \quad d\in \mathbb{N}.$$

\paragraph{Question:} Is $f_d$ homotopic or null-homotopic? 
\paragraph{Answer:} The answer is no. We will explain why in two weeks.\\\\
\begin{proposition}[Fascinating Fact]
    There exists essential aps $\mathbb{S}^{n+k} \to \mathbb{S}^k$ for $n,k$ with $k > 1$.
\end{proposition}

\begin{example}
    One such example of an essential map is the Hopf map from $\mathbb{S}^3 \to \mathbb{S}^2$.
\end{example}

\subsection{Cell Complexes}

\subsubsection{Cell Addition}

Let $f : \mathbb{S}^{n-1} \to A$ be a map. Let $D^n$ be the $n$0disk, and define 
$$A \cup_f D^n := (A \sqcup D^n)/\sim_f,$$
where $A \sqcup D^n$ is topologise by defining the open sets to be those either coming from $D^n$ or $A$, and
the equivalence relation
$\sim_f$ is given by 
$x \sim_f f(x)$ for all $x\in \mathbb{S}^{n-1}$.
Check that $a \sim a'$ if and only if $a = a'$, for $a,a' \in A, D^n\setminus \mathbb{S}^{n-1}$,
$x\sim x'$ if and only if $x=x'$ for $x,x' \in \on{Int}(D^n) := D^n\setminus \mathbb{S}^{n-1}$,
and $x\sim x'$ if and only if 
$f(x) = f(x')$ for $x,x' \in \mathbb{S}^{n-1}$.

\begin{example}
    We have that $$\mathbb{S}^1 = \on{pt} \cup_{\on{const}} D^1,$$
    where $\on{const}$ is the constant map.
\end{example}

\begin{example}
    Consider the $2$-torus 
    $T^2 = \mathbb{S}^1 \times \mathbb{S}^1$. 
    We may write this as 
    $$T^2 = (\mathbb{S}^1 \vee \mathbb{S}^1) \cup_\varphi D^2,$$
    where for $(X,x)$ and $(Y,x)$ pointed spaces, the product $\vee$ is the disjoint union
    $$X \vee Y := X \sqcup Y,$$
    obtained by setting $x=y$ and no other relations. Assume that 
    $\mathbb{S}^1 \times \mathbb{S}^1 \cong (I\times I)/\sim$,
    where $(x,0) \sim (x,1)$
    and $(0,y) \sim (1,y)$
    is the equivalence relation. Visually, this is the square construction where opposite ends are identified and glued together.
    Note that $I \times I \cong D^2$.
\end{example}

\begin{definition}[Informal Definition]
    $X$ is a \emph{CW-complex} if there is a fitration 
    $$X^0 \subset X^1 \subset X^2 \subset \cdots \subset  X,$$
    such that 
    \begin{itemize}
        \item[(i)] $X^0$ is discrete,
        \item[(ii)] (Inductive Step) $X^k = X^{k-1} \cup_{\varphi_i} D^k_i,$ where 
            $D^k_i$ are \emph{affine $k$-cells},
        \item[(iii)] $\dim X = \max \lbrace k : X^k \neq X^{k-1}\rbrace.$ If $\dim X$ is infinite, then $U\subset X$ is open if and only if $U\cap X^k$ is open for all $k$.
    \end{itemize}
\end{definition}

\begin{remark}[Diarmuid Wisdom]
    If you think about everything just purely in terms of CW-complexes, you are guaranteed to get a high distinction in this class!
\end{remark}

\begin{remark}[Fun Fact]
    Every manifold is a CW-complex except possibly in dimension $4$.
\end{remark}

\paragraph{Idea of the Class:} Start with topology. From our point of view, they are given by continuous maps between topological spaces $f : X \to Y$.\\\\
From this, we can get the ``algebra'' part of ``algebraic topology'' by considering homology groups $H_i(X)$ and $H_i(Y)$. Then, any continuous map $f : X \to Y$ 
induces a map $f_\ast : H_i(X) \to H_i(Y)$ in homology. Moreover,
if $f\sim g$ then $f_\ast \sim g_\ast$. From this, we can deduce that
homotopy equivalent spaces have isomorphic homotopy groups.\\\\
Given a map of pointed spaces 
$f : (X,x) \to (Y,y)$, we can associate to them homotopy groups $\pi_i(X)$ and $\pi_i(Y)$, which are ``similar'' to homology groups. It induces a map 
$f_\ast : \pi_i(X) \to \pi_i(Y)$. When $i=1$, we get an important class of groups called fundamental groups.

\chapter{Week Two}

\section{Lecture 1, 06/03/2024}

\subsection{Paths and Path Homotopy}

\begin{remark}[Unrecorded Diarmuid Wisdom]
    Should have at least $6$ hours of study time -- both prepping for lectures
    and working through exercises. Good idea to have a look at what lectures
    will be covering before the lectures.
\end{remark}

Let $X$ be a topological space. One simple way that we will make groups from
spaces is the idea of a path homotopy.

\begin{definition}
    A \emph{path} in $X$ is a continuous map $\gamma : I \to X$ such that 
    $f(0) = x_0$, and $f(1) = x_1$. 
\end{definition}

\begin{remark}
    The map does \emph{not} have to be injective -- i.e. it can have 
    self-intersections. 
\end{remark}

Many paths can have the same endpoints. To capture this idea, we may 
define the idea of a \emph{path homotopy}. In particular, 
we only consider paths up to homotopy.

\begin{definition}
    A \emph{path homotopy} $f_t$ is a homotopy fixed on 
    $I \times \lbrace 0,1\rbrace$.
    That is, it is a map $f_t : I \times I \to X$ defined by 
    $(s,t) \mapsto f_t(s)$.
\end{definition}

\begin{definition}
    $\pi_0(X)$ is the set of all equivalence classes of points 
    given by $x_0 \sim x_1$ if there is a path from $x_0$ to $x_1$.
\end{definition}

Indeed, $x_0 \sim x_0$ via the constant path $c_{x_0}$. Given $x_0 \sim x_1$,
we have $x_1 \sim x_0$ by taking the path going in the opposite direction.
If $f$ is a path from $x_0$ to $x_1$, we denote the path going in
the opposite direction by $\overline{f}(s) = f(1-s)$.
Given any a path $f$ given by $x_0 \sim x_1$, and $g$ given by $x_1 \sim x_2$, 
then we can glue the paths together the same way we glue continuous functions
together. In particular, $x_0 \sim x_2$ via 
$$f \cdot g (s) := \begin{cases}
    f(2s) \quad 0\leq s \leq \frac{1}{2}\\
    g(2s-1) \quad \frac{1}{2} \leq s \leq 1
\end{cases}.$$

\begin{definition}
    $X$ is \emph{path-connected} if $\vert \pi_0(X) \vert = 1$. 
    Equivalently, any pair of points on $X$ has a path between them.
\end{definition}

\begin{example}
    If $X = V$ is a topological vector space over $\mathbb{R}$ 
    (i.e. $V = \mathbb{R}^n$). If $f_0(0) = f_1(0)$, and 
    $f_0(1) = f_1(1)$, then $f_1$ and $f_2$ are path homotopic via
    the \emph{straight line homotopy}:
    $$f_t(s) = (1-t)f_0(s) + tf_1(s).$$
\end{example}

\begin{remark}
    The above construction does not make sense if we do not have a vector
    space structure on $X$. 
\end{remark}

\begin{proposition}
    Path homotopy defines an equivalence relation on the set of paths 
    from $x_0$ to $x_1$.
\end{proposition}

\begin{proof}
    Exercise. Use the gluing lemma.
\end{proof}

\subsection{The Fundamental Groupoid}

Recall that given $f_0(1) = f_1(0)$, we may \emph{concatenate} the paths by:
$$f_0 \cdot f_1 = \begin{cases}
    f_0(2s) \quad 0\leq s \leq 1/2\\
    f_1(2s-1) \quad 1/2 \leq s \leq 1
\end{cases}.$$

\begin{definition}
    The \emph{fundamental groupoid} -- denoted by $\pi_1(X)$ -- is the set 
    $$\pi_1(X) := \lbrace [f] : \text{path homotopy classes of all paths in $X$}\rbrace.$$
\end{definition}

Given a map 
$$\on{ep} : \pi_1(X) \longrightarrow X \times X,\quad [f] \longmapsto (f(0),f(1)),$$
then define 
$$\pi_1(X,x_0,x_1) := \on{ep}^{-1} (x_0,x_1).$$
Then, we have maps
$$\pi_1(X,x_0,x_1) \times \pi_1(X,x_1,x_2) \longrightarrow \pi_1(X,x_0,x_2),\quad \longmapsto ([f],[g]) \longmapsto ([f\cdot g]).$$
One can use this to show associativity.

\begin{definition}
    A path $f$ for which $f(0) = f(1)$ is called a \emph{loop}. 
    Let $\pi_1(X,x_0) := \pi_1(X,x_0,x_0)$ be the set of path homotopy
    classes of loops at $x_0$. This is called the \emph{fundamental group} of 
    $X$.
\end{definition}

We will now see why this is called a fundamental \emph{group}:

\begin{proposition}\label{prop_fund_group}
    $\pi_1(X,x_0)$ is a group, with identity given by the constant path 
    $[c_{x_0}]$ at $x_0$. Given any $[f] \in \pi_1(X,x_0)$, its inverse
    is given by $[\overline{f}]$ -- that is, $[f]^{-1} = [\overline{f}]$.
\end{proposition}

\begin{remark}
    Observe that $f \cdot (\cdot g \cdot h) \neq (f\cdot g)\cdot h$ in general, 
    since $f$ goes ``twice as fast'', but $g\cdot h$ goes four times 
    as fast. On the other hand, $f\cdot g$ goes four times as fast,
    but $h$ goes twice as fast. So they are not the same loop.
    Hatcher explains this idea much more nicely.
\end{remark}

First, a lemma:

\begin{lemma}[Reparametrisation]
    Let $\varphi : I \to I$ be a map such that 
    $\varphi(0) = 0$ and $\varphi(1) = 1$.
    Then, $f\circ \varphi \sim f$.
\end{lemma}

\begin{proof}
    Consider the straight line homotopy $\varphi_t(s) = (1-t) \varphi(s) - st$.
    Note that $\varphi_t(s)$ lies between $s$ and $\varphi(s)$ for all $t$.
    Then, $f_t = f \circ \varphi_t$ is a path homotopy from $f$ to 
    $f\circ \varphi$.
\end{proof}

\begin{proof}[Proof of Proposition \ref{prop_fund_group}]
    $f \cdot (g\cdot h)$ is obtained by a reparametrisation of 
    $(f\cdot g)\cdot h$ with a piecewise linear path (see Hatcher for the 
    picture). Call this graph $\Gamma(\varphi)$. Hence, 
    $[f\cdot (g\cdot h)] = [(f\cdot g)\cdot h]$. By definition,
    $([f] \cdot [g])\cdot [h] = [f] \cdot ([g] \cdot [h])$, 
    and this proves associativity.\\\\
    $f \cdot c$ just makes $f$ goes around twice as fast, and then does 
    nothing for the remaining time. $c\cdot f$ does nothing for half the time,
    then makes $f$ go aroudn twice as fast. To see this mathematically, 
    one reparametrises $\overline{c}$ as graphs (see Hatcher again for the 
    pictures).\\\\
    To show that $[f]^{-1} = [\overline{f}]$, one needs to show that 
    $\overline{f} \cdot f \sim c \sim f\cdot \overline{f}$. Set 
    $$f_t(s) = \begin{cases}
        f(t) \quad s \geq 1-t\\
        f(s) \quad 0\leq s \leq 1-t
    \end{cases}.$$
    Then, set $h_t := \overline{f}_t \cdot f_t$, which is always a loop.
    Indeed, $h_0 = \overline{f}_0 \cdot f_0 = \overline{f}\cdot f$,
    and similarly $h_1 = \overline{f}_1 \cdot f = \overline{c} \cdot c = c$.
    It is an exercise to check that continuity, and that $h_t$ defines a path 
    homotopy. Similarly, check that $\overline{h}_t = f_t \cdot \overline{f}_t$
    is a path homotopy from $f\cdot f$ to $c$.
\end{proof}

\begin{proposition}[Basepoint Independence]
    For any two points $x_0$ and $x_1$, $\pi_1(X,x_0) \cong \pi_1(X,x_1)$.
\end{proposition}


\begin{proof}
    If there exists a path $h$ from $x_0$ to $x_1$, then 
    we can define $$\beta_h : \pi_1(X,x_0) \longrightarrow \pi_1(X,x_1),\quad [f] \longmapsto [\overline{h}\cdot f \cdot h].$$
    If $[h] = [h'] \in \pi_1(X,x_0,x_1)$, then 
    $\beta_h = \beta_{h'}$. This allows us to define $\beta_{[h]}$ one the 
    equivalence classes of paths in $\pi_1(X,x_0,x_1)$.
    We show that $\beta_h$ is an isomorphism. 
    First, we see that 
    $$\beta_h ([f] \cdot [g]) = [\overline{h}\cdot f \cdot h\cdot \overline{h} \cdot g \cdot h] = [\overline{h} \cdot f] \cdot [\overline{h} \cdot g \cdot h] = \beta_h([f]) \cdot \beta_h([g]).$$
    We leave it as an exercise to check that 
    $\beta_{\overline{h}}$ is a suitable choice of inverse. That is,
    $$\beta_h \circ \beta_{\overline{h}} = \beta_{h\cdot \overline{h}} = \beta_{[c_{x_0}]} = \on{id}_{\pi_1(X,x_0)}.$$ 
    For the other direction, 
    $\beta_{\overline{h}} \circ \beta_h = \beta_{\overline{h}\cdot h}\on{id}_{\pi_1(X,x_1)}$.
\end{proof}

\begin{remark}[Diarmuid Wisdom]
    Learning when you have to care about basepoints and ignore them is an 
    important skill in algebraic topology.
\end{remark}

\section{Lecture 2, 07/03/2024}

\begin{definition}
    $X$ is called \emph{simply-connected} if 
    \begin{itemize}
        \item[(i)] $X$ is path-connected,
        \item[(ii)] $\pi_1(X,x_1,x_0) = 1$ -- that is, all choices of paths are 
            homotopic to one another
    \end{itemize}
\end{definition}

\begin{proposition}
    A path-connected space is simply-connected if and only if 
    $\pi_1(X)$ is trivial.
\end{proposition}

\begin{proof}
    The $\Longleftarrow$ direction is clear. In particular, 
    $\pi_1(X,x)$ acts freely and transitively on $\pi_1(X,x_0,x)$
    by concatenation. The proof of this is left as an exercise.
\end{proof}

\subsection{Fundamental Group of $\mathbb{S}^1$}

We wish to consider $\pi_1(\mathbb{S}^1,1)$.

For some $n \in \mathbb{Z}$, let $\omega_n(s) = e^{2\pi i ns},$ which is the loop that wraps around the circle $n$ times. Intuitively, this tells us that $\pi_1(\mathbb{S}^1)$ should be isomorphic to $\mathbb{Z}$. This turns out to be the case.

\begin{theorem}\label{thm_fundgrp_circle}
    $\pi_1(\mathbb{S}^1)\cong \mathbb{Z}$.
\end{theorem}

The proof uses the idea of a covering space.

\begin{definition}
    A \emph{covering space} of $X$ is a map $p : \widetilde{X} \to X$ 
    such that for every $x \in X$, there exists an open set $U$ containing $x$ such that $p^{-1}(U)$ is homeomorphic to a disjoint union $\sqcup_{\alpha \in \Lambda} U_\alpha$ such that $p\vert_{U_\alpha} : U_\alpha \stackrel{\simeq}{\to} U$.
    Such a $U$ is said to be \emph{evenly covered}.
\end{definition}

\begin{exercise}
    Show that $p : \mathbb{R} \to \mathbb{S}^1$ is a covering space.
\end{exercise}

We have some properties of covering spaces:
\begin{itemize}
    \item[(i)]
        Given a map $f : I \to X$ given by 
        $f(0) = x_0$, $f(1) = x_1$, then there 
        exists a unique $\widetilde{f} : I \to \widetilde{X}$ lifting $f$ -- that is, the lift $\widetilde{f}$ starts at 
        $\widetilde{f}(0) = \widetilde{x_0}$, and satisfies $p\circ \widetilde{f} = f$. That is, we have a commutative diagram:
        $$\begin{tikzcd}
                                                 & \widetilde{X}     \\
I \arrow[r, "f"] \arrow[ru, "\exists !", dashed] & X \arrow[u, "p"']
\end{tikzcd}$$
    \item[(ii)] For each homotopy $f_t : I \times I \to X$ of paths 
        starting at $x_0$ and $\widetilde{x_0} \in p^{-1}(x_0)$, there is a unique 
        lift $\widetilde{f}_t : I \to \widetilde{X}$ of paths starting at 
        $\widetilde{x_0}$ such that the diagram commutes  $$\begin{tikzcd}
                                                           & \widetilde{X}     \\
I \times I \arrow[r, "f_t"'] \arrow[ru, "\exists !", dashed] & X \arrow[u, "p"']
\end{tikzcd}$$
\end{itemize}

\begin{remark}
    The proof below is messily transcribed. You're much  better off just 
    going just reading page 30 of Hatcher, ``Proof of Theorem 1.7''.
\end{remark}

\begin{proof}[Proof of Theorem \ref{thm_fundgrp_circle}]
    Define a map 
    $$\Phi : \pi_1(\mathbb{S}^1) \longrightarrow \mathbb{Z},\quad [f]\longmapsto \widetilde{f}(1),$$
    which is well-defined by Property (ii) of covering spaces. Further, if $f \sim f'$ via $f_t$, then we have 
    $\widetilde{f}_t'(1) : I \to \mathbb{R}$, which is continuous, and $I$ is path-connected. Therefore, $f_t'(I)$ is path-connected and constant.
    This implies that 
    $\widetilde{f}_t'(1) = \widetilde{f}(1)$,
    and we thus have a well-defined map.
    By construction, we have $$\Phi([\omega_n]) = \widetilde{\omega_n}(1) = n.$$
    By Property (i), we then have that $\widetilde{\omega_n}(s) = ns$.\\\\
    Suppose that $\Phi([f]) = 0$. Then, 
    $\widetilde{f}(0) = 0 = \widetilde{f}(1)$.
    Take a straight line path homotopy from the loop $\widetilde{f}$ to the constant map $c_0$. Then, $p\circ \widetilde{f}_t$  is a path homotopy to $c_1$.
    \begin{exercise}
        Show that if $\Phi$ is a homotopy, then $\Phi$ is a bijection.
    \end{exercise}
    \begin{proof}
        Just translate lifts along to show this.
    \end{proof}

We have one more property of covering spaces:
\begin{itemize}
    \item[(iii)] 
    Given a homotopy $F : Y \times I \to X$, and a covering space $p : \widetilde{X} \to X$,
    there exists a unique lift $\widetilde{F}$ of $F$ wih $G : Y \times \lbrace 0 \rbrace \to \widetilde{X}$ 
    such that $p\circ G = F\vert_{Y \times \lbrace 0 \rbrace}$, and
    such that $\widetilde{F}\vert_{Y \times \lbrace 0 \rbrace} = G$.
    That is, we have a commutative diagram:
    $$\begin{tikzcd}
Y \times \lbrace 0 \rbrace \arrow[r, "G"] \arrow[d]       & \widetilde{X}     \\
Y\times I \arrow[r, "F"'] \arrow[ru, "\exists !", dashed] & X \arrow[u, "p"']
\end{tikzcd}$$
\end{itemize}

\begin{proof}[Proof of Property (iii)]
    Let $y \in Y$, and
    find a neighbourhood $y_0 \in N$ and construct $\widetilde{F}$ on $N\times I$. By continuity of $F$ and definition of product topologyon $N\times I$, there exists neighbourhoods
    $N_t \times (a_t,b_t)$ containing $(y_0,t)$
    such that $F(N_t \times (a_t,b_t)) \subset U_t$ for some evenly covered $U_t$. Since every point in $X$ has such a neighbourhood, we may take $F(y_0,t) = x_0$. By comppactness of $\lbrace y_0 \rbrace \times I$, a finite number of these cover $y_0 \times I$. There thus exists a partitionof $I$ $0 = t_0 < t_1 < \cdots < t_n = 1$ such that there exists a neighbourhood $N$ of $y_0$ such that $F(N\times [t_i,t_{i+1}])$ is evenly covered.\\\\
    Then, proceeding by induction, assume that $\widetilde{F}$ has been constructed on $N \times [0,t_1]$. Replace $N$ with $\widetilde{F}^{-1}\vert_{N \times [0,t_1]}(\widetilde{U}_t)$ for $\widetilde{F}(t_0,t_i) \subset \widetilde{U}_t$. Then, from here define $\widetilde{F}$ on $N \times [0,t_{i+1}]$ by $\widetilde{F} := (p\vert_{\widetilde{U}_{t_{i+1}}})^{-1} \circ F.$ Then, by gluing, $\widetilde{F}$ on $N \times [0,t_{i+1}]$ is continuous.
\begin{exercise}
    Show that $\widetilde{F}\vert_{y_0 \times I}$ is unique (or just look at Hatcher).
\end{exercise}
Now, just glue all these solutions together to obtain the cover 
$Y = \bigcup_{y_0}(I \times N_{y_0})$.
\end{proof}
\end{proof}

Observe that 
given the isomorphism $p : \mathbb{R} \to \mathbb{S}^1$, then observe that $\pi_1(\mathbb{S}^1) \cong \mathbb{Z} = p^{-1}(0)$. This is not a coincidence, and we ill come back in a month to see why this happens.

\begin{theorem}[Gauss' Fundamental Theorem of Algebra]
    Let $p(e)$ be a non-constant polynomial $p : \C \to \C$. Then, there exists $z_0 \in \C$ such that $p(z_0) = 0$.
\end{theorem}

\begin{proof}
    Without loss of generality, let $p(z) = z^n + a_1 z^{n-1} + \cdots + a_n$.
    For some $r \in [0,\infty)$, define 
    $$f_r(s) = \frac{p(re^{2\pi is})/p(r)}{\vert p(re^{2\pi is}/p(r)\vert}.$$
    Note here that $f_0 = c_1$. As $r$ varies continuously, $f_r$ is path-homotopic to $f_0$for all $r$.  Now, observe that $r > \vert a_1 \vert + \cdots + \vert a_n\vert$, and $r > 1$, and that $$\vert z^n \vert > (\vert a_1\vert + \cdots + \vert a_n \vert) \vert z \vert^{n-1} \geq \vert a_1 z^{n-1}\vert + \cdots + \vert a_n\vert \geq \vert a_1 z^{n+1} + \cdots + a_n\vert.$$
    It follows then that $p_t(z) = z^n +_ t (a_1z^{n-1} + \cdots + a_n)$. It follows then that 
    $p_t$ has no roots in $\mathbb{S}^1_r$ for $0\leq t \leq 1$.
    Then, $f_r = p_r(s)$, but $p_0(s) = \omega_n(s)$. It follows then that $0 = [f_0] \cdot [f_r] = [p_0] = [\omega_n] = n$.
    Thus $n=0$, and we are done.
\end{proof}

\section{Lecture 3, 08/03/2024}

Today, we will consider maps $\mathbb{S}^2 \to \mathbb{R}^2$, and we will
prove the Borsuk-Ulam theorem. 

\begin{theorem}[Borsuk-Ulam]
    Every continuous map $f : \mathbb{S}^1 \to \mathbb{R}^2$ maps
    pairs of antipodal points to the same point.
\end{theorem}

\begin{proof}
    Consider $g(x) = f(x) - f(-x)$. By contradiction, suppose that no such 
    $x$ exists for which $g(x) = 0$. 
    Then, $$g(x) = \frac{f(x)-f(-x)}{\vert f(x) - f(-x)\vert} \in \mathbb{S}^1,$$
    where by abuse of notation we denote the above function by $g$.
    set 
    $$\eta : \mathbb{S}^1 \to \mathbb{S}^1, \quad (x,y) \longmapsto (x,y,0)$$
    where the image of $\eta$ is a copy of a great circle in 
    $\mathbb{S}^2$. 
    Let $\widetilde{g\circ \eta}$ be the unique lift of $g\circ \eta$ -- 
    then, $\widetilde{g\circ \eta} (t + 1/2) = \widetilde{g\circ \eta}(t) + \frac{2k\pi + 1}{2}$, for some $k \in \mathbb{Z}$.
    Then, let us consider the continuous function 
    $$I \longrightarrow \frac{1}{2}\mathbb{Z},\quad t\mapsto \widetilde{g\circ \eta}(t + 1/2) - \widetilde{g\circ \eta}(t)$$.
    Then, since $\frac{1}{2}\mathbb{Z}$ is discrete, this function is constant.
    It follows then that $\widetilde{g\circ \eta}(1) = \widetilde{g\circ \eta}(1/2) + \frac{2k\pi + 1}{2}$, and so 
    $\widetilde{g\circ \eta}(0) = 0$, but 
    also $\widetilde{g\circ \eta}(0) = \frac{2k\pi + 1}{2}$, which is non-zero.
    It follows that 
    $g\circ \eta : I \to \mathbb{S}^1$ represents $[g\circ \eta] = 2k + 1\in \mathbb{Z} \cong \pi_1(\mathbb{S}^1)$, and moreover $2k+1 \neq 0$.
    But $\eta : I \to \mathbb{S}^1$ is path homotopic to a constant map.
    This then implies that $g\circ \eta$ is homotopic to a constant,
    and so $[g\circ \eta] = 0$, which is a contradiction.
\end{proof}

\begin{theorem}[Brower's Fixed Point Theorem, Thm 1.10, Hatcher]
    Let $h : D^2 \to D^2$ be a map. Then, there exists $x \in D^2$ such that
    $h(x) = x$, where
    $$D^2 = \lbrace x \in \mathbb{R}^2 : \vert x \vert \leq 1\rbrace.$$
\end{theorem}

\begin{definition}
    A \emph{retraction} if a map $r : X \to A$ 
    such that $r\vert_A = 1_A$.
\end{definition}

\begin{proof}[Proof Idea]
    We wish to construct a retraction $D^2 \to \mathbb{S}^1 \subset D^2$.
    Use the fact that $\pi_1(\mathbb{S}^1)\cong \mathbb{Z}$ to show this
    is impossible.\\\\
    Define $r : D^2 \to \mathbb{S}^1$ by $r(x) = L_x \cap \mathbb{S}^1$,
    where $L_x$ is the line passing through $x$.
    Then, we claim that if $h$ is continuous, then so is $r$.
    If $x \in \mathbb{S}^1$, then $r(x) = X$ as required. 
    See Hatcher for rest of the proof.
\end{proof}

\subsection{Induced Homomorphisms}

Given a map $\varphi: X \to Y$, the \emph{induced homomorphism} is the group 
homomorphism $$\varphi_\ast : \pi_1(X,x) \longrightarrow \pi_1(Y,\varphi(x)), \quad [f] \longmapsto [\varphi \circ f].$$
One checks that $\varphi_\ast$ is well-defined -- in particular, one checks 
that composition of maps commutes with compositions of path homotopy.

\begin{proposition}[Prop 1.12, Hatcher]
    Given two spaces $X$ and $Y$, we have
    $$\pi_1(X\times Y, (x,y)) \cong \pi_1(X,x) \times \pi_1(Y,y).$$
\end{proposition}

\begin{proof}
    Consider projections 
    $\on{pr}_X : X \times Y \to X$ and $\on{pr}_Y : X \times Y \to Y$.
    Then, any map $f : Z \to X\times Y$ is continuous if and only if 
    $\on{pr_X} \circ f$ and $f \circ \on{pr}_Y$ are continuous.
    Define $$\Phi : \pi_1(X\times Y,(x,y)) \longrightarrow \pi_1(X,x) \times \pi_1(Y,y),\quad [f] \longmapsto ([\on{pr}_X\circ f], [\on{pr}_Y\circ f]).$$
    $\Phi$ defines a group homomorphism since $(\on{pr}_X)_\ast$ and 
    $(\on{pr}_Y)_\ast$ define group homomorphisms.
\end{proof}

\begin{example}
    Let $\mathbb{T}^2$ be the $2$-torus. Then, 
    $\pi_1(\mathbb{T}^2) = \pi_1(\mathbb{S}^1\times\mathbb{S}^1) \cong \mathbb{Z} \times \mathbb{Z}$.
    Generally, $\pi_1(T^n) \cong \mathbb{Z}^n$.
\end{example}

\begin{definition}
    A map $r_t : I \times X \to A$ is a \emph{deformation retraction} if 
    \begin{itemize}
        \item[(i)] $r_0 = A$,
        \item[(ii)] $r_t\vert_A = 1_A$ for all $t$,
        \item[(iii)] $r_1 = A$ -- that is, $r_1$ is a retraction.
    \end{itemize}
\end{definition}

\begin{definition}
    If $\varphi_t$ is homotopic rel $x_0$, then 
    $\varphi_t\vert_{I \times \lbrace x_0 \rbrace} = c_{x_0}$.
    This implies then that $(\varphi_0)_\ast = (\varphi_1)_\ast$.
\end{definition}

\begin{proposition}[Prop 1.17, Hatcher]\label{prop1.17}
    If $A\subset X$ is a retraction, $x_0 \in A$, then 
    $\pi_1(A,x_0) \subseteq \pi_1(X,x_0)$ is a retraction.
    In particular, $i_\ast : \pi_1(A,x_0) \to \pi_1(X,x_0)$ is injective.
    Moreover, if $A\subseteq X$ is a deformation retract, then 
    $i_\ast : \pi_1(A,x_0) \to \pi_1(X,x_0)$ is an isomorphism.
\end{proposition}

\begin{remark}
    Note that $\pi_1(-)$ defines a functor 
    $$\pi_1 : \mathbf{Top} \longrightarrow \mathbf{Ab},$$
    where $\mathbf{Top}$ is the category of topological spaces,
    and $\mathbf{Ab}$ is the category of abelian groups.
\end{remark}

\begin{proof}[Proof of Proposition \ref{prop1.17}]
    $r\circ i = 1_A$, and thus we have that $(r\circ i)_\ast = 1$.
    It follows then that $r_\ast \circ i_\ast = 1$, and thus 
    $i_\ast$ is injective.\\\\
    If $r_t$ is a deformation retraction, then $i\circ r : X \to A$ 
    is homotopic rel $x_0$ to $1_X$.
\end{proof}

\begin{corollary}[Corollary 1.16, Hatcher]
    $\mathbb{R}^2 \not\cong \mathbb{R}^n$, for $n \neq 2$.
\end{corollary}

\begin{proof}
    By contradiction, suppose that 
    $$f : \mathbb{R}^2 \longrightarrow \mathbb{R}^n$$
    is a homeomorphism. Then, 
    $$f\vert_{\mathbb{R}^2\setminus \lbrace 0\rbrace} :\mathbb{R}^2 \setminus \lbrace 0 \rbrace \longrightarrow \mathbb{R}^n\setminus \lbrace f(0)\rbrace,$$
    is a homeomorphism.
    But then we have a map 
    $$(f\vert_{\mathbb{R}^2\setminus\lbrace0\rbrace})_\ast : \pi_1(\mathbb{S}^1) \longrightarrow \pi_1(\mathbb{S}^{n-1}),$$
    and by Proposition \ref{prop1.18}, $\pi_1(\mathbb{S}^{n-1}) = 0$
    for $n \geq 2$. But $\pi_1(\mathbb{S}^1)\cong \mathbb{Z}$, which is 
    not isomorphic to $0$. A contradiction.
\end{proof}

\chapter{Week Three}

\section{Lecture 1, 13/03/2024}

\subsection{Homotopy rel $A$}

Maps of pairs are given by maps
$$f : (X,A) \longrightarrow (Y,B),$$
such that the map $f : X \to Y$ has the property that 
$f(A) \subseteq B$. If we specalise $A = \lbrace x_0 \rbrace$ and 
$B = \lbrace y_0\rbrace$ to be singleton sets, then we get maps 
of pointed spaces, which are maps such that $f(x_0) = y_0$.
Then, there is a notion of a \emph{homotopy rel $A$}. There is a pointwise
and a set-wise definition of this notion.
Given maps $$f,g : (X,A) \longrightarrow (Y,B),$$
we have maps $$I \times A \longrightarrow I \times B, \quad H : I\times X \longrightarrow Y.$$
Typically, for a deformation retraction we require that $f\vert_A = g\vert_A$,
and $H\vert_{I\times A} = f\vert_A \circ \on{pr}_A$.
We mention this because the definition of a deformation retraction is nothing 
more than a pointwise homotopy rel $A$. For a setwise homotopy rel $A$, we 
simply require that $H\vert_{\lbrace t\rbrace \times X}$ is a map of 
pairs.

\subsection{Pointed Homotopy}

Let $$[X,Y] := \lbrace [f] : f : X \to Y\rbrace,$$
where $[f]$ denotes the homotopy class of $f$.
Given pointed spaces, we have:
$$[(X,x), (Y,y)]_\ast := \lbrace f_\ast : f : (X,x) \longrightarrow (Y,y)\rbrace,$$
where $[f]_\ast$ denotes the \emph{pointed homotopy class} of $f$.\\\\
The key point is that 
$$\pi_1(X,x_0) = [(\mathbb{S}^1,1), (X,x_0)]_\ast, [f] \longmapsto [\overline{f}],$$
where $\overline{f}$ is the induced map on 
$I\setminus \lbrace 0,1\rbrace \to \mathbb{S}^1$.

\begin{remark}
    We have:
    $$\pi_0(X) = [(\mathbb{S}^0,1), (X,x_0)]_\ast,$$
    $$\pi_1(X) = [(\mathbb{S}^1,1), (X,x_0)]_\ast,$$
    $$\pi_2(X) = [(\mathbb{S}^2,e_1), (X,x_0)]_\ast,$$
    which is pointed-homotopy invariant.
    Moreover, $\pi_3(X) = [(\mathbb{S}^3,e_1)$,
    and all $\pi_i(X)$ are abelian groups. These are examples
    of \emph{higher homotopy groups}, and they will be discussed more in Week 
    $11$.
\end{remark}

Hatcher has some sneaky proofs for the following results:

\begin{proposition}[Prop 1.14, Hatcher]\label{prop1.14}
    $\pi_1(\mathbb{S}^n,x_0)$ is trivial for all $n \geq 2$.
\end{proposition}

\begin{proposition}[Prop 1.18, Hatcher]\label{prop1.18}
    If $\varphi : X \to Y$ is a homotopy equivalence between
    path-connected spaces, then the induced map:
    $$\varphi_\ast :  \pi_1(X,x_0) \longrightarrow \pi_1(Y,\varphi(x_0)),$$
    is an isomorphism.
\end{proposition}

Before we can prove Prop \ref{prop1.18} from Hatcher, we need the following 
lemma:

\begin{lemma}[Lemma 1.19, Hatcher]\label{lem1.19}
    If $\varphi_t : X \to Y$ is a homotopy, then the diagram $$\begin{tikzcd}
                                                                              & {\pi_1(Y,\varphi(x_0))} \arrow[dd, "\beta_h"] \\
{\pi_1(X,x_0)} \arrow[ru, "(\varphi_0)_\ast"] \arrow[rd, "(\varphi_1)_\ast"'] &                                               \\
                                                                              & {\pi_1(Y,\varphi(x_1))}                      
\end{tikzcd}$$
    commutes.
\end{lemma}

\begin{proof}
    Go see Hatcher.
\end{proof}

\begin{proof}[Proof of Proposition \ref{prop1.18}]
    Consider the composition of maps 
    $X \stackrel{\varphi}{\to} Y \stackrel{\psi}{\to} X$,
    and let $\psi$ be a homotopy inverse to $\varphi$.
    That is, $\phi \circ \varphi \sim 1_X$, and 
    $\psi \circ \varphi \sim 1_Y$.
    Thus, there are induced maps
    $$\pi_1(X,x_0) \stackrel{\varphi_\ast}{\longrightarrow} \pi_1(Y,\varphi(x_0)) \stackrel{\psi_\ast}{\longrightarrow} \pi_1(X,\psi(\varphi(x_0))) \stackrel{\varphi_\ast}{\longrightarrow} \pi_1(Y,\varphi(\psi(\varphi(x_0)))),$$
    and by Lemma \ref{lem1.19},
    $\psi_\ast \circ \varphi_\ast$ is an isomorphism, and 
    $\varphi_\ast \circ \psi_\ast$ is an isomorphism.
\end{proof}


\paragraph{Final word on basepoints} There are maps
$$\pi_1(X,x_0) = [(\mathbb{S}^1,1),(X,x_0)]_\ast \longrightarrow [(\mathbb{S}^1,1), (X,x_0)] \longrightarrow [\mathbb{S}^1,X],$$
obtained by forgetting the relevant structure with each sucessive map.
In particular, we may show that the conjugacy classes of $\pi_1(X,x_0)$ 
is equal to $[\mathbb{S}^1,X]$.\\\\
Now we can prove our first negative result:

\begin{proof}[Proof of Proposition \ref{prop1.14}]
    We may cover $\mathbb{S}^n$ by $$\mathbb{S}^n = A \cup B,$$
    where $A = \mathbb{S}^n \setminus \lbrace \text{north pole}\rbrace$,
    and $B = \mathbb{S}^n \setminus \lbrace \text{south pole}\rbrace$.
    As spaces, $A$ and $B$ are homeomorphic to $\mathbb{R}^n$, and 
    thus retractible -- that is, $\pi_1(A) \cong \pi_1(B) \cong 1$.
    Given path $f$, if $f(I) \subseteq B$, then $[f]$ is in the 
    image of the map $\pi_1(B) \to \pi_1(\mathbb{S}^n)$, and thus $[f]$ 
    is trivial since the image of a trivial group is trivial.
    The same is true for $A$.\\\\
    The goal is to show that $f$ is path-homotopic to $[f_1]\ast \cdots \ast [f_n]$, where $f_i(I)$ lies either in $A$ or $B$. To do this, we wish to cover
    the interval $I$ by $f^{-1}(A)$ and $f^{-1}(B)$, and use the compactness
    of $I$ to find a partition of $I$ such that $f([t_i,t_{i+1}])$ is contained 
    in either $A$ or $B$.
    Suppose that $f_i([t_i,t_{i+1}])$ is contained in $A$, and assume
    inductively that $f(t_i) = x_0$. Since $A\cap B$ is path-connected,
    define a path $h$ from $f(t_{i+1})$ to $x_0$. Then, add
    $h_i\overline{h_i}$ to $f_i$. Then, 
    define $$f_i := f\vert_{[t_i,t_{i+1}]},$$
    from which we obtain:
    $$f = f_0 \ast f_1 \ast \cdots \ast f_m = (f_0 \ast h_0) \ast (\overline{h_0} \ast f_1 \ast h_1) \ast (\overline{h_1}) \ast f \ast h_2) \ast \cdots \ast (\overline{h_m}\ast f_m),$$
    and each $f_i \ast h_i$ has image in either $A$ or $B$.
\end{proof}

\section{The Vam Kampen Theorem}

\paragraph{Goal:} Suppose that $X = \bigcup_\alpha A_\alpha$ such that 
$x_0 \in A_\alpha$ is path-connected, and 
$A_\alpha \cap A_\beta$ is path-connected for all $\alpha,\beta$,
and $A_\alpha \cap A_\beta \cap A_\gamma$ is path connected for all
$\alpha,\beta,\gamma$.
Then, we wish to determine $\pi_1(X,x_0)$ in terms of the components
$\pi_1(A_\alpha, x_0)$.

\subsection{Some Algebra}
To achieve the goal, we need some algebra.

\begin{definition}
    The \emph{free product} of groups $G_\alpha$ is the group 
    $\ast_\alpha G_\alpha$, whose underlying set is given by:
    $$W  = \lbrace \text{reduced words from $G_\alpha$} : [\omega]\rbrace.$$
    In particular, a \emph{reduced word} is a string 
    $g_1\cdots g_m$ where 
    $g_i \in G_{\alpha_i}\setminus \lbrace e_{\alpha_i}\rbrace,$
    and if $g_i \in G_{\alpha_i}$, then $g_{i+1}\not\in G_{\alpha_i}$.
    such 
    An \emph{alphabet} is the set 
    $$\sqcup_\alpha (G_\alpha \setminus \lbrace e_\alpha\rbrace).$$
    In fact, reduced words can be thought of as equivalence classes of all words.
    The group operation $\ast$ is given by \emph{juxtaposition} and 
    \emph{reduction}: given two reduced words $\omega_1 = g_1\cdots g_m$ and 
    $\omega_2 = h_1\cdots h_n$, then 
    $$\omega_1 \omega_2 = g_1\cdots g_m h_1\cdots h_n.$$
\end{definition}

\section{Lecture 2, 14/03/2024}

Recall that the general setting of the van Kampen theorem is that we 
are trying to compute the fundamental group of a space after we have 
chopped it up into a bunch of pieces whose intersections are path-connected.
The required algebra is the algebra of the free product.\\\\
The empty word $\phi$ denotes the identity in the free group 
$\ast_\alpha G_\alpha$. One readily checks that inverses hold.
Indeed, associativity holds as well, and one can simply grind it out. 
But there is a much more clever way of proving this.
Let $\mathfrak{S}_W$ denote the permutation group on the set $W$ of reduced 
words, whose elements are bijections $K : W \to W$ under composition. Define
a map $$W \longrightarrow \mathfrak{S}_W,\quad g \longmapsto L_g,$$
where $L_g$ is the left-multiplication map acting on a reduced $\omega$ by
$L_g(\omega) = g\omega$. From this, one then checks that
$L_{gg'} = L_g L_{g'}$. From this, one can then define the map
$$\Phi : W\longrightarrow \mathfrak{S}_W, \quad g_1,\cdots,g_m \longmapsto L_{g_1\cdots g_m},$$
which is injective since $L_{g_1\cdots g_m}(\phi) = g_1\cdots g_m$.
It follows then that if $L_{g_1\cdots g_m} = L_{g_1'\cdots g_m'}$,
then $g_1\cdots g_m = g_1'\cdots g_m'$.
By construction, $\Phi$ maps juxtaposition in $W$ to composition in 
$\mathfrak{S}_W$. It is injective, and thus juxtaposition is associative
since composition is associative in $\mathfrak{S}_W$.
It thus follows that $\ast_\alpha G_\alpha$ is a group.

\begin{example}[The Free Product on $2$ Words]
    Let $$F_2 := \mathbb{Z} \ast \mathbb{Z} = \lbrace a_1^{k_1}b_1^{k_2}\cdots\rbrace,$$
    where each $k_i \in \mathbb{Z}\setminus \lbrace 0 \rbrace$ for each 
    $i > 2$.
    Then, we can define $F_n := \mathbb{Z} \ast \cdots \ast \mathbb{Z}$.
\end{example}

\subsection{Amalgamated Free Product}

We have that 
$\pi_1(A_\alpha \cap A_\beta) = \lbrace e\rbrace$. Then, we have a commutative
diagram $$\begin{tikzcd}
                                                         & {\pi_1(A_\alpha,x_0)} \arrow[rd] &                \\
{\pi_1(A_\alpha \cap A_\beta,x_0)} \arrow[ru] \arrow[rd] &                                  & {\pi_1(X,x_0)} \\
                                                         & {\pi_1(A_\beta,x_0)} \arrow[ru]  &               
\end{tikzcd}$$
The maps going into $\pi_1(A_\alpha,x_0)$ and $\pi_1(A_\beta,x_0)$ land in 
completely different homotopy classes. So, in order to ensure that this does
not happen, it is useful to define the \emph{amalgamated free product}.\\\\
In general, suppose that we have 
$$\begin{tikzcd}
                                                & G_\alpha \\
K \arrow[ru, "i_\alpha"] \arrow[rd, "i_\beta"'] &          \\
                                                & G_\beta 
\end{tikzcd}$$
Then, define 
$$G_\alpha \ast_K G_\beta := G_\alpha \ast G_\beta /N,$$
where $N$ is the smallest normal subgroup of $G_\alpha \ast G_\beta$
containing the elements of the form $i_\alpha(k) i^{-1}_\beta(k) = i_\alpha(k)i_\beta(k^{-1})$. That is, a typical element of $N$ has the form
$$h_1 i_\alpha(k_1) i_\beta(k_1^{-1}) h_1^{-1} h_2i_\alpha(k_2)i_\beta(k_2^{-1})h_2^{-1}\cdots$$
The grup $G_\alpha \ast_K G_\beta$ is called the \emph{amalgamated free product}
of $G_\alpha$ and $G_\beta$.\\\\
More generally, we have
$$\begin{tikzcd}
                                                                 & G_\alpha \\
    {K_{\alpha,\beta}} \arrow[ru, "i_{\alpha\beta}"] \arrow[rd, "i_{\beta\alpha}"'] &          \\
                                                                 & G_\beta 
\end{tikzcd}$$
for all $\alpha \neq \beta$. From this, we take the normal subgroup to be
$$N = \langle i_{\alpha\beta}(k)i_{\beta\alpha}(k^{-1}), i_{\beta\alpha}(k)i_{\alpha\beta}(k{-1})\rangle.$$
This then defines an amalgamated free product of $G_\alpha$ over all
$\alpha,\beta$. We denote this by $\ast_{K_{\alpha,\beta}}G_\alpha.$

\begin{remark}
    Can reduce this to $$K_\alpha /\ker (i_{\alpha\beta}\cap \ker (i_{\beta\alpha}),$$
    and we have maps induced maps $\overline{i_{\alpha\beta}}$, and 
    $\overline{i_{\beta\alpha}}$.
\end{remark}

\begin{example}
    There is an isomorphism
    $$\on{SL}_2(\mathbb{Z}) \cong \mathbb{Z}/4\mathbb{Z} \ast_{\mathbb{Z}/2\mathbb{Z}} \mathbb{Z}/6\mathbb{Z}.$$
    In this case, we have the diagram:
    $$\begin{tikzcd}
                                                                     & \mathbb{Z}/6\mathbb{Z} \arrow[rd] &                                                                             \\
\mathbb{Z}/2\mathbb{Z} \arrow[ru, "i_\alpha"] \arrow[rd, "i_\beta"'] &                                   & \mathbb{Z}/6\mathbb{Z} \ast_{\mathbb{Z}/2\mathbb{Z}} \mathbb{Z}/4\mathbb{Z} \\
                                                                     & \mathbb{Z}/4\mathbb{Z} \arrow[ru] &                                                                            
\end{tikzcd}$$
\end{example}

\begin{example}
    $$\begin{tikzcd}
    \pi_1(A_\alpha \cap A_\beta) \arrow[r] \arrow[d] & \pi_1(A_\alpha) \arrow[d]                                                             &          \\
    \pi_1(A_\beta) \arrow[r]                         & \pi_1(A_\alpha) \ast_{\pi_1(A_\alpha\cap A_\beta)} \pi_1(A_\beta) \arrow[r, "\simeq"] & \pi_1(X)
    \end{tikzcd}$$
    $$\begin{tikzcd}
    \mathbb{Z} \arrow[r] \arrow[d, "i"'] & \lbrace e \rbrace \arrow[d]        \\
    \pi_1(A_\beta) \arrow[r]             & \pi_1(A_\beta)/\langle i(1)\rangle
    \end{tikzcd}$$
    $$\begin{tikzcd}
    K \arrow[r, "\operatorname{id}_K"] \arrow[d] & K \arrow[d]        \\
    G \arrow[r]                                  & G \cong G \ast_K K
    \end{tikzcd}$$
\end{example}

\subsection{Presentations of Groups}

A \emph{presentation} of a group is given by:
$$G = \langle g_1,\cdots, g_m : r_1,\cdots,r_k\rangle,$$
where $g_1,\cdots, g_m$ are the generators, and 
$r_1,\cdots,r_k$ are relations that the generators satisfied.
One can re-write $G$ in terms of a free product by:
$$G = \ast_{i=1}^m \mathbb{Z}/ N\langle r_1,\cdots,r_k\rangle,$$
where $N\langle r_1,\cdots,r_k\rangle$ is the normal subgroup
generated by the relations.

\begin{example}
    $$\ast_2\mathbb{Z} = \langle g_1,g_2 : \text{no relations}\rangle.$$
    $$\mathbb{Z}^2 = \langle g_1,g_2 : [g_1,g_2] = g_1g_2g_1^{-1}g_2\rangle,$$
    $$\mathbb{Z}/d\mathbb{Z} = \langle g : g^d = 0\rangle.$$
\end{example}

\subsection{Van Kampen's Theorem}

\begin{theorem}[(Vam Kampem) Thm 1.20, Hatcher]
    Let $X = \bigcup_\alpha A_\alpha$, where each $A_\alpha$ is open,
    and each $A_\alpha \cap A_\beta$ is path-connected. Let 
    $x_0 \in A_\alpha$. 
    Then, $i_\alpha : A_\alpha \hookrightarrow X$ be the injection map.
    Then, we have maps: $$\ast_\alpha (i_\alpha)_\ast : \ast_\alpha \pi_1(A_\alpha,x_0) \longrightarrow \pi_1(X,0)$$
    and $$i_{\alpha\beta}: A_\alpha \cap A_\beta \longrightarrow A_\alpha.$$
    Then, $\Phi$ is surjective.\\\\
    Moreover, if $A_\alpha \cap A_\beta \cap A_\gamma$ are path-connected
    for all $\alpha,\beta,\gamma$, then $\ker\Phi$ is the subgroup normally
    generated by 
    $$(i_{\alpha\beta})_\ast(k) (i_{\beta\alpha})_\ast(k^{-1}), \quad (i_{\beta\alpha})_\ast(k) (i_{\alpha\beta})_\ast(k^{-1}).$$
\end{theorem}

\begin{example}
    Recall that we proved that $\pi_1(\mathbb{S}^1,1) = \lbrace e\rbrace$
    follows from van Kampen by setting covering $\mathbb{S}^2$ by 
    $U_N \cap U_S$, where $U_N$ and $U_S$ are the the sphere without 
    the north and south poles. 
    Then, we have a map 
    $$\pi_1(U_N) \ast_{\pi_1(U_N \cap U_S)} \pi_1(U_S) \longrightarrow \pi_1(\mathbb{S}^2),$$
    but since $U_N \cong U_S \cong \mathbb{R}^2$, it is thus retractible
    and has trivial fundamental group. 
    Thus, $\pi_1(\mathbb{S}^2)$ is trivial.
\end{example}

\begin{example}
    Cover $\mathbb{S}^1 = U_N \ast U_S$ similarly as before.
    We can argue by a similar argument that $U_N$ and $U_S$ 
    retractible, and thus conclude that $\pi_1(\mathbb{S}^1)$ is trivial.
    But this is not true, as we already know. 
    What went wrong here is that $U_N \cap U_S = (0,1) \sqcup (0,1)$, 
    which is not path-connected.
\end{example}

\subsection{Cell Addition and Cell Complexes}

Define a map 
$$\varphi : \mathbb{S}^{k-1} \longrightarrow X,$$
and define a path $\gamma$ going from $x_0$ to $\varphi(\ast)$.
Then, we can define a space
$$Y = X \cup_\varphi D^k.$$
Let us restrict ourselves to the case where $k=2$. We wish to consider 
maps $$\pi_1(X,x_0) \longrightarrow \pi_1(Y,x_0).$$
Our goal is to derive a van Kampen decomposition of 
$\pi_1(Y,x_0)$.
In this case, consider:
$$X \cup_\varphi \underbrace{( (1,1-\varepsilon) \times \mathbb{S}^{k-1})}_{\subseteq D^k},$$ 
which deformation retracts back onto $X$ by construction.
Let us define $$Y = X \cup_\varphi ( (1,1-\varepsilon) \times \mathbb{S}^{k-1}) \cup \on{Int}(D^k),$$ 
Then, we have the van Kampen diagram:
$$\begin{tikzcd}
{\pi_1( (-1,1-\varepsilon) \times \mathbb{S}^{k-1})} \arrow[r] \arrow[d] & \pi_1(D^k) \cong \lbrace e\rbrace                            \\
{\pi_1(X,x_0) \cong \pi_1(A_\alpha,x_0)} \arrow[r]                       & {\pi_1(X,x_0)/N\langle \mathbb{Z} (h\varphi \overline{h})\rangle} 
\end{tikzcd}$$
The point is that if we understand van Kampen's theorem, then we understand
how the fundamental group of cells change under cell addition.

\section{Lecture 3, 03/15/2024}

Given any map $$f: X \longrightarrow Y,$$ 
then the \emph{mapping cylinder} is 
$$\on{Cyl}(f) := ( (I\times X) \sqcup Y)/ \simeq,$$ where $(x,1) = f(x)$.
As Assignment One will show, the mapping cylinder is homotopy equivalent to $Y$.
The \emph{cone} $CX$ of $X$ is:
$$CX := (X\times I)/\sim,$$ where $(x,0) \sim (x',0)$.
As an example,
$$C\mathbb{S}^{n-1} \cong D^n.$$
The \emph{mapping cone} $\on{Cone}(f)$ of $f$ is given by:
$$\on{Cone}(f) := (CX \sqcup Y)/\simeq,$$
where $[x,1] \simeq f(x)$.

\begin{example}
    Cell attachment is equivalent to the mapping cone of a map
    $f : \mathbb{S}^{n-1} \to Y$.
    In particular,$$Y \cup_f D^n = \on{Cone}(f).$$
\end{example}

\subsection{Wedge Spaces}

\begin{definition}
    $(X,x)$ is \emph{well-pointed} if for each open neighbourhood $U$ of $x$
    is contractible.
\end{definition}

\begin{example}
    Manifolds $M$ are well-pointed for all $m \in M$.
\end{example} 
Suppose that $X = \sqcup_\alpha X_\alpha$ is a wedge product of well-pointed base points 
$x_\alpha \in X_\alpha$. Then, 
\begin{equation}\label{eqn_thing}
    \pi_1(X)  \cong \ast_\alpha \pi_1(X_\alpha, x_\alpha).
\end{equation}

\begin{proof}
    Set $A_\alpha = X_\alpha \bigcup_{\beta \neq \alpha} U_\beta,$
    of which $x_\beta$ is an element. Then,
    $X_\alpha \to A_\alpha$ is homotopy equivalent. We have:
    $$X_\alpha \cap X_\beta = U_\alpha \vee_{x_\alpha = x_\beta} U_\beta,$$
    is path-connected. Then, the triple intersections are given by
    $$X_\alpha \cap X_\beta \cap X_\gamma = U_\alpha \vee U_\beta \vee U_\gamma,$$
    which is also path-connected. Moreover, they are path-connected and contractible.
    Therefore, by van Kampen's theorem, 
    $$\Phi : \ast_\alpha (X_\alpha,x_\alpha) \longrightarrow \pi_1(X),$$
    is surjectivewith trivial kernel.
    Thus, the isomorphism \eqref{eqn_thing} holds.
\end{proof}

\begin{example}
    Generally, 
    $$\pi_1\left(\vee_{i=1}^n \mathbb{S}^1\right) \cong F_n,$$
    where $F_n$ is the free group on $n$ generators.
\end{example}

\subsection{Back to Cell Attachment}

Let $(X,x_0)$ be path-connected.
Then, let 
$$Y := X^{k+1} :=  X \cup_{\sqcup_i \varphi_i} \left(\bigsqcup_{i=1}^n D^k\right).$$

\begin{proposition}[Prop 1.26, Hatcher]
    In the situation above, suppose that $k\geq 2$,
    and consider the map
    $$i_\ast : \pi_1(X,x_0) \longrightarrow \pi_1(Y,x_0).$$
    \begin{itemize}
        \item[(a)] If $k = 2$, then $i_\ast$ is surjective, with kernel
        normally generated by the classes $[h_i\varphi \overline{h_i}]$,
        for $i=1,\cdots,n$.
    \item[(b)] If $k \geq 3$, then $i_\ast$ is an isomorphism.
    \item[(c)] The inclusion of the $2$-skeleton $X^2$ in 
        $X^2 \hookrightarrow X$ of a 
        CW-complex induces an isomorphism $\pi_1(X^2,x_0) \to \pi_1(X,x_0)$.
\end{itemize}
\end{proposition}

\begin{proof}
    \leavevmode
    \begin{itemize}
        \item[(a)] Let $0$ be the centre of $D^2$, and take 
            $$A_\alpha = (Y \setminus \lbrace 0 \rbrace) \cup_{\widetilde{h}} (I\times I),$$
            where $$\widetilde{h} : (I \times \lbrace 0 \rbrace) \cup (\lbrace 0 \rbrace \times I) \longrightarrow Y,$$
            and let $h$ be the map 
            $h : \lbrace 0 \rbrace \times I \to Y$.
            The map $\widetilde{h}$ pushes into the disc $\varphi^{-1}(\ast) = h(1)$.
            It follows then that $A_\beta = \on{Int}(D^2)$.
            First, let us define $$\widetilde{Y} := Y \cup (I \times I),$$
            and note that $Y \subseteq \widetilde{Y}$ is a deformation retract.
            Moreover, $X\subset A_\alpha$ as a deformation retraction.
            $A_\alpha \cap A_\beta$ contains $\mathbb{S}^1 \subseteq \on{Int}(D^2)$
            as a deformation retraction. In praticular, $A_\alpha \cap A_\beta$ is 
            homotopy equivalent to $\mathbb{S}^1$. By construction,
            $A_\beta \cong \on{Int}(D^2)$, which is contractible.
            From this, one then obtains the following van Kampen diagram:
            $$\begin{tikzcd}
\pi_1(A_\alpha \cap A_\beta) \arrow[r] \arrow[d] & \pi_1(A_\alpha) \arrow[d] \\
\pi_1(A_\beta) \arrow[r]                         & \pi_1(\widetilde{Y})     
\end{tikzcd}$$
            which is, explicitly:
            $$\begin{tikzcd}
                \mathbb{Z} \arrow[r, "i"] \arrow[d] & \pi_1(\widetilde{X}) \arrow[d]          \\
                \lbrace e \rbrace \arrow[r]                         & \pi_1(\widetilde{X})/N\langle i(1)\rangle
            \end{tikzcd}$$
            where we have that $\pi_1(A_\alpha \cap A_\beta) \cong \pi_1(\mathbb{S}^1)\cong \mathbb{Z}$,
            and $\pi_1(A_\beta) \cong 0$ since its retractible,
    \item[(b)] The proof is similar to the $k=2$ case.
        By Proposition \ref{prop1.14}, $\pi_1(\mathbb{S}^{k-1}) = \lbrace e\rbrace$.
        So, we obtain the van Kampen diagram:
        $$\begin{tikzcd}
    \lbrace e\rbrace \arrow[r, "i"] \arrow[d] & \pi_1(\widetilde{X}) \arrow[d]      \\
    \lbrace e \rbrace \arrow[r]               & \pi_1(\widetilde{Y}) \cong \pi_1(Y)
\end{tikzcd}$$
    \item[(c)] Follows from (b) via an inductive argument.
\end{itemize}
\end{proof}

\begin{example}[Closed Surfaces]
    Let $M_g$ be a compact, connected surface of genus $g$, and let
    $$N_n := \#_n \mathbb{R}P^2,$$
    where $\#$ is the connected sum of manifolds.
    Given two manifolds $M_1$ and $M_2$, then we take $M_1 \setminus \on{Int}(D^2)$,
    and $M_2 \setminus \on{Int}(D^2)$, into which we glue in an interval 
    $\mathbb{S}^{n-1}\times I$ between the holes we cut out of the manifolds.
    This is the check sum, and is defined as:
    $$M_1 \# M_2 := \left((M_1 \setminus \on{Int}(D^2)) \cup (M_2 \setminus \on{Int}(D^2)) \cup \mathbb{S}^{n-1} \times I\right)/ \simeq.$$
    \paragraph{Fact:} for $n=2$, the homotopy type of $M_1 \# M_2$
    is well-defined. For $n \geq 3$, one needs to take a choice of orientation.\\
    Recall that the $2$-torus can be given by 
    $$T^2 = (I\times I)/\simeq \cong (\mathbb{S}^1_a \times \mathbb{S}^1_b) \cup_\varphi D^2,$$
    where $a$ and $b$ are there to keep track of the two copies of the circle,
    and $[\varphi] \in \pi_1(\mathbb{S}^1 \vee \mathbb{S}^1) \cong F_2(a,b)$.
    If we travel once around the square defined by $(I\times I)/\simeq$, 
    we find that $[\varphi] = aba^{-1}b^{-1} = [a,b]$.
    Then, we have the van Kampen diagram:
    $$\begin{tikzcd}
    {\pi_1(T^2) \cong \langle a,b : [a,b] = 0 \rangle \cong \mathbb{Z}^2} \arrow[d] \\
    \pi_1(\mathbb{S}^1 \times \mathbb{S}^1) \arrow[d]                           \\
    \pi_1(\mathbb{S}^1) \times \pi_1(\mathbb{S}^1) \arrow[d, "\cong"]           \\
    \mathbb{Z} \times \mathbb{Z} \cong \mathbb{Z}^2                            
    \end{tikzcd}$$
\end{example}

\begin{example}
    Let us consider $$\pi_1(\mathbb{R}P^2).$$
    One finds that $\mathbb{R}P^2 \cong \mathbb{S}^1 \times_\varphi D^2$.
    Take some $[\varphi] \in \pi_1(\mathbb{S}^1)$.
    Drawing the picture out, we find that $[\varphi] = 2a$, 
    and thus by van Kampen, we have that
    $$\pi_1(\mathbb{R}P^2) = \langle a : 2a = 0 \rangle \cong \mathbb{Z}/2\mathbb{Z}.$$
\end{example}

\chapter{Week Four}

\section{Lecture 1, 20/03/2024}

TODO: watch first 15 minutes of lecture

\begin{proof}[Proof Idea]
    An element of $G \ast H$ is a reduced word 
    $$g_1\cdots g_m \longmapsto p_{\alpha_1} (g_1) \cdots p_{\alpha_m}(g_m),$$
    where $g_i \in G_{\alpha_i}$, and $G = G_{\alpha_1}$, and 
    $H = G_{\alpha_m}$. The exercise is to check that this map 
    defines a homomorphism.
\end{proof}

In the amalgamated case, if we are given maps 
$p : G \to U$ and $q : H \to U$, we have the commutative diagram:
$$\begin{tikzcd}
    K \arrow[r, "i"] \arrow[d, "j"]                 & G \arrow[d, "j_G"] \arrow[rdd, "p", bend left] &   \\
    H \arrow[r, "j_H"] \arrow[rrd, "q", bend right] & G \ast_K H                                     &   \\
                                                    &                                                & U
\end{tikzcd}$$
That is, we have the property that $$p\circ i = q \circ j.$$
Then, there exists a unique map $\phi : G \ast_K H \to U$
such that $$\phi \circ j_H = p, \quad \phi \circ j_H = q,\quad \phi \circ i_G \circ i = \phi \circ j_H \circ j$$.
That is, such that the diagram $$\begin{tikzcd}
    K \arrow[r, "i"] \arrow[d, "j"]                 & G \arrow[d, "j_G"] \arrow[rdd, "p", bend left] &   \\
    H \arrow[r, "j_H"] \arrow[rrd, "q", bend right] & G \ast_K H \arrow[rd, "\exists!\phi", dashed]  &   \\
                                                    &                                                & U
\end{tikzcd}$$ commutes.

\begin{proof}[Proof Idea]
    Observe that for $\phi : G \ast H \to U$, the maps
    $\phi(i(k)j(k^{-1})) = e$, and $\phi(j(k) i(k^{-1})) = e$. 
    From this, it follows then that the restriction map $\phi\vert_N$ is 
    trivial, where $$N = \langle i(k)j(k^{-1}) : k \in K \rangle.$$
    It follows then that $\phi$ descends to $\phi_\alpha : G \ast_K H \to U$.
\end{proof}

\subsection{Mapping Cylinders}

We now talk about assignment problems! Given a continuous map $f : X \to Y$,
$$M_f := ((X\times I) \sqcup Y)/((x,1) \simeq f(x).$$
In other words, we attach the endpoint of $x\in X$ to its image in $f$.
Then, from this, one may deduce that there is a retraction
$$r : M_f \longrightarrow Y, \quad [x,t] \longmapsto f(x),\quad y\longmapsto y.$$
One may check using that using the universal property of quotient spaces,
this is indeed a continuous map. From this, one may then put this into a 
deformation retraction.
There is an obvious inclusion $i : X \hookrightarrow M_f$ given by 
$x\mapsto (x,0)$. From this, we obtain two maps 
$$r \circ i : X \longrightarrow Y.$$ 
Then, one observes that $r \circ i = f$ by construction.

\begin{example}
    Consider the winding map 
    $$f_d : \mathbb{S}^1 \longrightarrow \mathbb{S}^1,\quad z\longmapsto z^d.$$
    If we take $M_{f_2}$, then one may show that $M_{f_2}$ is homeomorphic 
    to the M\"obius band. If we draw out the usual picture used to create the 
    M\"obius band, we see that we have one copy of $\mathbb{S}^1$ embedded 
    in the edges with arrows, and another copy of $\mathbb{S}^1$ along the 
    boundary of the M\"obius band that is twice as large as the other copy.\\\\
    Another hint! Problem 1.2 basically takes a bunch of these mapping cylinders 
    and then attaches them together, denoted by $X_{d,\infty}$.
    You are not being asked to determine $\pi_1(X_{d,\infty})$ precisely,
    you just need to show that it is different from $\pi_1(\mathbb{S}^1)$.
    A key hint is to show that given the embedding $i : \mathbb{S}^1_0 \hookrightarrow X_{d,\infty}$,
    the induced homomorphism 
    $$i_\ast : \pi_1(\mathbb{S}^1_0) \longrightarrow \pi_1(X_{d,\infty}),$$
    is non-zero. Once we show that, we are most of the way there, supposedly.
    Use the fact that $D^2$ is compact, and thus its image lies in some 
    $X_{d,k}$ for some sufficiently large $k$, because it cannot "fit" into all the 
    pieces of the mapping cone telescope because of its compactness.
\end{example}

A map $$\mathbb{S}^1 \longrightarrow X,$$ is \emph{nul-homotopic} if and only if 
$f$ extends to $\overline{f} : D^2 \to X$.

\begin{proof}
    There is a canonical identification 
    $$D^2 \cong C\mathbb{S}^1 \cong (\mathbb{S}^1 \times I)/(x_0 \sim (x',0)).$$
    Let $q : \mathbb{S}^1 \times I \to C\mathbb{S}^1$.
    Then, $\overline{f} \circ q \vert_{\mathbb{S}^1 \times \lbrace 1\rbrace} = f$,
    and $\overline{f}\circ q \vert_{\mathbb{S}^1 \times \lbrace 0 \rbrace} = c_x$,
    and thus $f$ is null-homotopic.
\end{proof}

\subsection{Discussion of van Kampen's Theorem (Thm 1.20, Hatcher)}

Recall that we have a map 
$$\Phi : \ast_\alpha \pi_1(A_\alpha, x_0) \longrightarrow \pi_1(X,x_0),$$
and we have shown that this is surjective. We wish to now analyse its kernel.

\begin{definition}
    A \emph{factorisation} of $[f] \in \pi_1(X,x_0)$ is a word 
    $[f_1] \cdots [f_m]$, where $[f_i]$ belongs to $\pi_1(A_{\alpha_i},x_0)$,
    and $$[f] = (i_{\alpha_1})_\ast([f_1])\cdot  (i_{\alpha_2})_\ast([f_2]) \cdots (i_{\alpha_m})_\ast ([f_m]).$$
    This is an \emph{unreduced} word for the groups 
    $\pi_1(A_\alpha,x_0)$.
\end{definition}
We consider the equivalence relation on factorisations 
generated by: \begin{itemize}
    \item[(i)] multiplying adjacent elements in the same group,
    \item[(ii)] if $$[f_i] \in \on{Im}((i_{\alpha,\beta})_\ast : \pi_1(A_\alpha \cap A_\beta, x_0) \to \pi_1(A_\alpha,x_0)),$$
    then we regard $[f_i]$ as being in the image of the map $$(i_{\beta\alpha})_\ast : \pi_1(A_\alpha \cap A_\beta, x_0) \longrightarrow \pi_1(A_\beta).$$
\end{itemize}
If two factorisations are equivalent, then they define the same element in the amalgamated product 
$\ast_\alpha \pi_1(A_\alpha,x_0)$.
We must show that path homotopic factorisations are equivalent.
We can define a map $$f_t : I \times I \longrightarrow X,$$
and then "chop up" the square $I\times I$ into little rectangles $[a_i,b_i] \times [a_j,b_j]$.
The point of this construction is that we can show the rectangles $R_i \subset I \times I$ 
such that $f(R_i) \subset A_{\alpha_i}$.

\section{Lecture 2, 21/03/2024}

\subsection{Proof Idea of van Kampen's Theorem}

Suppose that $[f]_1\ast \cdots \ast [f_m]$ factorises $[f]$, and 
each $[f_i] \in \pi_1(A_{\alpha_i},x_0)$. 
Suppose that $[f]_1'] \ast \cdots \ast [f_n']$ is another factorisation
of $[f]$, where $[f_j'] \in \pi_1(A_{\alpha_j},x_0)$.
Then, we chop up the square $I \times I$ into little rectangles 
such that we only have triple intersections (i.e. no quadruple intersections).
We also pick the rectanlges in such a way that the rectangles $R_i$ match 
up with the factorisations. 
\paragraph{Claim:} we obtain such rectangles $R_i$ such that there is a 
homotopy $F : I \times I \to X$ such that $F(R_i) \subset A_{\alpha_i}$.\\\\
In particular, the homotopy maps $F(\lbrace 0,1\rbrace \times I) = x_0$, so 
that two edges of the rectangle get mapped to a point, which gives us a loop.\\\\
Let $v \in I \times I$ be a vertex -- i.e. a point where vertical and horizontal lines meet. 
We know that $F(v)$ belongs to $A_{\alpha_1} \cap A_{\alpha_2} \cap A_{\alpha_3}$
(it is possible that $\vert \lbrace \alpha_1,\alpha_2,\alpha_3\rbrace \vert < 3$).
Now, take a path $h_v$ from $F(v)$ To $x_0$. Then, insert copies of $h_v\overline{h_v}$ into paths arriving at $v$. \\\\
Hence, the loop $F \circ \gamma_r$ -- where $\gamma_r$ separates 
$R_1,\cdots,R_r$ from the other rectangles -- defines a factorisation
of $[f]$, at least once choices are made to place loops in $A_{\alpha_i}$.
Finally, passing from $R_i$ to $R_{i+1}$, we obtain 
path homotopic loops.

\begin{remark}
    We must decide to place all relevant loops in the same $\pi_1(A_{\alpha_i},x_0)$.
\end{remark}

\begin{remark}
    \textbf{ALSO}, this is not examinable! We are only required to apply 
    van Kampen's theorem, not to prove it.
\end{remark}

\subsection{Assignment Tips and Tricks: Mapping Tori}

In the most general situation, we can consider an arbitrary continuous map 
$f : X \to X$. Then, define 
$$T_f := \frac{X\times I}{(x,1) \sim (f(x),0)}.$$
By construction, $T_{\on{id}} = X \times \mathbb{S}^1$.
Observe that there is always a map 
$$T_f \longrightarrow \mathbb{S}^1,\quad [x,t]\longmapsto [t],$$
under the identification $\mathbb{S}^1 \cong [0,1]/(0 \sim 1)$.
As an example, let us consider a map 
$f : \mathbb{S}^1 \to \mathbb{S}^1$ defined by $z\mapsto e^{i\theta}z$,
for some $\theta \in [0,2\pi]$. Then, the mapping torus 
$T_f \cong \mathbb{S}^1 \times \mathbb{S}^1$, the $2$-torus, since we can connect this to the identity 
map. Explicitly, the homeomorphism is given by $(z,[t]) \mapsto [ze^{i\theta t}, t]$. \\\\ 
Consider another map $f : \mathbb{S}^1 \to \mathbb{S}^1$ defined by $z\mapsto \overline{z}$,
which is an antiholomorphic map that flips the circle $180$ degrees around, 
and reverses the orientation. Then, $T_f$ is homeomorphic to the Klein bottle 
$K^2$. If $f(x_0) = x_0$, then 
$$\pi_1(T_f) \cong \pi_1(X,x_0) \ltimes \mathbb{Z},$$
where $\mathbb{Z}$ acts on $\pi_1(X,x_0)$ by $1$ acting by 
$f_\ast$. 
\begin{definition}
    Let $N$ and $A$ be groups, and $\phi : A \to \on{Aut}(N)$ be a 
    homomorphism. 
    Then, $$N \rtimes_\varphi A,$$ 
    is a group with underlying set $N\times A$, defined by 
    $$(n_1,a_1) (n_2,a_2) = (n_1 \varphi(a_1) n_2, a_1a_2).$$
    Note that if $\varphi$ is non-trivial, then $N\ltimes A$ is 
    \emph{non-abelian}.
\end{definition}
One checks that $N\rtimes_\varphi A$ defines a group.
Using the isomorphism, we see then that $\pi_1(K^2) \cong \mathbb{Z} \ltimes \mathbb{Z}$, 
where the semidirect product is defined by the homomorphism 
$\varphi : \mathbb{Z} \to \on{Aut}(\mathbb{Z}) \cong \mathbb{Z}/2\mathbb{Z}$.
So, if we have the map $f_{-1} : \mathbb{S}^1 \to \mathbb{S}^1$ 
defined by $z\mapsto \overline{z}$, then the induced homomorphism 
$$(f_{-1})_\ast : \pi_1(\mathbb{Z}) \longrightarrow \pi_1(\mathbb{Z}),$$
defined by multiplication by $-1$.\\\\
Moreover, another hint is that there is a canoincal relationship between 
$T_f$ and $T_{f^d}$. More on that tomorrow! If you can answer that you're basically 
done with question $4$.

\subsection{Covering Spaces (Chapter 1.2, Hatcher)}

Recall that $p : \widetilde{X} \to X$ is a \emph{covering space} if 
every $x\in X$ has an open neighbourhood $U$ such that 
$p^{-1}(U) \cong \sqcup_\alpha U_\alpha$, where
each $p\vert_{U_\alpha}$ is a homeomorphism onto $U$. 
From now on, we assume that $X$ is path-connected.
We have the result that we've already proved:
\begin{proposition}[Homotopy Lifting, Prop 1.30, Hatcher]
    Given a homotopy $F_t : Y \times I \to X$ and a covering map 
    $p : \widetilde{X} \to X$, there exists a 
    unique map $\widetilde{F}_t$ such that the diagram
    $$\begin{tikzcd}
        Y \times \lbrace 0 \rbrace \arrow[d, hook] \arrow[r, "\widetilde{F}_0"]   & \widetilde{X}    \\
        Y \times I \arrow[r, "F_t"] \arrow[ru, "\exists!\widetilde{F}_t", dashed] & X \arrow[u, "p"]
    \end{tikzcd}$$
    That is, such that $p \circ \widetilde{F}_t = F_t$.
\end{proposition}

\begin{proposition}[Prop 1.31, Hatcher]\label{prop13.1}
    Let $\widetilde{x_0} \in p^{-1}(x_0)$. 
    Then, the induced homomorphism 
    $$p_\ast : \pi_1(\widetilde{X}, \widetilde{x_0})  \longrightarrow \pi_1(X,x_0),$$
    is injective.
\end{proposition}

\begin{proof}
    Suppose that $p_\ast([\gamma]) = e$. Then, $p \circ \gamma$ is 
    path-homotopic to the constant loop $c_{x_0}$ via a homotopy 
    $F_t$. Then, by the lifting property, there exists a lift 
    $\widetilde{F}_t$ using $p$ as a lift. Then, we have the diagram:
    $$\begin{tikzcd}
        I \times \lbrace 0 \rbrace \arrow[d, hook] \arrow[r, "\widetilde{F}_0"]   & \widetilde{X}    \\
        I \times I \arrow[r, "F_t"] \arrow[ru, "\exists!\widetilde{F}_t", dashed] & X \arrow[u, "p"]
    \end{tikzcd}$$
    We claim that $\widetilde{F}_t$ is a path homotopy to 
    $c_{\widetilde{x_0}}$. Observe that 
    $$\widetilde{F}_t \vert_{\lbrace 1\rbrace \times I} : I \longrightarrow p^{-1}(x_0),$$
    where $p^{-1}(x_0)$ is discrete. It follows then that 
    $\widetilde{F}_t\vert_{\lbrace 1\rbrace \times I}$ is a path homotopy 
    that lifts the constant loop. Thus, it lifts the constant loop. More on this next time.
\end{proof}

\section{Lecture 3, 22/03/2024}

\begin{remark}[Diarmuid Wisdom]
    The hardest thing about covering spaces is trying to remember 
    that it's a map, not a space.
\end{remark}

We will continue our discussion of covering spaces 
$p : (\widetilde{X}_0, \widetilde{x_0}) \to (X,x_0)$.
Recall that we have Proposition \ref{prop13.1}, which is a
useful fact to have.

\begin{remark}
    Homotopy lifting means that path homotopies lift to path homotopies.
    Moreover, the uniqueness of lifting implies that constant 
    maps lift to constant maps.
\end{remark}

We want to relate the algebra of our fundamental group to the topology of 
our situation. Recall that if $H$ is a subgroup of $G$, then we 
can always write $$G = \sqcup_{g_\alpha} Hg_\alpha,$$ 
where $Hg$ is a right coset of $G$ -- that is, the right cosets of $G$ 
give a partition of $G$. Recall the index $[H:G]$ of $H$ in $G$ is the 
size of $H$ in $G$. In particular, 
$[H:G] = \vert \lbrace g_\alpha \rbrace\vert$.
As an example, $3\mathbb{Z}$ has index $3$ in $\mathbb{Z}$. 
The trivial subgroup $\lbrace 0 \rbrace$ has index $\infty$ 
in $\mathbb{Z}$.

\begin{proposition}[Prop 1.32, Hatcher]\label{prop1.32}
    Suppose that $\widetilde{X}$ and $X$ are path-connected 
    Then, the image of the induced homomorphism 
    $$\left[p_\ast\left(\pi_1(\widetilde{X},\widetilde{x_0})\right) : p_1(X,x_0)\right] = \vert p^{-1}(x_0)\vert,$$
    which is called the \emph{degree} of the cover $p$. Hatcher calls it the 
    ``number of sheets''.
\end{proposition}

\begin{proof}
    Consider a map 
    $$\Phi : \pi_1(X,x_0) \longrightarrow p^{-1}(x_0),\quad [\gamma] \longmapsto [\widetilde{\gamma}],$$
    where $\widetilde{\gamma}$ is the lift of $\gamma$ given by $\widetilde{\gamma}(0) = \widetilde{x_0}$.
    This is well-defined by the homotopy lifting property. In particular, we are applying 
    homotopy lifting to the case where $Y = \on{pt}$, and we thus can regard $\gamma$ as a 
    homotopy. \\\\
    We now wish to show that  $\Phi$ is constant on the right cosets of 
    $p_\ast(\pi_1(\widetilde{X},x_0))$. To see this, let 
    $\gamma = \delta \cdot \gamma'$, where $\delta \in \on{Im}(p_\ast)$.
    From this, we wish to determine $\widetilde{\gamma}(1)$. 
    By our construction, we have $\widetilde{\gamma}(1) = (\widetilde{\delta}\cdot\widetilde{\gamma'})(1)$.
    But since $\widetilde{\delta}_{\widetilde{x_0}}(1) = \widetilde{x_0}$,
    it follows that $\widetilde{\gamma}_{\widetilde{x_0}} (1) = \widetilde{\gamma}_{x_0}(1)$.
    This uses the fact that $[\gamma'] \in \on{Im}(p_\ast)$ if and only if 
    $\widetilde{\gamma}_{\widetilde{x_0}} = \widetilde{x_0}$ (see Hatcher).
    It follows then that $\Phi$ descends to a map on the left coset: 
    $$\Phi : H\setminus G \longrightarrow p^{-1}(x_0),$$ 
    where $G = \pi_1(X,x_0)$ and $H = p_\ast\left(\pi_1(\widetilde{X},\widetilde{x_0})\right)$.
    Since $\widetilde{X}$ is path-connected, there exists $\widetilde{\gamma'}$ such that 
    $\widetilde{\gamma'}(0) = \widetilde{x_0}$, and 
    $\widetilde{\gamma}(1) = x \in p^{-1}(x_0)$. 
    Now, $p\circ \gamma$ is a loop at $x_0$ such that $\widetilde{p\circ \gamma}_{x_0} = \widetilde{\gamma}$,
    amd thus $\Phi([p\circ \widetilde{\gamma}]) = x \in p^{-1}(x_0)$.
    This shows surjecitivty.\\\\ 
    Next, we wish to show injectivity. Suppose that $\Phi(H[\gamma_1]) = \Phi(H[\gamma_2])$.
    Then, $\widetilde{\gamma_1\overline{\gamma_2}}(1) = \widetilde{x_0}$.
    It follows then that $[\gamma_1\overline{\gamma_2}] \in \on{Im}(p_\ast)$,
    and therefore $H[\gamma_1] = H[\gamma_2]$ by property of cosets.
\end{proof}

Given a map $f : Y \to X$, and a covering space $p : \widetilde{X} \to X$, 
a natural question to ask is if there exists a lift 
$Y \to \widetilde{X}$ such that the diagram $$\begin{tikzcd}
    & \widetilde{X} \arrow[d, "p"] \\
Y \arrow[r, "f"'] \arrow[ru, dashed] & X                           
\end{tikzcd}$$
commutes. Observe that if such a lift does exist, then the induced 
homomorphisms give a commutative diagram: $$\begin{tikzcd}
    & {\pi_1(\widetilde{X},\widetilde{x_0})} \arrow[d, "p_\ast"] \\
{\pi_1(Y,y_0)} \arrow[r, "f_\ast"'] \arrow[ru, "\widetilde{f_\ast}", dashed] & {\pi_1(X,x_0)}                                            
\end{tikzcd}$$
That is, $f_\ast = p_\ast \circ \widetilde{f_\ast}$. 
It follows from this then that $\on{Im}(f_\ast) \subseteq \on{Im}(p_\ast)$.
However, this turns out to be an if and only if, as we will show later.

\begin{definition}[Template of a Definition]
    Let $P$ be a topological property (i.e. path-connectedness, compactness etc.),
    and $X$ a space. Then, we say that $X$ is \emph{locally $P$} if 
    for all $x \in X$, and for all neighbourhoods $U$ of $x$, 
    there exists a neighbourhood $x \in V \subset U$ such that $V$ is $P$.
\end{definition}

\begin{remark}
    You don't always want open neighbourhoods for the above definition -- i.e. if 
    we are concerned with local compacntess then we will require closed neighbourhoods
    contained in an open neighbourhood of $x$.
\end{remark}

\begin{remark}
    For some reason path-connected \emph{doesn't} imply locally path-connected???
    The topolgoists sine curve is an example of this.
\end{remark}

\begin{proposition}[Prop 1.33, Hatcher]
    If $X$ is path-connected and locally path-connected,
    then any map $f :Y \to X$ can be lifted to one on $\widetilde{X}$
    if and only if 
    $$f_\ast (\pi_1(Y,y_0)) \subseteq p_\ast(\pi_1(\widetilde{X},\widetilde{x_0})).$$
\end{proposition}

\begin{proof}
    The $\implies$ direction is clear, as we have already done that.
    Let us consider the other direction. Then, we have maps:
    $$\begin{tikzcd}
        & \widetilde{X} \arrow[d, "p"] \\
Y \arrow[r, "f"'] & X                           
\end{tikzcd}$$
    Let $y\in Y$. Then, since $X$ is path-connected, then 
    there exists a path $\gamma : I \to Y$ such that $\gamma(0) = y_0$
    and $\gamma(1) = y$.
    Then, the path $f\circ \gamma$ in $X$ can be lifted to a path 
    $\widetilde{f\circ \gamma}$ starting at $x-0$. Then, we can define 
    $$\widetilde{f}(y) := \widetilde{f\circ \gamma}_{x_0}(1).$$
    Let us check that $\widetilde{f}$ is well-defined. 
    If $\gamma'$ is path-homotopic to $\gamma$, then 
    $$\widetilde{f\circ \gamma}_{x_0}(1) = \widetilde{f \circ \gamma'}_{x_0}(1),$$
    by path-lifting.
    Now, we wish to check this for path-homotopy classes. 
    In general, let $\gamma'(0) = y_0$, and $\gamma'(1) = y$.
    Then, $\gamma \simeq \gamma' \overline{\gamma}\gamma$.
    Observe that $[\gamma'\overline{\gamma}] \in \pi_1(Y,y_0)$.
    Then, $f\circ \gamma' \simeq f\circ (\gamma' \overline{\gamma}\gamma) = f\circ (\gamma' \overline{\gamma}) \cdot f\circ \gamma$.
    Then, we see that $$\widetilde{f\circ \gamma'}(1) = \widetilde{f\circ \gamma}(1),$$
    since $[f\circ \gamma'] \in f_\ast(Y,y_0) \subset p_\ast(\widetilde{X},\widetilde{x_0})$,
    and thus the induced homomorphism is $\widetilde{f}_\ast$ is well-defined.\\\\
    We now show that $\widetilde{f}$ is continuous. Let $N\subseteq Y$ be a neighbourhood of 
    $y \in Y$ such that $f(N) \subset U$, where $U$ is evenly covered by $p$.
    (we're running out of time so just go see Hatcher)
\end{proof}

\begin{remark}
    Local path-connectedness is actually a stronger condition than path-connectedness, 
    since you can choose as small of a neighbourhood as you want 
    and then prove your result on that. Which is wild.
\end{remark}

\subsection{More Assignemnt Tips on Mapping Tori}

Suppose that we have a homeomorphism $f : X \to X$, we can raise 
to the $d$-th power by $f^d$. Then, we have $T_{f^d}$ and $T_f$.
Now that we know some more things about covering spaces, we can say more 
about how they may be related.\\\\
The answer is that they are related by a covering space. Suppose that 
$d=3$. Then, we can take three copies of the interval, and glue each 
copy of the interval to $f$. One can check that this is a three-fold covering.
Call this thing $T_{f,f,f}$. 
This construction is modelled on the $z\mapsto z^d$ map. 
We want to show that $$T_{f^3} \cong T_{f,f,f}.$$
This is what we have to show. If you glue two cylinders via a 
homeomorphism, it is still a homeomorphism. And you can always untwist it.
Moreover, you can say that you can assume that the homeomorphism has a 
fixed point.

\chapter{Week Five}

\section{Lecture 1, 27/03/2024}

Let us recall one of the key results that we unfortunately skipped. 

\begin{proposition}[Prop 1.33, Hatcher]
    Let $Y$ be path-connected, and locally path-connected, and consider 
    a covering space $p : \widetilde{X} \to X$ fitting into a diagram:
    $$\begin{tikzcd}
        & \widetilde{X} \arrow[d, "p"] \\
Y \arrow[r, "f"] \arrow[ru, "\widetilde{f}", dashed] & X                           
\end{tikzcd}$$
    Then, a lift $\widetilde{f}$ exists if and only if 
    $f_\ast(\pi_1(Y,y)) \subseteq p_\ast(\pi_1(\widetilde{X},\widetilde{x}_0))$.
\end{proposition}

\begin{example}
    As a nice application of Proposition 1.33, let us look at the covering space 
    $$p  : \mathbb{R} \longrightarrow \mathbb{S}^1.$$ 
    Then, for any simply-connected space $Y$, any map $Y \to \mathbb{S}^1$ is 
    null-homotopic, since 
    the fundamental group of $Y$ is trivial. Since $\mathbb{R}$ is contractible, 
    it follows that every map into $\mathbb{R}$ is null-homotopic via the straight 
    line homotopy. 
    Then, apply Proposition 1.33 to show that we have a lift 
    $\widetilde{f} : Y \to \mathbb{R}$, 
    which factors through the null-homotopic map $Y \to \mathbb{S}^1$. 
    Thus, the lift $\widetilde{f}$ is also null-homotopic.
\end{example}

\begin{proposition}[Hatcher, Prop 1.34]
    Consider maps 
    $$\begin{tikzcd}
        & {(\widetilde{X}, \widetilde{x_0})} \arrow[d, "p"] \\
{(Y,y)} \arrow[r, "f"] & {(X,x)}                                          
    \end{tikzcd}$$
    where $p$ is a covering space, and $f$ has lifts 
    $\widetilde{f}_1, \widetilde{f_2} : Y \to \widetilde{X}$.
    If $Y$ is path-connected, then $\widetilde{f_1} = \widetilde{f_2}$ 
    if and only if $\widetilde{f_1}(y_0) = \widetilde{f}_2(y_0)$.
\end{proposition}

\begin{proof}
    $\Longleftarrow$ Let 
    $$Z = \lbrace y : \widetilde{f_1}(y) = \widetilde{f_2}(y)\rbrace.$$
    Let $f(y) \in U$, where $U$ is evenly covered. Then, by continuity, there 
    exists a neighbourhood $N$ such that 
    $\widetilde{f_1}(N) \subset \widetilde{U_1}$, and $\widetilde{f_2}(N) \subset \widetilde{U_2}$
    If $\widetilde{U_1} \neq \widetilde{U_2}$, then we may conclude that 
    $N$ is not contained in $Z$, and thus $Z$ is closed. Conversely, if the two 
    agree, then we can agree that $Z$ is open, since $N \subset Z$. 
    If $Z \subset Y$ is non-empty, open and closed, then $Z=Y$ since $Y$ is 
    connected.
\end{proof}

\subsection{Classification of Covering Spaces}

Our goal now is to classify all pointed covering spaces of $(X,x)$, where 
$X$ is path-connected and locally path-connected. By a classification, we 
mean the following. Given two covering spaces $p_1$, and $p_2$ over $X$, 
there exists a pointed homeomorphism $h$ such that that the diagram
$$\begin{tikzcd}
    {(\widetilde{X}_1,\widetilde{x}_1)} \arrow[rd, "p_1"'] \arrow[rr, "h"] &         & {(\widetilde{X}_2,\widetilde{x}_2)} \arrow[ld, "p_2"] \\
                                                                           & {(X,x)} &                                                      
\end{tikzcd}$$ commutes.\\\\
Observe that if $p_1 \cong p_2$, then since $p_1 = p_2 \circ h$, 
it follows then that 
$$(p_1)_\ast (\pi_1(\widetilde{X}_1,\widetilde{x}_1)) = (p_2)_\ast (f_\ast (\pi_1(\widetilde{X}_1, \widetilde{x}_1))) = (p_2)_\ast (\pi_1(\widetilde{X}_2,\widetilde{x}_2)).$$
That is, $\on{Im}(p_1)_\ast = \on{Im}(p_2)_\ast$, and so $H = \on{Im}(p_1)_\ast$ is an invariant 
of covering spaces up to isommorphism.

\begin{example}
    Recall that for $X = \mathbb{S}^1 \vee \mathbb{S}^1$,
    $\pi_1(X) = \mathbb{Z} \ast \mathbb{Z} =: F_2$,
    which is the free group on two letters. 
    In this case, $\widetilde{X} = \mathbb{S}^1 \vee \mathbb{S}^1 \vee \mathbb{S}^1$,
    and $\pi_1(\widetilde{X}) = F_3$, and we have $F_3 \hookrightarrow F_2 = \pi_1(X)$.
    The group $H$ here is generated by $\langle a^2,b^2,ab\rangle$, which is a 
    subgroup of index $2$ in $F_2$. In particular, note that the index of the 
    subgroup is equal to how many sheets the cover has. Indeed, in this case, 
    $\widetilde{X} \to X$ is a two-sheeted cover.
\end{example}

\subsection{Existence}

We want, for every subgroup $H$ of $\pi_1(X,x)$, a covering space 
$p : (\widetilde{X}, \widetilde{x}_0) \to (X,x)$ such that 
$\on{Im}p_\ast = H$. The covering space corresponding to the case for which 
$H = \lbrace e\rbrace$ will be called the \emph{universal cover} 
$$X_{\on{univ}},$$ 
since every other covering space will arise as a quotient of the universal cover.
If such a cover exists, then any point $x\in X$, and evenly covered set $U$ 
containing $x$, the homomorphism 
$$\pi_1(U,x) \longrightarrow \pi_1(X,x),$$ is trivial. Choose any one of the 
even covers $\widetilde{U}_i$ in $\widetilde{X}$. Then, $p\vert_{\widetilde{U}_i} : \widetilde{U}_i \to U$
is a homeomorphism, and thus induces a group isomorphism.
We have the commutative diagram:
$$\begin{tikzcd}
    {\pi_1(\widetilde{U}_i,\widetilde{x}_i)} \arrow[r, "i"] \arrow[d, "(p\vert_{\widetilde{U}_i})_\ast"] & {\pi_1(\widetilde{X},\widetilde{x}_0)} \arrow[d, "p"] \\
    {\pi_1(U,x)} \arrow[r]                                                                               & {\pi_1(X,x)}                                         
\end{tikzcd}$$
However, the composition $p\circ i$ is trivial, and since $(p\vert_{\widetilde{U}_i})_\ast$ 
is an isomorphism, it follows that the map $\pi_1(U,x) \to \pi_1(X,x)$ must be trivial.

\begin{definition}
    A space $X$ is \emph{semi-locally simply-connected} if for all $x\in X$, 
    there exists a neighbourhood $U$ of $x$ such that for all $x' \in U$, 
    the map $\pi_1(U,x') \to \pi_1(X,x')$ is trivial -- that is, every loop in 
    $U$ must be contractible.
\end{definition} 

\begin{example}[A non-semi-locally simply-connected space]
    Consider an infinite shrinking wedge of circles (a.k.a a Haiwaiian earring),
    given by $$X = \bigcup_n \mathbb{S}_{1/n}^1 \subseteq \mathbb{R}^2,$$
    It is not well-behaved at the point where all the circles intersect, and 
    it is not semi-locally simply-connected.
\end{example}

\section{Lecture 2, 28/03/2024}

\begin{theorem}[Classification of Covering Spaces, Thm 1.38, Hatcher]
    Let $X$ be path-connected, locally path-connected, and semi-locally simply-connected.
    Then, there are bijections 
    $$\lbrace \text{isomorphism classes of pointed covering spaces}\rbrace \stackrel{p_\ast}{\longleftrightarrow} \lbrace \text{Subgroups of $\pi_1(X,x)$},$$ 
    where by $\stackrel{p}{\longleftrightarrow}$ we mean that the bijection is given by the induced 
    map $p_\ast$ of the covering space.
\end{theorem}

If we forget the basepoint of the covering spaces, then the bijection becomes 
$$\lbrace \text{isomorphism classes of covering spaces} \rbrace \longleftrightarrow \lbrace \text{conjugacy classes $[H]$ of subgroups of $\pi_1(X,x_0)$}\rbrace.$$
\begin{proof}[Proof Idea]
    Let us first begin by showing that existence of the covering space corresponding to 
    the group $H = \lbrace e\rbrace$. The resulting covering space will be called the \emph{universal covering space},
    and be denoted by $\widetilde{X}_{\on{univ}}$. All other covering spaces will arise as a quotient 
    of this covering space.\\\\ 
    Recall that $\pi_1(X,x_0)$ acts on $\Pi_1(X,x_0,x)$ freely and transitively. 
    Thus, let us set $$\widetilde{X} := \bigcup_{x\in X} \Pi_1(X,x_0,x),$$
    for some fixed $x_0 \in X$. Then, we take a map 
    $$p : \bigcup_{x\in X} \Pi_1(X,x_0,x) \longrightarrow X, \quad [\gamma] \longmapsto \gamma(1),$$
    where $\gamma(1) = x$. This is called the \emph{universal covering} of $X$. The next 
    step is to topologise $\widetilde{X}$ -- keep in mind that the whole idea of covering spaces is that their local 
    topology is the same as the topology on the base space $X$.\\\\ 
    Define $\mcal{U} \subseteq \tau_X$, where 
    $$\mcal{U} = \lbrace U \in \tau_X : \text{$U$ is path-connected and $\pi_1(U) \to \pi_1(X)$ is trivial}\rbrace,$$
    which are basepoint independent because if the map is trivial, then it is trivial 
    on every basepoint. Note that as a consequence, 
    two paths that are path-homotopic in $U$ 
    is also path-homotopic in $X$ by path-connectedness. Thus,
    $$i_\ast : \Pi_1(U,x_1,x_2) \longrightarrow \Pi_1(X,x_1,x_2),$$ 
    then $\vert \on{Im}i_\ast \vert = 1$.\\\\ 
    Note that if $V \subset U$, and $V$ is path-connected, and $U \in \mcal{U}$, then
    $V \in \mcal{U}$, since we the induced map of the inclusion $\pi_1(V,x) \to \pi_1(X,x)$
    is trivial, and the map $\pi_1(V,x) \to \pi_1(X,x)$ is also trivial, since 
    $X$ is locally path-connected. 
    This implies that $\mcal{U}$ is a \emph{basis} for $\tau_X$ -- that is, every 
    $U \in \tau_X$ can be written as a union of elements in $\mcal{U}$.
    
    \begin{example}
        Let $X$ be a metric space. Then, the set 
        $$\mcal{B} := \lbrace B_\varepsilon(x) : x \in X, \varepsilon > 0\rbrace,$$ 
        forms a basis for $\tau_d$, where $d : X \times X \to \mathbb{R}_{\geq 0}$ 
        is a metric on $X$.
    \end{example}
    
    A set $\mcal{B} \in \mcal{P}(X)$ defines $\tau_B$ for which $\mcal{B} \subset \tau_\mcal{B}$
    is a basis if \begin{itemize}
        \item[(i)] for all $X = \bigcup_\beta U_\beta$, $U_\beta \in \mcal{B}$,
        \item[(ii)] for all $U_{\beta_1}, U_{\beta_2} \in \mcal{B}$ and $x \in U_{\beta_1} \cap U_{\beta_2}$,
        there exists $U_{\beta_3} \in \mcal{B}$ such that $x \in U_{\beta_3} \subset U_{\beta_1}\cap U_{\beta_2}$.
    \end{itemize}
    Define $\tau_\mcal{B}$ by $U \in \tau_\mcal{B}$ if and only if 
    $$U = \bigcup_\beta U_\beta,$$
    for $U_\beta \in \mcal{B}$.\\\\ 
    To summarise, $\mcal{U}$ is a basis for $\tau_X$, and we define $\widetilde{\mcal{U}}$ satisfying 
    (i) and (ii) from $\mcal{U}$, and use $\widetilde{\mcal{U}}$ to topologise $\widetilde{X}$, 
    which we recall is 
    $$\widetilde{X} =  \bigcup_{x \in X} \Pi_1(X,x_0,x).$$
    Define 
    $$\widetilde{\mcal{U}} := \lbrace U_{[\gamma]} : \text{$U \in \mcal{U}$ , $[\gamma] \in \widetilde{X}$, $\gamma(1) \in U$}\rbrace,$$
    and $$U_{[\gamma]} := \lbrace [\gamma \cdot \eta] : \text{$\eta \in \Pi_1(U,\gamma(1),u)$, $u \in U$}\rbrace.$$
    \begin{remark}
        $U_{[\gamma]}$ depends only on $U$ and $[\gamma]$.
    \end{remark}
    Observe that the map 
    $$p : U_{[\gamma]} \longrightarrow U,\quad [\gamma\cdot \eta] \longmapsto (\gamma \cdot \eta)(1),$$
    is surjective, since $U$ is path-connected. Moreover, $p$ is injective, since different 
    choices of $\eta$ are path-homotopic in $X$.
    \paragraph{Claim:} $U_{[\gamma]} = U_{[\gamma']}$ if and only if $[\gamma'] \in U_{[\gamma]}$.
    \begin{proof}
        Suppose that $\gamma' = \gamma$. Then, elements of $U_{[\gamma']}$ 
        have the form 
        $[\gamma' \cdot \eta] = [\gamma \cdot \eta \cdot \mu]$. This proves the $\subseteq$ 
        direction. The other direction is analogous.
    \end{proof}
    \paragraph{Claim:} $\widetilde{\mcal{U}}$ is a basis for a topology.
    \begin{proof}
        Certainly, every $[\gamma] \in \widetilde{X}$ lies in some $U_{[\gamma]}$,
        just choose some $\gamma(1) \in U$. 
    \end{proof}
    Now, let us consider $U_{[\gamma]}$,
    and $V_{[\gamma']}$, and suppose that 
    $[\gamma''] \in U_{[\gamma]} \cap V_{[\gamma']}$. Then, $U_{[\gamma']} = U_{[\gamma'']}$,
    and similarly $V_{[\gamma']} = V_{[\gamma'']}$ by our previous claim. 
    Now, let us take an open, path-connected subset $W \subset U \cap V$ such that 
    $\gamma''(1) \in W$.
    Then, $W_{[\gamma'']} \subset U_{[\gamma']} = U_{[\gamma]}$, and $W_{[\gamma'']} \subset V_{[\gamma'']} = V_{[\gamma]}$.
    That is, 
    $$W_{[\gamma'']} \subset U_{[\gamma]} \cap V_{[\gamma']}.$$ 
    It follows then that $(\widetilde{X}, \tau_{\widetilde{\mcal{U}}})$ is a topological space.
    Now, we need to show the two facts:
    \begin{itemize}
        \item[(i)] $p$ is continuous, 
        \item[(ii)] $p$ is a covering space.
    \end{itemize}
    Recall the bijection $$p : U_{[\gamma]} \longrightarrow U.$$
    Then, one can check that $p$ defines a bijection to 
    $$\lbrace V_{[\gamma']} \rbrace \subseteq \widetilde{\mcal{U}},$$ 
    where $[\gamma'] \in U_{[\gamma']}$. This implies that $p$ is continuous. 
    To check that $p$ is a covering space, it remains to show that 
    $U_{[\gamma]} \cap U_{[\gamma']} = \emptyset$ if $[\gamma] \neq [\gamma']$,
    and $\gamma(1) = \gamma'(1)$. But, this holds since $U_{[\gamma]} = U_{[\gamma']}$ 
    if and only if $[\gamma'] \in U_{[\gamma']}$ by our previous claim. All 
    $[\gamma''] \in U_{[\gamma]}$ such that $\gamma''(1) = x$ are 
    path-homotopic.\\\\
    We must show now that $X$ is path-connected, and simply-connected. Recall that 
    $$\widetilde{X} = \lbrace [\gamma] : \text{base point $\widetilde{x} = [c_{x_0}]$}\rbrace.$$
    We wish to construct a path $\gamma_t$ going from $\gamma$ to $c_{x_0}$.
    So, let us set 
    $$\gamma_t(s) = \begin{cases}
        s \quad \text{if} \quad s \leq 1-t,\\
        1-t \quad \text{if} \quad s\geq 1-t,
    \end{cases}$$
    One can check that $\gamma_t(s)$ is continuous.
    But since $p_\ast : \pi_1(\widetilde{X},\widetilde{x}) \to \pi_1(X,x_0),$ is injective,
    it suffices to show that the image of $p_\ast$ is trivial.
    Let $\widetilde{\gamma}$ be a path on $\widetilde{X}$. Observe that $\widetilde{\gamma} \sim \widetilde{\gamma_1}$.
    But $\widetilde{\gamma_1} = c_{x_0}$, and so $[\widetilde{\gamma}] = [c_{\widetilde{x}}]$.
    The map $p\circ \widetilde{\gamma}$ has a lift, but since $\widetilde{\gamma}$ is the constant 
    path, it gets lifted to the constant path in $X$. Thus, the image of $p_\ast$ is trivial, since our choice of 
    $\gamma$ was arbitrary. Just go read Hatcher.
\end{proof}

\section{Lecture 3, 29/03/2024}

Good Friday. No lecture.

\chapter{Week Six}

\section{10/04/2024}

Before the easter break, we proved the following.
Let $X$ be a path-connected, locally path-connnected, and semi-locally simply connected 
(p.c., l.p.c., s.l.s.c.).
Then, there exists a covering space $$p  : \widetilde{X} \longrightarrow X,$$ 
such that $\pi_1(\widetilde{X}) = 0$ -- that is, $\widetilde{X}$ is simply-connected.
It follows then that $p_\ast$ is the trivial homomorphism.

\begin{proposition}[Prop 1.36, Hatcher]
    If $X$ is p.c., l.p.c., and s.l.s.c., then for all subgroups $H < \pi_1(X,x)$, 
    there exists a path-connected covering space 
    $$p_H : (\widetilde{X}_H, \widetilde{x}_0) \longrightarrow (X,x_0),$$
    such that the image of $(p_H)_\ast$ is $H$.
\end{proposition}

\begin{proof}
    Recall that 
    $$\widetilde{X}_{univ} := \bigcup_{x \in X} \Pi(X,x_0) = \lbrace [\gamma] : \gamma(0) = x_0\rbrace,$$
    such that there is a sequence of maps 
    $$\widetilde{X}_{univ} \longrightarrow \widetilde{X}_H \stackrel{p_H}{\longrightarrow} X.$$
    Then, define an equivalence relation on $\widetilde{X}_{univ}$ by
    $[\gamma] \sim [\gamma']$ if and only if $\gamma(1) = \gamma'(1)$ and 
    $[\gamma \cdot \bar{\gamma'}] \in H$.
    One checks that this defines an equivalence relation. 
    Define $$\widetilde{X}_H := \widetilde{X}_{univ} / \sim,$$
    and define the map $p_H$ by:
    $$\widetilde{X}_H \longrightarrow X,\quad [[\gamma]] \longmapsto \gamma(1),$$
    which makes sense since the equivalence relation on $\widetilde{X}_{univ}$ 
    does nothing to the end points of the path by construction.
    Further, note that $p_H$ is induced from the map $p_{univ} : \widetilde{X}_{univ} \to \widetilde{X}$.
    One readily checks that $p_H$ indeed defines a covering Space. 
    In particular, $\pi_1(X,x)$ labels the $U_{[\gamma]}$'s in $\widetilde{X}$, each of 
    which is a disjoint copy of the space $U$ in $X$. \\\\ 
    The last thing that remains to be checked is the fact the image of $(p_H)_\ast$
    is equal to $H$. We may argue this by lifting. This follows essentially by construction, 
    using the fact that there exists a path (that is not a loop) from a point in 
    $U_{[\gamma]}$ to $U_{[\gamma']}$. But under the quotient, all such points are identified,
    and each point corresponds to a loop in $H$.
\end{proof}

\begin{example}
    Consider a map 
    $$p : \mathbb{R} \longrightarrow \mathbb{S}^1,$$
    and let $H = d\mathbb{Z}$. We know that $\pi_1(\mathbb{S}^1,1) \cong \mathbb{Z}$.
    Then, the map $p$ factors through a map 
    $$\mathbb{R} \longrightarrow \mathbb{R}/d\mathbb{Z} \stackrel{p}{\longrightarrow} \underbrace{\mathbb{R}/\mathbb{Z}}_{\cong \mathbb{S}^1}.$$
\end{example}

\begin{proposition}[Prop 1.37, Hatcher]
    If $X$ is a path-connected, locally path-connected, then two 
    covering spaces $p_1 : (\widetilde{X}_1, \widetilde{x}_1) \to (X,x)$,
    and $p_2 : (\widetilde{X}_2,\widetilde{x}_2) \to (X,x)$
    are isomorphic if and only if $\on{Im}((p_1)_\ast) = \on{Im}((p_2)_\ast)$.
\end{proposition}

\begin{proof}
    Use the unique lifting property of Proposition 1.34. Given two covering spaces 
    $p_1$ and $p_2$, there exists a lift $\widetilde{h}_1 : \widetilde{X}_2 \to \widetilde{X}_1$, 
    by the existence of the lifting property.
    Conversely, there exists another lift $\widetilde{h}_2 : \widetilde{X}_1 \to \widetilde{X}_2$. 
    $$\begin{tikzcd}
        \widetilde{X}_1 \arrow[rd, "p_1"'] \arrow[rr, "h_1", bend left] &   & \widetilde{X}_2 \arrow[ld, "p_2"] \arrow[ll, "h_2", bend left] \\
                                                                        & X &                                                               
    \end{tikzcd}$$
    But by uniqueness of lifts, it follows that $\widetilde{h}_1 \circ \widetilde{h}_2$ is the identity 
    -- that is, $h_1$ and $h_2$ are inverses of one another.
\end{proof}

This proves the pointed classification of covering spaces in Theorem 1.38. 

\subsection{Classification of Pointed Covering Spaces}

Consider two covering spaces with different choices of base points:
$$\begin{tikzcd}
    {(\widetilde{X}, \widetilde{x}_1)} \arrow[r, "p"] & (X,x_0) \\
    {(\widetilde{X},\widetilde{x}_2)} \arrow[ru, "p"] &  
\end{tikzcd}$$
\paragraph{Question:} How are $p_\ast(\pi_1(\widetilde{X},\widetilde{x}_1))$ and 
$p_\ast(\pi_1(\widetilde{X},\widetilde{x}_2))$ related?\\\\
Let $\gamma : I \to \widetilde{X}$ be such that $\gamma(0) = \widetilde{x}_1$, 
and $\gamma(1) = \widetilde{x}_2$. Then, define a map 
$$\beta_\gamma : \pi_1(\widetilde{X}, \widetilde{x}_1) \longrightarrow \pi_1(\widetilde{X}, \widetilde{x}_2), \quad [\delta] \longmapsto [\overline{\gamma} \cdot \delta \cdot \gamma].$$
Under $p_\ast$, we conjugate by $[p \circ \gamma]$, and we obtain:
$$\on{Im}(p_1)_\ast = [p\circ \overline{\gamma}] \on{Im}(p_2)_\ast [p\circ \gamma].$$

\subsection{Deck Transformations and Group Actions}

\begin{definition}
    Let $p : \widetilde{X} \to X$ be a covering space. Then, 
    $$\mcal{G}(\widetilde{X}) := \lbrace h : X \stackrel{\simeq}{\to} X : p \circ h = p\rbrace = \on{Aut}(p).$$
    By construction, $\on{Aut}(p)$ is a subgroup of the group of homeomorphisms on 
    $\widetilde{X}$. The gruop $\mcal{G}(\widetilde{X})$ (or $\on{Aut}(p)$) is called the
    group of \emph{deck transformations} of $p$.
\end{definition}

\begin{example}
   Consider the covering space 
   $$p : \mathbb{R} \to \mathbb{S}^1, \quad t \longmapsto e^{2\pi it} = e^{2\pi i(t+n)},\quad n \in \mathbb{Z}.$$
   Then, $$\mcal{G}(\mathbb{R}) = \mathbb{Z},$$
   by the $\mathbb{Z}$-invariance of $p$. 
   Note that $\mcal{G}(\mathbb{R}) \cong \pi_1(\mathbb{S}^1,1)$. Later, we will learn that this 
   is not a coincidence. 
\end{example}

\begin{example}
    Consider the $d$-fold cover of $\mathbb{S}^1$, given by 
    $$p_d : \mathbb{S}^1 \longrightarrow \mathbb{S}^1,\quad z\longmapsto z^d.$$
    Then, $$\mcal{G}(p_d) \cong \mathbb{Z}/d\mathbb{Z}.$$
    Observe that 
    $$\mcal{G}(p_d) \cong \pi_1(\mathbb{S}^1,1) / \on{Im}(p_d)_\ast.$$
\end{example}

\begin{definition}[Normal Covering Space]
    Let $X$ be path-connected. 
    A map $p : \widetilde{X} \to X$ is a \emph{normal covering space} if 
    $\mcal{G}(\widetilde{X})$ acts transitively on $p^{-1}(x)$, for all $x\in X$.
\end{definition}

\begin{remark}
    If $X$ is path-connected, there is an embedding 
    $\mcal{G}(\widetilde{X}) \hookrightarrow p^{-1}(x)$, 
    given by $h \mapsto h(\widetilde{x}_0)$.
\end{remark}

\section{Lecture 2, 11/04/2024}

Recall for each covering space, 
$p : \widetilde{X} \to X$, that $\mcal{G}(\widetilde{X}) := \on{Aut}(p),$
the automorphism group of $p$. We say that $p$ is \emph{normal} if $\mcal{G}(\widetilde{X})$
acts transitively on the fibres $p^{-1}(x)$.

\begin{proposition}[Prop 1.39, Hatcher]
    Let $p : (\widetilde{X}, \widetilde{x}) \to (X,x)$ be a covering space with $\widetilde{X}$ path-connected, 
    and $X$ path-connected, and locally path-connected.
    Then, \begin{itemize}
        \item[(i)]  $p$ is normal if and only if $H:= p_\ast(\pi_1(\widetilde{X},\widetilde{x}))$ is 
        normal in $\pi_1(X,x)$,
        \item[(ii)] There is a group isomorphism: $$\mcal{G}(\widetilde{X}) \cong N(H)/H,$$ where 
        $$N(H) := \lbrace [\gamma]  \in \pi_1(X,x) : [\gamma] H [\overline{\gamma}] = H\rbrace,$$
        is the \emph{normaliser} of $H$.
    \end{itemize}
\end{proposition}

\begin{proof}
    \leavevmode 
    \begin{itemize}
        \item[(i)] Base-point change in $\widetilde{X}$ conjugates $H$ by the image of a path between 
        the basepoints. So, $[p\circ \gamma] \in N(H)$ if and only if 
        $p_\ast(\pi_1(\widetilde{X},\widetilde{x}_2)) = \pi_\ast(\pi_1(\widetilde{X},\widetilde{x}_2))$.
        Then, by lifting, there exists some 
        $$h : (\widetilde{X},\widetilde{x}_1) \longrightarrow (\widetilde{X}, \widetilde{x}_2).$$ 
        Then, using the uniqueness of liftings, we may find an inverse for $h$. Since every loop in $X$ 
        based at $x$ lifts to such a $\gamma$, we see that $p$ is normal if and only if 
        $N(H) = \pi_1(X,x)$ -- that is, $H$ is normal in $\pi_1(X,x)$. 
        \item[(ii)] Define $$\varphi : N(H) \longrightarrow \mcal{G}(\widetilde{X}), \quad [\gamma] \longmapsto h_{[\gamma]},$$
        where $h_{[p\circ \gamma]}.$ By the path-connectivedness of $\widetilde{X}$, $\varphi$ is surjective.
        Given some $h \in \mcal{G}(\widetilde{X})$, take $\gamma$ to be a path from $\widetilde{x}_1$ 
        to $\widetilde{x}_2 = h(\widetilde{x}_1)$. As an exercise, check that $\varphi$ is a 
        homomorphism and $\ker \varphi = H$.
    \end{itemize}
\end{proof}

\subsection{Group Actions}

Given a group $G$, and a space $Y$, an \emph{action} of $G$ on $Y$ is a group homomorphism 
$$\rho : G \longrightarrow \on{Homeo}(Y).$$
We will write group actions as follows: 
$$ G \times Y \longrightarrow Y,\quad (g,y) \longmapsto g \cdot y := \rho(g)y.$$
With this definition, for $g_1, g_2 \in G$, we have $g_1 \cdot (g_2\cdot y) = (g_1g_2)\cdot y$.

\begin{remark}
    If the multiplication and inverse maps $m : G \times G \to G$ and 
    $i : G \to G$ are continuous, then $G$ is called a \emph{topological group}.
    Moreover, if the group $G \times Y \to \on{Homeo}(Y)$ is continuous, then we have a 
    \emph{continuous group action}.
\end{remark}

\begin{definition}[Covering Space Action]
    A space $Y$ is equipped with a \emph{covering space action} if
    for all $y\in Y$, there exists a neighbhourhood $U$ containing $y$ such that 
    for all $g_1 \neq g_2 z \in G$, $g_1U \cap g_2U = \emptyset$.
\end{definition}

A group action defines an equivalence relation on $X$ by setting $x\sim x'$ if and only if 
$x = gx'$, for some $g\in G$. The equivalence classes under this equivalence relation are called 
the \emph{orbits} of $G$: 
$$X/G := [x] = \lbrace x' : x' \sim x\rbrace.$$ 
Then, we wish to determine when the map 
$$q : X \longrightarrow X/G,$$ is a covering space. 

\begin{definition}
    An action is \emph{free} if for all $g \in G$, $gx = x$ implies that $g =e$
    the identity element -- that is, the only fixed point is the identity.
    An action is \emph{transitive} if $X/G$ is a singleton set.
\end{definition}

Note that if $\widetilde{G}(\widetilde{X})$ acts on $\widetilde{X}$, then this action
is a covering space action.
To see this, take an evenly covered subset $U\subset X$, and let 
$\widetilde{x} \in \widetilde{U}_i \subset \widetilde{X}$. Then, 
$$h(\widetilde{x}) = \begin{cases}
    \widetilde{x} \Longleftrightarrow h = \on{id}\\ 
    \widetilde{x} \in \widetilde{U}_i, \quad \widetilde{U}_i \cap \widetilde{U}_j = \emptyset
\end{cases}$$

\begin{proposition}[Prop 1.40, Hatcher]
    \begin{itemize}
        \item[(i)] If $G$ acts on $Y$ as a covering space action, then $Y \to Y/G$ is a covering 
        \item[(ii)] If $Y$ is path-connected, then $G = \mcal{G}(Y)$. 
        \item[(iii)] If $Y$ is path-connecte and 
    locally path-connected, then $G \cong \pi_1(Y/G)/p_\ast(\pi_1(Y))$.

    \end{itemize}
    space.
\end{proposition}

\begin{proof}[Proof Idea]
    \begin{itemize}
        \item[(i)] We wish to show that $Y/G$ is evenly covered. Given $[y]$, take $Y \subset U$,
        where $U$ is as in the covering space action. Then, $[y] \in p(U)$ is 
        evenly covered by $p$. One checks that $p\vert_U : U \to p(U)$ is a 
        homeomorphism and $p^{-1}(p(U)) \cong \bigsqcup U$.
        \item[(ii)] Evidenlty, $G \subseteq \mcal{G}(Y)$ via the group action $G \times Y \to Y$.
        Each $g \in G$ defines a homeomorphism $g : Y \to Y$ that commutes with the projection 
        $Y \to Y/G$. In particular, the diagram: 
        $$\begin{tikzcd}
            Y \arrow[rd] \arrow[rr, "g"] &     & Y \arrow[ld] \\
                                         & Y/G &             
            \end{tikzcd}$$
        commutes. Then, if $Y$ is path-connected, $G \to \mcal{G}(Y)$ is surjective (exercise). 
        \item[(iii)] This is an immediate consequence of Proposition 1.31(ii) from Hatcher, which we proved.
    \end{itemize}
\end{proof}

\begin{example}
    Consider maps 
    $$\pi : \mathbb{S}^n \longrightarrow \mathbb{RP}^n,$$
    where $\mathbb{RP}^n$ is the space of lines pasing through the origin in 
    $\mathbb{R}^n$. There is a $\mathbb{Z}/2\mathbb{Z} \cong \lbrace \pm 1\rbrace$ action on $\mathbb{S}^1$ 
    given by $$\mathbb{S}^1 \times \mathbb{Z}/2 \longrightarrow \mathbb{S}^1,\quad (x,-1) \longmapsto -x.$$
    One readily checks that this defines a covering space action, and thus $\mathbb{S}^n$ defines a 
    covering space of $\mathbb{RP}^n$.
    For $n \geq 2$, we know that $\pi_1(\mathbb{S}^n)$ is trivial, and thus 
    $\mathbb{S}^n \to \mathbb{RP}^n$ is the universal covering. In particular, 
    $\pi_1(\mathbb{RP}^2) \cong \mathbb{Z}/2\mathbb{Z}$ by Proposition 1.40(iii).
\end{example}

In fact, this is a general phenomenon. If we are given a universal covering 
$$\widetilde{X}_{\on{univ}} \longrightarrow X,$$ then 
$$\mcal{G}(\widetilde{X}_{\on{univ}}) \cong \pi_1(X,x).$$
To see this, let us consider 
$$p : \widetilde{X}_{\on{univ}} \longrightarrow X.$$
Then, the fibres of this map are given by 
$$p^{-1}(x) = \Pi_1(X,x,x_0) \equiv \pi_1(X,x_0).$$ 
Since $\widetilde{X}_{\on{univ}} \to X$ is normal, it follows then that 
$$\widetilde{G}(\widetilde{X}_{\on{univ}}) \longrightarrow p^{-1}(x_0),$$ is a bijection 
(of sets). It thus follows that 
$\mcal{G}(\widetilde{X}_{\on{univ}}) \equiv \pi_1(X,x_0)$.

\begin{example}
    Given the covering space $p : \mathbb{R} \to \mathbb{S}^1$, we have that 
    $$\pi_1(\mathbb{S}^1,1) \cong \mathbb{Z} \subseteq \on{Homeo}(\mathbb{R}).$$
    Moreover, the map 
    $$\mathbb{R} \times \mathbb{R} \longrightarrow \mathbb{S}^1 \times \mathbb{S}^1 = T^2.$$ 
    Then, $\pi_1(T^2) = \mathbb{Z}^2 \subseteq \on{Homeo}(\mathbb{R}^2)$.
\end{example}

\subsection{Cayley Graphs and Cayley Complexes}

Consider a group $G$, together with a set of generators $g_1,\cdots, g_n$. 
That is, $$G = \langle g_1,\cdots, g_n : r_1,\cdots, r_m\rangle.$$
Let $S$ be the generating set of $G$. Then, the Cayley graph is constructed as follows:
\begin{itemize}
    \item each element $g\in G$ is assigned to a vertex, 
    \item each element $s \in S$ is assigned a colour $c_s$,
    \item for every $g\in G$ and $s \in S$, there is a directed edge of colour $c_s$ from the vertex
    $g$ to the vertex $gs$
\end{itemize}

\section{Lecture 3, 12/04/2024}

\subsection{Eilenberg-Maclane Spaces}

\begin{definition}
    A path-connected space $X$ is an \emph{Eilenberg-Maclane space} (or $K(G,1)$-space) if 
    $\pi_1(K) \cong G$, and its universal covering $\widetilde{K}_{\on{univ}} \simeq \on{pt}$ -- i.e. $\widetilde{K}$ is contractible.
    Recall that this means that $\pi_i(K) = 0$ for all $i \geq 2$.
\end{definition}

\begin{example}
    $\mathbb{S}^1$, $T^2$, $K^2$, and $F_g^2$ for $g \geq 1$ are all $K(G,1)$-spaces.
\end{example}

$K(G,1)$ always exists, and we can build it by taking $$G = \langle g_1,\cdots, g_n : r_1, \cdots, r_m\rangle.$$
From this, we can then build a space whose fundamental group is $G$ by taking 
$$\mathbb{S}^1_{g_1} \vee \cdots \vee \mathbb{S}^1_{g_n},$$ where the subscript is used to 
label each copy of $\mathbb{S}^1$ in the wedge product. The fundamental group of this is $F_n$, which is 
$G$ but if the relations $r_i$ were all empty. To obtain the relations, we attach $2$-cells to 
$\vee_{i=1}^m \mathbb{S}_{g_i}^1$ along maps generating the relation $r_j$.\\\\
Recall that $\pi_1(T^2) \cong \mathbb{Z}^2 = \langle a,b : [a,b] = 0\rangle$. We may construct 
$$X_G := \left(\vee_{i=1}^m \mathbb{S}^1_{g_i}\right) \cup \left(\bigcup_{i=1}^n D_{ij}^2\right),$$
whici is called the \emph{presentation $2$-complex}. By construction, one may see that 
$$\pi_1(X_G) \cong G.$$
From here, we can then attach $3$-cells to $X_G$ to kill $\pi_2$. Then, we can attach $4$-cells to kill $\pi_3$,
and so on.

\subsection{Homology}

\paragraph{Goal:} Given a space $X$, we wish to associate to it a family of abelian groups $H_i(X)$
that define functors $H(-) : \mathbf{Top} \to \mathbf{Ab}$. More generally, one can also define $H_i(X)$ 
to be $R$-modules. Then, one recovers abelian groups by taking $R = \mathbb{Z}$. To specialise what coefficients 
we are using, we will write $$H_i(X; \mathbb{Z}) = H_i(X).$$ If we are choosing coefficients from an 
$R$-module $M$, we write $$H_i(X;R).$$

\begin{definition}
    The \emph{standard $n$-simplex} is given by 
    $$\Delta^n = \lbrace t = (t_0,\cdots,t_n) \in \mathbb{R}^{n+1} : \sum_{i=0}^n t_i = 1, \quad t_i \geq 0\rbrace.$$
\end{definition}

\begin{example}
    $\Delta^0 = \lbrace 1\rbrace$. $\Delta^1 \cong [0,1]$, and is given by the line going fro $(1,0)$ to $(0,1)$. 
    $\Delta^2$ can be thought of as a triangle 
    with vertices given by the canonical basis vectors of $\mathbb{R}^3$.
\end{example}

More generally, given points $v_0,\cdots,v_n$ linearly independent in $\mathbb{R}^{n+1}$ 
(or any $(n+1)$-dimensional vector space), set 
$$[v_0, \cdots, v_n] := \left\lbrace \sum_{i=0}^n t_iv_i : t \in \Delta^n \right\rbrace.$$
The set $[v_0,\cdots,v_n]$ is called an \emph{affine $n$-simplex}.
Moreover, there is a map 
$$\Delta^n \longrightarrow [v_0,\cdots, v_n],\quad t\longmapsto \sum_{i=0}^n t_iv_i.$$
If we specify an ordering of vertices, then an affine $n$-simplex is canonically homeomorphic to 
$\Delta^n$. The image of the map $\Delta^n \to [v_0,\cdots,v_n]$ are called \emph{barycentric coordinates}.

\begin{definition}
    The \emph{$i$-th face} of $[v_0,\cdots,v_n]$ is the $(n-1)$-simplex 
    $$[v_0,\cdots, \widehat{v_i},\cdots, v_n],$$ where the 
    $\widehat{(-)}$ denotes omission.
    For $\Delta^n$, define a map:
    $$\partial_i \Delta^n = [e_0,\cdots, \widehat{e_i},\cdots,e_n],$$
    called the \emph{boundary map}.
\end{definition}

\begin{remark}
    $\Delta^n \cong D^n \cong B^n \cong \underbrace{I \times \cdots \times I}_{\text{$n$ times}}$.
\end{remark}

Define:
$$\partial \Delta^n := \bigcup_{i=0}^n \partial_i \Delta^n,$$ 
$$\partial [v_0,\cdots, v_n] = \bigcup_{i=0}^n [v_0,\cdots,\widehat{v_i},\cdots, v_n],$$
$$\on{Int}(\Delta^n)  = \Delta^n \setminus \partial \Delta^n,$$ 
hence why $\partial$ is the called the \emph{boundary operator}.

\subsection{Simplicial Homology for $\Delta$-Complexes}

\begin{definition}
    A \emph{$\Delta$-complex} structure on a space $X$ consists of a collection of maps 
    $$\sigma_\alpha : \Delta^n \longrightarrow X,$$ 
    where $n = n(\alpha)$, such that:
    \begin{itemize}
        \item[(i)] $$\sigma_\alpha\vert_{\on{Int}(\Delta^n)} : \on{Int}(\Delta^n) \longrightarrow X,$$
        is injective, and for all $x \in X$, there exists a unique $\alpha$ such that 
        $x \in \on{Im}(\sigma_\alpha \vert_{\on{Int}(\Delta^n)})$,
        \item[(ii)] $$\sigma_\alpha \vert_{\partial_i\Delta^n} \in \lbrace \sigma_\alpha : \text{identify $\partial_i\Delta^n$ with $\Delta^{n-1}$}\rbrace,$$
        \item[(iii)] $U \subset X$ is \emph{open} if and only if $\sigma_\alpha^{-1}(U) \subset \Delta^n$ is open in $\mathbb{R}^{n+1}$ for all $\alpha$.
    \end{itemize}
\end{definition}

\begin{remark}[Diarmuid Wisdom]
    $\Delta$-complexes are important, but they are not the main game. They're just a useful tool for getting us into 
    homology. However, this is not an excuse to fall asleep for the next $15$ minutes.
\end{remark}

\paragraph{Fact:} $$X \cong \left(\bigsqcup_\alpha \Delta_\alpha^{n(\alpha)} \right)/ \sim,$$ 
where $\sim$ is defined via canonical homeomorphisms between faces.

\begin{example}
    The $\Delta$-complex structure on $\mathbb{S}^1$ is given by 
    $$\lbrace \sigma_\alpha \rbrace = \lbrace \sigma^0,\sigma^1\rbrace,$$
    using the fact that $\mathbb{S}^1 \cong I /(0 \sim 1)$. In this case, we have one 
    $0$-simplex labelled by $v_0$, which is the image of $\sigma^0$, and an edge 
    from $v_0$ to itself, which is the image of $\sigma^2$.
\end{example}

\begin{example}
    For $T^2 \cong (I \times I)/\sim$, we have $1$-simplices labelled by 
    $v$, and $2$-simplices labelled by $a$ and $b$, and $3$-simplices labelled by $U$ and $L$
    (given by drawing a diagonal line $c$ -- which is a $1$-simplex -- across the square, and labelling the two partitioned areas 
    by $U$ and $L$). Together, this is written to be:
    $$\lbrace \sigma_v^0, \sigma_a^1, \sigma_b^1, \sigma_c^1, \sigma_U^2, \sigma_L^2\rbrace.$$
\end{example}

\begin{definition}[Homology of $\Delta$-complex Structure]
    Set $$\Delta_n(X) := \left\lbrace \sum_\alpha n_\alpha \sigma_\alpha^n \right\rbrace,$$
    the \emph{free abelian group on the $n$-simplices}, which consists of all formal linear combinations of 
    the $n$-simplices. Let $\Delta_n(X;M)$ be the free abelian group for which $n_\alpha \in M$, where $M$ is an $R$-module.
    That is, $\Delta_n(X;M)$ is a free $R$-module in $\sigma_\alpha^n$, tensored over $R$ 
    with respect to $M$. It follows then that 
    $$\Delta_n(X;M) \cong \bigoplus_{i=1}^{r_n} M,$$ where $r_n$ is the number of $n$-simplices 
    $\sigma_\alpha^n$. Define the \emph{boundary homomorphism} to be the map 
    $$\partial_n : \Delta_n(X) \longrightarrow \Delta_{n-1}(X),\quad \sigma_\alpha^n \longmapsto \sum_{i=0}^n (-1)^i\sigma_\alpha\vert_{\partial_i\Delta^n}.$$
\end{definition}

\begin{remark}
    The factor of $(-1)^i$ is just a miraculous addition that works. We will see in a bit why 
    we need this factor.
\end{remark}

Next, we wish to show that the boundary operator has the property that $\partial^2 = 0$, which we need 
to define homology.

\begin{example}
    Using the simplex $[v_0,v_1]$, we have: 
    $$\partial_1 \circ \partial_2 = \partial_1 ([v_0,v_1] - [v_0,v_2] + [v_1,v_2]) = -v_1 + v_2 -(-v_2 + v_2) + (-v_0 + v_1) = 0.$$
\end{example}

Indeed, we see that this is the case in general:

\begin{lemma}[Lemma 2.1, Hatcher]
    $\partial_{i-1} \circ \partial_i = 0$.
\end{lemma}

\begin{proof}
    \begin{align*}
        \partial_{n-1} \circ \partial_n (\sigma_\alpha^n) &= \partial_{n-1} \left(\sum_{i=0}^n (-1)^i \sigma_\alpha\vert_{[v_0,\cdots,\widehat{v_i},\cdots, v_n]}\right)\\
        &= \sum_{j < i} (-1)^{i+j} \sigma_\alpha\vert_{[v_0,\cdots,\widehat{v_j},\cdots,\widehat{v_i},\cdots, v_n]}
        + \sum_{j > i} (-1)^{i+j} \sigma_\alpha\vert_{[v_0,\cdots,\widehat{v_i},\cdots,\widehat{v_j},\cdots, v_n]}\\
        &= 0,
    \end{align*}
    since $[v_0,\cdots,\widehat{v_i},\cdots,\widehat{v_j},\cdots,v_n] = - [v_0,\cdots,\widehat{v_j},\cdots, \widehat{v_i},\cdots,v_n]$.
\end{proof}

\chapter{Week Seven}

\section{Lecture 1, 17/04/2024}

\begin{definition}
    A \emph{chain complex} (over $R = \mathbb{Z})$ is a sequence of free 
    $R$-modules 
    $$\cdots \longrightarrow C_{i+1} \stackrel{\partial_{i+1}}{\longrightarrow} C_i \stackrel{\partial_i}{\longrightarrow} C_i\longrightarrow \cdots,$$
    and homomorphisms $\partial_i : C_i \to C_{i-1}$ such that $\partial_{i-1} \circ \partial_i = 0$,
    for all $i\in \mathbb{Z}$. Usually, as convention we set $C_i = 0$ for all $i < 0$.
\end{definition}

\begin{definition}
    The $i$-th \emph{homology} of the complex $C_\ast$ is defined to be:
    $$H_i(C_\ast,\partial) := \frac{\ker \partial_i}{\on{Im} \partial_{i+1}}.$$
\end{definition}

\begin{example}
    If $X$ is a $\Delta$-complex, then 
    $$H_i(X) = H_i(\Delta_\ast(X)).$$
\end{example}

\begin{example}
    If $X = \mathbb{S}^1$, then $\Delta^1 = [v_0v_1]$. This is formed from 
    $\lbrace \sigma_0, \sigma_1\rbrace$.
    $f : \Delta^1 \to \mathbb{S}^1$ is the surjection. 
    Then, we have a boundary map 
    $$C_1 \stackrel{\partial_1}{\longrightarrow} C_0,$$
    which is 
    $$0 \longrightarrow \mathbb{Z} \stackrel{0}{\longrightarrow} \mathbb{Z}.$$
    Then, $\partial \delta_1 = f(v_1) - f(v_0) = v - v = 0$.
    Thus, 
    $$H_1(\mathbb{S}^1,\mathbb{Z}) = \begin{cases}
        \mathbb{Z},\quad i = 0,1,\\
        0,\quad \text{otherwise}.
    \end{cases}$$
\end{example}

\begin{example}
    Another $\Delta^1$-structure can be defined on $\mathbb{S}^1$ by taking 
    $\lbrace \sigma_a^1, \sigma_b^1, \sigma_v^0, \sigma^0_w\rbrace$.
    Then, we have a chain complex 
    \begin{align*}
        0 \longrightarrow \mathbb{Z}^2 &\longrightarrow \mathbb{Z}^2 \longrightarrow 0\\ 
        a &\longmapsto w - v\\ 
        b &\longmapsto v-w
    \end{align*}
    Then, 
    $$H_0(\mathbb{S}^1,\mathbb{Z}) \cong \mathbb{Z}^2(v,w)/(v-w) \cong \mathbb{Z},$$
    $$H_1(\mathbb{S}^1,\mathbb{Z}) \cong \ker \partial_1 = \lbrace ma + mb : m \in \mathbb{Z}\rbrace \cong \mathbb{Z},$$
    where the last isomorphism in $H_1(\mathbb{S}^1,\mathbb{Z})$ follows from the fact that 
    $\partial_1(a) = - \partial_1(b)$.
\end{example}

\begin{example}
    Consider the $2$-torus $T^2$. Recall that we have simplices 
    $$\lbrace \sigma_U^2, \sigma_L^2, \sigma_a^1, \sigma_b^1, \sigma_c^1\rbrace.$$
    Then, we have a chain complex 
    \begin{align*}
        0 \longrightarrow \mathbb{Z}^2 = \langle U,L\rangle \longrightarrow \mathbb{Z}^3 = \langle a,b,c\rangle&\longrightarrow \mathbb{Z} \longrightarrow 0\\
        a &\longmapsto v-v = 0\\ 
        b &\longmapsto 0\\
        c & \longmapsto 0
    \end{align*}
    The map $\mathbb{Z}^2 \to \mathbb{Z}^3$ is given by:
    $$U \longmapsto c -a - b,\quad L \longmapsto c-a-b.$$
    Then, we have 
    $$H_2(T^2,\mathbb{Z}) \cong \mathbb{Z}([U-L]),$$
    $$H_1(T^2,\mathbb{Z}) \cong \mathbb{Z}^2([a],[b]),$$
    where $[c] = [a] + [b]$.
    $$H_0(T^2,\mathbb{Z}) \cong \mathbb{Z}([v]).$$
\end{example}

\subsection{Remarks on Simplicial Complex and Homology}

It is important to know the difference between simplicial complex and singular 
complex.

\begin{remark}[Historical Remark]
    A \emph{simplicial complex} is a $\Delta$-complex, where each simplex embeds into 
    $X$. That is, in any a simplicial complex, we know identify faces from different 
    simplices. A \emph{finite simplicial complex} is homeomorphic to a subcomplex 
    of $\Delta^n$ for some $n$.
\end{remark}

\begin{remark}
    The minimal simplicial structure on $T^2$ has $18$ $2$-simplices.
    Chop up $T^2 = (I \times I)/\sim$ into $9$ squares, and divide each square 
    in half along the diagonal to get two triangles. All together, we will have $18$
    triangles, corresponding to $18$ simplices.
    So, we have a chain complex 
    $$0 \longrightarrow \mathbb{Z}^{18} \longrightarrow \mathbb{Z}^{27} \longrightarrow \mathbb{Z}^{9} \longrightarrow 0.$$
    Later we will see that the alternating sum of the dimensions of the complexes -- 
    in this case given by $18 - 27 + 9 = 0$ -- is equal to the Euler characteristic of the space.
    A priori, in this case the Euler characteristic of $T^2$ is $0$.
\end{remark}

\subsection{Singular Homology}

\begin{definition}
    Let $X$ be a space. A \emph{singular $n$-simplex} in $X$ is a map 
    $$\sigma : \Delta^n \longrightarrow X.$$
\end{definition}

\begin{definition}
    The \emph{$n$-th chain group} $C_n(X)$ is the free abelian group on the singular $n$-simplices 
    in $X$. 
    \begin{remark}
        Note that $C_n(X)$ is a large group -- in particular, it is uncountably generated.
    \end{remark}
    In particular, any $\zeta \in C_n(X)$ is a finite formal sum $$\zeta = \sum_i n_i\delta_i,$$
    for some $i \in \mathbb{Z}$, and $n_i \in M$, for some $R$-module $M$.
\end{definition}

\begin{definition}
    The \emph{boundary map} is defined to be: 
    $$\partial_n : C_n(X) \longrightarrow C_{n-1}(X),\quad \sigma \longmapsto \sum_{i=0}^n \sigma\vert_{[v_0,\cdots, \widehat{v_i},\cdots,v_n]},$$
    and extending linearly.
\end{definition}

The same proof seen in Lemma 2.1 shows that the boundary map has the property that $\partial^2 = 0$.\\\\
Let $C_\ast(X) = (C_\ast(X), \partial)$ be the singular chain complex of $X$. 
Then, the $n$-th singular homology group of $X$ is $$H_i(X,\mathbb{Z}) = \frac{\ker \partial_n}{\on{Im} \partial_{n+1}}.$$
If $\partial_\zeta = 0$, we call $\zeta$ a \emph{cycle}, and we write 
$$[\zeta] \in H_n(X,\mathbb{Z}),$$
and call it the \emph{homology class} represented by $\zeta$.

\begin{remark}[Relation to Geomtery]
    Given a cycle $\zeta = \sum_i n_i \sigma_i$, 
    re-write this as $\sum_j \varepsilon_j \sigma_j$, where $\varepsilon_j \in \lbrace \pm 1\rbrace$.
    Since $\partial(\zeta) = 0$, the $(n-1)$-dimensional faces, can be paired and identified 
    to give a $\Delta$-complex, 
    $$f_\zeta : K_\zeta \longrightarrow X.$$
    $K_\zeta$ has a lot of manifold points, and only fails to be a manifold around the $(n-1)$-skeleton.
    In particular, this is a finite $\Delta$-complex, and one can show that $H_n(K_\zeta,\mathbb{Z}) \cong \mathbb{Z}$,
    and $[\zeta] = (f_\zeta)_\ast (1)$, where
    $$(f_\zeta)_\ast : H_i(K_\zeta) \longrightarrow H_i(X).$$
\end{remark}

\paragraph{Slogan:} cycles give geometric carriers of homology classes. \\\\
So, for instance, we have $H_1(T^2,\mathbb{Z}) \cong \mathbb{Z}(a,b)$, 
where the generators correspond to the two copies of the circle embedded in $T^2$ 
given by $\mathbb{S}^1_a \hookrightarrow T^2$, and $\mathbb{S}^1_b \hookrightarrow T^2$,
where the subscript denotes the copy of $\mathbb{S}^1$ corresponding to one of the generators 
of the $1$-st homology.

\begin{proposition}[Prop 2.6, Hatcher]
    Write as a disjoint union of path-connected components:
    $$X = \sqcup_\alpha X_\alpha.$$
    Then, 
    $$H_n(X) = \bigoplus_\alpha H_n(X_\alpha).$$
\end{proposition}

\begin{proof}
    Each $\delta : \Delta^n \to X$ has path-connected image in some $X_\alpha$, and 
    $\partial\sigma$ has image in $X_\alpha$.
    So, $$C_\ast(X,\partial) = \bigoplus_\alpha C_\ast(X_\alpha,\partial_\alpha).$$
    Thus, 
    $$H_\ast(X) \cong \bigoplus_\alpha H_\ast(X_\alpha),$$
    where we can simply write $\ast$ as a variable for $i$, or we can take the 
    \emph{total homology}: 
    $$H_\ast(X) := \bigoplus_{i=0}^\infty H_i(X).$$
    The result works for both definitions of $H_\ast(X)$.
\end{proof}

\begin{proposition}[Prop 2.7, Hatcher]
    If $X \neq \empty$ is path-connected, then $$H_0(X) \cong \mathbb{Z}.$$
\end{proposition}

\begin{proof}
    By definition. $H_0(X) = C_0(X) / \on{Im}\partial_1$. 
    But by definition, the boundary of a $0$-simplex is always $0$, and so 
    $$C_1(X) \stackrel{\partial_1}{\longrightarrow} C_0(X) \longrightarrow 0.$$
    Define $$\varepsilon:  C_0(X) \longrightarrow \mathbb{Z},\quad \sum_i n_i\varepsilon_i \longmapsto \sum_i n_i.$$ 
    Since $X \neq \emptyset$, $\varepsilon$ is onto.
    \paragraph{Claim:} $\on{Im} \partial_1 = \ker \varepsilon$, which would imply that $\varepsilon$ 
    induces a pushforward map 
    $$\varepsilon_\ast : H_0(X) \longrightarrow \mathbb{Z},$$ 
    which is an isomorphism by the claim. We will finish the proof of this next time.
\end{proof}

\begin{remark}
    In general, 
    $$H_0(X) \cong \bigoplus_\alpha H_0(X_\alpha) = \bigoplus_\alpha \mathbb{Z},$$
    where the number of copies of $\mathbb{Z}$ is equal to the number of path-connected 
    components of $X$.
\end{remark}

\section{Lecture 2, 18/04/2024}

\begin{proposition}[Prop 2.7, Hatcher]
    If $X\neq \emptyset$ is path-connected, then $H_0(X) \cong \mathbb{Z}$.
\end{proposition}

\begin{proof}
    Define $$\varepsilon : C_0(X) \longrightarrow \mathbb{Z}, \quad \sum_i n_i \delta_i \longmapsto \sum_i n_i.$$
    \paragraph{Claim:} $\on{Im}\partial_1 = \ker \varepsilon$.\\\\
    Observe that $\on{Im}\partial_1 \subset \ker \varepsilon$, since  if $\delta' : \Delta' \to X$
    is a singular $1$-simplex, then 
    $$\varepsilon(\partial(\sigma')) = \varepsilon (\sigma(v_1) - \sigma(v_0)) = 1 - 1 = 0.$$
    Suppose now that $$\varepsilon\left(\sum_i n_i \sigma^0_i\right) = 0.$$
    Let $x_i = \sigma^0_i(\ast)$. Let $\sigma^1_i$ be paths from a basepoint $x_0$ to 
    $x_i$. 
    \paragraph{Claim:} $\partial_1\left(\sum_i n_i\sigma^1_i\right) = \sum_i n_i \delta_i^0$.\\\\
    To see this, observe that 
    $$\partial\left(\sum_i \sigma_i^1\right) = \sum_i n_i \sigma_i^0 - \underbrace{\sum_i n_i x_0}_{=0}. = \sum_i n_i \sigma_i^0,$$
    where the first equality follows from the fact that $\sigma_i^1(v_0) = x_0$.\\\\
    This claim thus implies the converse inclusion, and thus shows that 
    $\ker \varepsilon = \on{Im}\partial_1$.
\end{proof}

\begin{definition}
    The \emph{reduced homology} of $X$ is given by:
    $$\widetilde{H}_0(X) := H_0(C^{\on{aug}}_\ast),$$
    where $C^{\on{aug}}_\ast$ is the \emph{augmented chain complex}, which is given by:
    $$\cdots \longrightarrow C_n(X) \longrightarrow C_{n-1}(X) \longrightarrow \cdots \longrightarrow C_0(X) \stackrel{\varepsilon}{\longrightarrow} \mathbb{Z} \longrightarrow 0.$$
    This define a chain complex since 
    $\varepsilon \circ \partial_1 = 0$.
\end{definition}

\begin{remark}
    $$\widetilde{H}_i(X) = \begin{cases}
        H_i(X), \quad i > 0\\ 
        H_0(X) / \mathbb{Z},\quad i = 0
    \end{cases}.$$
\end{remark}

\begin{example}
    $$\widetilde{H}_i(\mathbb{S}^0) = \begin{cases}
        0, \quad i > 0,\\ 
        \mathbb{Z},\quad i = 0.
    \end{cases}$$
    Then, 
    $$H_1(\mathbb{S}^0) \cong \mathbb{Z} \oplus \mathbb{Z} \stackrel{\varepsilon}{\longrightarrow} \mathbb{Z}.$$
    Then, $\widetilde{H}_0(\mathbb{S}^0) \cong \mathbb{Z}^{\vert \alpha \vert - 1}$,
    where $\vert \alpha\vert$ denotes the number of path connected components of $X$.
\end{example}

\begin{proposition}[Prop 2.8, Hatcher]
    If $X = \lbrace \on{pt}\rbrace$, then 
    $$H_i(\on{pt}) = 0,\quad \text{for all} \quad i > 0,$$
    $$\widetilde{H}_i(\on{pt}) = 0,\quad \text{for all} \quad i \geq 0.$$
\end{proposition}

\begin{proof}
    $\Delta_n(\on{pt}) = \mathbb{Z}(c_{\on{pt}})$.
    This gives us an $n$-simplex 
    $$\sigma^n : \Delta^n \longrightarrow \on{pt}.$$
    Recall that an $n$-simplex has $(n+1)$-faces. Thus, there is precisely one 
    simplex here, given by 
    $$C_n(\on{pt}) \longrightarrow C_{n-1}(\on{pt}),\quad \sigma \longmapsto \sum_i (-1)^i \sigma^{n-1}.$$
    It follows then that 
    $$C_{2k+1}(\on{pt}) \longrightarrow C_{2k}(\on{pt}),$$
    is the zero map, and the next map 
    $$C_{2k}(\on{pt}) \longrightarrow C_{2k-1}(\on{pt}),$$
    is an isomorphism. This is because if we have a $2$-simplex $\sigma^2 : \Delta^2 \to \on{pt}$,
    then 
    $$\partial \sigma^2 = \underbrace{\sigma^2\vert_{[v_1v_2]} - \sigma^2\vert_{[v_0v_2]}}_{=0} + \sigma^2\vert_{v_0v_1}.$$
    More generally, 
    $$\sigma^n\vert_{[v_0 \cdots \widehat{v_i}\cdots v_n]} = \sigma^n\vert_{[v_0 \cdots \widehat{v_j}\cdots v_n]},$$
    for all $i,j$. So, the chain complex is given by 
    $$C_4 \stackrel{\simeq}{\longrightarrow} C_3 \stackrel{0}{\longrightarrow} C_2 \stackrel{\simeq}{\longrightarrow} C_1 \stackrel{0}{\longrightarrow} C_0 \longrightarrow 0.$$
    Thus, by inspection, 
    $$H_i(C_\ast(\on{pt})) = \begin{cases}
        0 \quad i > 0\\ 
        \mathbb{Z},\quad i = 0
    \end{cases}.$$
\end{proof}

\begin{remark}
    Let $X$ be path-connected, and consider $\pi_1(X,x_0)$. 
    Observe that $\gamma : I \to X$, for $\gamma(0) = \gamma(1) = x_0$, can be 
    identified with a singular $1$-cycle $\Delta^1$.
    So, there is a natural map 
    $$\rho_1 : \pi_1(X,x_0) \longrightarrow H_1(X),\quad [\gamma] \longmapsto [\sigma_\gamma^1],$$
    mapping a path homotopy class to a homology class.
    As we will see, $H_1(-)$ defines a functor as well. As such, what we see here is an example 
    of a natural transformation.
\end{remark}

One may show that the natural map $\rho_1$ is a surjective homomorphism, with kernel given by the 
normal subgroup generated by commutators: 
$$\langle [\gamma_1], [\gamma_2] \rangle.$$
That is, one may identify 
$$H_1(X) = \pi_1(X,x_0)_{\on{ab}},$$
where $\pi_1(X,x_0)_{\on{ab}}$ denotes the \emph{abelianisation} of $\pi_1(X,x_0)$.
In particular, given any group $G$, $$G_{\on{ab}} := G /\langle [g,h]\rangle.$$
(see Hatcher A.2 for more details). 
More generally, there exists maps 
$$\rho_n : \pi_n(X,x_0) \longrightarrow H_n(X),$$
called \emph{Hurewitz homomorphisms} (look up Hurewitz's theorem for more information, or 
wait until week 11). 

\subsection{Homotopy Invariants of Homology}


\begin{definition}
    Let $f : X \to Y$ be a continuous map. Then, $f$ induces a map of chain complexes:
    $$f_\# : C_n(X) \longrightarrow C_n(X), \quad \sigma_\alpha \longmapsto f \circ \sigma_\alpha.$$
    Extending linearly, the map is defined to be:
    $$\sum_\alpha n_\alpha \sigma_\alpha \longmapsto \sum_\alpha n_\alpha \left(f\circ \sigma_\alpha\right).$$
    The composition $f\circ \sigma : \Delta^n \to Y$ is continuous, and thus defines a singular complex in 
    $Y$. Further, a \emph{chain map} is given by:
    $$\begin{tikzcd}
        \cdots \arrow[r] & C_n(X) \arrow[r, "\partial"] \arrow[d, "f_\#"] & C_{n-1}(X) \arrow[r] \arrow[d, "f_\#"] & \cdots \\
        \cdots \arrow[r] & C_n(Y) \arrow[r, "\partial"]                   & C_{n-1}(Y) \arrow[r]                   & \cdots
    \end{tikzcd}$$
    such that each square commutes. Indeed, by construction the squares commute since:
    $$\partial (f_\# \sigma) = \partial (f \circ \sigma) = \sum_i f\circ \partial_i\sigma = f_\# \left(\sum_i \partial_i\sigma\right) = f_\#(\partial \sigma).$$
    Hence, $$\partial f_\# = f_\# \partial.$$
    In general, a collection of maps $f_i : C_i \to D_i$, for $i \in \mathbb{Z}$ is called a \emph{chain map} 
    if $f \partial f_i = f_{i-1}\partial$, for any abstract chain complexes $C_\ast$ and $D_\ast$.
\end{definition}

\begin{remark}
    The study of chain complexes and their properties is called \emph{homological algebra}.
\end{remark}

\begin{lemma}
    Any chain map induces a homomorphism 
    $$H_i(C_\ast) \longrightarrow H_i(D_\ast),\quad [c] \longmapsto [f_i(c)].$$
\end{lemma}
\begin{proof}
    $[c] \in H_i(C_\ast)$ if and only if $\partial c = 0$.
    Since $\partial f_i(c) = f_i (\partial c) = 0$, 
    it follows that $[f_i(c)] \in H_i(D_\ast)$.
    Moreover, 
    $[c] = c + \partial_{i+1}(C_{i+1}),$ 
    which gets mapped to $f_i([c]) + \partial_{i-1}(D_{i+1})$.
\end{proof}

It follows then that $f : X \to Y$ induces a map in homology:
$$f_\ast : H_i(X) \longrightarrow H_i(Y).$$

\paragraph{Basic Properties:} 
\begin{enumerate}
    \item The composition $X \stackrel{f}{\to} Y \stackrel{g}{\to} Z$, induces a map 
    $$(g\circ f)_\ast : H_i(X) \stackrel{f_\ast}{\longrightarrow} H_i(Y) \stackrel{g_\ast}{\longrightarrow} H_i(Z).$$
    \item The identity $1_X$ induces the identity map $1_{H_i(X)}$ in homology.
\end{enumerate}

The following is the non-trivial property:

\begin{theorem}[Thm 2.10, Hatcher, Homotopy Invariance]\label{thm2.10}
    Let $f,g : X \to Y$ be homotopic maps. 
    $$f_\ast = g_\ast : H_i(X) \longrightarrow H_i(Y).$$
\end{theorem}

\begin{corollary}[Cor 2.11, Hatcher]
    If $f : X \to Y$ is a homotopy equivalence, with homotopy inverse $g : Y \to X$,
    then, $f_\ast : H_i(X) \to H_i(Y)$ is an isomorphism with inverse 
    $g_\ast : H_i(Y) \to H_i(X)$.
\end{corollary}

\begin{proof}
    $g\circ f : X \to X$ is homotopic to the identity $1_X$. Hence, $(g \circ f)_\ast = (1_X)_\ast = 1_{H_i(X)}$.
    But on the other hand, $g_\ast \circ f_\ast = 1_{H_i(X)}$. Similarly, one shows that $f_\ast \circ g_\ast = 1_{H_i(Y)}$.
\end{proof}

\begin{proof}[Proof of Theorem \ref{thm2.10}]
    Let $H : X \times I \to Y$ be a homotopy from $f$ to $g$. 
    That is, we have maps: $$g : X \times \lbrace 1 \rbrace \to Y,\quad f : X \times \lbrace 0\rbrace \longrightarrow Y.$$
    We may define a simplex: $$\sigma \times \on{id}_I : \Delta^n \times I \longrightarrow X \times I,$$
    with $$\sigma^n : \Delta^n \longrightarrow X \times \lbrace 0 \rbrace.$$
    For $C_0(X)$, $\on{pt} \times X \cong \Delta^1$. For $C_1(X)$, 
    $\Delta^1 \times I \cong \Delta^1 \cup \Delta^1$. This is a square with vertices 
    given by $v_0, v_1, w_0, w_1$.
    Thus, we have 
    $$\Delta^1 \cup \Delta^1 \stackrel{g\circ \sigma^1}{\longrightarrow} X \times I \stackrel{H}{\longrightarrow} Y,$$
    the composition of which is $f\circ \sigma^1$. We will finish this proof of tomorrow. Or you can go read Hatcher.
\end{proof}

\section{Lecture 3, 19/04/2024}

\subsection{Assignment Tips}

\paragraph{Q 2.4} Uses cellular homology and tensor products. In the case of the $2$-torus, 
recall that we have a $\Delta$-complex given by 
$\lbrace \sigma_v^0, \sigma_a^1, \sigma_b^1, \sigma_c^1, \sigma_U^2, \sigma_L^2\rbrace$.
This gives us a chain complex $$\mathbb{Z}^2 \longrightarrow \mathbb{Z}^3 \longrightarrow \mathbb{Z}.$$
The Klein bottle $K$ has a chain complex given by 
$$\mathbb{Z} \longrightarrow \mathbb{Z}^2 \longrightarrow \mathbb{Z}.$$ 
The second map will be zero no matter how we define these maps.
Recall that $H_1(X) \cong \pi_1(X,x)_{\on{ab}}$. The cell attaching maps 
$\mathbb{S}^1 \to \mathbb{S}^1_a \vee \mathbb{S}^1_b$, we induce a map 
$$\underbrace{H_1(\mathbb{S}^1)}_{\cong \mathbb{Z}} \longrightarrow \underbrace{H_1(\mathbb{S}_a^1 \vee \mathbb{S}_b^\vee)_{\cong \mathbb{Z} \oplus \mathbb{Z}}}, \quad z \longmapsto aba^{-1}b, \quad 1 \longmapsto (0,2).$$
One checks that $H_1(K^2) \cong \pi_1(K^2) \cong (\mathbb{Z} \rtimes \mathbb{Z})_{\on{ab}}$.

\paragraph{Q 2.2} This is probably the hardest question on the assignment. A \emph{manifold with boundary} 
is a topological space $M$ such that $x\in \partial M$ is locally homeomorphic to the upper half plane 
$\mathbb{R}^{\geq 0} \times \mathbb{R}^{n-1}$. For every point in the interior of $M$ is locally homeomorphic 
to $\mathbb{R}^n$.\\\\ 
One can glue two copies of the manifold $M$ together via the antipodal map by 
$$\widetilde{X} = M \cup_a M,$$ where $a : \mathbb{S}^n \to \mathbb{S}^n$ is the antipodal map.
Then, $\mathbb{Z}/2$ acts on $\widetilde{X}$ by the twisting around the area where we glue.
We actually end up with a free $\mathbb{Z}/2$ action, and free $\mathbb{Z}/2$ actions are (inaudible).

\subsection{Homotopy Invariants}

\begin{theorem}
    If $f,g : X \to Y$ are homotopic maps, then $f_\ast = g_\ast : H_i(X) \to H_i(Y)$.
\end{theorem}

\begin{proof}
    Let $H : X \times I \to Y$ be the homotopy. Then, we have induced chain maps:
    $$f_\ast, g_\ast : C_\ast(X) \longrightarrow C_\ast(Y).$$
    Then, we have a map 
    $$\Delta^n \times I \stackrel{\sigma \times 1_XI}{\longrightarrow} X \times I \stackrel{H}{\longrightarrow} Y.$$
    Let us first restrict to the case where $n = 1$.
    In this case, we have $\Delta^1 \times I$, which is a square 
    whose vertices are labelled by $v_0,v_1,w_0,w_1$. We have maps:
    $$g : \Delta^1 \times \lbrace 1\rbrace \longrightarrow Y, \quad f : \Delta^1 \times lbrace 0 \rbrace \longrightarrow Y.$$
    Then, this induces maps 
    $$f_\#(\sigma^1) = f\circ \sigma\vert_{[v_0,v_1]},$$
    $$g_\#(\sigma^1) = g\circ \sigma\vert_{[w_0,w_1]}.$$
    Define $$P : C_1(X) \longrightarrow C_2(X), \quad \sigma \longmapsto H\circ \sigma^2\vert_{[v_0,w_0,w_1]} - H\circ \sigma\vert_{[v_0,v_1,w_1]},$$
    called the \emph{prism operator}.
    Then, applying the boundary map, we obtain:
    \begin{align*}
        \partial ([v_0,w_0,w_1] - [v_0,v_1,w_1]) &= ([v_0,w_0] - [v_0,w_1] + [w_0,w_1]) - ([v_0,v_1] - [v_0, w_1] + [v_1,w_1])\\ 
        &= [w_0, w_1] - [v_0,v_1] + [v_0,w_1] - [v_1,w_1]\\ 
        &= -P \partial
    \end{align*}
    What this shows us is that $\partial P = g_\# - f_\# = - P\partial$.
    Indeed, we can chop $\Delta^n \times I$ into $(n-1)$ $n$-simplices via 
    $$\bigcup_{i=0}^n [v_0,\cdots, v_i,w_i, \cdots, w_m].$$
\end{proof}

\begin{remark}
    More explicitly, in the example above, what we really have are maps:
    $$\begin{tikzcd}
        {\Delta^2_{[v_0,v_1,v_2]}} \arrow[r]  & Y \arrow[rd] \arrow[rr, "g"] &     & Y \arrow[ld] \\
        {\Delta^2_{[w_0,w_1,w_2]}} \arrow[ru] &                              & Y/G &             
        \end{tikzcd}$$
    and the prism operator just takes the difference betwen the two $2$-simplices that are being mapped in.
\end{remark}

\begin{definition}
    Generally, a \emph{prism operator} is a map 
    $$P_H :  C_n(X) \longrightarrow C_{n+1}(Y),$$ 
    such that $\partial P = g_\# - f_\# - P \partial$.
\end{definition}

\chapter{Week Eight}

\section{Lecture 1, 24/04/2024}

\subsection{Relative Homology} 

For any pair $(X,A)$, define: 
$$C_n(X,A) := \frac{C_n(X)}{C_n(A)}.$$
Recall that a singular pair is a map $\Delta^n \to X$, which might factor through $A$, in which 
case one may consider a singular pair in $A$. This leads us to consider pairs $(X,A)$.
Since we have boundary maps:
$$\begin{tikzcd}
    C_n(X) \arrow[r, "\partial"]           & C_{n-1}(X)           \\
    C_n(A) \arrow[u] \arrow[r, "\partial"] & C_{n-1}(A) \arrow[u]
\end{tikzcd}$$
we thus have an induced boundary map
$$\partial : \frac{C_n(X)}{C_n(A)} \longrightarrow \frac{C_{n-1}(X)}{C_{n-1}(A)} =: C_{n-1}(X,A).$$

\begin{lemma}
    There exists a long exact sequence 
    $$H_{k-1}(X,A) \stackrel{\partial}{\longrightarrow} H_k(A) \stackrel{i_\ast}{\longrightarrow} H_k(X) \longrightarrow H_k(X,A) \stackrel{\partial}{\longrightarrow} H_{k-1}(A).$$
\end{lemma}

\begin{remark}
    By abuse of notation, we are denoting all the boundary maps to be $\partial$, but in practise they 
    are different maps.
\end{remark}

\begin{proof}[Proof Sketch]
    We have a level-wise short exact sequence of chain complexes and chain maps, given by 
    the chain map 
    $$0 \to C_\ast(A) \stackrel{i_\ast}{\to} C_\ast(X) \stackrel{j_\ast}{\to} C_\ast(X,A) \to 0.$$
    Whenever we have a short exact sequence of chain complexes,  
    $$0 \to A_\ast \to B_\ast \to C_\ast \to 0,$$ we obtain a long exact sequence in homology 
    $$\cdots \to H_{i+1}(C) \stackrel{\partial}{\to} H_i(A) \stackrel{i_\#}{\to} H_i(B) \stackrel{j_\#}{\to} H_i(C) \to \cdots.$$
    The key point of the proof is the definition of $\partial : H_{i+1}(C) \to H_i(A)$.
    We have chain maps:
    $$\begin{tikzcd}
        A_{n+1} \arrow[r] \arrow[d] & A_n \arrow[r] \arrow[d] & A_{n-1} \arrow[r] \arrow[d] & A_{n-2} \arrow[d] \\
        B_{n+1} \arrow[r] \arrow[d] & B_n \arrow[r] \arrow[d] & B_{n-1} \arrow[r] \arrow[d] & B_{n-2} \arrow[d] \\
        C_{n+1} \arrow[r]           & C_n \arrow[r]           & C_{n-1} \arrow[r]           & C_{n-2}          
    \end{tikzcd}$$
    Using the fact that each vertical map is short exact, we may use a diagram chasing argument to conclude that 
    $\partial \widetilde{a} = 0$.
\end{proof}

\section{Lecture 2, 25/04/2024}

\section{Lecture 3, 26/04/2024}

\chapter{Week Nine}

\section{Lecture 1, 01/05/2024}

\section{Lecture 2, 02/05/2024}

\section{Lecture 3, 03/05/2024}

\chapter{Week Ten}

\section{Lecture 1, 08/05/2024}

\subsection{Equivalence of Simplicial and Singular Homology}

We wish to prove naturality: that is, a diagram as below induces a commutative diagram of $H_\ast$ 
long exact sequences:
$$\begin{tikzcd}
    \cdots \arrow[r] & {H^\Delta_{k+1}(X^{i+1},X^i)} \arrow[r] \arrow[d] & H^\Delta_{k}(X^i) \arrow[r] \arrow[d] & H_k^\Delta(X^{i+1}) \arrow[r] \arrow[d] & H_k^\Delta(X^i) \arrow[d] \arrow[r] & \cdots \\
    \cdots \arrow[r] & {H_{k+1}(X^{i+1},X^i)} \arrow[r]                  & H_k(X^i) \arrow[r]                    & H_k(X^{i+1}) \arrow[r]                  & {H_k(X^{i+1},X^i)} \arrow[r]        & \cdots
\end{tikzcd}$$

\paragraph{Strategy:} induction on $\dim X$ for $H_k^\Delta(X^i) \to H_k(X^i)$. Our basic input is something of the form 
$H_k^\Delta(X^{i+1},X^i) \to H_k(X^{i+1},X^i)$. These are isomorphisms:
$$H_k^\Delta(X^{i+1},X^i) \cong \bigoplus_\alpha \mathbb{Z}(\Delta_\alpha^k), \quad H_k(X^{i+1},X^i) \cong H_k(\vee_\alpha \mathbb{S}^k) \cong \bigoplus_\alpha \mathbb{Z}_\alpha.$$

\begin{lemma}\label{lem111}
    Singular homology $H_n(\Delta^n, \partial\Delta^n)$ is generated by $[\on{id} : \Delta^n \to \Delta^n]$, where 
    $$\partial \on{id} = \sum_{i=0}^n \on{id}\vert_{\partial_i\Delta^n} : \partial_i\Delta^n \longrightarrow \partial\Delta^n \subset \Delta^n.$$
\end{lemma}

\begin{proof}
    The proof is by induction. Let $\Lambda \supset \Delta^n$ be the union of the last $n$ faces.
    That is, 
    $$\Lambda = \bigcup_{i=1}^n [v_0,\cdots,\widehat{v_i},\cdots,v_n].$$
    \begin{example}
        For $n = 2$, we have a triangle with vertices labelled by $v_0,v_1,v_2$ with that orientation. 
        Then, $\Lambda$ is given by omiiting $v_0v_1$, and $v_0v_2$. In fact, $\Lambda$ is a cone on the simplex of the boundary 
        of the simplex. Hence, $\Lambda$ is contractible, since all cones are contractible, and $\Lambda$ has the homology 
        of a point.
    \end{example}
    Generally, $\Lambda$ is a cone, or a "horn" as Diarmuid calls it because of its shape. Thus, let us consider the triple $(\Delta^n, \partial\Delta^n, \Lambda)$.
    Consider the map on relative homology:
    $$H_n(\Delta^n, \partial\Delta^n) \longrightarrow H_{n-1}(\partial \Delta^n, \Lambda) \longrightarrow \underbrace{H_{n-1}(\Delta^n, \Lambda)}_{=0},$$
    where the last homology group goes to $0$ since $\Delta^n$ and $\Lambda$ have the same homotopy type. 
    Thus, $H_n(\Delta^n,\partial\Delta^n) \cong H_{n-1}(\partial \Delta^n,\Lambda)$. Additionally, we have a map:
    $$\begin{tikzcd}
        0 \arrow[r] & {H_n(\Delta^n,\partial\Delta^n)} \arrow[r, "\partial"] & {H_{n-1}(\partial\Delta^n,\Lambda)} \arrow[r]          & 0 \\
                    &                                                        & {H_{n-1}(\Delta^{n-1},\partial\Delta^{n-1})} \arrow[u] &  
    \end{tikzcd}$$
    Since there are relative homeomorphisms:
    $$\begin{tikzcd}
        {(\partial \Delta^n, \Lambda)} \arrow[r]                   & (\partial\Delta^n/\Lambda)                                  \\
        {(\Delta^{n-1}, \partial\Delta^{n-1})} \arrow[r] \arrow[u] & \Delta^{n-1}/\partial\Delta^{n-1} \arrow[u, "\text{homeo}"]
        \end{tikzcd}$$
    there are additions that we can make to the sequence:
    $$\begin{tikzcd}
        0 \arrow[r] & {H_n(\Delta^n,\partial\Delta^n)} \arrow[r, "\partial"] & {H_{n-1}(\partial\Delta^n,\Lambda)} \arrow[r] \arrow[rd, no head, Rightarrow] & 0                                 \\
                    &                                                        & {H_{n-1}(\Delta^{n-1},\partial\Delta^{n-1})} \arrow[u] \arrow[d]              & H_{n-1}(\partial\Delta^n/\Lambda) \\
                    &                                                        & H_{n-1}(\Delta^{n-1}/\partial\Delta^{n-1}) \arrow[ru, "\cong"']               &                                  
        \end{tikzcd}$$
    This will be the inductive step that we will perform.
    Reduce to the $n=1$ case. Then, by Lemma \ref{lem111}, we have that there is a diagram:
    $$\begin{tikzcd}
        {H_i^\Delta(X^i,X^i) \cong \mathbb{Z}([\on{id}:\Delta_\alpha^i \to \Delta_\beta^i])} \arrow[r, "\cong"] \arrow[rdd, bend right=49] & {H_i(X^i,X^{i-1})} \arrow[d] \\
                                                                                                                                           & {H_i(X^i,X^{i-1})} \arrow[d] \\
                                                                                                                                           & H_i(\vee_\alpha\mathbb{S}^i)
        \end{tikzcd}$$
    By induction on the $5$-lemma, it follows that 
    $$H_k^\Delta(X) \longrightarrow H_k(X),$$ is an isomorphism for all $k$. If $(X,A)$ is a $\Delta$-complex pair, then 
    $H_k^\Delta(X,A) \to H_k(X,A)$ is an isomorphism for all $k$.
    Then, by the $5$-lemma:
    $$\begin{tikzcd}
        H_i^\Delta(A) \arrow[r] \arrow[d, "\cong"] & H_i^\Delta(X) \arrow[r] \arrow[d, "\cong"] & {H_i^\Delta(X,A)} \arrow[d, "\text{$\cong$ by $5$-lemma}"] \\
        H_i(A) \arrow[r]                           & H_i(X) \arrow[r]                           & {H_i(X,A)}                                                
        \end{tikzcd}$$
\end{proof}

\subsection{CW-Homology}
A CW-complex of $X$ is a chain complex 
$$C_\ast^{\on{CW}}(X) = \left(\bigoplus_{i=0}^\infty C_i^{\on{CW}}(X), \partial_i\right),$$ 
where each $$C_i(X) = \mathbb{Z}(e_\alpha^i).$$
We are going to make the identifications $$C_i(X) \cong H_i(X^i,X^{i-1}) \cong H_i(X^i/X^{i-1}).$$

\begin{theorem}\label{thm_comparison}
    There exists a natural isomorphism $$H_i^{\on{CW}}(X) \longrightarrow H_i(X).$$
\end{theorem}

\begin{lemma}[Lemma 2.34, Hatcher]\label{lem2.34}
    \leavevmode 
    \begin{itemize}
        \item[(i)]  $$H_k(X^, X^{i-1}) = \begin{cases}
            0,\quad k \neq n\\ 
            \mathbb{Z}|^{c_n}, \quad k =n
        \end{cases}$$
        \item[(b)] $H_k(X^n) = 0$ for all $k > n$,
        \item[(c)] $X^n \hookrightarrow X$ induces $H_i(X^n) \to H_i(X)$, 
        which is an isomorphism for all $i < n$, and surjection for all $i = n$.
    \end{itemize}
\end{lemma}
\begin{proof}
    (i) follows by construction. For (ii) and (iii), consider a long exact sequence:
    $$\begin{tikzcd}
        \cdots \arrow[r] & {H_{k+1}(X^n,X^{n-1})} \arrow[r] & H_k(X^{n-1}) \arrow[r] & H_k(X^n) \arrow[r] & {H_k(X^n,X^{n-1})} \arrow[r] & H_{k-1}(X^{n-1}) \arrow[r] & \cdots
    \end{tikzcd}$$
    If $k > n$, 
    $$H_k(X^n) \cong H_k(X^{n-1}) \cong H_k(X^{n-2}) \cong \cdots \cong H_k(X^0) \cong 0,$$
    where $X^0 = \on{pt}$. This proves (ii).
    Note that if $k < n$, we have an isomorphism $$H_k(X^{n-1}) \to H_k(X^n) \to H_k(X^{n+1}) \to \cdots,$$
    unless $\dim X$ is infinite, in which case just go see Hatcher. For $k = n$, we argue similarly.
\end{proof}

Now, we want to construct a map between cellular homology, and argue that it is an isomorphism.

\begin{proof}[Proof of Theorem \ref{thm_comparison}]
    Could repeat $H_\ast^\Delta$-argument. Assume that $$H_n(X^{n+1}) \longrightarrow H_n(X),$$ 
    is an isomorphism. Then, consider the following diagram:
    $$\begin{tikzcd}
        H_n(X^{n-1}) \arrow[rd, "\text{$=0$ by Lemma \ref{lem2.34}(b)}"]                     &                                           & H_n(X^{n+1}) \cong H_n(X) \arrow[r]                                        & {H_n(X^{n+1},X^n) = 0}     \\
                                                                                             & H_n(X^n) \arrow[ru, two heads] \arrow[rd] &                                                                            &                            \\
        {H_{n+1}(X^{n+1},X^n)} \arrow[ru, "\partial"] \arrow[rr, "\partial_{n+1}^{\on{CW}}"] &                                           & {H_n(X^n,X^{n-1})} \arrow[r, "\partial_n^{\on{CW}}"] \arrow[d, "\partial"] & {H_{n-1}(X^{n-1},X^{n-2})} \\
                                                                                             &                                           & H_{n-1}(X^{n-1}) \arrow[ru, "j_{n-1}"']                                    &                            \\
        0 \arrow[r]                                                                          & H_{n-1}(X^{n-2}) \arrow[ru]               &                                                                            &                           
    \end{tikzcd}$$
    By a diagram chase argument, we have isomorphisms:
    $$H_n(X) \cong H_n(X^{n+1}) \cong H_n(X^n)/ \on{Im}(\partial_{n+1}) \cong j (H_n(X^n)) / \on{Im}\partial_n^{\on{CW}} \cong \ker (\partial_n)/\on{Im}(\partial_n^{\on{CW}}).$$
    But since $j_{n-1}$ is injective, we thus have that $$\frac{\ker\partial_n}{\on{Im}\partial_{n+1}^{\on{CW}}} \cong \frac{\ker\partial_n^{\on{CW}}}{\on{Im}(\partial_{n+1}^{\on{CW}})} = H_n^{\on{CW}}(X^n,X^{n-1}).$$
\end{proof}

It follows that map $H_i^{\on{CW}}(X) \to H_i(X)$ is a zig-zag:
$$\begin{tikzcd}
    & H_i(X) &                    \\
    H_i(X^i) \arrow[ru, "i"] \arrow[rr] &        & {H_i(X^i,X^{i-1})} \\
    {[c]} \arrow[rr, maps to]           &        & {[c]_{\on{CW}}}   
\end{tikzcd}$$ It is an exercise to check that this is well-defined.

\subsection{Cellular Boundary Map}

We have:
$$H_i(X^i,X^{i-1}) \longrightarrow H_{i-1}(X^{i-1}) \stackrel{j}{\longrightarrow} H_{i-1}(X^{i-1}, X^{i-2}),$$
for an $i$-cell $e_\alpha^i = \psi(D_\alpha^i)$. Using the fact that $(D^i,\mathbb{S}^{i-1}) \cong (\Delta^i, \Delta^{i-1})$ and 
$[\on{id} : (\Delta^i,\partial\Delta^i) \to (\Delta^i, \partial\Delta^i)]$ generates 
$H_i(\Delta^i, \partial\Delta^i) \cong H_i(D^i, \mathbb{S}^{i-1})$, we see that 
$\partial_{\on{CW}}(e^i_\alpha)$ is given by taking 
$$H_{i-1}(\mathbb{S}^{i-1}_\alpha) \stackrel{\varphi_\alpha}{\longrightarrow} H_{i-1}(X^{i-1}) \stackrel{q}{\longrightarrow} H_{i-1}(X^{i-1}/X^{i-2}) \cong H_i(\vee_\beta \mathbb{S}_\beta^i) \stackrel{\on{pr_\beta}}{\longrightarrow} \mathbb{S}_\beta^{i-1},$$
where $\on{pr}_\beta$ is the projection map.
Recall that $\partial(e_\alpha^i) = \sum n_{\alpha\beta} e^{i-1}_\beta$, where 
$$n_{\alpha\beta} = \deg \left(\varphi_\alpha \circ q  \circ \on{pr}_\beta \right).$$

\section{Lecture 2, 09/05/2024}

\subsection{Degree and Local Degree}

Suppose that $y \in \mathbb{S}^n$ is such that $$\vert f^{-1}(y)\vert < \infty.$$
That is, we are supposing that $y$ is a regular value of a smooth map. 
We wish to show that for each $x_i \in f^{-1}(y)$, we may assign a degree 
$\deg_{x_i} (f) \in \lbrace \pm 1\rbrace$ such that $$\deg(f) = \sum_i \deg_{x_i}(f).$$
Let $V \subset \mathbb{S}^n$ be a neighbourhood of $y$ such that $V \cong \mathbb{R}^n$, and 
$x_i \in U_i$ neighourhoods such that $f\vert_{U_i} : U_i \to V_i$.
On homology, we have:
$$H_n(\mathbb{R}^n, \mathbb{R}^n\setminus \lbrace \on{pt}\rbrace) \stackrel{\partial}{\longrightarrow} H_n(\mathbb{R}^n\setminus \lbrace \on{pt}\rbrace) \cong H_{n-1}(\mathbb{S}^{n-1}) \cong \mathbb{Z},$$
where the isomorphism follows since $\mathbb{R}^n \setminus \lbrace \on{pt}\rbrace$ deformation retracts to 
$\mathbb{S}^{n-1}$. Thus, for our choice of open sets, we have:
$$\begin{tikzcd}
    & {H_n(U_i, U_i\setminus\lbrace x_i\rbrace)} \arrow[r, "f_\ast"] \arrow[d] \arrow[ld, "\cong"] & {H_n(V,V\setminus \lbrace y\rbrace)} \arrow[d]            \\
{H_n(\mathbb{S}^n,\mathbb{S}^n\setminus\lbrace x_i\rbrace)} \arrow[r] & {H_n(\mathbb{S}^n,\mathbb{S}^n\setminus f^{-1}(y))} \arrow[r]                                & {H_n(\mathbb{S}^n,\mathbb{S}^n\setminus\lbrace y\rbrace)} \\
    & H_n(\mathbb{S}^n) \arrow[r] \arrow[u] \arrow[lu, "\cong"]                                    & H_n(\mathbb{S}^n) \arrow[u, "\cong"]                     
\end{tikzcd}$$
The left two isomorphisms follow by excision.
Explicitly, we have:
$$H_n(\mathbb{S}^n, \mathbb{S}^n \setminus f^{-1}(y)) \cong \bigoplus_{\vert f^{-1}(y)\vert} \mathbb{Z}.$$
Let $C = \mathbb{S}^n/mH(U_i)$, which is homotopy equivalent to $\mathbb{S}^n \setminus f^{-1}(y)$. 
Thus, on homology, we have: 
$$H_n(\mathbb{S}^n, \mathbb{S}^n\setminus f^{-1}(y)) \cong H_n(\mathbb{S}^n, C) \cong H_n(\mathbb{S}^n/C) \cong H_n(V\mathbb{S}^n).$$
We have a map: $H_n(\mathbb{S}^n) \to H_n(\mathbb{S}^n/C)$, which becomes $$\Delta : \mathbb{Z} \longrightarrow \mathbb{Z}^{\vert f^{-1}(y)\vert},$$ 
and 
$$H_n(\mathbb{S}^n, \mathbb{S}^n\setminus f^{-1}(y)) \longrightarrow H_n(\mathbb{S}^n,\mathbb{S}^n\setminus \lbrace y\rbrace),$$ 
whichb ecomes a map 
$$\mathbb{Z}^{\vert f^{-1}(y)\vert} \longrightarrow \mathbb{Z}, \quad (d_1,\cdots,d_m) \longmapsto d_1 + \cdots + d_m,$$
where for simplicity we are writing $m = \vert f^{-1}(y)\vert$.
This fact follows from the commutativity of the diagram above.

\paragraph{Question:} Is it true that $\deg_{x_i}(f) \in \lbrace \pm 1\rbrace$? This would imply that if $y$ 
is regular, then $f\vert_{U_i} : U_i \to V$ is a diffeomorphism.

\begin{example}
    Consider the map $\C \to \C$ defined by $z\mapsto z^2$. We have that $f^{-1}(0) = 0$, thus regular. 
    Then, $\deg_0(f) = 2$. 
\end{example}

\begin{example}
    Consider $f_d : \mathbb{S}^1 \to \mathbb{S}^1$ given by $z\mapsto z^d$. Choose an orientation for the 
    sphere. We have that $\deg(f_d) = d$. By construction, $f_{-1} : z \mapsto \overline{z}$, 
    and we have that $\deg(f_{-1}) = -1$.
\end{example}

\paragraph{Properties of Degrees}
\begin{enumerate}
    \item $\deg(R) = -1$ for $R$ a reflection, 
    \item If $f$ has no fixed points, then $\deg(f) = \deg(A) = (-1)^{n+1}$
\end{enumerate}

\chapter{Week Eleven}

\section{Lecture 1, 15/05/2024}

\begin{proposition}[Prop 2.30 (extended)]
    Let $f : M \to N$ be a map between closed, connected, oriented $n$-manifolds 
    (or a self-map of a closed, connected, orientable $n$-manifold $h : M \to M$), then 
    $$H_n(M) \cong \mathbb{Z}([M]), \quad H_n(N) \cong \mathbb{Z}([N]),$$ 
    and define $\deg(f)$ by $f_\ast[M] \deg(f) \cdot [N]$.
    Then,
    $$\deg(f) = \sum_{x_i \in f^{-1}(y)} \deg_{x_i}(f),$$
    where $\deg_{x_i}(f)$ is the local degree of $f$ at $x_i$.
\end{proposition}

\begin{proof}
    Take $M/ \left(M\setminus \bigcup_{i=1}^n U_i \right) \cong \vee \mathbb{S}^n$.
\end{proof}

\end{document}