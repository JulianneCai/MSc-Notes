\documentclass[a4paper]{report}
\usepackage[usenames,dvipsnames]{xcolor}
\usepackage{hyperref, lineno} % TODO REMOVE
\usepackage{mystyle}
%
% \graphicspath{{images/}}
% \SetWatermarkAngle{90} % TODO REMOVE
% \SetWatermarkScale{0.3} % TODO REMOVE
% \SetWatermarkHorCenter{1cm} % TODO REMOVE
\setcounter{tocdepth}{3} % SUBSECTIONS IN TOC
%
\author{
  Julianne Cai\\
} 
\title{}
\date{\today}

\usepackage[myheadings]{fullpage}
\usepackage{fancyhdr}
\usepackage{lastpage}
\usepackage{graphicx, wrapfig, subcaption, setspace, booktabs}
\usepackage[protrusion=true, expansion=true]{microtype}
\usepackage[english]{babel}
\usepackage{sectsty}
\usepackage{titling}
\usepackage{ytableau}

\theoremstyle{definition}
\newtheorem{definition}{Definition}
\theoremstyle{remark}
\newtheorem{remark}{Remark}
\theoremstyle{proposition}
\newtheorem{proposition}{Proposition}
\theoremstyle{conjecture}
\newtheorem{conjecture}{Conjecture}
\theoremstyle{lemma}
\newtheorem{lemma}{Lemma}
\theoremstyle{corollary}
\newtheorem{corollary}{Corollary}
\theoremstyle{exercise}
\newtheorem{exercise}{Exercise}
\theoremstyle{example}
\newtheorem{example}{Example}

\newcommand{\R}{\mathbb{R}}
\newcommand{\C}{\mathbb{C}}

\newcommand{\mcal}{\mathcal}

\newcommand{\Ell}{\mcal{E}\ell\ell}

\newcommand{\on}{\operatorname}

\newcommand{\spec}{\on{Spec}}
\newcommand{\proj}{\on{Proj}}
\newcommand{\bspec}{\on{\mathbf{Spec}}}
\newcommand{\sym}{\on{Sym}}

\newcommand{\D}{\on{D}}
\newcommand{\bounded}{\on{b}}

\newcommand{\qcoh}{\on{\mathbf{QCoh}}}
\newcommand{\coh}{\on{\mathbf{Coh}}}
\newcommand{\Mod}{\on{\mathbf{Mod}}}
\newcommand{\Vect}{\on{\mathbf{Vect}}}
\newcommand{\fin}{\on{fin}}

\newcommand{\fuch}{\on{fuch}}
\newcommand{\DiffEq}{\on{\mathbf{DiffEq}}}
\newcommand{\DiffMod}{\on{\mathbf{DiffMod}}}

\newcommand{\supp}{\on{supp}}

\newcommand{\Hom}{\on{Hom}}
\newcommand{\RHom}{\on{\textit{R}Hom}}
\newcommand{\Ext}{\on{Ext}}

\newcommand{\HRule}[1]{\rule{\linewidth}{#1}}

\title{ \normalsize \textsc{Live-TeXed Lecture Notes for My Dumbass}
		\\ [2.0cm]
		\HRule{0.5pt} \\
		\LARGE \textbf{MAST90017 - Representation Theory}
		\HRule{2pt} \\ [0.5cm]
        }

\date{}
\author{
		Julianne Cai \\\\
        }

\begin{document}


\maketitle
\tableofcontents

\chapter{Week One}

\textbf{Office Hours}: Peter Hall 203 (starting week 3) --- see Canvas\\
\textbf{Assessment}: $50$\% assignments, and $50\%$ final exam (3 hours)

\paragraph{This week, we learned about \ldots}definition of a 
representation, subrepresentation. Examples of a representation of 
$G$ are the regular representation (two definitions), and the permutation
representation. These representations can be obtained given any group $G$.\\\\
Given two representations, we can make more representations by taking
their direct sums, or taking their tensor product. The tensor product 
representation decomposes into two subrepresentations: the symmetric and 
alternating power representations.\\\\
A nice way of characterising isomorphic reprensentations is by studying 
their characters, since isomorphic representations have the same 
character. \\\\
More broadly, characters are examples of class functions, which are 
functions that take constant values on conjugacy classes. 

\section{Lecture 1, 24/07/2023}
This course will mostly deal with the representation theory of finite groups, 
and of finite-dimensional algebras. 

\subsection{Basic on Linear Representations of Finite Groups}


\begin{definition}
    Let $G$ be a finite group.
    A \emph{linear representation} of $G$ on a $\C$-vector space $V$ is 
    a group homomorphism 
    $$\rho: G \longrightarrow \on{Aut}_\C(V),$$
    where $\on{Aut}(V)$ is the group of invertible linear maps.
    If $\dim_\C V = n$, then we may choose a basis, and $$\on{Aut}(V) \cong \on{GL}_n(\C).$$
We say that $V$ is a \emph{representation space} of $G$ (or simply,
a \emph{representation of $G$}). 
\end{definition}

\begin{exercise}
    What happens if we replace $\C$ by some other field?
\end{exercise}

\begin{proof}
    It better be algebraically closed.
\end{proof}

From now on, we assume that $V$ is finite-dimensional. Serre explains that 
this is not a very severe condition. Often, it if possible to look at the 
finite points of $V$, and then study representations generated by those points. 
Further, since $G$ is finite, this is even less of a problem. \\\\
The dimension of $V$ --- denoted $\dim_\C V$ --- is called the \emph{degree} 
of the representation $V$.

\begin{definition}
    A homomorphism $\varphi$ between two $G$-representations $(\rho_1,V)$,
    and $(\rho_2,W)$ is a linear map $\varphi: V \to W$ such that 
    the diagram commutes for all $g \in G$:
    $$\begin{tikzcd}
        V \arrow[r, "\varphi"] \arrow[d, "\rho_1(g)"'] & W \arrow[d, "\rho_2(g)"] \\
        V \arrow[r, "\varphi"]                         & W                       
        \end{tikzcd}$$
    That is, $$\varphi(\rho_1(g)v) = \rho_2(g)\varphi(v).$$
    The two representations $(\rho_1,V)$ and $(\rho_2,W)$ are \emph{isomorphic}
    if there exists a homomorphism of $G$-representations which is a linear 
    isomorphism. We write $\on{Hom}_G(V,W)$ to denote the set of 
    homomomorphisms of $G$-representations ($G$-representations for short).
\end{definition}

The above information allows us to define a category of finite-dimensional 
$G$-representations, denoted by $\mathbf{Rep}_G(\C)$. It is an abelian category. 

\begin{exercise}
    Think about what other ways you can define a morphism of representations.
\end{exercise}

\begin{proof}
    You can define a representation of $G$ as a $G$-module, and then define
    morphisms as morphisms between $G$-modules. There is a bijection between 
    $G$-representations and $G$-modules, anyway, since the map 
    $\rho$ can just be defined as 
    $$G \longrightarrow \on{Aut}(V), \quad g\longmapsto (v\longmapsto gv).$$
\end{proof}

\begin{example}[Examples of Representations of $G$]
%    Let $G$ be any finite group. We give some examples of simple represntations.
    \leavevmode
    \begin{itemize}
        \item[(a)] \textbf{One-Dimensional Representations} Any degree one representation of $G$, given by 
            $$\rho : G \longrightarrow \on{Aut}_\C(\C) \cong \C^\times = \on{GL}_1(\C).$$
\paragraph{Question:} How do we find all degree one representations?
\paragraph{Answer:}
Since $G$ is finite, for all $g \in G$, there exists some $m \geq 1$ for which 
$g^m = 1$, which implies that $\rho(g)$ is an $m$-th root of unity. This does 
not give us the whole answer, though. Think about it!
\paragraph{Question:} What are degree $1$ representations of 
$\mathbb{Z}/n\mathbb{Z}$?
\paragraph{Answer:} Since $\mathbb{Z}/n\mathbb{Z}$ is a cyclic group,
any generator gets mapped to an $n$-th root of unity under $\rho$. There 
are $n$ such elements that we can map to an $n$-root of unity.
\item[(b)] \textbf{Regular Representation of $G$} Let 
    $$V = \bigoplus_{g \in G} \C e_g,$$
        be a vector space of dimension $\dim_\C V =\vert G \vert$.
        Define maps:
        $$\rho_h : V \longrightarrow V, \quad e_g \longmapsto e_{hg},$$
        and $$R : G \longrightarrow \on{GL}(V),\quad h\longmapsto \rho_h.$$
        This representation $R$ is called the \emph{regular representation}
        of $G$. 
    \item[(c)] \textbf{Another Definition of Regular Representation} 
        Let $W$ be the vector space of complex-valued functions on $G$.
        That is, the space of functions $f : G \to \C$. 
        %This space has a basis given by indicator functions. 
        Define a linear map 
        $$\tau : G \longmapsto \on{GL}(W), \quad g \longmapsto (f \longmapsto gf),$$
        where $f: G \to \C$, and $gf: G \to \C$. In particular, the action 
        $gf$ is given by 
        $$g\cdot f(h) = f(g^{-1}h),$$
        for $g,h \in G$. 
        The factor of $g^{-1}$ is added so that the map
        $f \mapsto g\cdot f$ is an isomorphism in $W$. 
        \begin{exercise}
            \leavevmode 
            \begin{itemize}
                \item[(a)] Check that this is a representation of $G$ (pay attention to the factor of $g^{-1}$ instead of $g$).
                \item[(b)] Show that $V$ and $W$ are isomorphic representations 
                    of $G$.
            \end{itemize}
        \end{exercise}
        \begin{proof}
            \leavevmode 
            \begin{itemize}
                \item[(a)] Let $g,h,k\in G$, and $f\in W$. Then,
                    $$\tau(gh)(f)(k) = f(h^{-1}g^{-1}k) = \tau(h)(f)(g^{-1}k) = \tau(h) (\tau(g)(f))(k) = (\tau(h) \circ \tau(g))(k)),$$
                    and it follows that $\tau$ is a group homomorphism. 
%                    Indeed, $\tau(1)(f)(k) : f(k) \mapsto f(k)$ for each 
%                    $k \in G$, and so
%                    $\tau(1) = \on{id}_W$. 
%                    For any $g\in G$, the map $f \mapsto g^{-1}\cdot f$ is an
%                    inverse to the map $f \mapsto g\cdot f$ in $\on{Aut}(W)$,
%                    and since $\tau(g^{-1})(f)(k) = g^{-1}\cdot f = f^{-1}$,
%                    we have that $\tau(g^{-1}) = \tau(g)^{-1}$. 
                \item[(b)] Define a map $1_g : G \to \C$ by: 
                    $$1_g(x) = \begin{cases}
                        1, \quad \text{if}\quad x=g,\\
                        0,\quad \text{otherwise}
                    \end{cases}.$$
                    We see from this that 
                    $W = \on{Span}_\C \lbrace 1_g:g\in G\rbrace$.
                    For any $f = \sum_{g\in G}a_g 1_g \in W$, 
                    we observe that if $a_g \neq 0$ for all $g \in G$,
                    then $f(G) \neq 0$. 
                    Thus, $f$ defines the zero function if and only 
                    if $a_g = 0$ for all $g \in G$, and it follows that the 
                    $1_g$'s are linearly independent. It follows then that 
                    $\lbrace 1_g: g\in G\rbrace$ form a basis for $W$,
                    and $\dim_\C W = \vert G \vert$.\\\\
                    Define a map $$\varphi : V \longrightarrow W,\quad e_g \longmapsto 1_g.$$
                    It is sufficient to show that $\varphi$ gives a morphism
                    of $G$-representations, as bijectivity is clear.
                    We have a diagram:
                    $$
                    \begin{tikzcd}
                    V \arrow[r, "\varphi"] \arrow[d, "R(h)"'] & W \arrow[d, "\tau(h)"] \\
                    V \arrow[r, "\varphi"']                   & W
                    \end{tikzcd}$$
                    The composition gives: $$\varphi(R(h)(e_g))=\varphi(\rho_h(e_g) = \varphi(e_{hg}) = 1_{hg},$$
                    and the other composition gives:
                    $$\tau(h)(\varphi(e_g))(x) = \tau(h) 1_g(x) = h\cdot 1_g(x) = 1_g(h^{-1}x) = 1_{hg}(x),$$
                    and thus the diagram commutes and we have 
                    $$\varphi \circ R(h) = \tau(h) \circ \varphi.$$
                    Thus, $\varphi$ gives a morphism
                    of $G$-representations, and is in fact an isomorphism of 
                    $G$-representations since it maps basis vectors to basis
                    vectors.
            \end{itemize}
        \end{proof}
    \item[(c)] \textbf{Permutation Representation} 
        Suppose that $G$ acts on a finite set $X$. Let 
        $$V = \bigoplus_{x \in X} \C e_x,$$
        be a finite-dimensional vector space with $\dim_\C V = \vert X \vert$.
        Then, there is a $G$-representation given by:
        $$\rho : G \longrightarrow \on{GL}(V), \quad g\longmapsto (e_x \longmapsto e_{gx}).$$
        This representation is called the \emph{permutation representation}
        associated with $X$.
\end{itemize}
\end{example}

\section{Lecture 2, 27/07/2023}

\paragraph{Question:} Can you define a degree one representation $\rho$ of $G$ such that 
$\rho(G)= \lbrace \pm 1\rbrace$?
\paragraph{Answer:} No! For instance, we cannot do this for 
$\mathbb{Z}/3\mathbb{Z}$.

\begin{definition}[Subrepresentations]
    Let $\rho : G \to \on{GL}(V)$ be a representation of $G$.
    A subspace $W$ of $V$ is \emph{$G$-stable} if for any $w\in W$, 
    and any $g\in G$, $\rho(g)(w) \in W$.
    A $G$-stable subspace $W$ of $V$ called a \emph{subrepresentation}
    if there is a map $\rho_W :G \to \on{GL}(W)$ given by 
    $g\mapsto \rho(g)\vert_W$.
\end{definition}

\begin{example}
    Consider the regular representation $$V = \bigoplus_{g \in G} \C e_g.$$
    Then, the $\C$-span of the element $\sum_{g\in G}e_g$ is a $G$-stable 
    subspace, and is in fact a degree $1$ trivial subrepresentation.
    Write this subrepresentation as 
    $$W = \C \sum_{g\in G}e_g.$$
\end{example}

\paragraph{Question:} Given a representation of $G$, what are all of 
its subrepresentations?\\\\
Before we answer this, we will recall some linear algebra. 
\subsection{Linear Algebra Recap}

Let $V$ be a $\C$-vector space, and $W\subset V$ a subspace. Then, we 
have the following bijection:
$$\left\lbrace \text{Projections of $V$ onto $W$} \right\rbrace \longleftrightarrow \left\lbrace \text{Complements of $W$ in $V$}\right\rbrace.$$
Recall that a \emph{projection} of $V$ onto $W$ is a linear map 
$p: V \to W$ such that $p\vert_W = \on{id}_W$.
The bijection is given explicitly by the assignments $$p\longmapsto \ker p,$$
$$V = W \oplus W' \longmapsto p(w+w')=w.$$
(TODO: make this look neater).
The first map makes sense since $V = \ker p \oplus W$. 

\begin{exercise}
    Check that this is a bijection.
\end{exercise}

\begin{proof}
    Given a projection $p: V \to W$, there is a short 
    exact sequence 
    $$0 \longrightarrow \ker p \longrightarrow V \stackrel{p}{\longrightarrow} W \longrightarrow 0.$$
    Since, by definition $p\vert_W = \on{id}_W$, it follows that the 
    exact sequence is split, and thus $V \cong W \oplus \ker p$, and it
    follows that the map $p \longmapsto \ker p$ is a map from projections 
    of $V$ onto $W$ to complements of $W$ in $V$.\\\\
    Conversely, given a decomposition $V = W\oplus W'$, there is a map
    $p : V \to W$ defined by $p(w+w') = w$, for $w\in W$, $w' \in W'$.
    This is clearly a projection map. 
\end{proof}

\subsection{Back to Representation Theory}

\begin{theorem}
    Let $\rho : G \to \on{GL}(V)$ be a $G$-representation, and $W$ 
    a $G$-subrepresentation. There exists a complementary subrepresentation
    $W'$ such that $V = W\oplus W'$. 
\end{theorem}

\begin{proof}
    Let $U$ be an arbitrary complementary subspace of $W$ --- that is, 
    such that $V = W \oplus U$. Let $p_0: V=W\oplus U \to W$ be the projection 
    onto $U$. Howevef, this map is not necessarily a morphism of 
    $G$-representations. Thus, we define an \emph{averaging map}: 
    $$p := \frac{1}{\vert G\vert} \sum_{g\in G} \rho(g) \circ p_0\circ \rho(g)^{-1},$$
    which defines a linear map $V \to W$. We now verify that $p$ is a projection.
    Given some $w\in W$, then 
    $$p(w) = \frac{1}{\vert G\vert} \sum_{g \in G} \rho(g) \circ \rho(g)^{-1}(w) = w,$$
    and it follows that $p$ is indeed a projection of $V$ 
    onto $W$. Using the aforementioned bijection, we obtain a complement of $W$ 
    given by $W' := \ker p$. 
    It remains to show that $W'$ is $G$-stable. Let $w'\in W'$. Then, 
    $p(v) = 0$ by definition. Now, given any $h\in G$, 
    \begin{align*}
        p(\rho(h)(v)) &= \frac{1}{\vert G\vert} \sum_{g\in G} \rho(g) \circ p_0 \circ\rho(g)^{-1}(\rho(h)(v))\\ 
                      &= \frac{1}{\vert G\vert} \sum_{g\in G} \rho(h) \circ (\rho(h^{-1}g) \circ p_0 \circ \rho(g^{-1}h))(v)\\ 
                      &= \frac{1}{\vert G \vert} \rho(h) \sum_{g \in G} \rho_{h^{-1}g} \circ p_0 \circ \rho_{g^{-1}h}(v)\\ 
                      &= 0,
\end{align*}
and it follows that $W' = \ker p$ is $G$-stable.
\end{proof}


\begin{proof}[Another Proof]
    There exists a sesquilinear form $\langle -,-\rangle$ on $V$ 
    given by $$\langle -,-\rangle : V\times V \longrightarrow \C,$$
    such that $$\langle av,w\rangle = a\langle v,w\rangle, \quad \langle v,aw\rangle = \bar{a} \langle v,w\rangle, \quad \langle v,v\rangle > 0,\quad \langle v+v',w\rangle = \langle v,w\rangle + \langle v',w\rangle,$$
    for all $v\neq 0,w \in V$, and $a\in \C$. 
    Make it $G$-invariant by using the averaging trick from the previous 
    proof. Let $$\langle v,w\rangle_G := \sum_{g\in G} \langle gv, gw\rangle.$$
    Let $$W^c := \lbrace v\in V: \text{$\langle v,w\rangle_G = 0$ for all $w \in W$}\rbrace.$$
    Then, it follows that 
    $$V = W \oplus W^c.$$
    \begin{exercise}\label{exercise5}
        Show that $W^c$ is $G$-stable.
    \end{exercise}
    \begin{proof}[Proof of Exercise \ref{exercise5}]
        Let $v \in W^c$. Then, for some $x \in G$, and any $w\in W$:
        $$\langle \rho(x)v,w\rangle_G = \sum_{g \in G} \langle \rho(x) \circ \rho(g)v, \rho(g)w\rangle,$$
        and so $\rho(x)v \in W^c$, and thus $W^c$ is $G$-stable. (TODO: 
        finish this proof)
    \end{proof}
\end{proof}

\begin{remark}
    What the above shows is that this decomposition is in general not unique.
\end{remark}

\begin{definition}
    A linear representation $\rho : G \to \on{GL}(V)$ is irreducible (or simple)
    if $V \neq \lbrace 0\rbrace$, and there exist no non-trivial 
    subrepresentations --- that is, the only subrepresentations are 
    $\lbrace 0 \rbrace$ and $V$.
\end{definition}

\begin{corollary}[Complete Reducibility/Semisimplicity]
    Every $\C$-linear representation of $G$ is a direct sum of irreducible
    representations. 
\end{corollary}

\begin{proof}
    We proceed by inducting on the dimension of our representation.
    Given a $G$-representation $V$ with $\dim_\C V = n$, if $V$ 
    is irreducible then we are done. If not, then there exists a 
    subrepresentation $W$, and the Theorem tells us that $V = W \oplus W'$,
    for some other subrepresentation $W'$. We continue this process,
    finding smaler and smaller subrepresentations of $W$ and $W'$, until
    we the subrepresentation is irreducible. Eventually, we end up with a 
    direct sum of irreducible representations.
\end{proof}

\begin{remark}[Caution!]
    This may break down if we consider $k$-linear representations, when 
    $\on{char}k \neq 0$. 
\end{remark}
\begin{exercise}
    Can you think of an example?
\end{exercise}

\begin{example}
    Consider the additive group of real numbers $(\mathbb{R},+)$, 
    and a group homomorphism
    $$\mathbb{R} \longrightarrow \on{GL}_2(\mathbb{R}), \quad a\longmapsto \begin{pmatrix}
        1&a\\
        0&1
    \end{pmatrix}.$$
    There is a subrepresentation given by 
    $\mathbb{R}e_1 \subset \mathbb{R}^2$.
    \begin{exercise}
        Find the complement of this representation.
    \end{exercise}
    \begin{proof}
        Here, we have $V = \mathbb{R}$. For $v,w \in \mathbb{R}$,
        $\langle v,w\rangle = vw$, given by mutiplication in $\mathbb{R}$.
    \end{proof}
\end{example}

\paragraph{Question:} Is this decomposition unique? (We will discuss more 
about this later).

\section{Lecture 3, 28/07/2023}

Recall from last time that we wanted to try and find ways of producing 
representations. 
\begin{definition}
    Let $(V,\rho)$ and $(V',\rho')$ be two representations of $G$. 
    Then, 
    \begin{itemize}
        \item[(a)] $V\oplus V'$ is also a representation of $G$ given by 
            $$\tau : G \longrightarrow \on{GL}(V\oplus V'), \quad g \longmapsto 
            \tau_g(v+v') = \rho(g)v + \rho'(g)v'.$$
        \item[(b)] $V\otimes V'$ is also a representation of $G$ given by:
            $$\gamma : G \longrightarrow \on{GL}(V\otimes V'), \quad 
            g\longmapsto \gamma_g(v\otimes v') = \rho_g(v) \otimes \rho'_g(v').$$
        \item[(c)] One can take a \emph{dual representation}
            $V^\ast := \on{Hom}_\C(V,\C)$, given by:
            $$\pi : G \longrightarrow \on{GL}(V^\ast), \quad g \longmapsto\pi_g,$$
            where $\pi_g : V^\ast \to V^\ast$ is given by $f\mapsto g\cdot f,$
            where $g$ acts on $f$ by $g\cdot f(v) = f(g^{-1}v)$.
        \item[(d)] The space $\on{Hom}_\C(V,V')$ defines a representation
            given by: 
            $$\pi : G \longrightarrow \on{GL}(\on{Hom}_\C(V,V')), \quad 
            g\longmapsto \pi_g,$$
            where $$\pi_g : \on{Hom}_\C(V,V') \longrightarrow \on{Hom}_\C(V,V'),\quad f\longmapsto g\cdot f,$$
            where the action is given by $$g\cdot f(v) = \rho'_gf(\rho_{g^{-1}}v).$$
    \end{itemize}
\end{definition}

\begin{remark}
    Note here that (c) is a special case of (d), given by replacing 
    $V'$ by the trivial representation.
\end{remark}
\begin{exercise}
    Check that all of these defines a representation!
\end{exercise}

\begin{proof}
    \leavevmode
    \begin{itemize}
        \item[(a)] Using the fact that $\rho$ and $\rho'$ are 
            group homomorphisms for $v\in V$, $v'\in V'$, and $g,h\in G$:
            $$\tau_{gh}(v+v') = \rho_{gh}(v) + \rho'_{gh}(v') = (\rho_g\circ \rho_h)(v) + (\rho'_g \circ \rho'_h)(v') = (\tau_g\circ \tau_h)(v+v').$$
        \item[(b)] Once again, let $v\in V$, $v'\in V'$, and $g,h\in G$:
            $$\gamma_{gh}(v\otimes v') = \rho_{gh}(v)\otimes \rho_{gh}(v') = (\rho_g \circ \rho_h)(v) \otimes (\rho'_g\circ \rho'_h)(v') = (\gamma_g \circ \gamma_h)(v\otimes v').$$
        \item[(c)] Computing directly,
            $$\tau_{gh}(f)(v) = (gh)\circ f(v) = f(h^{-1}g^{-1}v) = \pi_h(g\dot f)(v) = \pi_h(\pi_g(f))(v) = (\pi_h\circ \pi_g)(f)(v).$$
        \item[(d)] (TODO: check your previous calculation before attempting 
            this one. I think you did something wrong)
    \end{itemize}
\end{proof}

\begin{exercise}
    Show that $$\on{Hom}_G(V,V') \cong \on{Hom}_\C(V,V')^G,$$
    where the left-hand side denotes the $G$-fixed points of $\on{Hom}_\C(V,V')$.
\end{exercise}

\begin{proof}
    Define a map 
    $$\Phi : \on{Hom}_G(V,V') \longrightarrow \on{Hom}_\C(V,V')^G,\quad \varphi \longmapsto g\cdot \varphi,$$
    where $g\cdot \varphi(v) = \rho'_g\varphi(\rho_{g^{-1}}v)$. 
    We show that $g\cdot \varphi$ is a $G$-invariant linear map.
    Since $\varphi$ is a morphism of $G$-representations,
    $\rho_g'\varphi(v) = \varphi(\rho_g(v))$, and so
    $$g\cdot \varphi(v) = \rho'_g\varphi(\rho_{g^{-1}}(v)) = \varphi(\rho_g\circ \rho_{g^{-1}}(v)) = \varphi(v),$$
    and it follows then that $g\cdot \varphi \in \on{Hom}_\C(V,V')^G$.
    It is clear that $\varphi$ is a linear map. The map 
    $\varphi \mapsto g^{-1}\cdot \varphi$ is a suitable choice of 
    inverse for $\Phi$, and thus $\Phi$ is an isomorphism of vector spaces.
\end{proof}

\subsection{Symmetic Square/Alternating Square}

Let $(\rho,V)$ be a $G$-representation, and let $\lbrace e_i\rbrace_{i=1}^n$
be a basis for $V$. Then, $\lbrace e_i\otimes e_j\rbrace_{i,j=1}^n$ is a
basis for $V\otimes V$. Then, one can decompose the tensor product 
into two subrepresentations in the following way: 
$$V\otimes V \cong \on{Sym}^2(V) \oplus \Lambda^2(V).$$
Define an \emph{involution}:
$$\theta : V\otimes V \longrightarrow V\otimes V, \quad e_i\otimes e_j \longmapsto e_j \otimes e_i.$$
Extending by linearity, $\theta(v\otimes w) = w\otimes v$. 
Let $$\on{Sym^2}(V) := \lbrace x \in V\otimes V : \theta(x) = x\rbrace,$$
$$\Lambda^2(V) := \lbrace x \in V\otimes V : \theta(x) = -x\rbrace.$$
A basis for $\on{Sym}^2(V)$ (resp. $\Lambda^2(V)$) is then given by $\lbrace e_i\otimes e_j + e_j \otimes e_i\rbrace$  (resp. $\lbrace e_i\otimes e_j - e_j\otimes e_i\rbrace$). It has dimensions 
$$\dim_\C \on{Sym}^2(V) = \frac{n^2+n}{2},\quad \dim_\C\Lambda^2(v) = \frac{n^2-n}{2}.$$
\begin{exercise}
    Show that $\on{Sym}^2(V)$ and $\Lambda^2(V)$ are subrepresentations of 
    $V\otimes V$. 
\end{exercise}

\begin{proof}
    Indeed, $\on{Sym}^2(V)$ is $G$-stable since for $v,w \in V$, and 
    $vw \in \on{Sym}^2(V)$, we have 
    $$\rho(g)(vw) = (gv)(gw),$$
    and since each $gv \in V$, and $gw\in V$, it follows that 
    $(gv)(gw) \in\on{Sym}^2(V)$. The map 
    $\rho(g)\vert_{\on{Sym^2}(V)}$ which sends $vw$ to $(gv)(gw)$ has an 
    inverse given by $vw \mapsto (g^{-1}v)(g^{-1}w)$, and thus 
    $\rho(g)\vert_{\on{Sym}^2(V)} \in \on{GL}(\on{Sym}^2(V))$, and thus 
    there is a map 
    $$\rho_{\on{Sym}^2(V)} : G \longrightarrow \on{GL}(\on{Sym}^2(V)),\quad 
    g\longmapsto (vw \longmapsto (gv)(gw)),$$
    and thus $\on{Sym}^2(V)$ is a subrepresentation of $V\otimes V$.\\\\
    Similarly, $G$-stability of $\Lambda^2(V)$ follows from the fact that 
    $$\rho(g)(v\wedge w) = gv \wedge gw \in \Lambda^2(V).$$
    The restriction $\rho(g)\vert_{\Lambda^2(V)}$ defines a map 
    $v\wedge w \mapsto gv\wedge gw$, with inverse given by 
    $v\wedge w\mapsto g^{-1}v\wedge g^{-1}w$, and so 
    there is a map $$\rho_{\Lambda^2(V)} : G \longrightarrow \on{GL}(\Lambda^2(V)), \quad g\longmapsto (v\wedge w \longmapsto gv\wedge gw),$$
    and so $\Lambda^2(V)$ is a subrepresentation of $V\otimes V$.
\end{proof}

Inductively, we can construct exterior powers $\Lambda^n(V)$ and 
symmetric powers $\on{Sym}^n(V)$ (see Fulton and Harris Appendix B.1, B.2).

\subsection{Characters}

\begin{definition}
    Let $(\rho,V)$ be a representation of $G$. Then, a \emph{character} of 
    $\rho$ is given by the $\C$-valued function 
    $$\chi_\rho : G \longrightarrow \C,\quad g\longmapsto \on{tr}(\rho_g).$$
\end{definition}

Recall from linear algebra that the trace is the sum of the eigenvalues 
of a matrix (with multiplicities). We will show that representations 
are isomorphic if and only if their characters are the same,
and that characters play well with direct sums and tensor products.

\begin{exercise}
    Show that isomorphic representations have the same character. 
\end{exercise}

\begin{proof}
    Let $(\rho,V)$ and $(\tau,W)$ be two isomorphic $G$-representations ---
    that is, there is a $\C$-linear isomorphism $\varphi : V\to W$ such that the 
    diagram
    $$\begin{tikzcd}
V \arrow[r, "\varphi"] \arrow[d, "\rho_g"] & W                     \\
V \arrow[r, "\varphi"]                     & W \arrow[u, "\tau_g"]
\end{tikzcd}$$
    commutes. Then,
    $$\chi_\rho(g) = \on{tr}(\rho_g) = \on{tr}(\varphi^{-1} \circ \tau_g\circ \varphi) = \on{tr}(\tau_g) = \chi_\tau,$$
    where the second equality follows from the commutativity of the diagram, 
    and the third equality follows by the invariance of trace.
\end{proof}

\subsubsection{Basic Properties of Characters}

\begin{lemma}
    Let $(\rho,V)$ be a representation of $G$, and $\chi_\rho$ its character.
    \begin{itemize}
        \item[(a)] $\chi_\rho(1) = \dim_\C V.$
        \item[(b)] $\chi_\rho(g^{-1}) = \overline{\chi_\rho(g)}$.
        \item[(c)] $\chi_\rho(hgh^{-1}) = \chi_\rho(g)$, for all $h,g\in G$.
        \item[(d)] $\chi_{\rho\oplus \rho'} = \chi_\rho +\chi_{\rho'}$.
    \end{itemize}
\end{lemma}

\begin{proof}
    \leavevmode
    \begin{itemize}
        \item[(a)] Immediate.
        \item[(b)] Suppose $\rho_g$ has eigenvalues $\lambda_1,\cdots,\lambda_n$, where $n = \dim_\C V$. Then, $\rho_{g^{-1}}$ has eigenvalues 
            $\lambda_1^{-1},\cdots,\lambda_n^{-1}$, since 
            $\rho_g$ is a root of unity. Since $\lambda_i \in \C$,
            we have that $\lambda_i^{-1} = \overline{\lambda_i}$,
            and so $$\chi_\rho(g^{-1}) = \lambda_1^{-1} + \cdots + \lambda_n^{-1} = \overline{\lambda_1} + \cdots + \overline{\lambda_n} = \overline{\chi_\rho(g)}.$$
            \begin{exercise}
                Show that $\rho_g$ is diagonalisable.
            \end{exercise}
        \item[(c)] Immediate.
        \item[(d)] Immediate.
    \end{itemize}
\end{proof}

\begin{definition}
    A function $f : G \to \C$ is a \emph{class function} if 
    $$f(ghg^{-1}) = f(h),$$
    for all $h,g\in G$. That is, $f$ takes constant values on conjugacy classes
    of $G$.
\end{definition}

It follows then that the space of class functions has dimension equal to the 
number of conjugacy classes.

\begin{remark}
    Characters of representations of $G$ are an example of class functions.
\end{remark}

%\bibliography{bibliography}
%\bibliographystyle{alpha}


\chapter{Week Two}

\paragraph{This week, we learned about \ldots} various properties of characters. 
In particular, properties of characters with respect to direct sum representations, and 
tensor representations. In particular, characters characterise (no pun intended) irreducible 
representations of $G$ up to isomorphism.\\\\
An important result that we use all the time is Schur's lemma, which says that
any morphism of two irreducible representations of $G$ are either the zero map,
or isomorphic. If the two representations are isomorphic, then
$\dim_\C\on{Hom}_G(V_1,V_2) = 1$.\\\\
We also learned that irreducible characters have norm $1$, and 
are orthogonal to one another. In fact, they form an orthonormal basis for the space 
of class functions of $G$. A corollary of this fact is that --- up to isomorphism --- the number 
of irreducible representations of $G$ is equal to the number of conjugacy classes.

\section{Lecture 1, 31/07/2023}

\subsection{More Properties of Characters}

Let $(V,\rho)$ and $(V',\rho')$ be two representations of $G$. We use the 
notations $\chi_V$ and $\chi_\rho$ interchangeably. 
Recall then that $\chi_{\rho\oplus \rho'} = \chi_\rho + \chi_{\rho'}$.
Additionally, we have the following property:
$$\chi_{V\otimes V'} = \chi_V \chi_{V'},$$
and also we have that: 
$$\chi_{V^\ast}(g) = \chi_V(g^{-1}) = \overline{\chi_V(g)}.$$
\begin{exercise}
    \leavevmode
    \begin{itemize}
        \item[(a)] Check these properties! Also, think about when 
            $V \cong V^\ast$ as $G$-representations.
        \item[(b)] Show that 
            $$\on{Hom}_\C(V,V') \cong V^\ast \otimes V',$$
            as $G$-representations.
            It follows then that $$\chi_{\on{Hom}_\C(V,V')}(g) = \overline{\chi_V(g)} \cdot \chi_{V'}(g).$$
        \item[(c)] Express $\chi_{\on{Sym}^2(V)}$ and $\chi_{\Lambda^2(V)}$ 
            in terms of $\chi_V$. (Hint: observe first that 
            $\chi_{\on{Sym}^2(V)} + \chi_{\Lambda^2(V)} = (\chi_V)^2$).
    \end{itemize}
\end{exercise}

\begin{proof}
    \leavevmode 
    \begin{itemize}
        \item[(a)] Let $\lbrace v_i\rbrace_{i=1}^k$, and 
            $\lbrace v_j \rbrace_{j=1}^\ell$ be an eigenbasis for 
            $V$ and $V'$, respectively. Then, the eigenbasis 
            corresponding to $V\oplus V'$ and $V\otimes V'$ would be 
            $\lbrace v_i + v_j\rbrace$ and $\lbrace v_i\otimes v_j\rbrace$. 
            The dimension of the eigenbases are $k+\ell$ and $k\ell$, 
            respectively. It follows then that 
            $\chi_{\rho\oplus \rho'} = \chi_\rho + \chi_{\rho'}$, and 
            $\chi_{\rho\otimes\rho'} = \chi_\rho \chi_{\rho'}$.
        \item[(b)] Define a map 
            $$\on{Hom}_\C(V,V') \longrightarrow \on{Hom}_\C(V,\C) \otimes_\C V',\quad f \longmapsto \varphi \otimes f.$$
        \item[(c)] 
    \end{itemize}
\end{proof}

\subsection{Schur's Lemma}

\begin{proposition}[Schur's Lemma]\label{prop_schurs_lemma}
    Let $(V_1,\rho_1)$ and $(V_2,\rho_2)$ be two irreducible representations of 
    $G$. Let $f \in \on{Hom}_G(V_1,V_2)$. Then, 
    \begin{itemize}
        \item[(a)] $f =0$ or $f$ is an isomorphism of representations.
        \item[(b)] if $V_1=V_2$, and $\rho_1=\rho_2$, then 
            $f =\lambda \on{id}_{V_1}$, for some $\lambda \in \C$.
            That is, $\dim_\C \on{Hom}_G(V_1,V_2) = 1$.
    \end{itemize}
\end{proposition}
\begin{proof}
    \leavevmode
    \begin{itemize}
        \item[(a)] Assume $f\neq 0$. Then, $\on{Im}(f) \neq 0$, and it 
            follows then that $\on{Im}(f)$ is a subrepresentation of $V_2$.
            Since $V_2$ is irreducible, $\on{Im}(f) = V_2$ necessarily, and 
            it follows that $f$ is surjective. Similarly, $\ker f$ is a 
            subrepresentation of $V_1$, and by irreducibility, and the fact
             that $f\neq 0$, we conclude that $\ker f = 0$. This shows that 
             $f$ is an isomorphism. 
         \item[(b)] Suppose that $V=V_1=V_2$, and $\rho=\rho_1=\rho_2$. 
             Let $f : V \to V$ be a homomorphism of $G$-representations. 
             Then, $f$ has an eigenvalue $\lambda \in \C$ (this is guaranteed 
             since $\C$ is algebraically closed). Consider the morphism of 
             irreducible $G$-representations:
             $$f' := (f-\lambda \on{id}_V):  V\to V,$$
             and we observe that there exists an eigenvector $0\neq v\in V$
             such that $f(v) = \lambda v$. Then, $v \in \ker (f')$,
             and it follows then that $\ker f' \neq 0$, and thus $f'=0$
             by part (a). It follows then that $f = \lambda \on{id}_V$,
             as claimed. 
    \end{itemize}
\end{proof}

\subsection{Orthogonality Property of Characters}

\begin{definition}
    Let $\phi,\psi : G \to \C$ be $\C$-valued functions on $G$. Define 
    an inner product given by:
    $$(\phi\vert\psi) := \frac{1}{\vert G \vert} \sum_{g\in G} \phi(g)\overline{\psi(g)}.$$
    This product is linear in $\phi$, and sesqui-linear in $\psi$, and 
    $$(\phi\vert\phi) > 0, \quad \phi\neq 0$$
\end{definition}

\begin{remark}
    We may also define a bilinear form:
    $$\langle \phi,\psi\rangle :=\frac{1}{\vert G \vert} \sum_{g\in G} \phi(g)\psi(g^{-1}).$$
    Now, this form is linear in both $\phi$ and $\psi$. Indeed, 
    $$\langle \phi,\psi\rangle = \langle \psi,\phi\rangle,$$ which one may show 
    by making the variable change $g\mapsto g^{-1}$, making this 
    a symmetric bilinear form. \\\\
    In the case that $\phi$ and $\psi$ are characters of $G$,
    $$(\phi\vert\psi) = \langle\phi,\psi\rangle.$$
\end{remark}

\begin{lemma}\label{lemma_fixed_pt_dim}
    Let $(\rho,V)$ be a $G$-representation, and $V^G$ be the subrepresentation
    of $G$-fixed points of $V$. Then, 
    $$\frac{1}{\vert G \vert} \sum_{g\in G}\chi_\rho(g) = \dim_\C V^G.$$
\end{lemma}

\begin{proof}
    Consider a map 
    $$\varphi : V \longrightarrow V^G,\quad v \longmapsto \frac{1}{\vert G\vert}\sum_{g\in G} \rho(g)v.$$
    The result follows by showing that $\on{tr}(\varphi) = \dim_\C V^G$. 
    We wish to show that $\varphi$ is a projection onto $V^G$. 
    First, we show projectivity. Let $h \in G$.
    Then, $$\rho_h(\varphi(v)) = \frac{1}{\vert G \vert}\sum_{g\in G}\rho(hg)(v) = \varphi(v),$$
    for all $v\in V$. Let $v \in V^G$. Then,
    $$\varphi(v) = \frac{1}{\vert G \vert} \sum_{g\in G} \rho(g)(v) = \frac{1}{\vert G \vert} \sum_{g \in G} v = v,$$
    and it follows then that $\varphi$ is a projection. It follows then that 
    $\on{tr}(\varphi) = \dim_\C V^G$ --- that is, 
    $$\frac{1}{\vert G \vert} \sum_{g \in G} \on{tr}(\rho(g)) = \frac{1}{\vert G\vert}\sum_{g\in G} \chi_\rho(g) = \dim V^G.$$
\end{proof}

\begin{remark}
    See Serre's book for a more elementary, linear algebraic proof of this
    result.
\end{remark}

\begin{theorem}[Orthogonality of Characters]
    \leavevmode
    \begin{itemize}
        \item[(a)] If $\chi$ is the character of an irreducible 
            $G$-representation, then $\langle\chi,\chi\rangle = 1$. 
        \item[(b)] If $\chi = \chi_\rho$, and $\chi' = \chi_{\rho'}$ 
            for two non-isomorphic irreducible $G$-representations 
            $\rho$ and $\rho'$, then $$\langle \chi,\chi'\rangle = 0.$$
    \end{itemize}
\end{theorem}

\begin{proof}
    \leavevmode
    \begin{itemize}
        \item[(a), (b)] Let $(\rho,V)$ and $(\rho',V')$ be two irreducible,
            $G$-representations. Recall that 
            $\chi_{\on{Hom}_\C(V,V')} = \overline{\chi_V(g)}\chi_{V'}(g)$. 
            Then,
            $$\langle \chi_V,\chi_{V'}\rangle = \frac{1}{\vert G\vert} \sum_{g\in G} \chi_{\on{Hom}_\C(V,V')}(g).$$
            By Schur's lemma, 
            $$\on{Hom}_G(V,V') = \begin{cases}
                0, \quad V\not\cong V',\\
                \C, \quad V \cong V'
            \end{cases}.$$
            Recall that $$\on{Hom}_G(V,V') \cong \on{Hom}_\C(V,V')^G.$$
            The result follows by applying Lemma \ref{lemma_fixed_pt_dim} to
            $\on{Hom}_\C(V,V')$.
    \end{itemize}
\end{proof}

\section{Lecture 2, 03/08/2023}

\subsection{Orthogonality of Irreducible Characters: Another Proof}

Recall from last time:
\begin{theorem}
    If $\chi$ and $\chi'$ are both irreducible representations of $G$
    given by $\rho$ and $\rho'$, then $$\langle\chi,\chi\rangle = 1, \quad 
    \langle \chi,\chi'\rangle = 0.$$
\end{theorem}

Our last proof used the fact that $\on{Hom}_\C(V,V')^G \cong \on{Hom}_G(V,V')$.
We do something different this time.
\begin{proof}
    Let $(\rho,V)$ and $(\rho',V')$ be irreducible representations.
    Consider a function $f : V \to V'$ be a $\C$-linear map, and 
    define a $G$-invariant function by:
    $$f' := \frac{1}{\vert G \vert} \sum_{g \in G} \rho_g'\circ f \circ \rho_{g^{-1}}.$$
    It follows by construction that $f' \in \on{Hom}_G(V,V')$. Then, by 
    Schur's lemma, $f' = 0$ if $V\not\cong V'$, and $f' = \lambda \on{id}_V$
    if $V \cong V'$. In particular, the constant $\lambda$ is given by:
    $$\on{tr}(f') = \lambda \dim V = \on{tr}(f).$$
    So, $f' = \on{tr}(f) \cdot\on{id}_V$ if $V\cong V'$. 
    Assume that $\dim V = n$, and $\dim V' = m$, and write everything in matrix
    form. It follows then that $$f ' = (f_{ij}')_{m\times n}, \quad f = (f_{ij})_{m\times n},\quad \rho_g' = (\rho_{ij}'(g))_{m\times m}, \quad \rho_g = (\rho_{ij}(g))_{n\times n}.$$
    Then, by construction we have: 
    $$f_{ij}' = \frac{1}{\vert G \vert} \sum_{g \in G} \sum_{k,\ell} \rho_{ik}'(g) f_{k\ell} \rho_{\ell j}(g^{-1}) = \begin{cases}
        0,\quad &\text{if} \quad \rho\not\cong \rho'\\
        \frac{\on{tr}(f)}{n}\delta_{ij} \quad &\text{if}\quad \rho\cong\rho'
    \end{cases}.$$
    We may re-write the second case to be 
    $$\frac{\sum_kf_{kk}}{n}\delta_{ij},$$
    and since it is equal $f_{ij}'$ when $V\cong V'$, the coefficients are equal,
    and thus we have that 
    $$\frac{1}\vert{G} \sum_{g\in G} \rho_{ik}(g)\rho_{\ell j}(g^{-1}) = 0,$$
    when $k \neq \ell$. 
    Conversely, when $k=\ell$, we have 
    $$\frac{1}{\vert G\vert} \sum_{g\in G} \rho_{ik}(g) \rho_{kg}(g^{-1}) = \frac{\delta_{ij}}{n}.$$
    This implies then that 
    \begin{align*}
        \langle\chi_\rho,\chi_\rho\rangle &= \frac{1}{\vert G\vert}\sum_{g \in G}
        \chi_\rho(g)\chi_\rho(g^{-1}) \\
                                          &= \frac{1}{\vert G \vert} \sum_{g\in G}\sum_{i=1}^n \rho_{ii}(g) \sum_{j=1}^n\rho_{jj}(g^{-1})\\
                                          &=\frac{1}{\vert G \vert} \sum_{g\in G} \sum_{i=1}^n\rho_{ii}(g)\rho_{ii}(g^{-1})\\
                                          &= \frac{1}{\vert G\vert}\sum_{g\in G}\sum_{i=1}^n\frac{1}{n}\\
                                          &=1
    \end{align*}
    where the third equality follows from the aforementioned fact that the 
    sum vanishes when $i \neq j$. Suppose now that $V\not\cong V'$.
    Then, 
    $$\frac{1}{\vert G \vert} \sum_{g\in G} \rho_{ik}'(g)\rho_{\ell j}(g^{-1}) = 0,$$
    for all $i,k,\ell,j$,
    from which it follows that $\langle\chi_\rho,\chi_{\rho'}\rangle = 0$.
\end{proof}

\subsection{Characters of $G$ Characterises $G$-Representations (no pun intended)}

\begin{theorem}
    Suppose $V$ is a representation of $G$ with character $\phi$. Let
    $$V = W_1\oplus \cdots \oplus W_s,$$ be a decomposition of $V$ into irreducible
    representations. Let $W$ be an arbitrary irreducible representation of $G$,
    with character $\chi$. 
    Then, the inner product $\langle \phi,\chi\rangle$ gives the amount of 
    $W_i$'s that are isomorphic to $W$.
\end{theorem}

\begin{proof}
    By the decomposition of $V$, we have that 
    $$\phi = \sum_{i=1}^s \chi_{W_i}.$$
    By linearity, 
    $$\langle\phi,\chi\rangle = \left\langle \sum_{i=1}^s \chi_{W_i},\chi\right\rangle = \sum_{i=1}^s \langle \chi_{W_i},\chi\rangle = \sum_{W_i\cong W} \langle \chi_{W_i},\chi\rangle,$$
    where the third equality follows by the orthogonality property 
    of characters. 
\end{proof}

\begin{corollary}
    $\# \lbrace i: W_i\cong W\rbrace$ does not depend on the decomposition of 
    $V$.
\end{corollary}

\begin{corollary}
    Two $G$-representations with the same character are isomorphic as 
    $G$-representations. 
\end{corollary}

\begin{proof}
    Let $\lbrace V_i\rbrace$ be a set of pairwise 
    non-isomorphic irreducible representations of $G$, and let 
    $V$ be a representation with character $\phi$. Then:
    $$V \cong \bigoplus_iV_i^{\oplus \langle \phi,\chi_{V_i}\rangle}.$$
\end{proof}

\begin{corollary}
    The number of non-isomorphic irreducible representations of $G$ is equal 
    to the number of irreducible characters of $G$, which is less than 
    or equal to the number of conjugacy classes of $G$.
\end{corollary}

Later, we will concern ourselves with finding all the irreducible representations
of a given group $G$. But we now have an upper bound for the amount of them.

\begin{theorem}
    If $\phi$ is the character of a $G$-representation, then 
    $\langle \phi,\phi\rangle \in \mathbb{Z}_{>0}$. Moreover,
    $\phi$ is irreducible if and only if $\langle\phi,\phi\rangle = 1$.
\end{theorem}

\begin{proof}
    If $\phi$ is irreducible, then we are done. Otherwise,
    every $G$-representation decomposes as a direct sum of irreducible 
    representations --- that is, 
    $$\phi = \sum_{i=1}^km_i\chi_i,$$where
    $\chi_i\neq \chi_j$ for $i\neq j$, and each $\chi_i$ is an irreducible 
    character. Here, $m_i$ are the multiplicities of $V_i$ in the 
    direct sum decomposition.  It follows then that 
    $$\langle \chi,\chi\rangle = \left\langle \sum_{i=1}^k m_i\chi_i,\sum_{i=1}^km_i\chi_i\right\rangle = \sum_{i=1}^k m_i^2 \langle \chi_i,\chi_i\rangle = \sum_{i=1}^km_i^2,$$
    where the second and third equality follows from the orthogonality of 
    characters.
    At least one $m_i$ is greater than $0$, and thus $\langle\phi,\phi\rangle \in \mathbb{Z}_{>0}$.
\end{proof}

\subsection{Characters of The Regular Representation}

Recall that the regular representation $R_G$ is given by the space:
$$R_G:= \bigoplus_{g\in G}\C e_g,$$ 
and $\rho_h(e_g) = e_{hg}.$ The character $r_G$ is given by: 
$$r_G(g) = \begin{cases}
    \vert G \vert \quad &\text{if} \quad g = 1,\\
    0, \quad &\text{if}\quad g \neq 1
\end{cases}.$$
This follows since the representation $\rho$ takes the basis of $R_G$ to
itself. It follows then that the resulting matrix will have $0$ on its 
diagonal entries.\\\\
Let $\chi$ be an irreducible character of $G$. Then, 
$$\langle \chi,r_G\rangle = \frac{1}{\vert G \vert} \sum_{g \in G} \chi(g)r_G(g^{-1}) = \chi(1) = \dim_\C V,$$
where $V$ is the irreducible representation corresponding to $\chi$. 

\begin{corollary}
    The regular representation 
    $$R_G \cong \bigoplus_{i=1}^k W_i^{\oplus \dim W_i},$$
    where $W_1,\cdots, W_k$ are pairwise non-isomorphic irreducible 
    representations of $R_G$. 
\end{corollary}

\begin{corollary}
    Let us write $n_i:= \dim W_i$, where $W_i$ is as before, and 
    $i=1,\cdots, k$. Then, 
    \begin{itemize}
        \item[(a)] $\sum_{i=1}^k n_i^2 = \vert G \vert$.
        \item[(b)] $\sum_{i=1}^k n_i\chi_{W_i}(g) = 0$.
    \end{itemize}
\end{corollary}
\begin{proof}
    Use the fact that $r_G = \sum_i n_i\chi_i$. 
\end{proof}

\section{Lecture 3, 04/08/2023}

\subsection{Number of Irreducible Representations of $G$ (up to isomorphism)}

Let $H$ be the space of class functions of $G$. We have seen that irreducible
characters of $G$ form a set of orthonormal vectors in $H$.

\begin{exercise}
    Check that distinct irreducible characters of $G$ are linearly 
    independent.
\end{exercise}

\begin{proof}
    Let $G$ be finite group, and $V_1,\cdots, V_k$ distinct irreducible 
    $G$-representations with characters $\chi_1,\cdots,\chi_k$.
\end{proof}

\begin{proposition}\label{prop3}
    Let $f\in H$, and let $(\rho,V)$ be a $G$-representation.
    Consider a map $$\rho_f := \sum_{g\in G} f(g)\rho_g :V \longrightarrow V.$$
    If $V$ is irreducible of degree $n$ with character $\chi$, 
    $\rho_f = \lambda \on{id}_V$, where $$\lambda = \frac{\vert G\vert}{\dim_\C V} (f\vert\overline{\chi}),$$ where $\overline{\chi}(g) = \chi(g^{-1})$.
\end{proposition}

\begin{proof}
    We first show that $\rho_f \in \on{Hom}_G(V,V)$ --- that is, 
    for some $h \in G$, $\rho_h\circ \rho_f = \rho_f \circ \rho_h$.
    Let $h \in G$. Then, 
    \begin{align*}
            \rho_f \circ \rho_h (v) = \sum_{g \in G} f(g) \rho_g\rho_h(v) &= \sum_{g \in G} f(g) \rho_{gh}\\ &= \sum_{g \in G} f(gh^{-1}) \rho_g(v)\\ 
                                                                          &= \sum_{g \in G} f(h^{-1}gh^{-1})\rho_g(v)\\ &= \sum_{g\in G} f(h^{-1}g)\rho_g(v)\\ 
                                                                          &= \rho_h \circ \rho_f(v).
\end{align*} 
    Since $V$ is irreducible, then this implies that that 
    $\rho_f = \lambda \on{id}_V$ by Schur's lemma, and further that 
    $$\lambda = \frac{\on{tr}(\rho_f)}{\dim_\C V} =\frac{1}{\dim_\C V} \sum_{g \in G}f(g)\on{tr}(\rho_g) = \frac{1}{\dim_\C V}\sum_{g \in G}f(g)\chi(g) = \frac{\vert G\vert}{n} (f\vert \overline{\chi})$$
\end{proof}

\begin{theorem}\label{thm_chars_forms_basis}
    The distinct irreducible characters of $G$, written as 
    $\chi_1,\cdots,\chi_k$, form an orthonormal basis of $H$.
\end{theorem}

\begin{proof}
    We have shown that $\chi_1,\cdots,\chi_k$ are linearly independent and 
    orthonormal, and we know that $\chi_1,\cdots,\chi_k$ are class functions. 
    It remains to show that $\chi_1,\cdots,\chi_k$ span $H$.
    Let $f\in H$. Then, it suffices to show if 
    $(f\vert \chi_i) = 0$, for all $i=1,\cdots,k$, then $f = 0$. \\\\
    Recall that 
    $$(f\vert\chi_i) = \frac{1}{\vert G \vert} \sum_{g \in G} f(g)\chi_i(g^{-1}).$$
    Now, let $(\rho,V)$ be any $G$-representation. Then, there is a direct sum
    decomposition of $V$ into irreducibles:
    $V \cong W_1\oplus\cdots\oplus W_m,$ with $\rho_1,\cdots,\rho_m$ the 
    representations corresponding to the components in the decomposition
    of $V$. Let $f \in H$ be such that $(f\vert\chi_i) = 0$ for all
    $i=1,\cdots,k$. Define a map 
    $$\rho_f : = \sum_{g\in G}\rho_g : V \to V,$$
    It follows then that $\rho_f = (\rho_1)_f + \cdots + (\rho_m)_f$. 
    Then, by Proposition 
    \ref{prop3}, it follows then that $(\rho_i)_f = 0$ for each $i$, and so 
    $\rho_f = 0$. It remains to show that this implies that $f = 0$.\\\\
    Now, choose $V$ to be the regular representation. Then, 
    $V = \bigoplus_{g\in G}\C e_g$. It follows then that 
    $$\rho_f(e_1) = \sum_{g\in G} f(g) \rho_g(e_1) = \sum_{g\in G} f(g)e_g = 0,$$
    and it follows then that $f =0$, since $e_g \neq 0$ since it is a basis 
    by construction.
\end{proof}

\begin{corollary}\label{cor_class_func_decomp}
    Let $f \in H$. Then, 
    $$f = \sum_i (f\vert \chi_i) \chi_i.$$
\end{corollary}

\begin{theorem}
    The number of isomorphism classes of irreducible representations of $G$ is 
    equal to the number of conjugacy classes of $G$.
\end{theorem}

\begin{proof}
    There is an obvious basis of $H$ given by characteristic functions 
    of the form $1_{C_i}$, where $C_i$ denotes a conjugacy class of $G$.
    It follows then that $\dim_\C H$ is equal to the number of 
    conjugacy classes of $G$, and by Theorem \ref{thm_chars_forms_basis}, 
    it follows that the number of conjugacy classes is equal to the irreducible
    characters of $G$. Irreducible characters of $G$ characterise irreducible
    $G$-representations up to isomorphism, and thus we have our result.
\end{proof}

\begin{corollary}\label{cor_cor8}
    Let $g \in G$ and $C(g)$ denote the conjugacy class of an element $g\in G$.
    Then, 
    \begin{itemize}
        \item[(a)] $$\sum_{i=1}^k \overline{\chi_i(g)}\chi_i(g) =\frac{\vert C(g)\vert}{\vert G\vert}.$$
        \item[(b)] For $h$ not in the conjugacy class of $g$,
            $\sum_{i=1}^k \overline{\chi_i(g)}\chi_i(h) = 0$.
    \end{itemize}
\end{corollary}

\begin{proof}
          $$1_{C(g)}(h) = \begin{cases}
               0,\quad &\text{if}\quad h\not\in C(g)\\
               1,\quad &\text{if}\quad h \in C(g)
           \end{cases}.$$
           Since $1_{C(g)}$ is a class function, we know that 
           $$1_{C(g)} = \sum_{i=1}^k (1_{C(g)}\vert \chi_i) \chi_i,$$
           by Corollary \ref{cor_class_func_decomp}. 
           Computing the coefficients,
           $$(1_{C(g)}\vert \chi_i) = \frac{1}{\vert G\vert}\sum_{h\in G}1_{C(h)} \chi_i(h^{-1}) = \frac{1}{\vert G \vert} \sum_{h \in C(g)}\chi_i(h^{-1}) = \frac{1}{\vert G \vert} \sum_{h\in C(g)} \chi_i(g^{-1})=\frac{\vert C(g)\vert}{\vert G \vert} \cdot \chi_i(g^{-1}),$$
           where the third equality follows since $h^{-1}$ is in the 
           conjugacy class of $g^{-1}$, and the fact that $\chi_i$ is a 
           class function. It follows then that 
           $$1_{C(g)} = \sum_{i=1}^k \frac{\vert C(g)\vert}{\vert G \vert} \cdot \chi_i(g^{-1}) \chi_i.$$
           Taking some $h\in G$, 
           $$1_{C(g)}(h) = \sum_{i=1}^k \frac{\vert C(g)\vert}{\vert G \vert} \chi_i(g^{-1})\chi_i(h) =
           \begin{cases}
               1,\quad &\text{if} \quad h \in C(g),\\
               0, \quad &\text{if} \quad h\not\in C(g)
           \end{cases}.$$
\end{proof}

\chapter{Week Three}

\paragraph{This week, we learned about \ldots} 

\section{Lecture 1, 07/08/2023}

\subsection{Canonical Decomposition of a $G$-representation}

Let $\chi_1,\cdots,\chi_k$ be the distinct irreducible characters 
of irreducible representations $W_1,\cdots,W_k$ of $G$, where $k$ 
is the number of conjugacy classes of $G$. Let $V$ be a representation of $G$, 
and $$V = U_1\oplus\cdots\oplus U_s,$$ a decomposition of $V$ into irreducible 
subrepresentations. For $i = 1,\cdots,k$, let 
$$V_i = \bigoplus_{\text{$j$ such that $U_j\cong_G W_i$}} U_j,$$
where $\cong_G$ denotes isomorphisms as $G$-representations. 
Then, $$V \cong \bigoplus_{i=1}^k V_i.$$

\begin{example}
    If $W_1 = \C$ be the trivial representation, then, $V_1 = V^G$ since 
    $G$ acts trivially on $V^G$. 
\end{example}

\begin{theorem}
    \leavevmode
    \begin{itemize}
        \item[(a)] The decomposition $V = V_1\oplus\cdots\oplus V_k$ 
            does not depend on the decomposition 
            $V \cong U_1\oplus\cdots\oplus U_s$. 
        \item[(b)] The projection $p_i : V \to V_i$ is given by
            $$p_i = \frac{n_i}{\vert G \vert} \sum_{g \in G}\overline{\chi_i(g)}\rho_g,$$
            where $n_i = \dim_\C W_i$.
    \end{itemize}
\end{theorem}

\begin{remark}
    Recall from before, we used the fact that $V^G = \varphi(V)$, where 
    $$\varphi := \frac{1}{\vert G \vert}\sum_{g \in G} \rho_g,$$
    to prove the orthogonality of the irreducible characters. The projection map
    seen in part (b) of the theorem is similar to this construction.
\end{remark}

\begin{proof} 
    (b) $\implies$ (a), and thus it suffices to show (b). Let 
    $$q_i := \frac{n_i}{\vert G \vert} \sum_{g \in G} \overline{\chi_i(g)}\rho_g,$$
    which defines a map $q_i : U_j \to U_j$. Since $U_j$ is irreducible, it 
    follows by Proposition \ref{prop3} that 
    $$q_i = \frac{\dim_\C W_i}{\vert G \vert} (\overline{\chi_i} \vert \overline{\chi_{U_j}}) = \begin{cases}
        1,\quad &\text{if} \quad U_j \cong W_i,\\
        0, \quad &\text{if} \quad U_j\not\cong W_i
    \end{cases},$$
    where the last equality follows by the orthogonality of characters.
    This tells us that $q_i\vert_{V_i} = 1$, and $q_i\vert_{V_j} = 0$, for 
    $i\neq j$ --- this follows by construction of $V$.
    It follows then that $q_i$ is equal to the projection $p_i : V\to V_i$.
\end{proof}

The aforementioned decomposition $$V \cong \bigoplus_{i=1}^kV_i,$$
is thus called the \emph{canonical decomposition} of $V$. The 
$V_i$'s appearing in the canonical decomposition are called the 
\emph{isotypic components}.

\begin{example}
    Let $G = \lbrace 1,\sigma\rbrace \cong \mathbb{Z}/2\mathbb{Z}$.
    There are only two irreducible representations, since there are only 
    two conjugacy classes. Let $W^+ = \C$ be the trivial representation,
    with $\rho_\sigma^+ = 1$. The other representation is given by
     $W^- \cong \C$, where $\rho_\sigma^- = -1$.\\\\
     Given any $\mathbb{Z}/2\mathbb{Z}$-representation $(\rho,V)$, 
     the canonical decomposition is given by 
     $$V \cong V^+ \oplus V^-,$$
     where $V^+$ collects all the trivial representations, and 
     $V^-$ collects the non-trivial ones. 
     In particular, 
     $$V^+ = V^G = \lbrace v \in V : \rho_\sigma v = v\rbrace,$$
     and $$V^- = \lbrace v\in V:  \rho_\sigma v = -v\rbrace.$$
     There are two projection maps, given by: 
     $$p^+ = \frac{1}{2}(\rho_1 + \rho_\sigma) : V \longrightarrow V^+,$$ 
     $$p^- = \frac{1}{2}(\rho_1-\rho_\sigma):  V\longrightarrow V^-.$$
     It follows that $V^+$ and $V^-$ both decompose into a direct sum of 
     one-dimensional subspaces.
\end{example}

\subsection{Explicit Decomposition of a Representation}

Let $(V,\rho)$ be a representation of $G$, and let 
$V = V_1\oplus\cdots\oplus V_k$ be the canonical decomposition of $V$.
We know that $V_i \cong W_i^{\oplus k_i}$, where $W_i \not\cong W_j$
for $i \neq j$ is an irreducible $G$-representation.
Fixing $i$, let $W := W_i$ be such that $\dim_\C W = n$, 
and consider a representation 
$$\pi : G \longrightarrow \on{GL}(W),\quad g\longmapsto (\Gamma_{\alpha\beta}(g))_{n\times n}.$$
Let $p_{\alpha\beta}: V \to V$, where $$p_{\alpha\beta} := \frac{n}{\vert G \vert}\sum_{g \in G} \Gamma_{\beta\alpha}(g^{-1})\rho_g.$$

\begin{proposition}
    \leavevmode 
    \begin{itemize}
        \item[(a)] $p_{\alpha\alpha}\vert_{V_j} = 0$ for all $j\neq i$.
            Moreover, $V_{i,\alpha} = \on{Im}(p_{\alpha\alpha}) \subset V_i$.
            Then, $$V_i = \bigoplus_{\alpha = 1}^n V_{i,\alpha},$$
            and $p_i = \sum_{\alpha=1}^np_{\alpha\alpha}$.
        \item[(b)] $p_{\alpha\beta}\vert_{V_j} = 0$ for all $j\neq i$,
            and $p_{\alpha\beta}\vert_{V_{i,\gamma}} = 0$ for all 
            $\gamma \neq \beta$. It follows then that there is a bijection
            $$p_{\alpha,\beta} : V_{i,\beta} \longrightarrow V_{i,\alpha}.$$
        \item[(c)] 
            There is a chain of projections:
            $$V_{i,1} \stackrel{p_{12}}{\longrightarrow} V_{i,2} \stackrel{p_{23}}{\longrightarrow} V_{i,3} \stackrel{p_{34}}{\longrightarrow} \cdots \stackrel{p_{n-1,n}}{\longrightarrow} V_{i,n}.$$
            Now, let $v_1 \in V$, $v_1\neq 0$, and let 
            $v_\alpha := p_{\alpha,1}(v_1) \in V_{i,\alpha}$, and 
            $W(v_1) := \on{Span}_\C(\lbrace v_1,\cdots,v_n\rbrace)$. Then,
            $W(v_i) \subset V_i$ is a subrepresentation, and $\dim_\C W(v_1)=n$,
            and $W(v_1) \cong W_i$ as $G$-representations.
        \item[(d)] Let $\lbrace v_1^1,\cdots, v_1^m\rbrace$ be a basis of 
            $V_{i,1}$, and $m = \dim_\C V_{i,1}$. Then,
            $$V_i = \bigoplus_{j=1}^m W(v_1^j),$$
            gives an explicit decomposition of $V_i$.
            (TODO: draw the tower picture that Ting drew in the lectures).
    \end{itemize}
\end{proposition}

\begin{proof}
    Read Serre's book (or try it yourself!). Uses the relations we derived 
    in the alternate proof of orthogonality of characeters.
\end{proof}

\section{Lecture 2, 10/08/2023}

\subsection{Character Tables}

Recall that given a group $G$, the number of irreducible characters 
are given by the amount of conjugacy classes up to isomorphism.
A character table of $G$ is a $k\times k$ table with conjugacy classes
and irreducible characters of the form:\\\\
$$\begin{centering}
    \begin{tabular}{|c|c|c|c|c|}
        \hline
        &$C_1$&$C_2$&$\cdots$&$C_k$\\
        \hline
        $\chi_1$ &$\ast$&$\ast$&$\cdots$&$\ast$\\
        \hline
        $\vdots$ &$\ast$&$\ast$&$\cdots$&$\ast$\\
        \hline
        $\chi_k$ &$\ast$&$\ast$&$\cdots$&$\ast$\\
        \hline
    \end{tabular}
\end{centering}$$
where $C_i$'s are the conjugacy classes and $\chi_i$'s are the characters 
corresponding to that conjugacy class.

\begin{example}
    (TODO:) rewatch this example
    Let $G = \mathfrak{S}_3$. It has six elements given by:
    $$(1), \quad (12)(3), \quad (13)(2), \quad (1)(23), \quad (123), \quad (132),$$
    and we observe that $$C( (12) ) = \lbrace (1), (12), (132), (13), (23)\rbrace,$$
    and $$C( (123) = \lbrace (1), (123), (132)\rbrace.$$ 
    $C(1)$ always defines a conjugacy class in any group.
    An example of a $\mathfrak{S}_3$-representation is the sign representation:
    given by 
    $$\on{sgn} : G \longrightarrow \on{GL}(\C^3) \stackrel{\det}{\longrightarrow} \C^\times.$$
    The mapping is given in the following way:
    $$(12)(3) \longmapsto \begin{pmatrix}
        0&1&0\\
        0&0&0\\
        0&0&1
    \end{pmatrix} \longmapsto -1.$$
    Similarly, 
    $$(123) \longmapsto \begin{pmatrix}
        &&1\\
        &1&\\
        1&&
    \end{pmatrix} \longmapsto 1.$$
    Alternatively, we can define the \emph{sign} of an element 
    $s\in \mathfrak{S}_3$ to be $\on{sgn}(s) = (-1)^m$, where $m$ is the 
    number of adjacent transpositions appearing in its decomposition into 
    generators (recall that symmetric groups are generated by its adjacent 
    transpositions).\\\\
    It follows then that 
$\on{sgn}((12) )) = (-1)^1 = -1$, and $\on{sgn}( (123)) = \on{sgn}( (12)(13)) = (-1)^2 = 1.$\\\\
    Suppose now that the last representation we are looking for has dimension 
    $n$. Then, we know that $$\vert G \vert = 1^2 + 1^2 + n^2 = 6,$$
    and it follows that $n=2$ --- that is, $\chi_3(1) = 2$.
    Let us consider the regular representation --- recall that 
    the character is given by
    $$r_{\mathfrak{S}_3} = \begin{cases}
        6,\quad &\text{if} \quad g =1,\\
        0, \quad &\text{otherwise}
    \end{cases}.$$
    It follows then that 
    $r_G = \chi_1 + \chi_2 + 2\chi_3$, and thus 
    $$\chi_3 = \frac{r_G - \chi_1 - \chi_2}{2}.$$
    Let us verify that this is infact an irreducible character. By
    orthogonality, we know that $\langle \chi_3,\chi_3\rangle =1$ if 
    it is irreducible. So, computing directly:
    \begin{align*}
        \langle \chi_3,\chi_3\rangle &= \frac{1}{6} \sum_{s \in \mathfrak{S}_3} \chi_3(s) \overline{\chi_3(s)} \\
                                     &= \frac{1}{6}\sum_{s\in\mathfrak{S}_3} \frac{r_G-\chi_1-\chi_2}{2}\cdot \frac{\overline{r_G-\chi_1-\chi_2}}{2}\\
                                     &= \frac{(6 - 1 - 1)(6-1-1)}{24} + \frac{1}{6}\sum_{1 \neq s\in\mathfrak{S}_3} \frac{(\chi_1+\chi_2)(\overline{\chi_1}+\overline{\chi_2})}{4}\\  
                                     &= \frac{2}{3} + \frac{1}{24} \sum_{1\neq s\in\mathfrak{S}_3}\left(\vert \chi_1\vert^2 + \overline{\chi_1}\chi_2 + \chi_1\overline{\chi_2} + \vert\chi_2\vert^2\right) \\
                                     &= \frac{2}{3} -\frac{1}{24}\left(\vert\chi_1\vert^2(1) + \overline{\chi_1}(1)\chi_2(1) + \chi_1(1)\overline{\chi_2}(1) + \vert\chi_2\vert^2(1)\right) +\frac{1}{24}\sum_{s\in\mathfrak{S}_3} (\vert \chi_1\vert^2 + \overline{\chi_1}\chi_2 + \chi_1\overline{\chi_2} + \vert\chi_2\vert^2)\\
                                     &= \frac{2}{3} - \frac{1}{24} (1^2 + 1\cdot 1 + 1\cdot 1 + 1^2) + \frac{1}{24}(\langle \chi_1,\chi_1\rangle + \langle \chi_2,\chi_1\rangle + \langle \chi_1,\chi_2\rangle + \langle \chi_2,\chi_2\rangle)\\
                                     & = \text{not one!! $>:($}
    \end{align*}
    (TODO: come back to this calculation and do it properly)
    We end up with the character table:
    $$\begin{tabular}{|c|c|c|c|}
        \hline
        &$C( (1)) =C_1$ & $C( (12)) = C_{\on{sgn}}$ & $C( (123)) =C_{\on{reg}}$\\
        \hline
        $\chi_1$ & $1$ & $1$ & $1$\\
        \hline
        $\chi_2 = \chi_{\on{sgn}}$ & $1$ & $-1$ & $1$\\
        \hline
        $\chi_3=\chi_{\on{std}}$ & $2$ & $0$ & $-1$\\
        \hline
    \end{tabular}$$
    \begin{exercise}
        Check orthogonality: $(\chi_i\vert\chi_j) = \delta_{ij}$.
    \end{exercise}
    \begin{proof}
        
    \end{proof}
    In fact, it turns out that $\chi_3$ is a character of the \emph{standard
    representation}. In particular,
    the representation space is given by
    $$\lbrace (x_1,x_2,x_3) \in \C^3 : x_1+x_2+x_3 = 0\rbrace,$$
    which we equip with a basis $v_1 = e_1-e_2$, and $v_2 = e_2-e_3$.
    Then, $\mathfrak{S}_3$ acts by permuting the canonical basis vectors 
    $e_1,e_2,e_3$.
    If $s = (12)(3)\in\mathfrak{S}_3$, then 
    $$sv_1 = -v_1,\quad sv_2 = v_1+v_2.$$
    It follows then that 
    $$s\longmapsto \begin{pmatrix}
        -1&1\\
        0&1
    \end{pmatrix},$$
    under this two dimensional representation. Similarly, for $c = (123)$, 
    we see that 
    $$c\longmapsto \begin{pmatrix}
        0&-1\\
        1&-1
    \end{pmatrix}.$$
\end{example}

\begin{example}
    Now, let $G = \mathfrak{S}_4$. Then, it has $5$ conjugacy classes, given by
    $$1=1=(1)(2)(3)(4), \quad t_1=(12)(3)(4), \quad t_2=(12)(34), \quad t_3=(123)(4), \quad t_4=(1234).$$
    The sum of the lengths of each cycle gives a partition of $4$.
    In particular, these classes correspond to the partitions
    $$1+1+1+1, \quad 2+1+1, \quad 2+2, \quad 3+1, \quad 4.$$
    Consider the standard representation given by 
    $\mathfrak{S}_4$ acting on $$\lbrace (x_1,x_2,x_3,x_4) \in \C^4 : x_1+x_2+x_3+x_4 = 0\rbrace.$$
    We perform the same computation as before, and we find that 
    $$\rho_3 : \mathfrak{S}_4 \longrightarrow \on{GL}(\C^3),$$
    has trace given by 
    $$\on{tr}(\rho_3(t_2)) = 1, \quad \on{tr}(\rho_3(t_3)) = -1,\quad \on{tr}(\rho_3(t_4)) = 0,\quad \on{tr}(\rho_3(t_5)) = -1,$$
    where the sizes of each conjugacy class is $6$, $3$, $8$, and $6$, 
    respectively.
    We verify irreducibility by checking that its inner product with itself 
    is $1$:
    $$\langle \chi_{\rho_3}, \chi_{\rho_3}\rangle = \frac{1}{24} \left( 3^2 + 1^2\cdot 6 + (-1)^2 \cdot 3 + (-1)^2 \cdot 6\right) = 1,$$
    and it follows that this representation is irreducible.\\\\
    There remain two representations, $\chi_4$, and $\chi_5$, 
    suppose they have dimensions $a$ and $b$, respectively. 
    We know that $1^2 + 1^2 + 3^2 + a^2 + b^2 = 24$, and thus 
    the dimensions must satisfy $a^2+b^2 = 13$. 
    Let us consider the representation given by 
    $$\chi_{\on{sgn}} \otimes \chi_{\on{std}},$$
    which we can verify is an irreducible representation. This is given by 
    $\chi_4$. The last representation $\chi_5$ is given by the regular
    representation.
    $$\begin{tabular}{|c|c|c|c|c|c|}
        \hline
        & $C(1)$ & $C(t_1)$ & $C(t_2)$ & $C(t_3)$ & $C(t_4)$\\
        \hline 
        $\chi_1$ & $1$ & $1$ & $1$ & $1$ & $1$\\
        \hline
        $\chi_2 = \chi_{\on{sgn}}$ & $1$ & $-1$ & $1$ & $1$ & $-1$\\
        \hline 
        $\chi_{\on{std}} = \chi_3$ & $3$ & $1$ & $-1$ & $0$ & $1$\\
        \hline
        $\chi_{\on{sgn}} \otimes \chi_{\on{std}}$ & $3$ & $-1$ & $-1$ & $0$ & $1$\\
        \hline
        $\chi_{5}$ & $2$ & $0$ & $2$ & $-1$ & $0$\\
        \hline
    \end{tabular}$$
\end{example}

\section{Lecture 3, 11/08/2023}

Recall from last time that we looked at the irreducible representations of 
$\mathfrak{S}_4$, whose irreducible characters are given by 
$\chi_{\on{triv}}$, $\chi_{\on{sgn}}$, $\chi_{\on{std}}$, 
$\chi_{\on{sgn}\otimes \on{std}}$, and $\chi_{5}$ 
corresponding to conjugacy classes
$(1)$, $(12)$, $(12)(34)$, $(123)$, and $(1234)$.
We did not have time to determine what $\chi_5$ was, and so we will do that 
today.\\\\
Let us consider the element $t_2 = (12)(34)$. We know that 
$\chi_5(t_2) = 2$. Since $t^2 = 1$, we know that 
$\rho(t_2)^2 = I_2$, the $2\times 2$ identity matrix. From this, we can deduce
that $\rho(t_2) = I_2$, since after diagonalisation, the trace must 
add up to $2$, and has to have the property that $\rho(t_2)^2 = I_2$.\\\\
This then implies that $\rho$ is the identity if $s \in \mathfrak{S}_4$ 
is a transposition. It follows thus that $\rho$ is trivial on the subgroup
with elements $$G_0 = \lbrace 1, (12)(34), (13)(24), (14)(23)\rbrace.$$
That is, $\rho\vert_{G_0} = 1$. In particular, $G_0$ is a normal 
subgroup.
\begin{exercise}
    Check that $G_0$ is a normal subgroup of $\mathfrak{S}_4$.
\end{exercise}

\begin{proof}
    We recall the fact that $\mathfrak{S}_4$ is generated by adjacent 
    transpositions. That is, 
    $$\mathfrak{S}_4 = \left\langle (12), (13), (14), (23),(24),(34) : \text{some braid relations}\right\rangle.$$
    Observe that elements $G_0$ is made up of products of adjacent 
    transpositions. It 
    follows then that $G_0$ is invariant under conjugation by all elements of 
    $\mathfrak{S}_4$.
\end{proof}
Then, one may check that 
$$\mathfrak{S}_4/G_0 \cong \mathfrak{S}_3.$$
\begin{exercise}
    Check that this isomorphism is true. (Hint: $\mathfrak{S}_4$ acts on 
    $(12)(34)$, $(13)(24)$, $(14)(23)$ by conjugation.)
\end{exercise}


It follows then that a $\mathfrak{S}_4$-representation $W$ factors through 
$\mathfrak{S}_4/G_0\cong\mathfrak{S}_3$: 
$$\rho : \stackrel{\simeq}{\longrightarrow} \mathfrak{S}_4/G_0\cong \mathfrak{S}_3 \longrightarrow \on{GL}_2(W),$$
and it follows then that the representation $\chi_5$ should correspond
to the standard representation of $\mathfrak{S}_3$ that we deduced 
from the $\mathfrak{S}_3$ example.

\subsection{Representations of Abelian Groups}

Since $G$ is abelian, it follows that each conjugacy class consists of one 
element. 

\begin{theorem}\label{thm_abgrps_onedim}
    The following properties are equivalent:
    \begin{itemize}
        \item[(a)] $G$ is abelian 
        \item[(b)] all irreducible $G$-representations have dimension one
    \end{itemize}
\end{theorem}

\begin{proof}
    If $n_i = \dim_\C V_i$, for distinct, non-isomorphic, irreducible 
    $G$-representations for $i=1,\cdots,k$, then we know that 
    $\sum n_i^2 = \vert G\vert$. It follows then that $n_i = 1$ if and only 
    if the number of conjugacy classes of $G$ is equal to $\vert G \vert$.
    This can only happen if $G$ is abelian.
\end{proof}

\begin{corollary}
    Let $A$ be an abelian subgroup of a group $G$. Then, each irreducible 
    representation of $G$ has degree $\leq \frac{\vert G\vert}{\vert A \vert}$.
\end{corollary}

\begin{proof}
    Let $(\rho,V)$ be an irreducible $G$-representation. Let 
    $$\rho_A : A \hookrightarrow G \stackrel{\rho}{\longrightarrow} \on{GL}(V).$$
    Then, $(\rho_A,V)$ defines a representation of $A$. Let $W\subset V$ 
    be an irreducible subrepresentation. It follows then that $\dim_\C W =1$
    by Theorem \ref{thm_abgrps_onedim}. Now, let 
    $$V' := \sum_{g\in G}\rho_gW,$$
    which is a $G$-stable subspace of $V$. Since $V' \neq \lbrace 0\rbrace$,
    it follows then that $V' = V$. For any $h\in G$, and $a \in A$, 
    we know that $\rho_{ha}W = \rho_h\rho_aW = \rho_h\rho_a W$,
    and it follows then that $V'$ is a sum of $A$-cosets 
    $$V' = \sum_{\text{left cosets $gA$, $g\in G$}}\rho_{gA} W,$$
    and it follows that $$\dim_\C V = \dim_\C V' \leq \frac{\vert G\vert}{\vert A \vert}.$$
\end{proof}

\begin{remark}
    Note that we do not require $A$ to be a normal subgroup, so we are not
    taking quotients. As such, we are simply taking left cosets.
\end{remark}

\subsection{Representations of Products of Groups}

Let $G_1$ and $G_2$ be two finite groups, from which we may form the 
product of two groups given by $G_1 \times G_2$ with group 
structure given by applying group operation component-wise.
Given two representations $(\rho^1,V_1)$, and $(\rho^2,V_2)$ of 
$G_1$ and $G_2$, respectively. Then, we define a 
$(G_1\times G_2)$-module in the following way:
$$\rho : G_1\times G_2 \longrightarrow \on{GL}(V_1\otimes V_2), \quad (g_1,g_2)\longmapsto \left(v_1 \otimes v_2 \longmapsto \rho^1_{g_1}v_1 \otimes \rho^2_{g_2}v_2\right).$$
The characters are thus given by:
$$\chi_{\rho^1\otimes \rho^2}(g_1,g_2) = \chi_{\rho^1}(g_1) \chi_{\rho^2}(g_2).$$
\begin{exercise}
    Check that this holds.
\end{exercise}

\begin{proof}
    For each $g = (g_1,g_2) \in G$, let 
    $\lbrace w^1_i\rbrace_{i=1}^k$, and $\lbrace w^2_j\rbrace_{j=1}^\ell$ 
    be an eigenbasis of $\rho^1_{g_1}$, and $\rho^2_{g_2}$, respectively.
    Then, the representation $\rho_g$ 
    will have an eigenbasis given by $\lbrace w^1_i\otimes w^2_j\rbrace_{i,j}$,
    which will have dimension $k\ell$. It follows then that for each 
    $g = (g_1,g_2)$, we have that 
    $$\on{tr}(\rho_g) = \on{tr}(\rho^1_{g_1}\otimes\rho^2_{g_2}) = \on{tr}(\rho^1_{g_1})\on{tr}(\rho_{g_2}^2),$$
    and it thus follows that 
    $$\chi_{\rho^1\otimes\rho^2}(g_1,g_2) = \chi_{\rho^1}(g_1)\chi_{\rho^2}(g_2),$$
    as claimed.
\end{proof}

\begin{remark}
    When $G_1=G_2 = G$, we defined the tensor product representation 
    as a $G$-representation, \emph{not} a $(G\times G)$-representation.
    However, we may view $G$ as a subgroup of $G\times G$ via the diagonal 
    map, and view the tensor $G$-representation as the 
    representation $\rho^1\otimes \rho^2$ restricted to $G$ --- that is,
    $\rho^1\otimes\rho^2\vert_G$.
\end{remark}

\begin{theorem}\label{thm_prod_groups}
    \leavevmode
    \begin{itemize}
        \item[(a)] If $\rho^1$ is an irreducible $G_1$-representation,
            and $\rho^2$ is an irreducible $G_2$-representation, then 
            $\rho^1\otimes \rho^2$ is an irreducible 
            $(G_1\times G_2)$-representation.
        \item[(b)] Each irreducible representation of $G_1\times G_2$ 
            is isomorphic to a representation $\rho^1\otimes \rho^2$,
            where $\rho^i$ is an irreducible $G_i$-representation 
            for $i=1,2$.
    \end{itemize}
\end{theorem}

\begin{proof}
    \leavevmode
    \begin{itemize}
        \item[(a)] Using the orthonormality of characters, it is sufficient
            to show that $\chi_{\rho^1\otimes\rho^2}$ has norm $1$.
            Thus, computing directly:
            \begin{align*}
                \langle \chi_{\rho^1\otimes\rho^2},\chi_{\rho^1\otimes\rho^2} \rangle&= \frac{1}{\vert G_1\vert \vert G_2\vert}\sum_{\substack{g_1\in G_1\\ g_2 \in G_2}} \chi_{\rho^1\otimes\rho^2}(g_1,g_2) \overline{\chi_{\rho^1\otimes\rho^2}(g_1,g_2)}\\
                                                                                     &= \frac{1}{\vert G_1\vert \vert G_2\vert} \sum_{\substack{g_1 \in G_1\\ g_2 \in G_2}} \chi_{\rho^1}(g_1) \chi_{\rho^2}(g_2) \overline{\chi_{\rho^1}(g_1)}\overline{\chi_{\rho^2}(g_2)}\\
                                                                              &= \left(\frac{1}{\vert G_1\vert}\sum_{g_1 \in G_2}\chi_{\rho^1}(g_1)\overline{\chi_{\rho^1}(g_1)}\right) \left(\frac{1}{\vert G_2\vert} \sum_{g_2 \in G_2} \chi_{\rho^2}(g_2)\overline{\chi_{\rho^2}(g_2)}\right)\\
                                                                              &=\langle \chi_{\rho^1},\chi_{\rho^1}\rangle \cdot \langle\chi_{\rho^2},\chi_{\rho^2}\rangle\\
                                                                              &=1.
            \end{align*}
        \item[(b)] 
            Suppose that the irreps of $G_1$ have dimensions $n_i$, and 
            irreps of $G_2$ have dimensions $m_j$, for some $1\leq i \leq k$,
            and $1\leq j \leq \ell$. It follows then that 
            $\rho_i^1\otimes \rho^2_j$ gives an irreducible representation of 
            dimension $n_im_j$. Observe that 
            $$\sum_{i,j} (n_im_j)^2 = \left(\sum_{i=1}^k n_i\right)\cdot \left(\sum_{j=1}^\ell m_j\right) = \vert G_1\vert \cdot \vert G_2\vert = \vert G_1\times G_2\vert,$$
            and it follows that we have a collection of irreducible 
            $(G_1\times G_2)$-representations of the form 
            $\rho^1_i\otimes \rho^2_j$. It remains to show of these 
            irreducible representations are distinct.
            \begin{exercise}
                Show that these $(G_1\times G_2)$-irreps are distinct
                (Hint: use orthonormality property).
            \end{exercise}
            \begin{proof}
                Let $\rho_1\otimes \rho_2$ be an irreducible 
                $(G_1\times G_2)$-representation that is distinct from 
                $\rho^1\otimes \rho^2$. Then, computing directly:
                \begin{align*}
                    \langle \chi_{\rho^1\otimes\rho^2},\chi_{\rho_1\otimes\rho_2}\rangle &= \frac{1}{\vert G_1\vert \cdot\vert G_2\vert} \sum_{\substack{g_1\in G_1\\g_2 \in G_2}} \chi_{\rho^1\otimes\rho^1}(g_1,g_2)\overline{\chi_{\rho_1\otimes \rho_2}(g_1,g_2)}\\
                                                                                         &=\frac{1}{\vert G_1\vert\cdot\vert G_2\vert} \sum_{\substack{g_1\in G_1\\g_2 \in G_2}} \chi_{\rho^1}(g_1) \chi_{\rho^2}(g_2) \overline{\chi_{\rho_1}(g_1)} \overline{\chi_{\rho_2}(g_2)}\\
                                                                                         &= \left(\frac{1}{\vert G_1\vert} \sum_{g_1\in G_1} \chi_{\rho^1}(g_1)\overline{\chi_{\rho_1})(g_1)}\right)\left(\frac{1}{\vert G_2\vert} \sum_{g_2\in G_2} \chi_{\rho_2}(g_2)\overline{\chi_{\rho_2}(g_2)}\right)\\
                                                                                         &= \langle \chi_{\rho^1},\chi_{\rho_1}\rangle \cdot \langle \chi_{\rho^2},\chi_{\rho_2}\rangle\\
                                                                                         &= 0\cdot 0 \\
                                                                                         &=0,
                \end{align*}
                where the second-last equality follows by orthogonality 
                of irreducible characters.
            \end{proof}
    \end{itemize}
\end{proof}

\chapter{Week Four}

\paragraph{This week, we learned \ldots}

\section{Lecture 1, 14/08/2023}

We give an alternative proof this result from last time:
\begin{theorem}
    Every irreducible representation of $G_1\times G_2$ is of the form 
    $\rho^1\otimes \rho^2$, where $\rho^i$ is an irreducible representation
    of $G_i$ for $i=1,2$.
\end{theorem}

\begin{proof}
    It suffices to show that for any class function $f$ on $G_1\times G_2$, 
    if $f$ is orthogonal to all characters of the form 
    $\chi_{\rho_1\otimes\rho_2}(g_1,g_2)=\chi_{\rho_1}(g_1)\chi_{\rho_2}(g_2)$,
    then $f=0$. Let us write $\chi_1 = \chi_{\rho_1}$, and 
    $\chi_2= \chi_{\rho_2}$. Suppose that 
    $\langle f,\chi_1\otimes\chi_2\rangle = 0$. Then, 
    \begin{equation}\label{eqn_4.1}
        0=\frac{1}{\vert G\vert} \sum_{\substack{g_1 \in G_1\\ g_2 \in G_2}} f(g_1,g_2) \overline{\chi_1(g_1)}\overline{\chi_2(g_2)}.
    \end{equation}
    Let $$\widetilde{f}(g_1) = \sum_{g_2 \in G_2} f(g_1,g_2) \overline{\chi_2(g_2)},$$
    which defines a class function on $G_1$. (\ref{eqn_4.1}) then implies that 
    $\langle\widetilde{f},\chi_1\rangle = 0$ for all irreducible characters
    $\chi_1$ of $G_1$. It follows then that $\widetilde{f}(g_1) = 0$ for all 
    $g_1 \in G_1$.
    That is, $$\sum_{g_2 \in G_2}f(g_1,g_2) \overline{\chi_2(g_2)}= 0,$$
    for all $g_1\in G_1$. This thus implies then that 
    $f(g_1,-)$ is a class function on $G_2$. 
    It follows then that 
    $$\langle f(g_1,-),\chi_2\rangle = 0,$$
    for all irreducible characters $\chi_2$ of $G_2$. It follows thus
    that $f(g_1,-) = 0$ for all $g_1\in G_1$, and it thus follows
    that $f=0$.
\end{proof}

Theorem \ref{thm_prod_groups} thus tells us that the representations of 
$G_1\times G_2$ can be reduced to the representation theory of 
$G_1$ and $G_2$ separately. 

\subsection{Induced Representations}

This gives us a way to produce a representation of a group $G$ from a 
known representation of a subgroup $H$.
Let us first recall that given a subgroup $H\leq G$, the \emph{left cosets}
are given for $g\in G$ by:
$$gH := \lbrace gh : h\in H\rbrace.$$
Two cosets $gH$ and $g'H$ are the same if and only if 
$g^{-1}g'\in H$. The set of left cosets is denoted by $G/H$.
\begin{remark}
    This is \emph{not} a quotient of $G$ by $H$. In order for that to happen,
    $H$ must be a \emph{normal subgroup} --- that is, a subgroup invariant
    under conjugation by $G$.
\end{remark}
For $R$ a set of representatives on $G/H$, we obtain a partition of $G$ 
into left cosets:
$$G = \bigsqcup_{s \in R} sH.$$
Further, $$\vert G/H\vert = \frac{\vert G\vert}{\vert H\vert} := [G:H],$$
the \emph{index} of $H$ in $G$. Now, let $(\rho,V)$ be a $G$-representation,
and define the \emph{restriction representation} by:
$$\on{Res}_H^G(V) := \rho\vert_H : H\hookrightarrow G \longrightarrow \on{GL}(V).$$
Let $W\subset V$ be a subrepresentation of $H$, and let 
$$\theta : H \longrightarrow \on{GL}(W),$$
be the corresponding representation.
Let $\sigma \in G/H$. Then, by the $H$-stability of $W$, it follows then 
that $\rho_sW = \rho_{s'}W$ if $s,s'\in\sigma$. So, for some $s\in\sigma$,
let us define $$W_\sigma:= \rho_s(W).$$
\begin{remark}
    A priori, we do not know if the $W_\sigma$'s have trivial intersection.
    So, we cannot take direct sums.
\end{remark}
It folows then that $$\sum_{\sigma \in G/H} W_\sigma \subset V,$$
is a subrepresentation. 
It follows that $G$ acts on $W_\sigma$ by permuting the cosets. That is,
$$gW_\sigma = \rho_g \rho_sW = \rho_{gs}W,$$
which maps it to another coset, say $g\sigma = \sigma' \in G/H$

\begin{definition}
    We say that the representation $(\rho,V)$ of $G$ is 
    \emph{induced} by the representation $(W,\theta)$ if:
    $$V = \bigoplus_{\sigma \in G/H}W_\sigma = \bigoplus_{r \in R} \rho_rW.$$
    We write $$\on{Ind}_H^G W := V.$$
    We see from this definition that $$\dim \on{Ind}_H^GW = [G:H]\dim W.$$
\end{definition}

\begin{example}\label{example_ind_properties}
    \leavevmode 
    \begin{itemize}
        \item[(a)] Let $H\leq G$ a subgroup, and let $V = R_G$ 
            and $W = R_H$, the regular representations of $G$ and 
            $H$, respectively. Then, 
            $$R_G = \on{Ind}_H^GR_H.$$
            \begin{exercise}
                Verify this.
            \end{exercise}
            \begin{proof}
                Let $R_G : G \to \on{GL}(V)$ and $R_H : H \to \on{GL}(W)$ 
                be the regular representations corresponding to $G$ and $H$,
                respectively. Then, by definition,
                $$V = \bigoplus_{g \in G}\C e_g, \quad W = \bigoplus_{h \in H}\C e_h.$$
                It follows by construction that $W$ is a $H$-subrepresentation 
                of $V$ (where $V$ is identified as a $H$-representation 
                by the restriction $\on{Res}_H^G$).
                Choosing a set of representatives $R$ in the coset $G/H$, 
                we recall that there is a partition of $G$:
                $$G = \bigsqcup_{s \in R} sH.$$
                Then, we may re-write $V$ as:
                $$V = \bigoplus_{g\in G}\C e_g = \bigoplus_{s\in R}\bigoplus_{\sigma \in sH} \C e_\sigma.$$
                Define $$W_\sigma := \bigoplus_{\sigma \in sH} \C e_\sigma = \bigoplus_{\lbrace sh : h\in H\rbrace} \C e_{sh} = \bigoplus_{h\in H} \C e_{sh} = \bigoplus_{h\in H} R_G(s)(\C e_h)=R_G(s)(W).$$
                It follows then that for each $\sigma \in sH$,
                $$V = \bigoplus_{s\in R} W_\sigma,$$
                and it follows by definition then that 
                $\on{Ind}_H^GW = V$, and so
                $$R_G = \on{Ind}_H^G R_H,$$
                as claimed.
            \end{proof}
        \item[(b)] Let $$V = \bigoplus_{\sigma \in G/H} \C e_\sigma,$$
            which we equip with a $G$-representation by
            $$\rho : G \longrightarrow \on{GL}(V),\quad g\longmapsto (e_\sigma \longmapsto e_{g\sigma}).$$
            This action is called the \emph{permutation representation of 
            $G$ on $G/H$}.
            Observe here that $H$ acts trivially on $e_1$, for $\sigma=1H$.
            It follows then that 
            $$V = \on{Ind}_H^G(\C e_1).$$
        \item[(c)] Given two induced representations 
            $\rho_1 = \on{Ind}_H^G \theta_1$, and 
            $\rho_2 = \on{Ind}_H^G\theta_2,$
            $$\rho_1\oplus \rho_2 = \on{Ind}_H^G(\theta_1\oplus \theta_2).$$
        \item[(d)] Let $(V,\rho) = \on{Ind}_H^G(W,\theta)$.
            If $W_1 \subset W$ is a subrepresentation of $H$, then 
            $$V_1 = \sum_{r \in R} \rho_rW_1\subset V,$$
            is a subrepresentation of $G$. Further, 
            $$V_1 = \on{Ind}_H^G W_1.$$
        \item[(e)] If $\rho = \on{Ind}_H^G\theta$, and $\rho'$ is a 
            representation of $G$, then 
            $$\rho\otimes \rho' = (\on{Ind}_H^G\theta)\otimes \rho' = \on{Ind}_H^G(\theta\otimes \on{Res}_H^G\rho').$$
            It follows then that 
            $$\on{Ind}_H^G(\on{Res}_H^G) = \rho' \otimes \on{Ind}_H^G\C_{\on{triv}},$$
            where $\C_{\on{triv}}$ is the trivial representation, 
            as seen in part (b) of this example.
    \end{itemize}
\end{example}

\subsection{Existence and Uniqueness of Induced Representation}

We wish to show that given any subgroup $H$ of $G$, and a representation of 
$H$, there always exists a unique $G$-representation given by the induced 
representation.

\begin{lemma}\label{lemma_extend_to_ind}
    Suppose that $(\rho,V ) = \on{Ind}_H^G(\theta,W)$. 
    Let $\rho' : G = \on{GL}(V')$ be a representation of $G$, and let 
    $f : W \to V'$ be an element of $\on{Hom}_H(W,\on{Res}_H^GV')$.
    Then, $f$ can be extended uniquely
    to a linear map $F : V \to V'$ such that 
    $F\vert_W = f$, and $F \in \on{Hom}_G(V,V')$. In particular, the
    diagram $$% https://tikzcd.yichuanshen.de/#N4Igdg9gJgpgziAXAbVABwnAlgFyxMJZABgBpiBdUkANwEMAbAVxiRADUBeAHW4OACSYKAF8A+gAkAegHEAFAHVSvHAAsYOOgEoQI0uky58hFGQCMVWoxZsFu-SAzY8BImdIXq9Zq0QcA5LqWMFAA5vBEoABmAE4QALZI7iA4EEhkKXRYDGyqEBAA1iBe1r4gWPbRcYmIyalIAEwlPmxRlSCxCenU9YhNVi1+vDAAHlhwOHAAhAAEAGK8WGC8-BIJ4vLspOz+OiIUIkA
\begin{tikzcd}
{V=\on{Ind}_H^G(W,\theta)} \arrow[rd, "{\exists! F\in\on{Hom}_G(V,V')}"] &    \\
W \arrow[u, "i", hook] \arrow[r, "f"]                                    & V'
\end{tikzcd}$$
commutes
\end{lemma}

\begin{proof}
    We will prove this later!
\end{proof}

\begin{theorem}
    Let $(W,\theta)$ be a $H$-representation. Then, there exists a unique
    (up to isomorphism) $G$-representation $(V,\rho)$ which is induced by 
    $H$. That is, $V = \on{Ind}_H^GW$.
\end{theorem}

\begin{proof}
    (Existence) By additivity of $\on{Ind}_H^G$, as seen in Example 
    \ref{example_ind_properties}(c), we can assume that $(W,\theta)$ 
    is irreducible without loss of generality since we can extend linearly.
    It follows then that $(W,\theta)$ is a subrepresentation of $R_H$.
    Since $R_G = \on{Ind}_H^GR_H$, it follows from 
    Example \ref{example_ind_properties}(d) that $V$ can be 
    induced. This proves existence.\\\\
    (Uniqueness) Suppose $(V,\rho)$ and $(V',\rho')$ are both 
    $G$-representations induced from $(W,\theta)$. Let us consider 
    a map $W \hookrightarrow V'$. It follows from Lemma \ref{lemma_extend_to_ind}
    that there exists a map $F: V \to V'$ such that $F \in \on{Hom}_G(V,V')$,
    and $F\vert_W = \on{id}_W$. We will finish the proof of this next time!
\end{proof}

\section{Lecture 2, 17/08/2023}

\subsection{Uniqueness of Induced Representation}

Last time, we proved that the induced representation exists. Today, we
will prove existence. We also has Lemma \ref{lemma_extend_to_ind},
which we did not prove.

\begin{proof}[Proof of Lemma \ref{lemma_extend_to_ind}]a
    By definition, 
    $$V = \bigoplus_{s \in R} \rho_s(W).$$
    Then, define $$F(\rho_s(W)) = \rho'_sF(w) = \rho_s'f(w),$$
    for all $w\in W$ by property of a $G$-homomorphism.
    \begin{exercise}
        Check that $F \in \on{Hom}_G(V,V')$.
    \end{exercise}
\end{proof}

\begin{theorem}
    Let $(W,\theta)$ be a $H$-representation. Then, there exists a unique
    (up to isomorphism) $G$-representation $(V,\rho)$ which is induced by 
    $H$. That is, $V = \on{Ind}_H^GW$.
\end{theorem}

\begin{proof}[Proof of Uniqueness]
    Suppose that $(\rho,V)$ and $(\rho',V')$ are both $G$-representations 
    induced by representation of $H$. By Lemma \ref{lemma_extend_to_ind},
    there exists a unique map $F$: 
    $$\begin{tikzcd}
        V \arrow[rd, "\exists! F", dashed]            &    \\
        W \arrow[r, "i'"', hook] \arrow[u, "i", hook] & V'
    \end{tikzcd}$$
    such that $F\vert_W = \on{id}$.  By definition, we know that 
    $$V' = \bigoplus_{g\in G/H}\rho_g'W,$$
    and since $F \in \on{Hom}_G(V,V')$, 
    $$F(\rho_gW) = \rho_g' F(w)) = \rho_g'w,$$
    and it thus follows that $\on{Im}F \supset \bigoplus_{g\in G/H}\rho_g'W = V',$
    and it follows that $\on{Im}F = V'$. Since $\dim V = \dim V'$, we have that 
    $F$ is an isomorphism. This shows uniqueness.

\end{proof}

\subsection{Characters of Induced Representations}


\begin{theorem}
    Let $(V,\rho) = \on{Ind}_H^G (W,\theta)$. 
    Then, for all $g\in G$, 
    $$\chi_\rho (g) = \frac{1}{\vert H\vert} \sum_{\substack{s\in G\\s^{-1}gs\in H}} \chi_\theta(s^{-1}gs).$$
\end{theorem}

\begin{proof}
    We recall that by definition, $$V = \bigoplus_{r \in R} \rho_r(W),$$
    where $R$ is a set of representatives of $G/H$. Then, the element 
    $g\in G$ maps $g : \rho_r(W) \mapsto \rho_{gr}W$, 
    for $gr \in r'H$. That is, elements of $G$ act by permuting the cosets.
    It follows then that 
    $$\chi_\rho(g) = \on{tr}(\rho_g) = \sum_{r' = r} \on{tr}\left(\rho_g\vert_{\rho_rW}\right).$$
    Recall that $r=r'$ if and only if $r^{-1}gr \in H$. So,
    $$\chi_\rho(g) = \sum_{\substack{r\in R\\r^{-1}gr\in H}} \on{tr}\left(\rho_g\vert_{\rho_r(W)}\right).$$
    Consider the composition 
    $$\begin{tikzcd}
\rho_r(W) \arrow[r, "\rho_r^{-1}"] & W \arrow[r, "\theta_{r^{-1}gr}"] & W \arrow[r, "\rho_r"] & \rho_r(W)
\end{tikzcd}$$
    which evaluates to 
    $$\rho_r\circ \theta_{r^{-1}gr}\circ \rho_r^{-1} = \rho_g,$$
    since $\theta$ is just the restriction of $\rho$ to $H$.
    It follows then that 
    $$\on{tr}(\rho_g\vert_{\rho_r(W)}) = \on{tr}(\theta_{r^{-1}gr}) = \chi_\theta(r^{-1}gr).$$
    Then,
    $$\chi_\rho(g) = \sum_{\substack{r\in R\\r^{-1}gr\in H}} \chi_\theta(r^{-1}gr).$$
    Now, if $s\in rH$, and $r^{-1}gr \in H$, then 
    $$\chi_\theta(s^{-1}gs) =\chi_\theta\left( (r^{-1}s(s^{-1}gs)s^{-1}r)\right)= \chi_\theta(r^{-1}gr),$$
    where the last equality follows since $\chi$ is a class function.
    It follows then that this funtion is independent of our choice of 
    representative. We may thus re-write the equation for $\chi_\rho(g)$
    in the following way:
    $$\chi_\rho (g) = \frac{1}{\vert H\vert} \sum_{\substack{s\in G\\s^{-1}gs\in H}} \chi_\theta(s^{-1}gs).$$
\end{proof}

\subsection{Representations of the Dihedral Group}

Let us consider the dihedral group $D_n$, which is the group of 
rotations and reflecitons of the plane that preserves a
regular $n$-gon. The group has order $2n$. 
It is generated by a rotation element $r$ that rotates the $n$-gon
by $\frac{2\pi}{n}$, and a relfection
element $s$ of order $2$. That is, 
$$D_n = \left\langle r,s : \text{$r^n = 1$, $s^2 = 1$, $rs = sr^{-1}$} \right\rangle.$$
\subsubsection{Conjugacy classes of $D_n$}
From this, we see that $r^\ell$ is conjugate to $r^{n-\ell}$, for 
$1\leq \ell \leq \frac{n}{2}$, and $sr^k$ is conjugate to 
$sr^{k-2\ell}$. It follows then that 
$$\vert C(r^\ell)\vert = \begin{cases}
    \text{$\frac{n}{2}+1$ classes if $n$ even}\\
    \text{$\frac{n+1}{2}$ classes if $n$ odd}
\end{cases},$$
and $$\vert C(sr^k) \vert  = \begin{cases}
    \text{$2$ classes if $n$ even}\\
    \text{$1$ class if $n$ odd}
\end{cases}.$$
It follows then that the total number of conjugacy classes is given by 
$$\begin{cases}
    \text{$\frac{n}{2}+3$ classes if $n$ even}\\
    \text{$\frac{n+3}{2}$ classes if $n$ odd}
\end{cases}.$$
\subsubsection{Irreducible Representations of $D_n$}
Let $C_n = \langle r\rangle \cong \mathbb{Z}/n\mathbb{Z} \hookrightarrow D_n$ 
be the subgroup of index $2$. Since $C_n$ is abelian, it follows that 
all irreducible representations of $D_n$ have degree $\leq 2$.
\paragraph{Case One: $n$ even} We begin by finding all one-dimensional
representations.
We have: $$\chi(r^n) = 1 = \chi(r)^n,$$
$$\chi(s^2) = 1 =\chi(s)^2 \implies \chi(s) = \pm1,$$
$$\chi(rs) = \chi(sr^{-1}) \implies \chi(r)\chi(s) = \chi(s)\chi(r^{-1}).$$
Generally, given a one-dimensional representation $\rho : G \to \C^\times$, 
then $\rho(gh) = \rho(hg)$ implies that $\rho(ghg^{-1}h^{-1}) = 1$, which
implies that $\rho\vert_{[G,G]} = 1$, where $[G,G]$ is the commutator 
(or derived) subgroup. 
From this, we obtain an abelian group representation $$\rho : G/[G,G]\longrightarrow \C^\times.$$
This implies then that 
$$\chi(r) = \chi(r^{-1}) \implies \chi(r) = \pm 1.$$
From this, we obtain $4$ irreducible characters of dimension one.
$$\begin{tabular}{|c|c|c|c|}
    \hline 
    & $r^k$ & $s$ & $sr$\\
    \hline 
    $\chi_1$ & $1$ & $1$ & $-1$ \\
    \hline
    $\chi_2$ & $1$  & $-1$  & $-1$ \\
    \hline
    $\chi_3$ & $(-1)^k$ & $1$ & $-1$ \\
    \hline
    $\chi_4$ & $(-1)^k$ & $-1$ & $1$\\
    \hline
\end{tabular}$$
It follows then that there are 
$$\frac{n}{2} + 3 - 4 = \frac{n}{2} - 1,$$
irreducible representations of degree $2$.
Now, let $$\phi_m = \on{Ind}_{C_n}^{D_n}\chi_m,$$
where $$\chi_m  : C_n \longrightarrow \C^\times,\quad r\longmapsto \zeta_n^m,$$
where $$\zeta_n = \exp\left(\frac{2\pi i}{n}\right).$$
Computing directly, then 
$$\phi_m(r^k) = \frac{1}{n} \sum_{\substack{g\in D_n\\ g^{-1}r^kg\in C_n}}\chi_m(g^{-1}r^kg) = \zeta_n^{km} + \zeta_n^{-km}.$$
This is because for $g = sr^\ell$, we have:
$$g^{-1}r^k g = r^{-\ell}sr^ksr^\ell = r^{-\ell} ssr^{-k}r^\ell = r^{-k},$$
but if $g = r^\ell$, then $g^{-1}r^kg = r^k$.
\begin{exercise}
    Show that $\phi_m(sr^k) = 0$.
\end{exercise}
\begin{proof}
    If $g = sr^\ell$, then
    $$(r^{-\ell}s )sr^k sr^\ell = r^{-2\ell + k}s = sr^{2\ell-k},$$
    and if $g = r^\ell$, then
    $$r^{-\ell}sr^k r^\ell = sr^{2\ell + k}.$$
    That is, $g^{-1}sr^kg$ is never in $C_n$, and thus 
    $\phi_m(sr^k)=0$.
\end{proof}
We will finish this next time.

\section{Lecture 3, 18/08/2023}

Recall from last time that we attempted to classify the irreducible 
representations of the dihedral group $D_n$. It has a presentation
given by 
$$D_n = \left\langle s,r : \text{$r^n = 1$, $s^2 = 1$, $rs=sr^{-1}$}\right\rangle.$$
If $n$ is even, then 
$$\phi_m = \on{Ind}_{C_n}^{D^n} \chi_m,$$
where $C_n := \langle r\rangle \cong \mathbb{Z}/n\mathbb{Z}$, and
$\chi_m : C_n \to\C^\times$ is the character defined by 
$r\mapsto \zeta_n^m,$ for $0\leq m \leq n-1$. On the generators, 
$$\phi_m(r^k) = \zeta_n^{km} + \zeta_n^{-km},$$
$$\phi_m(sr^k) = 0,$$
since for any $g\in D_n$, $gsr^k g^{-1} \not\in C_n$.
Further, note that 
$$\phi_{m} = \phi_{n-m}.$$
We wish to show that the $\phi_m$'s are irreducible  
representations of $D_n$.  We do this using the inner product. 
Computing directly:
\begin{align*}
    \langle \phi_m,\phi_m\rangle &= \frac{1}{2n} \sum_{k=0}^{n-1}\phi_m(r^k) \phi_m(r^{-k})\\
                                 &= \frac{1}{2n} \sum_{k=0}^{n-1} \left(\zeta_n^{km} + \zeta_n^{-km}\right)^2\\
                                 &= \frac{1}{2n} \sum_{k=0}^{n-1} \left(\zeta_n^{2km} + 2 + \zeta_n^{-2km}\right)\\
                                 &= \begin{cases}
                                     \sum_{k=0}^{n-1}\zeta_n^{2km}  = 1 \quad &\text{if}\quad m\neq 0,\frac{n}{2}\\
                                     2 \quad &\text{if}\quad m = 0, \frac{n}{2}
                                 \end{cases}.
\end{align*}
\begin{exercise}
    Check that the last equality in the above calculation is true.
\end{exercise}

%\begin{proof}
%    Indeed, if $m = 0,n/2$, then $\zeta_n^{2km} = \zeta^{-2km} = 1$,
%    and so the sum becomes
%    $$\frac{1}{2n}\sum_{k=0}^{n-1}(1+2+1) = \frac{4n}{2n} = 2.$$
%\end{proof}

So, $\phi_1,\cdots,\phi_{\frac{n}{2}-1}$ are all irreducible, distinct
two-dimensional representations of $D_n$. 

\begin{exercise}
    Check that these characters are distinct.
\end{exercise}

\begin{proof}
    Computing directly,
    \begin{align*}
        \langle \phi_k,\phi_\ell\rangle &= \frac{1}{2n}\sum_{m=0}^{n-1}\phi_k(r^m)\phi_\ell(r^{-m})\\
                                        &= \frac{1}{2n}\sum_{m=0}^{n-1}(\zeta_n^{mk} + \zeta_n^{-mk})(\zeta_n^{m\ell} + \zeta_n^{-m\ell})\\
                                        &= \frac{1}{2n} \sum_{m=0}^{n-1} (\zeta_n^{m(k+\ell)} + \overline{\zeta_n^{m(k+\ell)}} + \zeta_n^{m(k-\ell)} + \overline{\zeta_n^{m(k-\ell)}})\\
                                        &=0,
    \end{align*}
    and thus the representations are disctinct by orthogonality.
\end{proof}

So, now we have the following character table:
$$\begin{tabular}{|c|c|c|c|}
    \hline 
    & $r^k$ & $s$ & $sr$\\
    \hline 
    $\chi_1$ & $1$ & $1$ & $-1$ \\
    \hline
    $\chi_2$ & $1$  & $-1$  & $-1$ \\
    \hline
    $\chi_3$ & $(-1)^k$ & $1$ & $-1$ \\
    \hline
    $\chi_4$ & $(-1)^k$ & $-1$ & $1$\\
    \hline
    $\text{$\phi_m$ ($1\leq m \leq \frac{n}{2}-1$)}$ & $\zeta_n^{km} + \zeta_n^{-km}$ & $0$ & $0$\\
    \hline
\end{tabular}$$

In particular, the two-dimensional representation given by $\phi_m$ is the 
representation given by 
$$\rho_m : r\longmapsto \begin{pmatrix}
    \zeta_n^m &\\
              &\zeta_n^{-m}
    \end{pmatrix}, \quad s\longmapsto \begin{pmatrix}
              &\zeta_n^{-m}\\
    \zeta_n^m &
\end{pmatrix}.$$

\begin{exercise}
    Check that $\chi_{\rho_m} = \phi_m$ (in particular, check that the 
    $\rho_m$'s are irreducible). Revisit this calculation once more 
    after we discuss more about induced representations.
\end{exercise}

\begin{proof}
    Indeed, by definition
    $$\chi_{\rho_m}(r^k) = \on{tr}\left(\rho(r)^k\right) = \zeta_n^{mk} + \zeta_n^{-mk} = \phi_m(r^k).$$
    Further, we have:
    $$\rho(sr^k) = \rho(s)\rho(r)^k = \begin{pmatrix}
        &\zeta_n^{-m}\\
        \zeta_n^m &
        \end{pmatrix} \cdot \begin{pmatrix}
        \zeta_n^{mk} &\\
                     & \zeta_n^{-mk}
        \end{pmatrix} = \begin{pmatrix}
        0 & \ast\\
        \ast & 0
    \end{pmatrix},$$
    and thus $\rho_m(sr^k) = 0 = \phi_m(sr^k).$
    The irreducibility of $\rho_m$ follows from the irreducibility of 
    $\phi_m$.
\end{proof}

\begin{exercise}
    Show that for $n$ odd, there are two irreducible one-dimensional 
    irreducible representations, and $\frac{n-1}{2}$ irreducible 
    two-dimensional representations. In particular,
    $$\begin{tabular}{|c|c|c|}
        \hline
        & $r^k$ & $s$\\
        \hline
        $\chi_1$ & $1$ & $1$\\
        \hline 
        $\chi_2$ & $1$ & $-1$\\
        \hline
    \end{tabular}$$
\end{exercise}

\begin{proof}
    We recall that $C(r^\ell)$ has $\frac{n+1}{2}$ classes
    if $n$ is odd, and $C(sr^k)$ has $1$ class for $n$ odd.
\end{proof}



\subsection{Representations of the Alternating Group $A_4$}

Recall that the alternating group is a subgroup $A_4$ of 
$\mathfrak{S}_4$ of index $2$. Specifically, it is the subgroup of 
even permutations. As such, it has order $\vert A_4\vert = 12$,
and is generated by elements 
$$x = (12)(34),\quad y=  (13)(24),\quad z = (14)(23), \quad t = (123), \quad\cdots,$$
which gives us four conjugacy classes in $A_4$, 
$$\lbrace 1\rbrace, \quad \lbrace x,y,z\rbrace,\quad \lbrace t,tx,ty,tz\rbrace,\quad \lbrace t^2,t^2x,t^2y,t^2z\rbrace.$$
Recall from our classification of $\mathfrak{S}_4$ representations, that we 
had a normal subgroup given by $$H = \lbrace 1,x,y,z\rbrace.$$
Indeed, one verifies that 
$$txt^{-1} = z, \quad tzt^{-1} = y,\quad tyt^{-1} = x.$$
Its quotient $\mathfrak{S}_4/H$ is isomorphic to $\mathfrak{S}_3$.
Further, one has another subgroup given by 
$$K = \lbrace 1,t,t^2\rbrace \cong \mathbb{Z}/3\mathbb{Z}.$$
Since $K\cap H = \lbrace 1 \rbrace$, the above properties above thus 
imply that there is a well-defined semidirect product:
$$A_4 \cong K\rtimes H,$$
which means that every element in $A_4$ can be written uniquely 
as $kh$, where $k\in K$ and $h\in H$. Closure under the group operation
follows from the following fact:
$$(k_1h_1)(k_2h_2) = k_1k_2\underbrace{(k_2^{-1}h_1k_2)}_{\in H}h_2,$$
which means that the product of two elements $k_1h_1$ and $k_2h_2$, 
produces another element of the form $kh$. In this way, $K$ \emph{acts on} 
$H$ by conjugation --- this, together with the fact that their intersection 
is trivial, allows us to define this semidirect product,\\\\
Now, for some $k\in K$, and $h \in H$, define 
a character $$\chi : K \longrightarrow \C^\times,\quad t\longmapsto \zeta_3^i,$$
and observe that
$$\chi_i(kh) = \chi_i(k),$$
which follows since $\chi_i$ is a class function.
From before, we know that 
$$\chi_i( (k_1h_1)(k_2h_2)) = \chi_i(k_1h_1) \cdot \chi_i(k_2h_2),$$
which implies that 
$$\chi_i(k_1k_2) = \chi_i(k_1)\cdot \chi_i(k_2).$$
From this, we obtain the following character table:
$$\begin{tabular}{|c|c|c|c|c|}
    \hline 
    & $1$ & $x$ & $t$ & $t^2$\\
    \hline 
    $\chi_0$ & $1$ & $1$ & $1$ & $1$\\
    \hline 
    $\chi_1$ & $1$ & $1$ & $\zeta_3$ & $\zeta_3^2$\\
    \hline 
    $\chi_2$ & $1$ & $1$ & $\chi_3^2$ & $\chi_3$\\
    \hline
\end{tabular}$$

There remains one more representation --- call it $\psi$ --- which we have 
to find. The formula $\sum n_i^2 = \vert A_4\vert$ 
($n_i$'s are dimensions of all irreps) tells us that this last 
representation has dimension $3$. 

\begin{exercise}
    Show that $$\psi = \on{Ind}_H^{A_4} \theta,$$
    where $\theta$ is the representation defined by 
    $$1 \longmapsto 1, \quad x\longmapsto 1,\quad y\longmapsto -1,\quad z\longmapsto -1.$$
\end{exercise}

Putting this all together, we have the following complete character table 
for $A_4$.

$$\begin{tabular}{|c|c|c|c|c|}
    \hline 
    & $1$ & $x$ & $t$ & $t^2$\\
    \hline 
    $\chi_0$ & $1$ & $1$ & $1$ & $1$\\
    \hline 
    $\chi_1$ & $1$ & $1$ & $\zeta_3$ & $\zeta_3^2$\\
    \hline 
    $\chi_2$ & $1$ & $1$ & $\chi_3^2$ & $\chi_3$\\
    \hline
    $\psi$ & $3$ & $-1$ & $0$ & $0$\\
    \hline
\end{tabular}$$

Let us compare this with the character table for $\mathfrak{S}_4$:


$$\begin{tabular}{|c|c|c|c|c|c|}
    \hline
    & $1$ & $t_1 = (12)(34)$ & $t_2 = (123)$ & $t_3 = (12)$ & $t_4=(1234)$\\
    \hline
    $\widetilde{\chi}_{\on{triv}}$ & $1$ & $1$ & $1$ & $1$ & $1$\\
    \hline 
    $\widetilde{\chi}_1$ & $1$ & $1$ & $1$ & $-1$ & $-1$\\
    \hline
    $\widetilde{\chi}_2$ & $2$ & $2$ & $-1$ & $0$ & $0$\\
    \hline 
    $\widetilde{\chi}_3$ & $3$ & $-1$ & $0$ & $1$ & $-1$\\
    \hline
    $\widetilde{\chi}_4$ & $3$ & $-1$ & $0$ & $-1$ & $1$\\
    \hline
\end{tabular}$$

Observe here that 
$$\widetilde{\chi}_3\vert_{A_4} = \widetilde{\chi}_4\vert_{A_4},$$
$$\widetilde{\chi_2}\vert_{A_4} = \chi_1 + \chi_2,$$
$$\chi\vert_{A_4} = \widetilde{\chi}_1\vert_{A_4} = \chi_0.$$

\subsection{Brief Remark on Compact Groups}

\begin{definition}
    A \emph{topological group} $G$ is a group with a topology such that
    the multiplication $m : G \times G \to G$ and the inversion map
    $\iota : G \to G$ mapping $g\mapsto g^{-1}$ are both continuous maps.
    The group $G$ is called \emph{compact} if it is compact with respect
    to its topology.
\end{definition}
In the case of compact groups, we replace the averaging maps by maps
of the form $$\int_G f(g) \,\mathrm{d}g,$$
where $\mathrm{d}g$ is the Haar measure on $G$, and $f$ is a 
function on $G$. 
It satisfies the following properties:
\begin{enumerate}
    \item (Right-invariance) $$\int_G f(g) \,\mathrm{d}g = \int_G f(gh) \,\mathrm{d}g,$$
        for all $h \in G$.
    \item $$\int_G \,\mathrm{d} g = 1,$$
\end{enumerate}

\begin{example}
    $G = \mathbb{S}^1$, where group elements are given by 
    $g=e^{i\theta}$, where $0\leq \theta\leq\pi$, and the Haar measure 
    is given by 
    $$\mathrm{d}g = \frac{1}{2\pi} \,\mathrm{d}\theta.$$
\end{example}

\begin{example}[Non-example]
    Let $G$ be the group that preserves the form $(t,x,y,z) \in \mathbb{R}^4$ 
    given by $x^2 + y^2 + z^2 - t^2$ --- called the \emph{Lorentz group}.
\end{example}


\begin{definition}
    A linear representation of a compact group $G$ in a finite-dimensional 
    $\C$-vector space is a group homomorphism
    $$\rho : G\longrightarrow \on{GL}(V),$$ which is 
    continuous --- that is, the map 
    $$G \times V \longrightarrow V, \quad (g,v)\longmapsto \rho_g(v),$$
    is continuous.
\end{definition}

In this case, we may define a scalar product of representations given by:
$$(\phi\vert \psi) = \int_G \phi(g)\overline{\psi(g)}\,\mathrm{d}g.$$
From this, one may similarly show that complete reducibility,
orthogonality of characters, and irreducibility criterion hold.\\\\
Further, there exists a notion of a regular representation in this case too.
As before, the regular representation $\rho_s$ acts on functions 
$f\in L^2(G)$ by $$\rho_gf(t) = f(g^{-1}t).$$
There is a theorem that says that this contains, in an appropriate sense, 
all the irreducible finite-dimensional representations of $G$.
This is something called the Peter-Weyl theorem.

\begin{example}
    Let $G = \mathbb{S}^1$. Then, since $G$ is abelian, all irreducible 
    representations have dimension one, and are given by 
    $$\chi_n(e^{i\theta}) = e^{in\theta}.$$
    The orthogonality of each of these characters follows by a direct 
    computation.
    The fact that these characters span the space of functions on $G$
    follows from the fact that periodic has a Fourier series decomposition.
\end{example}

\chapter{Week Five}

\paragraph{This week, we learned \ldots}

\section{Lecture 1, 21/08/2023}

Let $\mathbf{G}$ be a split, linear reductive group scheme over 
$\mathbb{F}_q$. Since irreducible characters characterise irreducible 
$\mathbf{G}$-representations, it follows that there is a ring isomorphism
$$K(\mathbf{Rep}_\C(\mathbf{G})) \stackrel{\simeq}{\longrightarrow} \on{ClFun}(\mathbf{G},\overline{\mathbb{Q}_\ell}).$$
We will now devote the remaining eight weeks to classifying all the finite
groups of Lie type using Deligne-Lusztig theory. Our first step will
be to develop the theory of character sheaves, and $\ell$-adic cohomology
blah blah blah blah blah

\subsection{The Group Algebra}
Just kidding. Let $G$ be a finite group, and $K$ be a commutative 
ring of characteristic $0$. Then, the \emph{group ring} $$K[G],$$
is a ring where every element $f\in K[G]$ has the form
$$f = \sum_{g\in G} a_g g.$$
Addition is given by 
$$\sum_{g\in G}a_g g + \sum_{g\in G} b_g g = \sum_{g\in G}(a_g+b_g)g,$$
and multiplication is given by 
$$\left(\sum_{g\in G} a_gg \right)\left(\sum_{g \in G} b_hh\right) = \sum_{\substack{g\in G\\h\in G}} a_gb_h gh.$$
Now, let $k$ be a field, and $V$ a $k$-vector space, and let 
$$\rho : G \longrightarrow \on{GL}(V),$$ be 
a linear representation of $G$. Then, $V$ obtains the 
structure of a \emph{left $k[G]$-module}.
For all $f = \sum_{g\in G}a_gg$, $f$ acts on an element $v\in V$ by:
$$f\cdot v = \sum_{g\in G}a_g\rho_g(v).$$
Conversely, given a left $k[G]$-module, there is a linear $G$-representation
given by 
$$\rho : G\longrightarrow \on{GL}(V),\quad g \longmapsto(v\longmapsto g\cdot v).$$
It follows then that there is a one-to-one correspondence between
linear representations of $G$ over $k$, and left $k[G]$-modules.
\begin{proposition}[Complete Reducibility of Modules]
    If $k$ is a field of characteristic $0$, then the algebra 
    $k[G]$ is semisimple --- that is, every $k[G]$-module is semisimple.
    This means that every $k[G]$-submodule admits a complementary submodule.
\end{proposition}

\begin{proof}
    To make our lives easier, we will do this in the case for $k=\C$.
    Let $\rho^i : G \to \on{GL}(W_i)$ be pairwise non-isomorphic irreducible
    representations of $G$, for $i=1,\cdots,m$. Extending, by linearity,
    we obtain a $\C[G]$-representation:
    $$\widetilde{\rho^i} : \C[G] \longrightarrow \on{End}(W_i).$$
    Choosing a basis for $W_i$, we may identify 
    $\on{End}(W_i)$ with $\on{Mat}_{n_i}(\C)$, where $n_i = \dim W_i$.
    From this, one obtains a homomorphism of algebras 
    $$\widetilde{\rho} : \C[G] \longrightarrow \prod_{i=1}^m \on{End}(W_i) = \prod_{i=1}^m \on{Mat}_{n_i}(\C), \quad g\longmapsto (\widetilde{\rho^1}(g),\cdots,\widetilde{\rho^m}(g)) = (\rho^1(g),\cdots,\rho^m(g)).$$
    That is, 
    $$\widetilde{\rho}\left(\sum_{g\in G} a_gg\right) := \sum_{g \in G}a_g\widetilde{\rho}(g).$$
    \begin{exercise}
        Check that this defines a $\C$-algebra homomorphism.
    \end{exercise}
    \begin{proof}
        It suffices to check this for $a_gg \in \C[G]$, and 
        $b_hh \in \C[G]$, and then extend by linearity.
        We have:
        \begin{align*}
            \widetilde{\rho}( (a_gg)\cdot (b_hh)) &= a_gb_h\widetilde{\rho}(gh)\\
                                                  &= a_gb_h\cdot (\rho^1(gh),\cdots,\rho^m(gh))\\
                                                  &= a_gb_h \cdot (\rho^1(g)\rho^1(h),\cdots,\rho^m(g)\rho^m(h))\\
                                                  &= a_gb_h\cdot (\rho^1(g),\cdots,\rho^m(g))\cdot (\rho^1(h),\cdots,\rho^m(h))\\
                                                  &= a_gb_h\widetilde{\rho}(g)\widetilde{\rho}(h)\\
                                                  &= \widetilde{\rho(a_gg)}\cdot\widetilde{\rho}(b_hh),
        \end{align*}
        as claimed.
    \end{proof}
    We now wish to show that $\widetilde{\rho}$ is an algebra isomorphism,
    from which our result will follow.
    Let us first show that $\widetilde{\rho}$ is injective. Suppose 
    otherwise --- then, there exists some element 
    $\sum_{g\in G}a_gg \in\C[G]$ such that $\sum_{g \in G} a_g\rho_g^i = 0,$
    for all $i =1,\cdots,m$. Recall that the 
    regular representation has a decomposition 
    $$R_G \cong \bigoplus_{i=1}^m W_i^{\oplus n_i},$$
    which implies that $\sum_{g\in G}a_g R_G(g) = 0$. Applying 
    this to the element $e_1$, we thus have that 
    $$\sum_{g \in G} a_g R_G(g)e_1 = \sum_{g \in G} a_g e_g = 0,$$
    which thus implies that $e_g = 0$ for all $g \in G$, since 
    $e_g$ is a basis of the regular representation by construction.\\\\
    Surjectivity follows since 
    $$\dim_\C \C[G] = \vert G \vert = \sum_{i=1}^m n_i^2 = \dim_\C \left(\on{Mat}_{n_i}(\C)\right).$$
    It follows thus that $\widetilde{\rho}$ is an isomorphism of 
    algebras.
\end{proof}

\begin{proof}[Alternative Proof of Surjectivity for $\widetilde{\rho}$]
    We may also show directly that $\widetilde{\rho}$ is surjective.
    Proceed by contradiction, and suppose otherwise. Then, 
    $\on{Im}\widetilde{\rho}$ is a strict subset of 
    $\prod_{i=1}^m\on{Mat}_{n_i}(\C)$.
    That is, there exists a non-zero linear form 
    $f$ on $\prod_{i=1}^m \on{Mat}_{n_i}(\C)$ such that 
    $f\vert_{\on{Im}(\widetilde{\rho})} = 0$. This means that 
    $$\sum_{i_k,j_k} a_{i_k,j_k}^k (\rho^k_g)_{i_k,j_k} = 0,$$
    for $1\leq i_k,j_k\leq n_k$, for $1\leq k \leq m$. 
    The above notation basically just means that we are taking 
    entries of elements of $\prod_{i=1}^m\on{Mat}_{n_i}(\C)$.
    This implies that 
    $$\sum_{i_k,j_k} a^k_{i_k,j_k} \rho^k_{i_k,j_k} = 0,$$
    as a function $G \to \C$, where in particular
    $$\rho^k_{i_k,j_k} : G \longrightarrow \C, \quad g \longmapsto (\rho_g^k)_{i_k,j_k}.$$
    We now wish to show that 
    $$(\rho_{i_k,j_k}^k) : G \to \C,\quad 1\leq k \leq m, \quad 1\leq i_k,j_k \leq n_k,$$ form a basis of $\C$-valued functions on $G$, since this would imply then that
     $a^k_{i_k,j_k} = 0$. Recall the inner product given by 
     $$\langle \varphi,\psi\rangle = \frac{1}{\vert G\vert} \sum_{g \in G} \varphi(t) \psi(t^{-1}).$$
     Then, by orthogonality of irreducible characters, it would then follow
     that 
     $$\left\langle \rho_{ik}^a, \rho_{j\ell}^b\right\rangle = \frac{\delta_{k\ell}\delta_{ij}}{n_a}.$$
     The required result thus follows.
\end{proof}


\section{Lecture 2, 24/08/2023}

\begin{proposition}[Fourier Inversion Formula]
    Let $(u_1,\cdots,u_m) \in \prod_{i=1}^k\on{End}(W_i)$, and let 
    $$u = \sum_{g \in G} u(g)g \in \C[G],$$ be such that 
    $\widetilde{\rho}(u) = (u_1,\cdots,u_k)$, where $\widetilde{\rho}$ 
    is the $\C$-algebra isomorphism 
    $$\widetilde{\rho} : \C[G] \longrightarrow \prod_{i=1}^k \on{End}(W_i).$$
    Then,
    $$u(g) = \frac{1}{\vert G \vert} \sum_{i=1}^m n_i \on{tr}_{W_i}\left(\rho^i(g^{-1})u_i\right).$$
\end{proposition}

\begin{proof}
    By definition, we know that 
    $$\widetilde{\rho}(u) = \sum_{g \in G} u(g) \widetilde{\rho}(g) = \left(\sum_{g\in G} u(g) \rho^1(g), \cdots, \sum_{g \in G}u(g)\rho^k(g)\right) = (u_1,\cdots,u_k).$$
    It follows then that 
    $$u_i = \sum_{g \in G} u(g) \rho^i(g).$$
    Plugging this into the formula for the 
    Fourier inversion formula, we see that 
    \begin{align*}
        &\,\,\,\,\,\,\,\,\frac{1}{\vert G\vert} \sum_{j=1}^k n_j \on{tr}_{W_j} (\rho^j(g^{-1}u_j)\\
        &= \frac{1}{\vert G\vert} \sum_{j=1}^k n_j \on{tr}_{W_j} (\rho^j(g^{-1}))\sum_{h\in G} u(h)\rho^j(h)\\
        &= \frac{1}{\vert G\vert} \sum_{j=1}^k n_j\sum_{h\in G} \on{tr}_{W_j}(\rho^j(g^{-1})\cdot \rho^j(h))\\
        &= \frac{1}{\vert G\vert} \sum_{h \in G} u(h) \sum_{j=1}^k n_j \on{tr}_{W_j}(\rho^j(g^{-1})\cdot\rho^j(h))\\
        &= \frac{1}{\vert G\vert} \sum_{h \in G} u(h) \sum_{j=1}^k \chi_j(1)\chi_j(g^{-1}h)\\
        &= \frac{1}{\vert G\vert}\sum_{h\in G} u(h) \delta_{gh}\vert G\vert\\
        &=u(g),
    \end{align*}
    where the second-last equality follows by Corollary \ref{cor_cor8}.
\end{proof}

\subsection{The Center of $\C[G]$}

By definition, $$\on{Cent}(\C[G]) := \lbrace x \in \C[G] : \text{$xy=yx$ for all $y\in \C[G]$}\rbrace = \lbrace x \in \C[G] : \text{$xg = gx$ for all $g\in G$}\rbrace,$$
where the second equality follows by linearly extending.
Let $C\subset G$ be a conjugacy class. Define 
$Z_C := \sum_{g \in C}g \in \C[G]$.
Then, for any $h\in G$, 
$$hZ_Ch^{-1} = h\left(\sum_{g \in C} g\right)h^{-1} = \sum_{g\in C} hgh^{-1} = \sum_{h\in C} h = Z_C,$$
and so $Z_C \in \on{Cent}(\C[G])$.

\begin{exercise}
    $\lbrace Z_C : \text{$C$ a conjugacy class of $G$}\rbrace$ 
    is a basis for $\on{Cent}(\C[G])$.
    It follows then that 
    $$\dim_\C(\on{Cent}(\C[G])) = \text{no. of conjugacy classes of $G$}.$$
\end{exercise}

\begin{proof}
    Let $C_1,\cdots,C_k$ be a complete list of conjugacy classes for $G$.
    Then, it is clear that the $Z_{C_i}$'s are linearly independent,
    since conjugacy classes give a partition of $G$.
    For some central element $x\in \C[G]$, we have that 
    $x = g^{-1}xg$, for any $g\in \C[G]$. Thus, writing 
    $$x = \sum_{h\in G} a_h x = \sum_{h\in G} a_h g^{-1}xg,$$
    which implies that $a_h = a_{g^{-1}hg}$ for all $g$ --- 
    that is, $a_h$ is constant on conjugacy classes.
    It follows then that we may write any central element as 
    a linear combination of the form $\sum_{i=1}^k \lambda_iZ_{C_i},$
    and it follows that the $Z_{C_i}$'s spans $\on{Cent}(\C[G])$.
\end{proof}

\begin{proposition}\label{prop_prop6}
    The homomorphism $\widetilde{\rho}_i : \C[G] \to \on{End}(W_i)$
    gives rise to an algebra homomorphism 
    $$\omega_i = \widetilde{\rho}_i\vert_{\on{Cent}\C[G]} : \on{Cent}\C[G]\longrightarrow \on{Cent}(\on{End}(W_i)) \cong \C.$$
    In particular, $\on{Cent}(\on{End}(W_i))$ is isomorphic to the space 
    of scalar matrices --- that is, scalar multiples of $\on{id}_{W_i}$.
    Further, the family of $\omega_i$'s defines an isomorphism 
    $$(\omega_i)_{1\leq i \leq k} : \on{Cent}\C[G] \longrightarrow \C^k \subset \prod_{i=1}^k \on{End}(W_i).$$
    Moreover, if $u = \sum_{g \in G} u(g)g \in \on{Cent}\C[G]$, then
    $$\omega_i(u) = \frac{1}{n_i}\sum_{g\in G}u(g)\chi_i(g)$$
\end{proposition}

\begin{proof}
    Let $z \in \on{Cent}\C[G]$. Then, from this we obtain a morphism of 
    $G$-representations given by 
    $$\widetilde{\rho}_i(z) : W_i \longrightarrow W_i.$$
    Since $z = \sum_{g \in G} a_gg$, we may extend linearly and write 
    $$\widetilde{\rho}_i(z) = \sum_{g \in G}a_g \rho_i(g).$$
    We wish to check that $$\widetilde{\rho}_i(z) = h(\rho_i(h) w) = \rho_i(h)\widetilde{\rho}_i(h)w,$$
    for all $h \in H$. But since $z$ is in the center, we thus have that 
    $$\widetilde{\rho_i}(zh) = \widetilde{\rho}_i(hz),$$
    and by Schur's lemma (Proposition \ref{prop_schurs_lemma}), 
    we deduce that $\widetilde{\rho}_i(z) = \lambda_z\on{id}_{W_i}$.
    If $u\in \on{Cent}\C[G]$, then 
    \begin{align*}
        \omega_i(u) &= \frac{1}{n_i} \on{tr}_{W_i}(\widetilde{\rho}_i(u))\\ 
                    &= \frac{1}{n_i} \on{tr}_{W_i}\left(\sum_{g\in G}u(g)\rho_i(g)\right)= \frac{1}{n_i}\sum_{g \in G}u(g)\chi_i(g)
    \end{align*}
\end{proof}

Next, our goal is to show that the dimensions of irreducible representations
divides the order of $G$. More generally, they divide $\frac{\vert G\vert}{\vert Z(G)\vert}$, where $Z(G)$ is the center of $G$. The proof of this will use 
the theory of algebraic integers, which we define below.

\subsection{Algebraic Integers}

Let $R$ be a commutative ring of characteristic $0$. Let $x\in R$. We 
say that $x$ is \emph{integral} over $\mathbb{Z}$ if there exists an integer
$n \geq 1$ and $a_1,\cdots, a_n\in\mathbb{Z}$ such that $x$ is the root 
of a monic polynomial with $\mathbb{Z}$-coefficients.
$$x^n + a_1x^{n-1} + \cdots + a_n = 0.$$

\begin{definition}
    A complex number which is integral over $\mathbb{Z}$ is called an 
    \emph{algebraic integer}. 
\end{definition}

\begin{example}
    The roots of integers are algebraic integers.
\end{example}

\begin{exercise}
    If $x \in \mathbb{Q}$, and $x$ is an algebraic integers, then 
    $x\in\mathbb{Z}$.
\end{exercise}

\begin{proof}
    Any polynomial $x^n + a_1x^{n-1} + \cdots + a_n = 0$
    over $\mathbb{Q}$ can be made into a polynomial over $\mathbb{Z}$ 
    by multiplying out its denominators, thus making it a 
    polynomial over $\mathbb{Z}$.
\end{proof}

\section{Lecture 3, 25/08/2023}

\begin{proposition}
    Let $R$ be a commutative ring (of characteristic $0$). Given some 
    $x \in R$, the following statements are equivalent:
    \begin{itemize}
        \item[(i)] $x$ is integral over $\mathbb{Z}$,
        \item[(ii)] The subring $\mathbb{Z}[x]$ of $R$ generated by $x$ 
            is finitely-generated as a $\mathbb{Z}$-module,
        \item[(iii)] there exists a finitely generated $\mathbb{Z}$-submodule 
            of $R$ which contains $\mathbb{Z}[x]$.
    \end{itemize}
\end{proposition}

\begin{proof}
    (ii) $\Longleftrightarrow$ (iii). $\mathbb{Z}$ is a Noetherian ring, since
    it is a principal ideal domain.
    It follows then that every submodule of a finitely generated 
    $\mathbb{Z}$-module is finitely generated.\\\\
    (i) $\implies$ (ii). If $x$ is integral, then it always has the form
    $x^n + a_{n-1}x^{n-1} + a_0 = 0$
    for some $a_i \in \mathbb{Z}$. And so $\mathbb{Z}[x] = \mathbb{Z}\langle 1,x,\cdots,x^{n-1}\rangle$.\\\\
    (ii) $\implies$ (i) Suppose $\mathbb{Z}[x]$ is finitely generated. 
    Let $R_n = \mathbb{Z}\langle 1,x,\cdots,x^{n-1}\rangle$. 
    We thus have a chain of ideals 
    $$\cdots\subseteq R_n \subseteq R_{n+1}\subseteq \cdots,$$
    for which there exists some $k$ such that 
    $R_k \cong \mathbb{Z}[x]$.
\end{proof}

\begin{corollary}\label{cor_cor10}
    If $R$ is a finitely-generated $\mathbb{Z}$-module, then every element of 
    $R$ is integral over $\mathbb{Z}$.
\end{corollary}

\begin{corollary}
    The elements of a commutative ring $R$ of characteristic zero which 
    are integral over $\mathbb{Z}$ form a subring of $R$.
\end{corollary}

\begin{proof}
    Let $x,y\in R$ integral elements over $\mathbb{Z}$. Then, it follows
    that $\mathbb{Z}[x]\otimes \mathbb{Z}[y]$ is finitely generated over 
    $\mathbb{Z}$. It follows then that $\mathbb{Z}[x,y]$ is finitely 
    generated over $\mathbb{Z}$.
\end{proof}

\subsection{Integrality of Characters}

\begin{proposition}[Integrality of Characters]
    Let $\rho$ be a representation of $G$, and $\chi_\rho$ its character.
    Then, $\chi_\rho(g)$ is an algebraic integer over $\C$ for any $g\in G$.
\end{proposition}

\begin{proof}
    By definition, $\chi_\rho(g) = \on{tr}(\rho(g))$, which are
    the sum of eigenvalues of $\rho(g)$, which are roots of unity. 
    Since roots of unity are algebraic integers over $\C$, the result follows.
\end{proof}

\begin{proposition}
    Let $u =\sum_{g\in G} u(s)s \in \on{Cent}\C[G]$ such that 
    the coefficients $u(s)$ are algebraic integers. Then, 
    $u$ is integral over $\mathbb{Z}$.
\end{proposition}

\begin{proof}
    Since $\on{Cent}\C[G]$ is a commutative ring, it has the structure of 
    a $\mathbb{Z}$-module. Let $e_i = \sum_{g\in C_i}g,$ where $C_i$ is a 
    conjugacy classs of $G$, and $i=1,\cdots,k$. 
    These form a basis for $\on{Cent}\C[G]$ as a $\C$-vector space.
    Thus, we may write $u$ as 
    $$u = \sum_{i=1}^k u_i e_i,$$
    where $u_i\in\C$ is an algebraic integer.  
    By Corollary \ref{cor_cor10}, it suffices to show that the $e_i$'s are 
    integral over $\mathbb{Z}$. 
    Since $e_i\cdot e_j$ is an integral combination of the basis elements
    $\lbrace e_j\rbrace_{j=1}^k$.
    It follows then that 
    $$\mathbb{Z} e_1\oplus \cdots \oplus \mathbb{Z}e_k,$$
    is a subring of $\on{Cent}\C[G]$, and is finitely generated over 
    $\mathbb{Z}$. It follows thus by Corollary \ref{cor_cor10}
    that $e_i$ is integral for each $i=1,\cdots,k$.

\end{proof}

\begin{corollary}\label{cor_cor12}
    Let $(\rho,W)$ be an irreducible representation of $G$ of dimension $n$,
    and character $\chi$. If $u = \sum_{s \in G}u(s)s \in \on{Cent}\C[G]$
    such that $u(s)$ are algebraic integers, then 
    $$\frac{1}{n} \sum_{s \in G} u(s) \chi(s),$$
    is an algebraic integer.
\end{corollary}

\begin{proof}
    Recall that there is an algebra isomorphism
    $$\tilde{\rho} : \C[G] \longrightarrow \on{End}(W),$$
    which induces an algebra isomorphism (as in Proposition \ref{prop_prop6})
    $$\omega:\on{Cent}\C[G] \longrightarrow \C,\quad \sum_{g\in G} a_gg \longmapsto \frac{1}{n}\sum_{g\in G}a_g\chi(g),$$
    where $\C$ is identified with the scalar matrix subring in
    $\on{End}(W)$. 
    Since $\omega$ is a homomorphism of algebras, the fact that 
    $u = \sum_{s\in G}u(s)s$ is integral over $\mathbb{Z}$ (by Corollary 
    \ref{cor_cor10}), implies the fact that 
    $\omega(u) = \frac{1}{n}\sum_{s\in G} u(s)\chi(s)$ is integral over 
    $\mathbb{Z}$.

\end{proof}

\begin{corollary}\label{cor_cor13}
    The dimension of irreducible representations of $G$ divides $\vert G\vert$.
\end{corollary}

\begin{proof}
    We want $u = \sum_{s\in G}u(s) s\in\on{Cent}\C[G]$ such that 
    $u(s)$ are algebraic integers, and 
    $\frac{1}{n}\sum_{s \in G}u(s)\chi(s) = \frac{\vert G\vert}{n}$.
    Let us try $u(s) = \chi(s^{-1})$, which is an algebraic integer.
    Then,$$\frac{1}{n}\sum_{s \in G} \chi(s^{-1})\chi(s) = (\chi\vert\chi) = \frac{\vert G\vert}{n},$$
    but also $(\chi\vert\chi) = 1$ by the irreducibility of $\chi$.
    By Corollary \ref{cor_cor12}, it follows that 
    $\frac{\vert G\vert}{n}$, is an algebraic integer. Further, 
    $\frac{\vert G\vert}{n} \in \mathbb{Q}$, which implies that 
    $\frac{\vert G\vert}{n} \in \mathbb{Z}$.

\end{proof}

\begin{exercise}
    Let $a = \frac{\lambda_1+\cdots+\lambda_n}{n}$ and algebraic 
    integer such that the $\lambda_i$'s are integer. Then,
    $a = 0$, or $\lambda_i = \lambda_j$.
\end{exercise}

\begin{proof}

\end{proof}

\chapter{Week Six}

\paragraph{This week, we learned \ldots}

\section{Lecture 1, 28/08/2023}

\begin{proposition}
    Let $Z(G)$ be the centre of $G$. Then, the dimension of an 
    irreducible representation of $G$ divides $[G:Z(G)] = \frac{\vert G\vert}{\vert Z(G)\vert}.$
\end{proposition}

\begin{proof}
    Let $\rho : G \to \on{GL}(W)$ be an irreducible representation of 
    $G$ of degree $n$. Let $s \in Z(G)$. Then, by Schur's Lemma
    $\rho(s) = \lambda(s) \on{id}_W\in \C^\times$, which gives 
    rise to a group homomorphism $\lambda : Z(G) \to \C^\times$, defined by
    $s \mapsto \lambda(z)$.\\\\
    Let $m\geq 0$ be an integer, and consider the representation
    $$\rho^m : G^m := \prod_{i=1}^m G \longrightarrow \on{GL}(V^{\otimes m}),$$
    which is irreducible by the irreduciblity of $\rho$.
    Consider the subgroup $Z(G)^m \leq G^m$, and $(s_1,\cdots,s_m) \in Z(G)^m$.
    Then, $$\rho^m(s_1,\cdots,s_m) = \lambda(s_1)\cdots \lambda(s_m)\on{id}_{W^{\otimes m}} = \lambda(s_1\cdots s_m) \on{id}_{W^{\otimes m}}.$$
    Let $H \leq Z^m$ be a subgroup defined by 
    $$H = \lbrace (s_1,\cdots,s_m)\in Z(G)^m : s_1\cdots s_m = 1\rbrace,$$
    so that $\rho^m$ restricts to the identity on $W^{\otimes m}$ --- that is,
    $\rho^m\vert_H = \on{id}_{W^{\otimes m}}$. Equivalently, one says that 
    $H$ acts trivially on $W^{\otimes m}$. Further, $H$ is normal in 
    $G^m$ since it is in the centre of $G$, and thus the 
    quotient subgroup $G^m/H$ is well-defined.
    It follows then that the representation $\rho^m$ factors through 
    the quotient $G^m /H$:
    $$\rho^m  :G^m \longrightarrow G^m/H \stackrel{\overline{\rho}}{\longrightarrow} \on{GL}(W^{\otimes m}).$$
    This means that $W^{\otimes m}$ is irreducible as a $G^m/H$-representation,
    since by construction there exists no non-trivial stable under 
    $G^m/H$, otherwise it would also be stable under $G$.
    This implies that 
    $\dim_\C W^{\otimes m} = n^m$, which divides $\frac{\vert G^m\vert}{\vert H\vert} = \frac{\vert G\vert^m}{\vert Z(G)\vert^{-1}}$, 
    and it follows thus that 
    $$\left(\frac{\vert G\vert}{\vert Z(G)\vert \cdot n}\right)^m \in \frac{1}{\vert Z(G)\vert}\mathbb{Z},$$ for all $m\geq 0$.
    Thus, $\mathbb{Z}\left[\frac{\vert G\vert}{\vert Z(G)\vert n}\right]$ 
    is a finitely-generated $\mathbb{Z}$-module --- specifically, 
    it is generated by $\frac{1}{\vert Z(G)\vert}$. It follows then that 
    $\frac{\vert G\vert}{\vert Z(G)\vert \cdot n}$ is integral over $\mathbb{Z}$,
    and since it is rational, it follows that $\frac{\vert G\vert}{\vert Z(G)\vert\cdot n} \in \mathbb{Z}$, and it follows therefore that $n$ divides 
    $\frac{\vert G\vert}{\vert Z(G)\vert}$, as claimed.

\end{proof}

\subsection{Induced Representations, Re-visited}

Let $H\leq G$ be a subgroup, and $R$ a system of left coset representatives 
of the coset $G/H$. We now interchange between the language of 
$G$-representations and $\C[G]$-modules. Let $V$ be a $\C[G]$-module,
which also inherits the structure of a $\C[H]$-module by 
restriction. Let $W$ be a $\C[H]$-submodule of $V$.\\\\
Recall that we said that $V$ is \emph{induced} by $W$ if 
$$V = \on{Ind}_H^G W := \bigoplus_{s \in R} s\cdot W.$$
However, we want a more canonical construction of the induced module, which we
will outline now.\\\\
Recall that given a $k$-algebra $A$, and $V$ a right $A$-module,  
and $W$ a left $A$-module, the \emph{tensor product over $A$} --- denoted by $\otimes_A$ --- is given by 
$$V\otimes_A W := V\otimes W / \langle (v\cdot a)\otimes w - v\otimes (a\cdot w) : v\in V, w\in W, a\in A\rangle.$$
Now, consider $W' = \C[G] \otimes_{\C[H]} W$. Then, $W'$ is a $\C[G]$-module:
with action given by 
$$\C[G] \times W' \longrightarrow W', \quad (g,x\otimes w)\longmapsto (g\cdot x) \otimes w.$$

\begin{exercise}
    Check that this is well-defined.
\end{exercise}

\begin{proof}
    Let $g,g' \in \C[G]$ be such that 
    $gx\otimes w = g'x\otimes w$.
    Then, $$ (gx\otimes w)\cdot (g^{-1}x\otimes w) = (g'x\otimes w) (g^{-1}x\otimes w),$$
    gives $x\otimes w = g'g^{-1}x\otimes w$, which implies that 
    $g'g^{-1} = 1$, and thus $g = g'$.
\end{proof}

From this, we now have a map 
$$\C[G]\otimes_{\C[H]} W \longrightarrow V,$$
defined by $x\otimes w \mapsto x\cdot w,$ and extending linearly.

\begin{proposition}
    $V$ is induced by $W$ if and only there is an isomorphism
    $$\varphi :\C[G]\otimes_{\C[H]}W \stackrel{\simeq}{\longrightarrow} V.$$
\end{proposition}

\begin{proof}
    Observe that 
    $\dim_\C \C[G]\otimes_{\C[H]} W = \frac{\vert G\vert}{\vert H\vert}$.
    A basis of $\C[G]\otimes_{\C[H]}W$ is given by elements of the form 
    $r\otimes w_i$, where $r\in R$, and $\lbrace w_i\rbrace$ is a basis of 
    $W$. Then, the map $\varphi$ is defined by $r\otimes w \mapsto r\cdot w$.
    \begin{exercise}
        Convince yourself that this proves the statement.
    \end{exercise}
    \begin{proof}
        The proof works because $\varphi$ maps basis elements to
        basis elements, hence is an isomorphism.
    \end{proof}
\end{proof}

\begin{exercise}
    Show that $$\on{Ind}_H^G W \cong \on{Hom}_H(\C[G], W) = \lbrace f: G \to W :\text{$f(gh) = h\cdot f(g)$, for all $h\in H$, $g\in G$}\rbrace.$$
    $\on{Hom}_H(\C[G],W)$ obtains the structure of a $\C[G]$-module 
    by the action 
    $$g' \cdot f(g) = f(gg'),$$
    for all $g,g'\in G$.
\end{exercise}

\begin{proof}

\end{proof}

\paragraph{Properties of Induction}
\begin{itemize}
    \item[(i)] If $V = \on{Ind}_H^GW$, and $E$ is a $\C[G]$-module, then 
        there is a canonical isomorphism
        $$\on{Hom}_H(W,\on{Res}_H^GE) \cong \on{Hom}_G(\on{Ind}_H^GW,E).$$
        That is, $\on{Res}_H^G$, and $\on{Ind}_H^G$ are adjoint 
        to each other as functors in the category of $G$-representations.
        \begin{exercise}
            Produce an isomorphism 
            $$\on{Hom}_{\C[H]}(W,E) \cong \on{Hom}_{\C[G]}(\C[G]\otimes_{\C[H]}W, E),$$
            which is equivalent to proving the above property.
        \end{exercise}
        \begin{proof}
            Define a map 
            $$\Phi : \on{Hom}_{\C[H]}(W,E) \longrightarrow \on{Hom}_{\C[G]}(\C[G]\otimes_{\C[H]}W,E),$$
            by $$f \longmapsto (g\otimes w \longmapsto g\cdot f(w)).$$
            Then, we claim that the map in the other direction given by 
            $$\Phi^{-1} : \varphi \longmapsto (w\longmapsto \varphi(1\otimes w)),$$
            is a suitable choice of inverse for $\Phi$.
            Indeed,
            $$(\Phi \circ \Phi^{-1})(\varphi)(w) = \Phi(\varphi(1\otimes w)) = 1\cdot \varphi(w) = \varphi(w),$$
            and 
            $$(\Phi^{-1}\circ \Phi)(f)(g\otimes w) = \Phi^{-1}(g\cdot f(w)) = g\cdot f(1\otimes w)  g\cdot f(w).$$
        \end{proof}
    \item[(ii)] There is an isomorphism of $\C[G]$-modules:
        $$V\otimes_\C \on{Ind}_H^G W \cong \on{Ind}_H^G (\on{Res}_H^GV\otimes_\C W).$$
        \begin{exercise}
            Show that there is a $\C[G]$-module isomorphism
            $$V \otimes_\C (\C[G] \otimes_{\C[H]} W) \cong \C[G]\otimes_{\C[H]} (V\otimes W),$$
            and deduce the above property from this.
        \end{exercise}
        \begin{proof}
            It is clear by construction that $\C[G]$
            is a $(\C[G],\C[H])$-bimodule.
            Let us equip $W$ with a right $H$-action by:
            $$w\cdot h = h^{-1}w.$$
            Thus, $W$ is a $\C[H]$-bimodule.
            Similarly, one makes $V$ a $(\C[H],\C[G])$=bimodule.
            Then, by associativity of tensor products, we thus 
            have an isomorphism:
            $$V \otimes_\C (\C[G]\otimes_{\C[H]}W) \cong \C[G]\otimes_{\C[H]}(V\otimes_\C W).$$
        \end{proof}
    \item[(iii)] (Transitivity of Induction) 
        Let $H\leq K \leq G$. Then, there is an isomorphism of 
        $G$-representations:
        $$\on{Ind}_K^G(\on{Ind}_H^GW) \cong \on{Ind}_H^GW.$$
        \begin{exercise}
            Show that $$\C[G] \otimes_{\C[K]}(\C[K] \otimes_{\C[H]} W) \cong \C[G] \otimes_{\C[H]} W.$$
        \end{exercise}
        \begin{proof}
            Since $\C[G]$ is a $(\C[G],\C[K])$-bimodule, we have:
            $$\C[G]\otimes_{\C[K]}(\C[K] \otimes_{\C[H]} W) = (\C[G] \otimes_{\C[K]} \C[K]) \otimes_{\C[H]} W = \C[G]\otimes_{\C[H]} W.$$
        \end{proof}
\end{itemize}

\subsection{Frobenius Reciprocity}

Let $H\leq G$ be a subgroup, and let $f$ be a class function on $H$.
Let $f ' : G \to \C$ be defined by 
$$f'(g) = \frac{1}{\vert H\vert} \sum_{\substack{g \in G\\ s^{-1}gs\in H}} f(s^{-1}gs).$$
We say that $f'$ is \emph{induced} by $f$ and we write 
$$f' = \on{Ind}_H^G (f).$$

\begin{proposition}
    \leavevmode
    \begin{itemize}
        \item[(i)] $\on{Ind}_H^Gf$ is a class function,
        \item[(ii)] If $f = \chi_W$, a character of a $H$-representation $W$,
            then $\on{Ind}_H^G f = \chi_{\on{Ind}_H^GW}$.
    \end{itemize}
\end{proposition}

\begin{proof}
    \leavevmode
    \begin{itemize}
        \item[(i)] Every class function is a linear combination of characters.
            So, (ii) $\implies$ (i).
        \item[(ii)] Already proved.
    \end{itemize}
\end{proof}


Let $V_1$, $V_2$ be two $\C[G]$-modules. Set $$\dim_\C \on{Hom}_G(V_1,V_2) =: \langle V_1,V_2\rangle_G.$$

\begin{lemma}
    $$\langle V_1,V_2\rangle_G = \langle \chi_{V_1},\chi_{V_2}\rangle_G,$$
    where $$\langle \varphi,\psi\rangle_G := \frac{1}{\vert G\vert} \sum_{g\in G} \varphi(g^{-1})\psi(g).$$
\end{lemma}

\begin{proof}
    Each $\C[G]$-module admits a decomposition into irreducible 
    $\C[G]$-modules by:
    $$V_1 \cong \bigoplus_{i=1}^k W_i^{\oplus a_i}, \quad V_2 =\bigoplus_{i=1}^k W_i^{\oplus b_i},$$
    where $W_i$ are pairwise non-isomorphic irreducible representations.
    Then, it follows by Schur's Lemma that 
    $$\langle V_1,V_2\rangle_G = \sum_{i=1}^k a_ib_i.$$
    But we also have that 
    $$\langle \chi_{V_1},\chi_{V_2}\rangle_G = \left\langle \sum_i a_i\chi_i,\sum_ib_i\chi_i\right\rangle = \sum_ia_ib_i,$$
    and the result follows.
\end{proof}

\begin{theorem}[Frobenius Reciprocity]\label{thm_frob_reciprocity}
    If $\psi$ is a class function on $H$, and $\varphi$ is a 
    class function on $G$, then 
    $$\left\langle \psi, \on{Res}_H^G\varphi\right\rangle_H = \left\langle \on{Ind}_H^G \psi,\varphi\right\rangle.$$
\end{theorem}

\begin{proof}
    As before.
\end{proof}

\begin{corollary}
    If $W$ is an irrep of $H$, and $E$ is an irrep of $G$,
    then the number of times that $W$ occurs in 
    $\on{Res}_H^G E$ is equal to the number of times that 
    occurs in $\on{Ind}_H^GW$.
\end{corollary}

\section{Lecture 2, 31/08/2023}

\subsection{Restriction to Subgroups}

Let $H\leq G$ and $K\leq G$ subgroups. 
If $\rho : H \to \on{GL}(W)$ is a representation of $G$,
and $V = \on{Ind}_H^G W$, then how does one define $\on{Res}^G_KV$ 
from this?\\\\
Consider double cosets $K\backslash G / H$, and choose a set of representatives
such that $G = \sqcup_{s \in S} KsH$, where 
$$KsH = \lbrace ksh : k\in K, h\in H\rbrace.$$
For $s\in S$, define a subgroup given by $$Hs := sHs^{-1}\cap K \leq K.$$
From this, consider the representation given by 
$$\rho^s : Hs \longrightarrow \on{GL}(W), \quad x\longmapsto \rho(s^{-1}xs).$$
Let $W_s := (W,\rho^s)$ be the corresponding representation of $Hs$.

\begin{proposition}\label{prop_res_of_ind}
    $$\on{Res}_K^G\on{Ind}_H^GW = \bigoplus_{s\in S \cong K\backslash G/H} \on{Ind}_{Hs}^K(W_s).$$
    In particular,
    $$\frac{\vert G\vert}{\vert H\vert} \dim_\C W= \sum_{s \in S} \frac{\vert K\vert}{\vert Hs\vert} \dim_\C W.$$
\end{proposition}

\begin{proof}
    By definition,
    $$V = \on{Ind}_H^GW = \bigoplus_{r \in G/H} \rho_r(W).$$
    Let $s \in S$, and let $$V(s) := \sum_{x \in KsH} \rho_x(W),$$
    which is a $K$-stable $K$-representation under the left action of $K$.
    We now wish show that there is an isomorphism of $K$-representations:
    $$V(s) \cong \on{Ind}_{Hs}^K W_s.$$
    Let $x_1 \in KsH$, and $x_2 \in KsH$. Then, $\rho_{x_1}(W) = \rho_{x_2}(W)$
    if and only if $x_2^{-1}x_1\in H$.
    Suppose that $x_1 = k_1sh_1$, and $x_2 = k_2sh_2$. 
    Then, $x_2^{-1}x_1 = h_2^{-1} s^{-1}k_2^{-1}k_1sh_1,$ is an element of $H$
    if and only if $k_2^{-1}k_1 \in sHs^{-1}\cap K = Hs$.
    This implies that 
    $$V(s) = \bigoplus_{z \in K\backslash Hs} z(sW) \cong \on{Ind}_{Hs}^KW_s,$$
    where the isomorphism follows since $sW$ is $Hs$-stable, and the 
    definition of an induced representation.
    Since $V = \bigoplus_{s \in S} V(s)$, it now remains to show that 
    $sW \cong Ws$ as a $Hs$-representation. 
    Indeed, there is a map of $Hs$-representations
    $$Ws \longrightarrow sW,\quad w \longmapsto sw = \rho^s(w).$$
    Let $h_s \in sHs^{-1}\cap K$, so that 
    $h_s = shs^{-1}$. Then, $h_s\cdot w = h\cdot w$, by construction of 
    $Ws$. This gets mapped to $sh\cdot w = h_s s \cdot w$ under the above
    map, and it thus follows that the map is a homomorphism of $Hs$-modules.
    Bijectivity follows since $w\mapsto s^{-1}w$ is a suitable inverse of the 
    map. 

\end{proof}

\subsection{Mackey's Irreducibility Criterion}
Consider $K = H$, and $Hs = sHs^{-1}\cap H$, where $s\in G$.
Then, $Hs \leq H$. Given a representation $\rho : H \to \on{GL}(W)$, 
we have two representations of $Hs$, given by 
$$\rho^s: Hs \longrightarrow \on{GL}(W),\quad x\longmapsto \rho(s^{-1}xs),$$
and 
$$\on{Res}_{H_s}^H\rho : Hs \longrightarrow \on{GL}(W),\quad x\longmapsto \rho(x).$$

\begin{proposition}[Mackey's Irreducibility Criterion]\label{thm_mackeys_thm}
    The induced representation $V = \on{Ind}_H^GW$ is irreducible if and only 
    if the following conditions are satisfied:
    \begin{itemize}
        \item[(i)] $W$ is an irreducible $H$-representation
        \item[(ii)] for all $s \in G -H$ (this is the set-theoretic
            minus), $\rho^s$, and $\on{Res}_{Hs}^H\rho$ are \emph{disjoint}
            as $Hs$-representations (two representations $V_1$, $V_2$ of 
            $G$ are \emph{disjoint} if they have no irreducible components 
            in common --- i.e. $\langle V_1,V_2\rangle_G = 0$)
    \end{itemize}
\end{proposition}

\begin{proof}
    $V$ is irreducible if and only if $\langle V,V\rangle_G = 1$.
    By Frobenius reciprocity, and applying Proposition \ref{prop_res_of_ind}:
    \begin{align*}
        \left\langle \on{Ind}_H^GW,V\right\rangle_G &= \left\langle W, \on{Res}_H^GV\right\rangle_H\\ &= \left\langle W, \bigoplus_{s \in H\backslash G/H}\on{Ind}_{H_s}^H \rho^s\right\rangle_H \\ 
                                                    &= \sum_{s \in H\backslash G/H} \left\langle W,\on{Ind}_{Hs}^H\rho^s\right\rangle_H\\
                                                    &= \sum_{s \in H\backslash G/H} \left\langle \on{Res}_{Hs}^HW,\rho^s\right\rangle_{Hs}.
    \end{align*} 
    Let $s=1$, then $HsH=H$. Then,
    $$\left\langle \on{Res}_{Hs}^H W, \rho^s\right\rangle_{Hs} = \langle W,W\rangle_H = \langle \rho,\rho\rangle_H \geq 1.$$
    So, $\langle V,V\rangle_G = 1$ if and only if $\langle \rho,\rho\rangle_H=1$
    and all other $\left\langle \on{Res}_{Hs}^HW,\rho^s\right\rangle_{Hs} = 0$
    for all $s\neq 1$ in $H \backslash G/H$.

\end{proof}

\begin{corollary}
    Suppose that $H$ is a normal subgroup of $G$. Then, 
    $\on{Ind}_H^G\rho$ is irreducible if and only if $\rho$ is irreducible 
    and $\rho \not\cong \rho^s$ for all $s\not\in H$.
\end{corollary}

\section{Lecture 3, 01/09/2023}

\subsection{Revisiting Characters of $A_4 \leq \mathfrak{S}_4$}

In light of what we learned last time, we want to re-visit the representations
of the alternating group $A_4$ in $\mathfrak{S}_4$.
Recall that the character table of $\mathfrak{S}_4$ is given by 
$$\begin{tabular}{|c|c|c|c|c|c|}
    \hline
    & $1$ & $(12)$ & $(12)(34)$ & $(123)(4)$ & $(1234)$\\
    \hline
    $\chi_{\on{triv}}$ & $1$ & $1$ & $1$ & $1$ & $1$\\
    \hline
    $\chi_{\on{sgn}}$ & $1$ & $-1$ & $1$ & $1$ & $-1$\\
    \hline
    $\chi_{\on{std}}$ & $3$ & $1$ & $-1$ & $0$ & $-1$\\
    \hline
    $\chi_{\on{std}\otimes\on{sgn}}$ & $3$ & $-1$ & $-1$ & $0$ & $1$\\
    \hline
    $\chi_5$ & $2$ & $0$ & $2$ & $-1$ & $0$\\
    \hline
\end{tabular}$$
Recall that $A_4 \cong K \rtimes H$, where $K = \lbrace 1,t,t^2\rbrace$,
for $t=(123)$, and $H = \lbrace 1,x,y,z\rbrace$.
From this, we obtain the character table:
$$\begin{tabular}{|c|c|c|c|c|}
    \hline
    & $1$ & $x=(12)(34)$ & $t= (123)$ & $t^2$\\
    \hline
    $\chi_1$ & $1$ & $1$ & $1$ & $1$\\
    \hline
    $\chi_2$ & $1$ & $1$ & $\omega$ & $\omega^2$\\
    \hline
    $\chi_3$ & $1$ & $1$ & $\omega^2$ & $\omega$\\
    \hline
    $\chi_4$ & $3$ & $-1$ & $0$ & $0$\\
    \hline
\end{tabular}$$
where $\omega$ is the third root of unity.
Note here in particular that 
$\rho_{\on{triv}}\vert_{A_4} \cong \rho_{\on{sgn}}\vert_{A_4}$,
and $\rho_{\on{std}}\vert_{A_4} \cong \rho_{\on{std}\otimes\on{sgn}}\vert_{A_4}$,
just by inspecting the respective character tables.
\paragraph{Question:} Is the restriction $\on{Res}_{A_4}^{\mathfrak{S}_4}\rho_{\on{std}}$ irreducible?\\\\
Let $\rho$ be an irreducible representation of $\mathfrak{S}_4$. 
Then, consider $\on{Res}_{A_4}^{\mathfrak{S}_4}\rho$. Computing directly,
$$\left\langle \on{Res}_{A_4}^{\mathfrak{S}_4}\rho, \on{Res}_{A_4}^{\mathfrak{S}_4}\rho\right\rangle_{A_4} = \frac{1}{\vert A_4\vert} \sum_{x \in A_4} \chi_\rho(x^{-1})\chi_\rho(x)  =\frac{2}{\vert\mathfrak{S}_4\vert} \sum_{x \in A_4}\chi(x^{-1})\chi(x) \leq 2 \langle \chi_\rho,\chi_\rho\rangle_{\mathfrak{S}_4} = 2,$$
where the second equality follows since $A_4$ is a subgroup of index $2$
in $\mathfrak{S}_4$.
From this, we conclude that either $\on{Res}_{A_4}^{\mathfrak{S}_4}$ is 
irreducible, or it is the direct sum of two non-isomorphic irreducible
representations (have to be non-isomorphic otherwise the inner product 
evaluates to $4$)--- that is, $\on{Res}_{A_4}^{\mathfrak{S}_4} \rho\cong\rho_1\oplus\rho_2$, where $\rho_1 \not\cong \rho_2$. \\\\
From the above calculation, we see that 
$$\left\langle\chi_{\on{Res}_{A_4}^{\mathfrak{S}_4}\rho},\chi_{\on{Res}_{A_4}^{\mathfrak{S}_4}\rho}\right\rangle_{A_4} = 2,$$
if and only if $\chi_\rho(x) = 0$ for all $x\not\in A_4$. So, looking at 
the table, we see that $\on{Res}_{A_4}^{\mathfrak{S}_4}\rho_{\on{std}}$ 
is therefore irreducible. Further,
$\on{Res}_{A_4}^{\mathfrak{S}_4} \rho_5$ decomposes as the sum of two 
irreducible $A_4$-representations. Looking at the character table, 
we see that 
$$\on{Res}_{A_4}^{\mathfrak{S}_4} \rho_5 \cong \phi_1\oplus \phi_2,$$
where $\phi_1$ and $\phi_2$ are the representations corresponding
to $\chi_1$, $\chi_2$, respectively.

\paragraph{Question:} How do we find $3$-dimensional irreducible representations
of $A_4$?
One way is by the restriction functor $\on{Res}_{A_4}^{\mathfrak{S}_4}\rho_{\on{std}}$.\\\\
The second way is by inducing from $H$ to $A_4$.
$H = \lbrace 1,x,y,z\rbrace = \lbrace 1,(12)(34),(13)(24),(14)(23)\rbrace$.
Since every element has at most order $2$, we see that this is 
isomorphic to the Klein group --- that is, 
$$H \cong \mathbb{Z}/2\mathbb{Z} \times \mathbb{Z}/2\mathbb{Z}.$$
This is a subgroup of index $3$ in $A_4$.
It follows thus that $H$ has $4$ irreducible characters given by 
$$\begin{tabular}{|c|c|c|c|c|}
    \hline
    & $1$ & $x$ & $y$ & $z$\\
    \hline
    $\psi_1$ & $1$ & $1$ & $1$ & $1$\\
    \hline
    $\psi_2$ & $1$ & $-1$ & $1$ & $-1$\\
    \hline
    $\psi_3$ & $1$ & $1$ & $-1$ & $-1$\\
    \hline
    $\psi_4$ & $1$ & $-1$ & $-1$ & $1$\\
    \hline
\end{tabular}$$
Let us consider the inner product and apply Frobenius reciprocity, we 
obtain:
$$\left\langle \on{Ind}_H^{A_4}\psi_i, \chi_4\right\rangle_{A_4} = \left\langle \psi_i, \on{Res}_H^{A_4}\chi_4\right\rangle_H = \begin{cases}
    0, \quad &\text{if}\quad i =1\\
    1,\quad &\text{if}\quad i=2,3,4
\end{cases}.$$
\begin{exercise}
    Verify this calculation.
\end{exercise}
This shows us that 
$$\on{Ind}_H^{A_4}\psi_2 \cong \on{Ind}_H^{A_4} \psi_3 \cong \on{Ind}_H^{A_4}\psi_4 \cong \phi_4.$$

\begin{exercise}
    Calculate $\on{Ind}_H^{A_4}\psi_1$, and compute its decomposition
    into irreducibles.
\end{exercise}

Let us compute these induced representations explicitly  --- that is, 
explicitly compute the map $\rho : A_4 \to \on{GL}_3(W)$.
\paragraph{Question:} What is $\on{Ind}_H^{A_4}\psi_2$?\\\\
By definition,
$$\on{Ind}_H^{A_4} \psi_2 \cong \C[G] \otimes_{\C[H]}\C_{\psi_2}.$$
Let us choose a basis of $W$. Recall that there is an isomorphism
$A_4 \cong K \rtimes H$ --- that is, any element $x\in A_4$ can be 
written uniquely as $x = kh$, for some $k\in K$, and $h\in H$. 
Let $e_1$ be the basis for the one-dimensional space $\C_{\psi_2}$,
and let us choose a basis given by:
$$\lbrace 1\otimes e_1, t\otimes e_1, t^2 \otimes e_1\rbrace =: \lbrace v_1,v_2,v_3\rbrace.$$
From this, we see that $$tv_1 = v_2, \quad tv_2 = v_3, \quad tv_3 = v_1.$$
Under this map, $$t\longmapsto \begin{pmatrix}
    0&0&1\\
    1&0&0\\
    0&1&0
\end{pmatrix}.$$
Similarly,
$$xv_1 = x\otimes e_1 = 1\otimes xe_1 = -1\otimes e_1 = -v_1,$$
where we can move $x$ across the tensor product since $x \in \C[H]$.
$$xv_2 = xt\otimes e_1 = t(t^{-1}xt)\otimes e_1 = ty\otimes e_1 = t\otimes ye_1 = t\otimes e_1 = v_2.$$
$$xv_3 = xt^2\otimes e_1 = t^2z \otimes e_1 = t^2 \otimes ze_1 = -t^2\otimes e_1 = -v_3.$$
It follows thus that 
$$x\longmapsto \begin{pmatrix}
    -1&0&0\\
    0&1&0\\
    0&0&-1
\end{pmatrix}.$$

\begin{exercise}
    Complete this computation and check directly that $\chi_W = \chi_4$.
\end{exercise}

\begin{exercise}
    Use Mackey's irreducibility criterion  to deduce that 
    $\on{Ind}_{H}^{A_4}\phi_2$ is irreducible.
\end{exercise}

\subsection{More on Dimensions of Irreps of $G$}

\begin{proposition}\label{prop15}
    Let $H \leq G$ be a normal subgroup, and let $\rho: G \to \on{GL}(V)$ be 
    an irreducible representation, as before. Then,
    \begin{itemize}
        \item[(a)] Either $\rho\vert_H$ is isotypic --- that is, a direct 
            sum of isomorphic irreducible representations (i.e. the canonical 
            decomposition only has one summand),
        \item[(b)] or, there exists a proper subgroup $K$ of $G$ 
            containing $H$, and there exists an irreducible 
            representation $\sigma$ of $K$ such that 
            $\rho = \on{Ind}_K^G\sigma$.
    \end{itemize}
\end{proposition}

\begin{proof}
    Suppose that $$\on{Res}_H^G(\rho,V) = \bigoplus_i V_i,$$ where $V_i$ are 
    the isotypic components --- i.e. $\on{Hom}_H(V_i,V_j) = 0$ for all $i\neq j$.
    Then, for all $g \in G$, $g\cdot V_i$ is $H$-stable since 
    $h(g\cdot V_i) = g(g^{-1}hgV_i) = gV_i$, by the normality of $H$.
    Further, we see that $gV_i \subseteq V_j$ for some $j$. 
    Moreover, $V_i \subseteq g^{-1}V_i \subseteq V_k$ for some $k$.
    This implies that $i=k$ by property of isotypic components.
    It follows that $g$ permutes the $V_i$'s.\\\\
    Fixing $i_0$, if $V_{i_0} = V$, then (a) is fine --- that is, 
    $\on{Res}_H^G = V_{i_0}$.
    Now, fixing $i_0$, if $V_{i_0} = V$, then (a) is true.
    Otherwise, consider 
    $K = \lbrace g\in G :gV_{i_0} = V_{i_0}\rbrace \supseteq H$.
    It follows then that $K \neq G$, because $V$ is irreducible, and 
    $V_{i_0}\neq V$. Now, it follows then that 
    $$V = \bigoplus_{g\in G/K} gV_{i_0} =: \on{Ind}_K^GV_{i_0}.$$
    This tells us then that $V_{i_0}$ must be irreducible as a 
    $K$-representation, otherwise $V$ would not be irreducible.
\end{proof}


\chapter{Week Seven}

\section{Lecture 1, 04/09/2023}

We prove a consequence of Proposition \ref{prop15} from last time:
\begin{corollary}
    If $A$ is an abelian and normal subgroup of $G$, then 
    the dimension of each irreducible $G$-representation divides 
    $[G:A] = \frac{\vert G\vert}{\vert A \vert}$.
\end{corollary}

\begin{proof}
    We proceed by inducting on $\vert G\vert$. Let $(\rho,V)$ be an irreducible
    $G$-representation.
    Suppose we are in (a) of Proposition \ref{prop15} --- that is, 
    $\on{Res}_A^G\rho$ is isotypic --- and so 
    $\on{Res}_A^G\rho cong \chi^{\oplus N}$, where $\chi$ is a 
    $1$-dimensional representation of $A$. This then implies that 
    $\rho(A)\subset Z(\rho(G))$. Consider $\rho : G/A \to \rho(G)/\rho(A)$,
    which is a surjection.
    We thus have $\frac{\vert \rho(G)\vert}{\vert \rho(A)\vert}$.
    We think of $V$ as an irreducible representation of $\rho(G)$.
    Since $\rho(A)\subset Z(\rho(G))$, we have shown before that 
    $\dim_\C \rho = \dim V$, which divides 
    $[\rho(G):\rho(A)]$. It follows then that $\dim_\C\rho$ divides
    $[G:A]$. Suppose we are in case (b) of Proposition \ref{prop15}.
    Then, $\rho = \on{Ind}_K^G\sigma$, and $A\subseteq K \subset G$,
    and $\sigma$ is an irreducible $K$-representation. Then, inducting on 
    the order of $G$, $\dim_\C\sigma$ didivdes $[K:A]$, and thus 
    $\dim_\C\rho = [G:K] \cdot \dim_\C \sigma$, which then divides
    $[G:K][K:A] = [G:A]$.
\end{proof}

\subsection{Semidirect Product by an Abelian Subgroup}

Suppose that $G = H\ltimes A$, where $A$ is an abelian normal subgroup of $G$.
Using a method of Wigner and Mackey, one can construct irreps 
of $G$ from certain subgroups of $H$.
Recall that if $A$ is abelian, then all of its irreducible representations
are one-dimensional. All such representions form a group, denoted by 
$$X = \on{Hom}_\C(A,\C^\times),$$
with the group structure given by 
$$(\chi_1\chi_2)(a) = \chi_1(a)\cdot \chi_2(a),$$
for some $a\in A$.
Note that $G$ acts on $X$ --- for $\chi \in X$, and $g\in G$,
$$g\cdot \chi(a) = \chi(g^{-1}ag).$$
Let $\lbrace \chi_i\rbrace_{i\in X/H}$ be a system of representatives 
for $H$-orbits on $X$. For each $i\in X/H$, let $H_i := \lbrace h\in H:h\cdot \chi_i = \chi_i\rbrace$ be the stabiliser subgroup.
Define $$G_i:= H_i \ltimes A,$$ which is a subgroup of $G$.
For each $G_i$, define a one-dimensional representation given by 
$$\chi_i : G_i \to \C^\times,$$ given by $\chi_i(ha) = \chi_i(a)$,
for $h\in H$, and $a\in A$. We now have to check that this is a group
homomorphism. Indeed,
\begin{align*}
    \chi_i(h_1a_1h_2a_2) &= \chi_i(h_1h_2h_2^{-1}a_1h_2a_2)\\ 
                         &= \chi_i(h_2^{-1}a_1h_2a_2)\\ 
                         &= \chi_i(h_2^{-1}a_1h_2)\chi_i(a_2)\\ 
                         &= h_2\cdot\chi_i(a_1)\chi_i(a_2)\\ 
                         &= \chi_i(a_1)\chi_i(a_2) = \chi\\ 
                         &= \chi_i(h_1a_1)\chi_i(h_2a_2),
\end{align*}
using the fact that $h_2\in H_i$, and $h_2\cdot\chi_i = \chi_i$.
Let $(\rho,W)$ be an irreducible representation of $H_i$, and and consider
the composition:
$$\widetilde{\rho} : G_i \longrightarrow G_i/A \cong H_i \stackrel{\rho}{\longrightarrow} \on{GL}(W),$$
and we have $\widetilde{\rho}(ha) =\rho(h)$, for each $h\in H$, $a\in A$.
We now have an irreducible representation of $G_i$ given by $\widetilde{\rho}$.
Let $$Q_{i,\rho}:= \on{Ind}_{G_i}^G (\chi_i\otimes \widetilde{\rho}).$$

\section{Lecture 2, 07/09/2023}

\begin{proposition}
    \leavevmode
    \begin{itemize}
        \item[(i)] $Q_{i,\rho}$ is an irreducible $G$-representation,
        \item[(ii)] $Q_{i,\rho} \cong Q_{i',\rho'}$ as $G$-representations
            if and only if
            $i=i'$, and $\rho \cong \rho'$, as $H_i$-representations,
        \item[(iii)] Every irreducible representation of $G$ is isomorphic
            to one of $Q_{i,\rho}$.
    \end{itemize}
\end{proposition}

\begin{proof}
    \leavevmode 
    \begin{itemize}
        \item[(i)] Using Mackey's irreduciblity criterion. Let 
            $s \not\in G_i = H_i\ltimes A$. Let $K_s = G_i \cap s G_is^{-1}$.
            Consider the two representations, given by
            $\on{Res}^{G_i}_{K_s}(\chi_i\otimes\widetilde{\rho}),$
        and $(\chi_i\otimes \widetilde{\rho})^s : K_s \to \on{GL}(\widetilde{W})$, where the latter denotes a twisted action, given by 
        $x\mapsto \chi_i(s^{-1}xs)\widetilde{\rho}(s^{-1}xs)$.
        We wish to show that these two representations contain no common
        factors --- that is, they are disjoint representations.
        It is enough to consider their restriction to $A\subset K_s$.
        Then, 
        $$\on{Res}_A^{K_s}\on{Res}_{K_s}^{G_i}(\chi_i\otimes\widetilde{\rho}) = \chi_i^{\oplus \dim_\C W},$$
        and 
        $$\on{Res}_A^{K_s}(\chi_i\otimes \widetilde{\rho})^s = s\cdot \chi_i^{\dim_\C W}.$$
        But $s\cdot \chi_i\neq \chi_i$, since $s\not\in G_i$. Thus, they 
        are disjoint.
    \item[(ii)] 
        Observe that 
        $$\on{Res}_A^G Q_{i,\rho} = \on{Res}_A^G\on{Ind}_{G_i}^G(\chi_i\otimes\widetilde{\rho}) = \bigoplus_{s\in A\backslash G/G_i \cong G/G_i} s\cdot \chi_i^{\oplus \dim_\C\rho},$$
        where the isomorphism $A\backslash G/G_i \cong G/G_i$ follows
        since $AgG_i = gAG_i = gG_i$. Note also that 
        $G/G_i\cong H/H_i$ by the same argument.
        \begin{exercise}
            Check the above identity.
        \end{exercise}
        \begin{proof}
            We have: $$\on{Res}_A^G \on{Ind}_{G_i}^G (\chi_i\otimes\widetilde{\rho}) = \on{Res}_A^G \bigoplus_{s \in G/G_i} s\cdot \chi^{\oplus\dim_\C\rho} = \bigoplus_{s \in A\backslash G/G_i} s\cdot \chi^{\oplus\dim_\C\rho},$$
            as claimed.
        \end{proof}
        Observe that $A = sG_is^{-1} \cap A$, for each $s\in G$.
        It follows then that 
        $\on{Res}_A^GQ_{i,\rho}$ depends only on each $H$-orbit of $\chi_i$.
        It follows then that $Q_{i,\rho}$ determines $i$.
        Now, we wish to show that $Q_{i,\rho}$ determines $\rho$.
        Let $V$ be the representation space corresponding to $Q_{i,\rho}$,
        and define 
        $$V_i := \lbrace v\in V : \text{$Q_{i,\rho}(a)v = \chi_i(a)v$, for all 
        $a\in A$}\rbrace.$$
        We claim that $V_i$ is $H_i$-stable. Indeed,
        $$Q_{i,\rho}(a)h_i\cdot v = Q_{i,\rho}(a)\theta_{i,\rho}(h_i)v = Q_{i,\rho}(h_i) \theta_{i,\rho}(h_i^{-1}ah_i)v = Q_{i,\rho}(h_i) \underbrace{\chi_i(h_i^{-1}ah_i)}_{=h_i\cdot \chi_i(a) =\chi_i(a)}v = \chi_i(a) h_iv.$$
        It follows then that $V_i \cong \rho$ as $H_i$-representations,
        and it thus follows that $Q_{i,\rho}$ determines $\rho$ --- 
        this is because $\chi_i\otimes \widetilde{\rho} \cong V_i$ as 
        $H_i$-representations.
        \begin{exercise}
            Prove this isomorphism.
        \end{exercise}
        \begin{proof}

        \end{proof}
    \item[(iii)] Let $\sigma : G \to \on{GL}(V)$ be an irreducible 
        $G$-representation. Suppose 
        $$\on{Res}^G_A V = \bigoplus_{\chi \in X = \on{Hom}(A,\C^\times)}V_\chi,$$
        where $$V_\chi = \lbrace v\in V : av = \chi(a) v\rbrace,$$
        is the canonical decomposition of $\on{Res}^G_AV$.
        Then, there exists some $\chi \in X$ such that 
        $V_\chi \neq \lbrace0\rbrace$. Any element $s\in G$ acts 
        by permuting the isotypic components (recall this from our discussion 
        about normal subgroups)--- that is,
        $$\sigma(s) : V_\chi \longrightarrow V_{s\cdot \chi}.$$
        Suppose now that $\chi$ is in the $H$-orbit of $\chi_i$.
        Then, $V_{\chi_i}\neq 0$, and $V_{\chi_i}$ is $H_i$-stable.
        Let $W_i$ be an irreducible $\C[H_i]$-module of $V_{\chi_i}$,
        and $\rho : H_i \to \on{GL}(W_i)$ the corresponding 
        irreducible representation. Then, as a representation of 
        $G_i = H_i\ltimes A$, $$W_i \cong \chi_i\otimes \widetilde{\rho}.$$
        Then, computing, and using Frobenius reciprocity:
        $$\left\langle \sigma, \underbrace{\on{Ind}_{G_i}^G (\chi_i\otimes\widetilde{\rho})}_{=Q_{i,\rho}}\right\rangle_G = \left\langle \sigma, \on{Ind}_{G_i}^GW_i\right\rangle_G = \left\langle \on{Res}_{G_i}^G \sigma, W_i\right\rangle_{G_i} \neq 0,$$
        which is non-zero, since $W_i$ is a subspace of $\sigma$.
        It follows then that $\sigma \cong Q_{i,\rho}$, since 
        both $\sigma$ and $Q_{i,\rho}$ are irreducible.
    \end{itemize}
\end{proof}

\begin{exercise}
    Classify all irreducible representations of the group of signed permutations.
    This is the Weyl group of type $B$.
\end{exercise}

\subsection{Group Theory Review: $p$-Groups, Nilpotent, (Super)solvable}

\begin{definition}
    A group $G$ is \emph{solvable} if there exists a sequence
    $$\lbrace 1\rbrace \leq G_0 \leq G_1 \leq \cdots \leq G_n = G,$$
    such that $G_{i-1} \trianglelefteq G_i$, and 
    $G_i/G_{i-1}$ is abelian. A group $G$ is \emph{supersolvable} of 
    it is solvable, and moreover $G_i \trianglelefteq G$, and 
    $G_i/G_{i-1}$ is cyclic.
    A group $G$ is \emph{nilpotent} if it is solvable, and 
    $G_i/G_{i-1} \subset Z(G/G_{i-1})$.
\end{definition}

We note here that nilpotency is the strongest condition here. 
That is, any nilpotent group is supersolvable.

\begin{example}
    \leavevmode
    \begin{enumerate}
        \item $A_4$ is solvable, but not supersolvable,
        \item The dihedral group $D_n$ is supersolvable, and nilpotent 
            if and only if $n = 2^m$ for some $m$,
    \end{enumerate}
\end{example}

\begin{exercise}
    Prove the above statements.
\end{exercise}

\begin{definition}
    If $p$ is a prime number, then a \emph{$p$-group} is a group $G$ 
    whose order is a power of $p$.
\end{definition}

There is the following theorem, which we will state without proof:

\begin{theorem}
    Every $p$-group is nilpotent.
\end{theorem}

\begin{lemma}\label{lem5}
    Let $G$ be a $p$-group acting on a finite set $X$,
    and let $X^G$ be the set of fixed points of the $G$-action.
    Then, $$\vert X \vert \equiv \vert X^G\vert \pmod{p}.$$
\end{lemma}

\begin{proof}
    The set $X\setminus X^G$ is a union of non-trivial orbits of 
    $G$ --- that is, $Z_G(x) \neq G$, for $x$ in the orbit of $G$.
    Each of these orbits have size $p^a$, for some $a\geq 1$.
    It follows then that $\vert X - X^G\vert = \vert X \vert - \vert X^G\vert \equiv 0 \pmod{p}$.
\end{proof}

\begin{proposition}
    Let $V$ be a non-zero vector space over a field $k$ of characteristic 
    $p$, for $p$ a prime number. Assume that 
    $$\rho : G \longrightarrow \on{GL}(V),$$ is a linear representation of 
    $G$, where $G$ is a $p$-group. Then there exists, $0\neq v \in V$ 
    such that $$\rho(g)v = v,$$ for all $g\in G$.
\end{proposition}

\begin{proof}
    Let $0\neq x \in V$, and let 
    $$X := \on{Span}_{\mathbb{F}_p}(\rho(g)x : g\in G\rbrace.$$
    Then, $G$ acts on $X$. It follows then that 
    $\vert X \vert = p^\alpha$, for some $\alpha \geq 1$.
    By Lemma \ref{lem5}, it follows then that 
    $\vert X^G\vert \equiv 0 \pmod{p}$, from which it follows then that 
    there exists some $v \in X^G$, as claimed.
\end{proof}


\section{Lecture 3, 08/09/2023}

Recall from last time that if $V$ if a vector space over $\mathbb{F}_p$,
and $G$ is a $p$-group, then there exists a fixed point 
of the representation $\rho : G \to \on{GL}(V)$. We 
obtain the following corollary:

\begin{corollary}
    The only irreducible representation of a $p$-group $G$ in characteristic
    $p$ is the trivial reprensentation.
\end{corollary}

Let $p$ be a prime number, and $G$ a group of order $p^nm$, with 
$\gcd(m,p) = 1$. A subgroup of order $p^n$ is called a \emph{Sylow 
$p$-subgroup}. That is, Sylow $p$-subgroups are the largest 
$p$-subgroups of a $p$-grop. 
From this, we have the following theorem, known as Sylow's theorem:

\begin{theorem}[Sylow's Theorem]
    \leavevmode
    \begin{itemize}
        \item[(a)] There exists Sylow $p$-subgroups,
        \item[(b)] The Sylow $p$-subgroups are conjugate by inner automorphisms
            of $G$ --- i.e. if $P$ and $Q$ are Sylow subgroups, then there 
            exists $g\in G$ so that $gPg^{-1}=Q$,
        \item[(c)] Each $p$-subgroup of $G$ is contained in a Sylow 
            $p$-subgroup.
    \end{itemize}
\end{theorem}

\begin{proof}
    \leavevmode
    \begin{itemize}
        \item[(a)] Let $Z(G)$ be the center of $G$. Then, if 
            $\vert Z\vert\equiv 0\pmod{p}$, then $Z(G)$ contains a cyclic
            group of order $p$, denoted by $D\cong \mathbb{Z}/p\mathbb{Z}$. 
            By induction on the order of $G$, $G/D$ has a Sylow $p$-subgroup of 
            order $p^{n-1}$.
            Consider the map $G \to G/D$. The pre-image of $H$ in $G$ is a
            Sylow $p$-subgroup of $G$.\\\\
            We now consider the case when $\vert Z \vert \not\equiv \pmod{p}$.
            Consider the set $G\setminus Z$ (we are minusing the 
            sets, not taking quotients). Then,
            $$G\setminus Z = \prod_{\text{$C_i$ conjugacy class of $G$}}C_i.$$
            It follows then that $\vert G \setminus Z \vert\not\equiv\pmod{0}$,
            and thus $p$ does not divide $\vert C_i\vert,$ for some $i$.
            Let $x\in C_i$. Recall that 
            $$\vert C_i\vert = \frac{\vert G\vert}{\vert\on{Stab}_x\vert},$$
            and thus $\vert \on{Stab}_x\vert = p^nm'$, 
            for some $m'<m$. So, inducting on the order of $G$, we see that 
            $H= \on{Stab}_x$ has a Sylow $p$-subgroup.
        \item[(b), (c)] Let $P$ be a Sylow $p$-subgroup, and let $Q$ be a 
            $p$-subgroup of $G$. Let $Q$ act on the coset $X:=G/P$. Recall 
            from last time that 
            $$\vert X\vert = \vert X^Q\vert \pmod{p},$$
            by a Lemma from last time (TODO: add reference).
            Then, $\vert X^Q\vert \not\equiv 0\pmod{p}$. It follows therefore
            that $X^Q$ is non-empty, and there thus exists some 
            $g\in G$, for which $qgp_1 = gp_2$, and thus
            $q=gp_2p_1^{-1}g^{-1} \in gPg^{-1}$, for any $q\in Q$, and
            $p_1,p_2\in P$.
            This shows that $Q\subset gPg^{-1}$. It follows thus that
            any $p$-subgroup is contained in a Sylow $p$-subgroup.\\\\
            If $Q$ is a Sylow $p$-subgroup, then 
            $\vert Q \vert = \vert gPg^{-1}\vert = p^n$, and therefore 
            $Q = gPg^{-1}$, as claimed.
    \end{itemize}
\end{proof}

\begin{lemma}\label{lem_supsolv}
    Let $G$ be a non-abelian supersolvable group. Then, there exists a
    normal abelian subgroup of $G$ which is not contained the centre of 
    $G$.
\end{lemma}

\begin{proof}
    Let $H := G/Z(G)$, which is still supersolvable --- that is, there is a 
    descending series of strict inclusions
    $$0 \subset H_1 \subset H_2 \subset \cdots \subset H,$$
    where each $H_1$ is a cyclic normal subgroup of $H$.
    Let $G_1 := \pi^{-1}(H_1)$, where $\pi : G \to H$.
    Then, $G_1$ is as desired.
    \begin{exercise}
        Show this.
    \end{exercise}
\end{proof}

\begin{theorem}
    Let $G$ be a supersolvable group. Then, every irreducible representation 
    of $G$ is induced by a one-dimensional representation of a subgroup 
    of $G$.
\end{theorem}

\begin{proof}
    By induction on the order of $G$, let $\rho : G \to \on{GL}(V)$ be 
    an irreducible representation. We can assume that $\rho$ is faithful --- 
    if not, we can mod out by the kernel and argue again by induction.
    If $G$ is abelian, then we are done.\\\\
    Thus, suppose that $G$ is non-abelian. Then, by Lemma \ref{lem_supsolv},
    there exists a normal abelian subgroup $A$ of $G$ that is not 
    contained in $Z(G)$. Since $\rho$ is faithful, $\rho(A)$ is not 
    contained in $Z(\rho(G))$ --- the centre of $\rho(G)$.
    This thus implies that $\rho\vert_A$ is not isotypic. As such,
    it follows that $\rho$ is of the form $\rho = \on{Ind}_H^G\sigma$,
    for some proper subgroup $H$ of $G$, and $\sigma$ an irreducible
    $H$-representation.\\\\
    Inducting on the order of $G$, $\sigma = \on{Ind}_K^H \C_\chi$,
    where $K$ is a proper subgroup of $H$, and $\C_\chi$ is a 
    one-dimensional representation defined by a character $\chi$ of $K$.
\end{proof}

We state some important results without proof:

\begin{theorem}[Artin's Theorem]
    Each character of $G$ is a $\mathbb{Q}$-linear combination 
    of characters induced by characters of cyclic groups.
\end{theorem}

\begin{definition}
    A group $H$ is \emph{$p$-elementary} if $H = C \times P$, 
    where $C$ is cyclic of order prime up to $p$, and 
    $P$ is a $p$-group. A subgroup of $G$ is \emph{elementary}
    if it is $p$-elementary for at least one prime number $p$.
\end{definition}

\begin{definition}
    A character of $G$ is \emph{monomial} if it is induced 
    from a character of some subgroup.
\end{definition}

\begin{theorem}[Brauer's Theorem]
    \leavevmode
    \begin{itemize}
        \item[(i)] Each character of $G$ is a $\mathbb{Z}$-linear combination 
            of characters of elementary subgroups.
        \item[(ii)] Each character of $G$ is a $\mathbb{Z}$-linear 
            combination of monomial characters of.
    \end{itemize}
\end{theorem}

\subsection{Representations of $\mathfrak{S}_n$}

We begin by recalling some facts about the symmetric group. In particular,
the number of conjugacy classes are given by the number of partitions of 
$n$. The correspondence is given by the cycle types of elements of 
$\mathfrak{S}_n$.\\\\
Write $p(n)$ for the number of partitions of $n$. It has a generating function
given by $$\sum_{n=1}^\infty p(n)t^n = \frac{1}{\prod_{s=1}^\infty (1-t)^s}=(1+t+t^2+\cdots)(1+t^2+t^4+\cdots)(\cdots.$$
This function converges exactly in $\vert t\vert < 1$.
Moreover, $$p(n) \sim \left(\frac{1}{\alpha n}\right)e^{\beta \sqrt{n}},$$
where $\alpha = 4\sqrt{3}$, and $\beta = \pi \sqrt{2/3}$.

\chapter{Week Eight}

\section{Lecture 1, 11/09/2023}

\subsection{Young Diagrams}

Recall from last time that conjugacy classes of $\mathfrak{S}_n$ 
are in bijection with partitions of a positive integer $n$.
To a partition, one may associate a \emph{Young diagram}.
For instance, given a partition $\lambda = 3+2+1+1$ of $7$, 
the Young diagram is given by 
%$$\begin{ytableau}
%    &&\\
%    &&\\
%    &\\
%    &
%\end{ytableau}$$
The \emph{conjugate partition} of $\lambda$ --- denoted $\lambda'$ --- is given by the Young diagram
%$$\begin{ytableau}
%    &&&&\\
%    &&\\
%    &\\
%    &
%\end{ytableau}$$
A \emph{Young tableau} is a Young diagram with numbers $1,\cdots,n$,
with each number appearing exactly once. 
As an example, 
$$t_\lambda := \begin{ytableau}
    1&2&3\\
    4&5\\
    6\\
    7
\end{ytableau},$$
is a Young tableau corresponding to the partition 
$\lambda = 3+2+1+1$.\\\\
Fixing a tableau of shape $\lambda$, define 
$$P := P_{t_\lambda} = \lbrace g\in \mathfrak{S}_n : \text{$g$ preserves each row}\rbrace,$$
$$Q = Q_{t_\lambda} = \lbrace g\in \mathfrak{S}_n : \text{$g$ preserves each column}\rbrace.$$
For $\lambda = \lambda_1 + \cdots + \lambda_k$, for which 
$\lambda_1 \geq \cdots \geq \lambda_k \geq 0$, we see that 
$$P \cong \mathfrak{S}_{\lambda_1}\times\cdots \times \mathfrak{S}_{\lambda_k},$$
$$Q \cong \mathfrak{S}_{\lambda_1'} \times \cdots\times \mathfrak{S}_{\lambda_k'},$$
where $\lambda_i'$ correspond to the conjugate partition of $\lambda_i$.
These are called the \emph{Young subgroups} of $\mathfrak{S}_n$.\\\\
Let us now set 
$$a_\lambda = \sum_{g\in P} e_g \in \C[\mathfrak{S}_n],\quad b_\lambda := \sum_{g \in Q} \on{sgn}(g)e_g \in \C[\mathfrak{S}_n].$$
Let us also set 
$$c_\lambda := a_\lambda b_\lambda \in \C[\mathfrak{S}_n].$$
This element $c_\lambda$ is called the \emph{Young symmetriser}, and it turns out that 
it will give us all the irreducible representations of $\mathfrak{S}_n$.

\begin{theorem}\label{thm20}
    \leavevmode
    \begin{itemize}
        \item[(i)] Some scalar multiple of $c_\lambda$ is idempotent --- 
            that is, $c_\lambda^2 = n_\lambda c_\lambda$,
        \item[(ii)] $\C[\mathfrak{S}_n]c_\lambda$ is an irreducible
            representation of $\mathfrak{S}_n$,
        \item[(iii)] Every irreducible representation of $\mathfrak{S}_n$ 
            can be obtained this way.
    \end{itemize}
\end{theorem}

Let $V$ be any finite-dimensional $\C$-vector space. 
Then, $\mathfrak{S}_n$ acts on $V^{\otimes n}$ by permuting 
the factors of $V^{\otimes n}$. It follows then that there is 
a map 
$$\C[\mathfrak{S}_n] \longrightarrow \on{End}(V^{\otimes n}).$$
It follows then that $a_\lambda$ defines a map 
$a_\lambda : V^{\otimes n}\to V^{\otimes n}$. In particular,
\begin{equation}\label{eqn8.1}
    \on{Im}(a_\lambda) \cong \on{Sym}^{\lambda_1}V \otimes \on{Sym}^{\lambda_2} V \otimes \cdots \times \on{Sym}^{\lambda_k}V \subset V^{\otimes n},
\end{equation}
\begin{equation}\label{eqn8.2}
    \on{Im}(b_\lambda) \cong \Lambda^{\lambda_1}V \otimes \cdots \otimes\Lambda^{\lambda_k}V.
\end{equation}
To see why (\ref{eqn8.1}) and (\ref{eqn8.2}) hold, consider the example:
\begin{example}
    If we let $\lambda = n$, then we have a Young tableau
    given by 
    $$\begin{ytableau}
        1 & \cdots & n\\
    \end{ytableau}$$
    Then, $a_\lambda = \sum_{g\in \mathfrak{S}_n}e_g$,
    $b_\lambda = e_1$ since $Q$ is trivial.
    We can think of 
    $$\on{Sym}^nV = \lbrace T \in V^{\otimes n} : \text{$s(T) = T$, where $s\in\mathfrak{S}_n$ is a transposition.}\rbrace.$$
    It follows then that $\on{Im}a_\lambda = \on{Sym}^nV$.
    Then, $\C[\mathfrak{S}_n]c_\lambda = \C c_\lambda$, and 
    corresponds to the trivial $\C[\mathfrak{S}_n]$-module.
\end{example}
Let us consider the other extreme example:
\begin{example}
    $\lambda = (1+\cdots + 1)$, with tableau given by:
    $$\begin{ytableau}
        1\\
        \vdots\\
        1
    \end{ytableau}.$$
    Then, $a_\lambda = e_1$, and $b_\lambda = \sum_{g \in \mathfrak{S}_n} \on{sgn}(g) e_g$. 
    It follows then that 
    $$\on{Im}(b_\lambda) = \Lambda^nV = \lbrace T \in V^{\otimes n} : \text{$s(T) = -T$, where $s \in \mathfrak{S}_n$ is a transposition.}\rbrace.$$
    Then, $\C[\mathfrak{S}_n]c_\lambda = \C c_\lambda,$
    corresponding to the sign representation. This is 
    one-dimensional, hence irreducible.
\end{example}
Let us look at a non-trivial example:
\begin{example}
    $G = \mathfrak{S}_3$, and $\lambda = (2+1)$, with 
    Young tableau
    $$\begin{ytableau}
        1&2\\
        3
    \end{ytableau}.$$
    Then, $a_\lambda = e_1 + e_{(12)}$, and $b_\lambda = e_1-e_{(13)}$.
    It follows then that 
    $c_\lambda = e_1-e_{(13)} + e_{(12)} - e_{(132)}$.
    \begin{exercise}
        Calculate this at home. The answer is $\C[\mathfrak{S}_3]c_\lambda = \on{Span}_\C \lbrace c_\lambda, e_{(13)}c_\lambda\rbrace$, which is a 
        two-dimensional representation corresponding to the standard 
        representation.
    \end{exercise}
\end{example}

\begin{example}
    $$\rho_{\on{triv}} = \begin{ytableau}
        \ast & \ast & \ast & \ast
    \end{ytableau},$$
    $$\rho_{\on{sgn}} = \begin{ytableau}
        \ast\\
        \ast\\
        \ast\\
        \ast
    \end{ytableau},$$
    $$\rho_{\on{std}} = \begin{ytableau}
        \ast & \ast & \ast\\
        \ast 
        \end{ytableau},\quad \begin{ytableau}
        \ast & \ast\\
        \ast & \ast 
    \end{ytableau},\quad \text{($2$-dimensional)},$$
    $$\rho_{\on{std}}\otimes\rho_{\on{sgn}} = \begin{ytableau}
        \ast & \ast\\
        \ast\\
        \ast
    \end{ytableau}.$$
\end{example}

\begin{exercise}
    \leavevmode
    \begin{itemize}
        \item[(i)] Show that the standard representation
            $$V = \lbrace (x_1,\cdots,x_n)\in\C^n : x_1+\cdots+x_n = 0\rbrace \cong \C[\mathfrak{S}_n]c_\lambda,$$
            where $\lambda = ( (n-1) + 1)$.
        \item[(ii)] Show that $\Lambda^kV$ is irreducible if and only if 
            its corresponding Young tableau is of the form
            $$\begin{ytableau}
                \ast & \ast & \ast\\
                \ast\\
                \vdots\\
                \ast
            \end{ytableau},$$
            where the first column has $k+1$ entries.
    \end{itemize}
\end{exercise}

Observe that $P\cap Q = \lbrace 1 \rbrace$.
This means that every $g\in\mathfrak{S}_n$ can be written in at most
one way --- as $pq$, for $p\in P$, and $q\in Q$. This implies 
that $$c_\lambda = \left(\sum_{g\in P} e_g\right) \left(\sum_{h\in Q}\on{sgn}(h)e_{gh}\right) = \sum_{g \in PQ}(\pm) e_g,$$
where the sign of $e_g$ in the last equality depends on
the sign of $q$ in the decomposition $g=pq$. 

\begin{lemma}\label{lem6}
    Let $T$ be a tableau of shape $\lambda$
    $g \in \mathfrak{S}_n$, and let $T' = gT$ --- that is, we 
    replace $i$ by $g(i)$ in $T$. 
    Suppose there are no 
    pair of distinct integers so that they appear in the same row of $T$,
    and in the same column of $T'$.
    Then, there exists $p\in P$, $q\in Q$ such that $g = pq$.
\end{lemma}

\begin{proof}
    First note that $Q' = Q_{gT} = gQ_Tg^{-1}$ --- that is, 
    the columns of $gT$ are stabilised. We now show that there 
    exists some $p_1 \in P$, $q_1' \in gQg^{-1}$ such that 
    $p_1T$, and $q_1'T'$ have the same first row. Performing a column operation
    on $gT = T'$, we can move the $n_i$'s to the first column of $T'$ such that
    $q_1'T'$ has first row entries given by $n_1,\cdots,n_k$ in some order.
    Then, perform row operations on $T$ so that $p_1T$ and 
    $q_1'T'$ have the same first row. Repeating this procedure on the rest 
    of the tableau, we can find $p\in P$, and $q' \in Q'$ such that 
    $pT  = q'T'$. Since $pT = gqg^{-1}gT = gqT$, this implies that 
    $p = gq$, and thus $g = pq^{-1}$. 

\end{proof}


\section{Lecture 2, 14/09/2023}

\begin{lemma}\label{lem7}
    Let $a = \sum_{g \in P} e_g$, $b = \sum_{g \in Q}\on{sgn}(g)e_g$, and 
    $c=ab$.
    Then,
    \begin{itemize}
        \item[(i)] for all $p\in P$, $pa = a = ap$,
        \item[(ii)] for all $q\in Q$, $(\on{sgn}(q)q)b = b = b (\on{sgn}(q)q)$,
        \item[(iii)] for all $p\in P$, $q\in Q$, $p \cdot c \on{sgn}(q)q=c$,
        \item[(iv)] if $x\in\C[\mathfrak{S}_n]$ satisfies $px\on{sgn}(q)q=x$,
            then for all $p\in P$, $q\in Q$, then $x$ is a scalar multiple
            of $c$.
    \end{itemize}
\end{lemma}

\begin{proof}
    (i), (ii), and (iii) are clear. For (iv), let 
    $x = \sum_{g\in\mathfrak{S}_n}m_ge_g$, where $m_g \in \C$.
    Then,
    $$px\on{sgn}(q)q = \sum m_g \on{sgn}(q)e_{pgq} = \sum_{g\in\mathfrak{S}_n} m_ge_g,$$
    for any $p\in P$, $q\in Q$.
    This is true if and only if $m_g\on{sgn}(q) = m_{pgq}$, for any $p\in P$, 
    and $q\in Q$.
    Recall from last time, we proved:
    $$c = \sum_{\substack{p\in P\\q\in Q}}\on{sgn}(q)e_{pq}.$$
    It thus suffices to show that $m_g = 0$ if $g \not\in P \cdot Q$.
    Suppose $g\not\in PQ$. 
    Then, using Lemma \ref{lem6}, there exists two distinct integers
    $i\neq j$ such that $i,j$ are in the same row of $T$ and 
    same column of $gT = T'$. Let $t = (ij)$ be the transposition --- then,
    $t \in P_T$, and $t \in Q_{T'} = gQ_Tg^{-1}$. That is, 
    $t = gqg^{-1}$ for some $q\in Q$, and $t \in P$.
    This then implies that $g = tgq^{-1}$, and htus 
    $$m_g = m_{tgq^{-1}} = m_g\on{sgn}(q^{-1}) = m_g\on{sgn}(q) = m_g\on{sgn}(t) = m_g\cdot (-1),$$
    and thus $m_g = 0$.

\end{proof}

\subsection{Lexicographical Order on Partitions of $n$}

Given $\lambda = (\lambda_1 \geq \cdots\geq \lambda_k)$,
and $\mu = (\mu_1\geq \cdots \geq \mu_\ell)$,
then $\lambda > \mu$ if $\lambda_1 > \mu_1$, or $\lambda_i = \mu_i$
for $1\leq i \leq j$, and 
$\lambda_{j+1} > \mu_{j+1}$ for some $j$. This is a \emph{lexicographical
ordering on partitions of $n$}.

\begin{lemma}\label{lem8}
   Let $T$ is a tableau of shape $\lambda$, and $T'$ a tableau of shape $\mu$.
   Suppose that $\lambda > \mu$. Then, there exists $i\neq j$ such that 
   $i,j$ are in the same row of $T$, and the same column of $T'$.
\end{lemma}

\begin{proof}
    Assume otherwise. Suppose that the first row of $T$
    is $\lbrace n_1,\cdots,n_s\rbrace$, and that $n_a$'s are in 
    different columns of $T'$. Then, the $n_a$'s are in different columsn
    of $T'$, and it follows then that $\mu_1 \geq \lambda_1$.
    Since $\lambda > \mu$, this implies that $\mu_1 = \lambda_1$.
    Repeating this, we obtain $\lambda=\mu$, which is a contradiction.
\end{proof}

\begin{lemma}\label{lem9}
    \leavevmode
    \begin{itemize}
        \item[(i)] If $\lambda > \mu$, then for all $x \in \C[\mathfrak{S}_n]$,
            $$a_\lambda x b_\mu = 0.$$ In particular, if $\lambda > \mu$,
            then $$c_\lambda\cdot c_\mu = 0.$$
        \item[(ii)] For all $x \in \C[\mathfrak{S}_n]$, 
            $$c_\lambda x c_\lambda.$$ is a scalar multiple of $c_\lambda$.
            In particular, $c_\lambda^2 = n_\lambda c_\lambda$, for some 
            $n_\lambda \in \C$.
    \end{itemize}
\end{lemma}

\begin{proof}
   \leavevmode
   \begin{itemize}
       \item[(i)] It suffices to show that 
           $a_\lambda e_g b_\mu = 0$, for all $g \in \mathfrak{S}_n$.
           Since $gb_{T_\mu}g^{-1} = b_{gT_\mu}$, and it thus suffices
           to show that $a_{T_\lambda}b_{T_\mu} = 0$, for 
           any tableau $T_\lambda$ of shape $\lambda$, and $T_\mu$ of 
           shape $\mu$.\\\\
           Then, by Lemma \ref{lem8}, there exists $i\neq j$ such that 
           $i$ and $j$ are in the same row of $T_\lambda$, and the 
           same column of $T_\mu$. Let $t = (ij)$, then $t\in P_\lambda$,
           and $t\in Q_\mu$. It follows then that 
           $a_\lambda t = a_\lambda$, and $tb_\mu = -b_\mu$. 
           It follows then that $a_\lambda t \cdot t b_\mu = a_\lambda b_\mu = a_\lambda\cdot (-b_\mu)$,
           and it thus follows that $a_\lambda b_\mu = 0$.
       \item[(ii)] By Lemma \ref{lem7}(iv), for all $p\in P$, and $q\in Q$,
           we have that $pc_\lambda x c_\lambda \on{sgn}(q)q = c_\lambda xc_\lambda,$
           and thus $c_\lambda x c_\lambda$ is a scalar multiple of $c_\lambda$.
   \end{itemize}
\end{proof}

\begin{lemma}
    \leavevmode
    \begin{itemize}
        \item[(i)] Each $V_\lambda = \C[\mathfrak{S}_n]c_\lambda$ is an 
            irreducible representation of $\mathfrak{S}_n$.
        \item[(ii)] If $\lambda\neq \mu$, then $V_\lambda$ is not isomorphic
            to $V_\mu$.
    \end{itemize}
\end{lemma}

\begin{proof}
    \leavevmode 
    \begin{itemize}
        \item[(i)] Let $W\subset V_\lambda$ be a subrepresentation of $\mathfrak{S}_n$.
            Then, $c_\lambda W \subset c_\lambda V_\lambda \subset \C c_\lambda$,
            by Lemma \ref{lem9}.
            It follows then that either $c_\lambda W = 0$, or 
            $c_\lambda W = \C c_\lambda$.
            Suppose that $c_\lambda W = \C c_\lambda$.
            Then, $V_\lambda = \C[\mathfrak{S}_n]c_\lambda\subseteq \C[\mathfrak{S}_n]c_\lambda W \subseteq \C[\mathfrak{S}_n]W = W$,
            and thus $V_\lambda = W$.\\\\
            For the second case, suppose that $c_\lambda W = 0$. Then,
            $W\cdot W \subseteq \C[\mathfrak{S}_n]c_\lambda W = 0$.
            From this, we wish to claim that $W=0$.
            Recall that there is an orthogonal decomposition given by
            $\C[\mathfrak{S}_n] = W \oplus W'$, where $W'$ is a complementary
            subrepresentation. Consider the projection map 
            $p : \C[\mathfrak{S}_n] \to W$. It is a homomorphism of 
            $\mathfrak{S}_n$-representations mapping 
            $1 \mapsto w\in W$, and $w \mapsto w\cdot w = w$, since 
            $w$ must map to $w$.
            But $w^2 = 0$, and thus $w =0$. It follows then that 
            $W=0$.
            \begin{remark}
                It follows from this proof that $c_\lambda V_\lambda \neq 0$, and 
                $c_\lambda\cdot c_\lambda \neq 0$.
            \end{remark}
        \item[(ii)] We can assume that $\lambda > \mu$. Then, 
            $c_\lambda V_\lambda = \C c_\lambda \neq 0$.
            Further, $c_\lambda V_\mu = \C[\mathfrak{S}_n]c_\mu = 0$ by 
            Lemma \ref{lem9}(i).
            It follows then that $V_\mu \not\cong V_\lambda$.
    \end{itemize}
\end{proof}

\section{Lecture 3, 15/09/2023}

Recall from last time that $\C[\mathfrak{S}_n]c_\lambda$ is an irreducible
$\mathfrak{S}_n$-representation.

\begin{lemma}
    For any $\lambda$, $$c_\lambda\cdot c_\lambda = n_\lambda c_\lambda,$$
    where $$n_\lambda = \frac{n!}{\dim V_\lambda}.$$
\end{lemma}

\begin{proof}
    Consider a map $$F : \C[\mathfrak{S}_n]\longrightarrow \C[\mathfrak{S}_n],\quad x \longmapsto x\cdot c_\lambda.$$
    Then, $\on{Im}F = V_\lambda$, and it follows then that 
    $$\C[\mathfrak{S}_\lambda] \cong \ker F \oplus \on{Im}F \cong \ker F \oplus V_\lambda.$$ 
    As such, we may now view $F$ as a projection
    $F: \ker F \oplus V_\lambda \to V_\lambda$. 
    For some $x \in V_\lambda$, it follows then that $F(x) = x\cdot c_\lambda=n_\lambda x$. 
    It follows thus that 
    $$\on{tr}(F) = n_\lambda \dim V_\lambda.$$
    Now, for some basis element $e_g\in\C[\mathfrak{S}_n]$, 
    we have that $$F(e_g) = e_g \cdot \sum_{\substack{p\in P\\q \in Q}}\on{sgn}(q)e_{pq} = e_g + \text{things not containing $e_g$},$$
    which thus tells us that $\on{tr}(F) = \vert \mathfrak{S}_n\vert = n!$,
    and the result follows.

\end{proof}

\subsection{Frobenius Formula for Irreducible Characters of $\mathfrak{S}_n$}

Let $\widehat{i} := (i_1,\cdots,i_n)$, with $i_k\geq 0$ for each $1\leq k \leq n$. The components of $\widehat{i}$ are the \emph{parts} of the partition.
Then, let $$\lambda_{\widehat{i}} := n^{i_n} (n-1)^{i_{n-1}} \cdots 1^{i_1},$$
called the \emph{multiplicity} of the partition.
As an example, the partition of $n$ given by $(n) = \lambda_{(0,\cdots,0,1)}$,
and $(1+\cdots+1) = \lambda_{(n,0,\cdots,0)}$.
Generally, for some $\sum_{j=1}^n i_jj = n$, where $j$ appears $i_j$ times in $\lambda_{\widehat{i}}$.
Let $C_{\widehat{i}}$ be the conjugacy class corresponding to a partition.\\\\
Let $x_1,\cdots,x_k$ be independent variables, and let $k$ be a number 
that is greater than or equal than the number of rows in $\lambda_{\widehat{i}}$.
Define a polynomial
$$p_j(x) := x_1^j + \cdots + x_k^j,\quad j=1,\cdots,n,$$
called the \emph{power sum}, which is a symmetric polynomial in these variables.
The \emph{discriminant} or the \emph{van der Monde determinant} is given by:
$$\Delta(x) := \prod_{1\leq i < j \leq k} (x_i-x_j).$$
If $f(x)=f(x_1,\cdots,x_k)$ is a formal power series, and 
$(\ell_1,\cdots,\ell_k)$ a $k$-tuple of elements in $\mathbb{Z}_{\geq 0}$,
then denote 
$$[f(x)]_{(\ell_1,\cdots,\ell_k)} := \text{coefficient of $x_1^{\ell_1}\cdots x_k^{\ell_k}$ in $f$}.$$
Let $\lambda = (\lambda_1,\cdots,\lambda_k)$, where in this case we now allow 
$\lambda_k$ to be zero. Assume that $\lambda_1\geq \cdots \geq\lambda_k\geq 0$ be a partition of $n$. Set 
$\ell_1 : =\lambda_1 + k-1$, $\ell_2:= \lambda_2 + k -2$, and $\ell_k=\lambda_k$.
Then, $\ell_1 > \ell_2 > \cdots > \ell_k$.
\begin{theorem}[Frobenius Formula]
    $$\chi_\lambda(C_{\widehat{i}}) = \left[\Delta(x) \cdot \prod_{j=1}^n p_j(x)^{i_j}\right]_{(\ell_1,\cdots,\ell_k)}, \quad \ell_i = \lambda_i + k-i.$$
\end{theorem}

Given this, let us try to deduce a formula for the dimension of $V_\lambda$.
The conjugacy class corresponding the identity is $\widehat{i} = (n,0,\cdots,0)$.
It follows then that 
\begin{align*}
    \dim V_\lambda &= \chi_\lambda(C_{(n,0,\cdots,0)}) = \left[ \Delta(x) \cdot p_1(x)^n \right]_{(\ell_1,\cdots,\ell_k)}.
\end{align*}
$\Delta(x)$ is called the van der Monde determinant because it is the 
determinant of the matrix:
$$\Delta(x) = \det \begin{pmatrix}
    1 & x_k & \cdots & x_k^{k-1}\\
    \vdots  &\vdots & \ddots & \vdots  \\
    1 & x_1 & \cdots & x_1^{k-1}
\end{pmatrix} = \sum_{\sigma\in\mathfrak{S}_k} \on{sgn}(\sigma) x_k^{\sigma(1)-1}\cdots x_1^{\sigma(k)-1}.$$
We have that:
$$p_1(x)^n = (x_1+\cdots+x_k)^n = \sum_{r_1+\cdots+r_k=n}\frac{n!}{r_1!\cdots r_k!} x_1^{r_1}\cdots x_k^{r_k}.$$
We have the relation that 
$\sigma(k)-1+r_1 = \ell_1$, which implies that 
$r_1 = \ell_1 + 1 - \sigma(k) \geq 0$. Iterating this process over 
all the $r_i$'s, we get:
\begin{align*}
    \dim V_\lambda &= \sum_{\substack{\sigma \in \mathfrak{S}_k\\ \sigma(i) \leq \ell_{k+1-i} + 1\\1\leq i \leq k}}\on{sgn}(\sigma) \frac{n!}{(\ell_1-\sigma(k)+1)! \cdots (\ell_k + 1 -\sigma(1))!}\\
                   &= \frac{n!}{\ell_1!\cdots \ell_k!} \sum_{\sigma\in\mathfrak{S}_k} \on{sgn}(\sigma) \prod_{j=1}^k \ell_j(\ell_j-1)\cdots (\ell_j-\sigma(k-j+1)+2)\\
                   &= \frac{n!}{\ell_1!\cdots\ell_k!} \cdot\det \begin{pmatrix}
                       1 & \ell_k & \ell_k(\ell_k-1) & \cdots & (\ell_k)\cdots (\ell_k-k+2)\\
                       \vdots & \vdots & \vdots & \ddots & \vdots\\
                       1 & \ell_1 & \ell_1(\ell_1-1) & \cdots & (\ell_1)\cdots(\ell_1-k+2)
                   \end{pmatrix}\\
                   &= \frac{n!}{\ell_1!\cdots\ell_k!} \prod_{i<j} (\ell_i-\ell_j),
\end{align*}
where the second equality follows from multiplying the numerator and denominator
by $\frac{n!}{\ell_1!\cdots \ell_k!}$. The last equality 
follows by using the van der Monde determinant.\\\\
The above does not depend on the choice of $k$, as long as it is bigger than or 
equal to the number of rows in $\lambda_{\widehat{i}}$.
\begin{exercise}
    Convince yourself of this.
\end{exercise}

\begin{theorem}[Hook-Length Formula]
    $$\dim V_\lambda = \frac{n!}{\prod \text{hook lenghts in lambda}}.$$
\end{theorem}

\begin{example}
    Given the Young diagram
    $$\begin{ytableau}
        \times & \times & \times & \times\\
        \times & \ast & \ast\\
        \times & \ast & \ast\\
        \times
    \end{ytableau}.$$
    The $\times$ symbols give a \emph{hook} of length $7$. 
    $$\begin{ytableau}
        7 & 5 & 4 & 1\\
        5 & 3 & 2\\
        4 & 2 & 1\\
        1
    \end{ytableau}.$$
    Labelling the entries,
    we get 
    $$\dim V_\lambda = \frac{11!}{7\cdot 5^2 \cdot 4^2 \cdot 1 \cdot 3 \cdot 2^2},$$
    using the hook length formula.
\end{example}

\begin{exercise}
    Deduce the hook length formula.
\end{exercise}

\begin{proof}
    The proof is on wikipedia, and it looks long :(
\end{proof}

\subsection{Sketch of Proof of Frobenius' Formula}

Given a partition $\lambda$ of $n$, $\lambda = (\lambda_1,\cdots,\lambda_k)$,
define $$U_\lambda := \on{Ind}_{\mathfrak{S}_{\lambda_1}\times \cdots \times\mathfrak{S}_{\lambda_k}}^{\mathfrak{S}_n}\C_{\on{triv}}.$$
That is, we are inducing from the Young subgroup $P$ to $\mathfrak{S}_n$.
\paragraph{Claim:} $$U_\lambda \cong \C[\mathfrak{S}_n]\cdot a_\lambda.$$
Define a map:
$$U_\lambda = \C[\mathfrak{S}_n] \otimes_{\C[P_\lambda]} \C_{\on{triv}} \longrightarrow \C[\mathfrak{S}_n]\cdot a_\lambda,\quad g\otimes v_0 \longmapsto g\cdot a_\lambda.$$
\begin{exercise}
    Show that this map is well-defined. That is, $gh\otimes v_0 \mapsto gha_\lambda$ for some $h\in P$ is equal to the map 
    $g\otimes v_0 \mapsto ga_\lambda$.\\\\
    Further, show that this map is an isomorphism.
\end{exercise}

\begin{proof}
    By properties of tensor products over modules, we have that for some 
    $h\in P_\lambda$, 
    $$gh \otimes v_0 = g \otimes h\cdot v_0 = g\otimes v_0,$$
    which gets mapped to $g\cdot a_\lambda$. The second equality follows
    because $h$ acts by $1$ on $v_0$, because $v_0 \in \C_{\on{triv}}$ 
    is a trivial $\C[P_\lambda]$-module.
    The map
    $$\C[\mathfrak{S}_n]\cdot a_\lambda \longrightarrow \C[\mathfrak{S}_n]\otimes_{\C[P_\lambda]}\C_{\on{triv}},\quad g^{-1}\cdot a_\lambda \longmapsto g^{-1}\cdot a_\lambda\otimes v_0 = g^{-1}\otimes a_\lambda \cdot v_0 = g^{-1}\cdot v_0,$$
    is an inverse to the map $g\otimes v_0 \mapsto g\cdot a_\lambda$,
    and thus the map is an isomorphism.
\end{proof}

\begin{exercise}
    Let $\lambda = ( (n-1) + 1)$. Show that $$U_{(n-1,1)} \cong V_{(n-1,1)} \oplus V_{(n)}.$$
\end{exercise}

\begin{proof}
    
\end{proof}

\chapter{Week Nine}

\section{Lecture 1, 18/09/2023}

\subsection{Actual Proof of Frobenius' Formula}

Recall from last time that we defined an induced representation given by
$$U_\lambda := \on{Ind}_{P_\lambda}^{\mathfrak{S}_n} \C_{\on{triv}},$$
where $P_\lambda$ is the Young subgroup of $\mathfrak{S}_n$ corresponding
to a partition $\lambda$.
Further, there is an isomorphism 
$U_\lambda \cong \C[\mathfrak{S}_n]a_\lambda$, and a surjective 
homomorphism
$$\C[\mathfrak{S}_n] \cdot a_\lambda \longrightarrow V_\lambda = \C[\mathfrak{S}_n]c_\lambda.$$

\begin{exercise}
    Show that $V_\lambda \cong \C[\mathfrak{S}_n]\cdot a_\lambda b_\lambda \cong \C[\mathfrak{S}_n]b_\lambda a_\lambda$, and thus 
    $V_\lambda$ defines a $U_\lambda$-subrepresentation.
\end{exercise}

Now, let $\psi_\lambda = \chi_{U_\lambda}$ be a character of $U_\lambda$.
Recall that for any subgroup $H$ of $G$, and $\rho = \on{Ind}_H^G\sigma$,
$$\chi_\rho(g) = \frac{1}{\vert H\vert} \sum_{\substack{s\in G\\ s^{-1}gs\in H}}\chi_\sigma(s^{-1} gs).$$
If $\sigma$ is trivial, then $$\chi_\rho(g) = \frac{1}{\vert H\vert} \vert C(g)\cap H\vert \frac{\vert G\vert}{\vert C(g)\vert}.$$
Now, let $G = \mathfrak{S}_n$, and $H = P_\lambda$. Then, for some $g\in C_{\widehat{i}} = C(g)$, then
$$\vert C(g)\vert = \frac{n!}{i_1!\cdot  i_2!\cdots i_n!}.$$
Thus,
\begin{align*}
    \psi_\lambda(g) &= \frac{\vert\mathfrak{S}_n\vert}{\vert \mathfrak{S}_{\lambda_1}\times \cdots\times \mathfrak{S}_{\lambda_k}\vert \cdot \vert C_{\widehat{i}}} \cdot \vert C_{\widehat{i}} \cap \mathfrak{S}_{(\lambda)}\vert \\
                    &= \frac{i_1! \cdots i_n! 1^{i_1}\cdot 2^{i_2}\cdots n^{i_n}}{\lambda_1! \cdots \lambda_k!} \sum_{r_{pq}\in R_{p,q}}\prod_{p=1}^k\frac{\lambda_p!}{r_{p_1}! \cdots r_{pq}! 1^{r_{p1}} \cdots n^{r_{p_n}}}\\
                    &= \sum_{r_{pq}\in R_{p,q}} \frac{\prod_{i=1}^nq^{i_q}i_q!}{\prod_{p=1}^k \prod_{q=1}^n q^{r_{pq}} r_{pq}!}\\
                    &= \sum_{r_{pq}\in R_{p,q}} \frac{\prod_{q=1}^n q^{i_q}i_q!}{\prod_{q=1}^n q^{\sum_{p=1}^k r_{pq}} \prod_{p=1}^k \prod_{q=1}^nr_{pq}!}\\
                    &= \sum_{r_{pq}\in R_{p,q}} \prod_{q=1}^n \frac{i_q!}{r_{1q}! \cdot r_{2q}! \cdots r_{kq}!}\\
                    &= \text{coefficient of $x_1^{\lambda_1}\cdots x_k^{\lambda_k}$ in $P^{(\widehat{i})} := (x_1 + \cdots + x_k)^{i_1} (x_1^2+\cdots x_k^2)^{i_2} \cdots (x_1^n+\cdots +x_k^n)^{i_n}$.}
\end{align*}
where $R_{p,q}$ has elements $r_{pq}$ such that $1\leq p \leq k, 1\leq q \leq n$,
with $i_q = r_{1q}+\cdots + r_{kq}$, and $\lambda_p = r_{p_1} + 2 r_{p2} + \cdots + nr_{p_n}$. 

\begin{exercise}
    Go home and think about this calculation.
\end{exercise}

It follows then that 
$$\psi_\lambda (C_{\widehat{i}}) = \left[ P^{(\widehat{i})}\right]_{\lambda = (\lambda_1 \geq \cdots \geq \lambda_k)}.$$
To prove Frobenius' formula, let us define 
$$\omega_\lambda(\widehat{i}) := \left[ \Delta \cdot P^{(\widehat{i})}\right]_{\ell_1,\cdots,\ell_k}.$$
From this, we wish to show that:
$$\chi_\lambda(C_{\widehat{i}}) = \omega_\lambda(\widehat{i}),$$
where $\chi_\lambda$ is an irreducible character of $\mathfrak{S}_n$.

\subsubsection{General Identity of Symmetric Polynomials}

For any symmetric polynomial $P$. Then, 
\begin{equation}\label{eqn_sym_poly}
    [P]_{\lambda_1,\cdots,\lambda_k} = \sum_{\text{$\mu$ a partition of $n$}} k_{\mu\lambda} [\Delta \cdot P]_{\mu_1+k-1, \mu_2 + k -2, \cdots, \mu_k}.
\end{equation}
We will not give a proof of this fact, but it can be found in Fulton-Harris.
The coefficients $k_{\mu\lambda}$ admit an interesting combinatorial 
interpretation.\\\\ 
They are called \emph{Kostka numbers}, which are defined 
as the number of ways to fill the boxes of the Young diagram for $\mu$ 
with $\lambda_1$ many $1$'s, $\lambda_2$ many $2$'s, $\cdots$, and
$\lambda_k$ many $k$'s in such a way that the entries in each row 
are non-decreasing, and in each column strictly increasing.\\\\
Such tableaux are called \emph{semi-standard Young tableaux on $\mu$ 
of type $\lambda$}. In particular, $k_{\lambda\lambda} = 1$.
Thus, if $\mu < \lambda$, then $k_{\mu\lambda} = 0$. That is,
$k_{\mu\lambda}$ is an upper-triangular matrix.\\\\
By \eqref{eqn_sym_poly}, it follows thus that 
$$\psi_\lambda(C_{\widehat{i}}) = \omega_\lambda(\widehat{i}) + \sum_{\mu > \lambda} k_{\mu\lambda} \omega_\mu(\widehat{i}).$$

\begin{lemma}[Fulton-Harris, Lemma A.28]
    $$\frac{1}{n!}\sum_{\widehat{i}} \vert C_{\widehat{i}}\vert \cdot \omega_\lambda(\widehat{i}) \omega_\mu(\widehat{i}) = \delta_{\lambda\mu}.$$
    That is, $\lbrace \omega_\lambda\rbrace$ satisfies some orthogonal 
    relations on $\lbrace \chi_\lambda\rbrace$.
\end{lemma}

\begin{proposition}
    $\chi_\lambda(C_{\widehat{i}}) = \omega_\lambda(\widehat{i})$. 
    That is, $\chi_\lambda = \chi_{V_\lambda}$.
\end{proposition}

\begin{proof}
    Since $V_\lambda$ is a subrepresentation of $U_\lambda$, it follows
    that $$\psi_\lambda = \sum_{\text{$\mu$ a partition of $n$}} n_{\lambda\mu}\chi_\mu,\quad n_{\lambda\lambda} \geq 1, \quad n_{\lambda\mu}\geq 0.$$
    Thus, we may write $$\omega_\lambda = \sum m_{\lambda\mu}\chi_\mu,$$
    and $$\psi_\lambda = \omega_\lambda + \sum_{\mu > \lambda} k_{\mu\lambda}\omega_\mu = \sum_{\text{$\mu$ partition of $n$}} n_{\lambda\mu}\chi_\mu.$$
    Then, by inducting on order of $\lambda$, we can show that 
    $m_{\lambda\mu}\in\mathbb{Z}$. But 
    $$(\omega_\lambda,\omega_\lambda) = \sum m_{\lambda\mu}^2 = 1,$$
    and thus $$\omega_\lambda = \pm \chi,$$
    for some irreducible character $\chi$.
    We know that 
    $\omega_{(n)} = \chi_{(n)}$. Now, if we fix $\lambda$, then by induction
    $\chi_\mu = \omega_\mu$, for any $\mu > \lambda$. Then, 
    \begin{align*}
        \psi_\lambda &= \omega_\lambda + \sum_{\mu> \lambda}k_{\mu\lambda}\chi_\mu\\
                     &= n_{\lambda\lambda}\chi_\lambda + \sum_{\mu\neq \lambda} n_{\lambda\mu}\chi_\mu,
    \end{align*}
    which implies that 
    $\omega_\lambda  = n_{\lambda\lambda}\chi_\lambda + \cdots$,
    for $n_{\lambda\lambda}\geq 1$, and so $\omega_\lambda = \chi_\lambda$.
\end{proof}

\begin{corollary}[Young's Rule]
    $$U_\lambda \cong V_\lambda \oplus \bigoplus_{\mu>\lambda} V_\mu^{\oplus k_{\mu\lambda}}.$$
\end{corollary}

\section{Lecture 2, 21/09/2023}

\subsection{More Properties of Irreducible $\mathfrak{S}_n$-representations}

\begin{corollary}
    $\dim_\C V_\lambda$ is equal to the number of standard Young tableaux on 
    $\lambda$ --- that is, the number of ways to fill the Young diagram
    $\lambda$ with numbers $1,\cdots,n$ so that all rows and columsn are 
    increasing.
\end{corollary}

\begin{proof}
    Let $\lambda = \underbrace{(1,\cdots,1)}_{\text{$n$ times}}$. 
    Then,
    $$U_\lambda = \C[\mathfrak{S}_n],$$
    and using Young's rule,
    $$U_{(1^n)} \cong V_{(1^n)} \oplus \bigoplus_{\mu > 1^n} V_\mu^{k_{\mu,(1^n)}}.$$
    Since every partition is bigger than $1^n$, it thus follows that 
    $$k_{\mu,(1^n)} = \dim_\C V_\mu,$$
    which by construction of $k_{\mu,(1^n)}$ is the number of ways of filling 
    boxes in $\mu$ with $1,\cdots,n$ such that the rows are non-decreasing,
    and the columns are increasing, which is precisely equal to the number of 
    standard Young tableaux of shape $\mu$.

\end{proof}

What this tells us is that there is a bijection
$$\mathfrak{S}_n \longleftrightarrow \lbrace (A_\lambda,B_\lambda) : \text{both 
$A_\lambda$ and $B_\lambda$ are Young tableaux}\rbrace.$$
Given any element of $\mathfrak{S}_n$, one can produce such a pair of 
tableaux using the Robinson-Schensted algorithm.\\\\
Recall that we have a Young symmetriser $c_\lambda$, which a priori is an 
element of $\C[\mathfrak{S}_\lambda]$, but induces an action on $V^{\otimes n}$.
So,
$$c_\lambda : V^{\otimes n}\longrightarrow V^{\otimes n}.$$
Suppose that $N = \dim_\C V$.
\begin{definition}
    The \emph{Schur polynomial} $s_\lambda(x_1,\cdots,x_N)$
    is given by:
    $$s_\lambda(x_1,\cdots,x_N) := \frac{\det \left(x_j^{\lambda_i + N - i}\right)}{\Delta},$$
    where $\Delta$ is the van der Monde determinant.
\end{definition}
\begin{remark}
    This is related to the Weyl character formula from 
    Lie algebras.
\end{remark}
Further, 
$\on{GL}(V)$ acts on $V^{\otimes n}$ by 
$g(v_1\otimes\cdots\otimes v_n) = gv_1\otimes\cdots\otimes gv_n$.
Define a \emph{Schur functor}
$$\mathbb{S}_\lambda(V) := \on{Im}(c_\lambda).$$
Then,
\begin{enumerate}
    \item $\mathbb{S}_\lambda(V) = 0$ if $\lambda_{N+1} = 0$,
    \item otherwise, $$\dim_\C \mathbb{S}_\lambda(V) = \prod_{1\leq i \leq j \leq k}\frac{\lambda_i-\lambda_j + j - i}{j-i},$$ if $\lambda = (\lambda_1\geq \cdots\geq \lambda_N \geq 0)$. Moreover, 
        $$\chi_{\mathbb{S}_\lambda}(g) = s_\lambda(x_1,\cdots,x_N),$$
        where $x_1,\cdots,x_N$ are eigenvalues of $g\in\on{GL}(V)$
        on $V$, and $s_\lambda$ is a \emph{Schur polynomial}.
\end{enumerate}

\subsection{Schur-Weyl Duality}

As a representation of $\mathfrak{S}_n \times \on{GL}(V)$,
$$V^{\otimes n} \cong \bigoplus_\lambda V_\lambda \otimes L_\lambda,$$
where $L_\lambda$ is a $\on{GL}(V)$-representation that is irreudcible
when $\lambda\neq 0$, and $V_\lambda$ is the aforementioned 
irreducible $\mathfrak{S}_n$-representation that we have constructed.
In particular,
$$L_\lambda = \on{Hom}_{\mathfrak{S}_n}(V_\lambda,V^{\otimes n}),$$
which are distinct irreps of $\on{GL}(V)$ if $\lambda \neq 0$.
Varying $n$, we thus obtain infinitely many irreducible 
$\on{GL}(V)$-representations.

\begin{example}
    $L_{(n)} = \on{Sym}^nV$, and $L_{(1^n)} = \Lambda^nV$, both of which 
    vanishes when $n > \dim_\C V$.
\end{example}

\begin{remark}[Littlewood-Richardson]
    The representation
    $$\on{Ind}_{\mathfrak{S}_m \times \mathfrak{S}_k}^{\mathfrak{S}_{m+k}} V_\lambda \otimes V_\mu = \sum_{\text{$\nu$ partition of $m+k$}} N_{\lambda\mu}^\nu\nu,$$
    where $\lambda$ is a partition of $m$, and $\mu$ a partition of 
    $k$. The coefficients $N_{\lambda\mu}^\nu$ can be determined 
    using the \emph{Littlewood-Richardson rule}.
\end{remark}

\begin{remark}[Pieri's Formula]
    $$\on{Res}^{\mathfrak{S}_n}_{\mathfrak{S}_{n-1}} V_\lambda = \sum_{\text{$\mu$ : one box removed from $\lambda$}} V_\mu.$$
\end{remark}

\begin{remark}
    For $\lambda,\mu$ both partitions of $n$,
    $$V_\lambda \otimes V_\mu = \bigoplus_{\text{$\nu$ partitions $n$}} c_{\lambda\mu}^\nu V_\nu.$$
\end{remark}

\begin{remark}
    Let $R_n := \on{Rep}(\mathfrak{S}_n)$ be the representation ring of 
    $\mathfrak{S}_n$. 
    Then,
    $$R = \bigoplus_{n=0}^\infty R_n,$$
    has the structure of a graded Hopf algebra.
    \begin{exercise}
        Think about this.
    \end{exercise}
\end{remark}

\subsection{Alternating Groups}

For $n\geq 5$, the alternating group $A_n$ is a simple group.
Recall that $A_n$ is a subgroup of $\mathfrak{S}_n$ of index $2$.
Generally, let $H$ be an index $2$ subgroup of $G$. Then, $H$ is normal, 
since $G/H = \lbrace 1,r\rbrace$ such that $rh = h'r$.
$G$ acts on $G/H$, and the corresponding permutation representation
is given by 
$$\C_{\on{triv}} \oplus \C_{\on{non-triv}}.$$
We wish to use this fact to deduce some facts about $A_n$-representations.

\begin{proposition}\label{prop19}
    Let $V$ be an irreducible representation of $G$ and 
    $W = \on{Res}_H^GV$. Then, one of the following holds:
    \begin{itemize}
        \item[(i)] If $V' = V \otimes \C_{\on{non-triv}}$,
            then $V\cong V'$, $W$ is irreducible, and 
            $$\on{Ind}_H^GW \cong V\oplus V'.$$
        \item[(ii)] If $V\cong V'$, then 
            $W = W'\oplus W''$, where $W'$, and $W''$ are 
            irreducible, non-isomorphic $H$-representations.
            Moreover,
            $$\on{Ind}_H^GW' \cong V \cong \on{Ind}_H^GW''.$$
        \item[(iii)] Every irreducible representation of $H$ 
            arises uniquely this way.
    \end{itemize}
\end{proposition}

\begin{proof}
    Let $\chi = \chi_V$ be the character of $V$. 
    Then, $\langle\chi\vert\chi\rangle=1$. 
    Thus,
    $$\sum_{h\in H} \vert\chi(h)\vert^2 + \sum_{g\not\in H}\vert \chi(g)\vert^2 = \vert G\vert = 2\vert H\vert,$$ 
    which implies that $$\vert H\vert\cdot \langle \chi,\chi\rangle_H + \sum_{g\not\in H} \vert \chi(g)\vert^2 = 2\vert H\vert.$$
    It follows then that 
    $\langle\chi,\chi\rangle_H$ is either $1$ or $2$.
    If $\langle\chi,\chi\rangle_H=1$, then $W$ is irreducible.
    Then,
    $$\on{Hom}_G(V,\on{Ind}_H^GW) = \on{Hom}_H(W,W) = 1,$$
    by hom-tensor adjunction.
    It follows then that
    $$\on{Ind}_H^GW \cong \on{Ind}_H^G\on{Res}_H^GV \cong V\oplus V',$$
    by a previous result.\\\\
    If $\langle\chi,\chi\rangle_H = 2$, then $W$ is not irreducible, and
    and has two distinct components in its decomposition:
    $W \cong W' \oplus W''$, where $W' \not\cong W''$ are irreducible.
    Then, $\sum_{g\not\in H}\vert\chi(g)\vert^2 = 0$, which implies that 
    $\chi(g) = 0$ for all $g\not\in H$.
    We thus have that 
    $$\chi_{V'}(g) = \begin{cases}
        \chi_V(g),\quad &\text{if} \quad g\in H,\\
        -\chi_V(g),\quad &\text{if} \quad g\not\in H.
    \end{cases},$$
    and 
    $$\chi_{\C_{\on{non-triv}}}(g) = \begin{cases}
        1\quad &\text{if}\quad g\in H,\\
        -1 \quad &\text{if}\quad g \not\in H.
    \end{cases}.$$
    It thus follows that $\chi_{V'}(g) = \chi_V(g)$, and 
    so $V\cong V'$.
    Then,
    $$\on{Hom}_G(V,\on{Ind}_H^GW) = \on{Hom}_H(W,W') = 1 = \on{Hom}_G(V,\on{Ind}_H^GW'').$$
    Now, let $W$ be any irreducible representation. Then, 
    $$\on{Res}_H^G\on{Ind}_H^GW \cong W\oplus W^r,$$
    where for each $r \in G/H$, $W^r$ is the representation defined by 
    $$\rho^r : H \longrightarrow \on{GL}(W),\quad h\longmapsto \rho(r^{-1}hr).$$
\end{proof}



\section{Lecture 3, 22/09/2023}

\subsection{Conjugacy Classes in Subgroups of Index $2$}

There are two types: for $C$ a conjugacy class in $H$, we have 
$C\subset G$ a conjugacy class in $G$, 
or $C\cup C' \subset H \subset G$ a conjugacy class in $G$.
The latter type of conjugacy class are called \emph{split conjugacy 
classes}.
In particular, $C = rC'r^{-1}$, for $1\neq r \in G/H$.
There is a way to detect which conjugacy class splits, which we will cover 
more next time. In particular, the conjugacy class of an element 
$x\in C$ splits if $Z_G(x) \subset H$. More on this next time.\\\\
For $G = \mathfrak{S}_n$, we any conjugacy class $C_\lambda$ ---
where $\lambda$ is a partition of $n$ --- splits into two conjugacy
classes in $A_n$ if and only if 
$\lambda = \lambda_1+\lambda_2 + \cdots + \lambda_k$, where 
$\lambda_i\neq \lambda_j$, $i\neq j$, and $\lambda_i$'s are odd.\\\\
Let $V_\lambda$ be an irreducible representation of $\mathfrak{S}_n$,
and $U' = \C_{\on{sgn}}$ --- the sign representation --- which is the 
the representation obtained from the action of $\mathfrak{S}_n$
on $\mathfrak{S}_n/A_n$.
Then, there is an isomorphism $$V_\lambda \otimes \C_{\on{sgn}} \cong V_{\lambda'},$$
where $\lambda'$ is the conjugate of $\lambda$ --- that is, it is the 
\emph{dual partition} of $\lambda$.\\\\
In case (i) of Proposition \ref{prop19}, if $\lambda\neq\lambda'$,
then 
$$W_\lambda = \on{Res}_{A_n}^{\mathfrak{S}_n}V_\lambda \cong \on{Res}_{A_n}^{\mathfrak{S}_n}V_{\lambda'},$$
is an irreducible representation of $A_n$.
Then, 
$$\chi_{W_\lambda} = \chi_{V_\lambda}\vert_{A_n}.$$
In case (ii) of Proposition \ref{prop19}, then 
$$\on{Res}_{A_n}^{\mathfrak{S}_n}V_\lambda \cong \on{Res}_{A_n}^{\mathfrak{S}_n}V_{\lambda}' \cong W_{\lambda}'\oplus W_\lambda'',$$
where $W_\lambda'\not\cong W_\lambda''$, are non-isomorphic irreducible
$A_n$-representations. Let us write,
$$\chi_\lambda' := \chi_{W_\lambda'}, \quad \chi_\lambda'' := \chi_{W_\lambda''}.$$
The partitions in case (i) are called \emph{self-dual partitions of $n$}.
What we have showed is that there is a bijection
\begin{equation}\label{eqn_self_conj_part}
    \lbrace \text{self-conjugate partitions of $n$}\rbrace \longleftrightarrow \lbrace \text{partitions of $n$ into sum of distinct odd parts}\rbrace.
\end{equation}
\begin{example}
    The partition given by:
    $$\lambda = \begin{ytableau}
        \ast & \ast & \ast & \ast & \ast\\
        \ast & \ast & \ast & \ast\\
        \ast & \ast & \ast\\
        \ast & \ast\\
        \ast
    \end{ytableau},$$
    is a self-conjugate partition.
    The mapping of the bijection is given by: 
    $$(\lambda_1,\cdots,\lambda_k) \longmapsto (2\lambda_1-1,2\lambda_2-3,\cdots,2\lambda_i-(2i-1),\cdots).$$
\end{example}

\begin{proposition}
    Let $C,C'$ be a pair of split conjugacy classes obtained from 
    $q_1>\cdots>q_r$, for $q_i$ odd. Suppose that $\lambda=\lambda'$.
    Then,
    \begin{itemize}
        \item[(i)] If $C$ and $C'$ do not correspond to $\lambda$
            under the bijection \eqref{eqn_self_conj_part},
            then
            $$\chi_\lambda' (C) = \chi_\lambda'(C') = \chi_\lambda''(C') = \chi_\lambda''(C) = \frac{1}{2}\chi_\lambda(C\cup C').$$
        \item[(ii)] If $C$ and $C'$ correspond to $\lambda$ 
            under \eqref{eqn_self_conj_part}, then
            $$\chi_\lambda'(C) = \chi_\lambda''(C') = a,\quad \chi_\lambda'(C') = \chi_\lambda''(C) = b,$$
            where $a,b$ are
            $$\frac{1}{2}\left( (-1)^m \pm \sqrt{(-1)^mq_1\cdots q_r}\right),\quad m = \frac{1}{2}(n-r).$$
    \end{itemize}
\end{proposition}

\begin{exercise}
    Prove this proposition using the following steps:
    \begin{itemize}
        \item[(i)] Determine $\chi_\lambda'$ and $\chi_\lambda''$ on a 
            non-split conjugacy class.
        \item[(ii)] Prove the above proposition (follow the steps in
            Fulton-Harris).
    \end{itemize}
\end{exercise}

\subsection{Aside: Nilpotent Orbits in $\mathfrak{gl}_n(\C)$}

The group $G=\on{GL}_n(\C)$ acts on the Lie algebra $\mathfrak{gl}_n(\C)$
by conjugation. These are in bijection with partitions of $n$. 
From this, one obtains \emph{Springer representations}:
given a nilpotent element $x\in \mathfrak{gl}_n(\C)$, we consider the 
a subvariety $B_x \subset G/B$, where $B$ is the Borel subgroup of $G$.
The subvariety $B_x$ is called the \emph{Springer fibre}, which under
this correspondence will be labelled by Young tableaux.

\subsection{Representations of Algebras}

Let $k$ be an algebraically closed field.

\begin{definition}
    An \emph{associative algebra} over $k$ is a $k$-vector space 
    $A$ with an associative bilinear multiplication 
    $$A\times A \longrightarrow A,\quad (a,b)\longmapsto ab.$$
    Further, we will always assume that $A$ contains a unit --- that is,
    $A$ is \emph{unital}.
\end{definition}

\begin{example}
    Let $V$ be a $k$-vector space. Then, $\on{End}(V)$ is a $k$-algebra
    with respect to composition of linear maps. Equivalently,
    $\on{End}(V)\cong \on{Mat}_{\dim_kV}(k)$ is an algebra with respect
    to matrix multiplication.\\\\
    For a finite group, the group algebra $k[G]$ is a $k$-algebra if 
    $\on{char}k$ does not divide $\vert G\vert$.\\\\
    The universal enveloping algebra $\mcal{U}(\mathfrak{g})$ of a Lie algebra
    $\mathfrak{g}$ is a unital, associative algebra.\\\\
    $k[x_1,\cdots,x_n]$ is a $k$-algebra.\\\\
    For a finite-dimensional $k$-vector space, the tensor algebra 
    $T(V)$ is a $k$-algebra.\\\\
    The \emph{Weyl algebra} is a $k$-algebra of differential operators 
    on a polynomial ring $k[x_1,\cdots,x_n]$.\\\\
    One can also have an algebra defined using generators and relations.
    That is, 
    $$A = k\langle x_1,\cdots,x_n\rangle /\langle f_1,\cdots,f_m\rangle.$$
    Hecke algebras are the \emph{best} examples of this. Diagram algebras,
    Temperley-Lieb algebras, and Brauer algebras are other examples
    of such algebras.
\end{example}

\begin{definition}
    A \emph{left $A$-module} (or, a \emph{representation}) of an 
    associative algebra $A$ is a $k$-vector space $V$ equipped with a
    homomorphism
    $$\rho : A \longrightarrow \on{End}_k(V).$$
    That is, it is a linear map preserving multiplication and unit.
    Similarly a right $A$-module is a $k$-vector space equipped with a right
    action, and a homomorphism
    $$\rho : A \longrightarrow \on{End}_k(V),\quad \rho(ab) = \rho(b)\rho(a),\quad \rho(1)=1.$$
    An \emph{$A$-submodule} (or $A$-subrepresentation) of $V$ is a subspace 
    $U\subset V$ such that $\rho(a)U\subset U$ --- that is, it is 
    stable under the action of any $a \in A$.
\end{definition}

\begin{remark}
    If $A$ is commutative, then $V$ clearly defines an $A$-bimodule.
\end{remark}

\begin{definition}
    A non-zero $A$-module $V$ is \emph{irreducible} (or \emph{simple}) 
    if its only submodules are $\lbrace 0\rbrace$ and $V$.
    $V$ is \emph{indecomposable} if it cannot be written as a direct sum 
    of two non-zero submodules.
\end{definition}

\begin{remark}
    Irreducibility implies indecomposability, but the converse is not true 
    in general. That is, irreducibility is a much stronger condition than
    indecomposability.
\end{remark}

\begin{example}
    Consider $A = \C[t^\pm]$. Then, the two-dimensional representation given by 
    $$t\longmapsto \begin{pmatrix}
        1&1\\
        0&1
    \end{pmatrix},$$
    defines an indecomposable representation that is not irreducible,
    since $\C[t^\pm]\cong \C[\mathbb{Z}]$, and we know that all
    irreducible representations of $\mathbb{Z}$ are one-dimensional.
\end{example}

\begin{example}
    Let $A = k[x]$.
    Then, the representation given by sending $x$ to its Jordan decomposition
    given by block diagonal matrices of the form $\begin{pmatrix}
        \lambda & 1\\
        0 & \lambda
    \end{pmatrix}.$ This is an example of an indecomposable representation.
    Calling this representation $V_{\lambda,n}$, we see that 
    $V_{\lambda,n}\cong k^n$.
\end{example}

\chapter{Week Ten}

\section{Lecture 1, 02/10/2023} 

\begin{example}
    Let $A$ be a $k$-algebra, where $k$ is algebraically closed.
    \begin{enumerate}
        \item (Regular Representation) Let $V=A$, which defines an $A$-module
            structure by $\rho : A \to \on{End}_k(A)$, given by $\rho(a)b=ab$.
        \item Let $A=k$. Then, a $k$-module is given by a vector space 
            over $k$.
        \item Let $A = k\langle x_1,\cdots,x_n\rangle$, a free algebra
            generated by $n$ elements.
            Then, an $A$-module is a vector space over $k$, 
            with $\rho(x_i) : V \to V$ for each generator $x_i$.
    \end{enumerate}
\end{example}

\begin{definition}
    Let $V_1$, and $V_2$ be representations of $A$. Then, $A$ acts on the 
    direct sum $V_1\oplus V_2$ by $a(v_1+v_2) = av_1+av_2$, for 
    $a \in A$, $v_1\in V_1$, and $v_2\in V_2$.
    An \emph{$A$-homomorphism} $\phi : V_1 \to V_2$ is a linear map 
    such that $\phi(av_1) = a\phi(v_1)$ for $a\in A$, $v_1\in V_1$ --- that is,
    $\phi$ \emph{intertwines} the $A$-action.
    The map $\phi$ is an \emph{isomorphism} if it is also an isomorphism
    of $k$-vector spaces.
\end{definition}

Then, as before, we may write 
$$\on{Hom}_A(V_1,V_2) := \lbrace \text{homomorphisms of $A$-modules $V_1\to V_2$}\rbrace.$$

\subsection{Schur's Lemma for $A$-modules}

\begin{theorem}[Schur's Lemma]
    Let $V_1$, $V_2$ be $A$-modules, and $\phi : V_1 \to V_2$ a non-trivial
    $A$-homomorphism. Then,
    \begin{itemize}
        \item[(i)] If $V_1$ is irreducible, then $\phi$ is injective,
        \item[(ii)] If $V_2$ is irreducible, then $\phi$ is surjective,
        \item[(iii)] If $V_1$ and $V_2$ are irreducible, then 
            $\phi$ is an isomorphism,
    \end{itemize}
    Moreover, if $V$ is a finite-dimensional irreducible $A$-module, then
    for any $A$-homomorphism $\phi : V \to V$, we have that 
    $\phi = \lambda \on{id}_V$, where $\lambda \in k$ (note that $k$ is 
    algebraically closed).
\end{theorem}

\begin{proof}
    Basically same as proof of Schur's lemma for finite groups.
\end{proof}

\begin{corollary}
    Let $A$ be a commutative algebra. Then, every irreducible finite-dimensional
    $A$-module is one-dimensional.
\end{corollary}

\begin{proof}
    Let $V$ be a finite-dimensional irreducible $A$-module, 
    and for each $a\in A$, define $\phi_a : V \to V$ by 
    $v\mapsto av$. Then, $\phi_a$ is a homomorphism, and by 
    Schur's lemma, we have that $\rho_a = \lambda_a\on{id}_V$ for all $a\in A$.
\end{proof}

\begin{example}
    Let $A = k[x]$. This is a commutative $k$-algebra. Let $V = k$,
    and define a map 
    $\rho_\lambda : A \to k$ mapping $x\mapsto \lambda$, for some $\lambda\in k$.
    Then, the indecomposable representations of $A$ are given by 
    $$V_{\lambda,n} = k^n,$$
    following the theory of Jordan normal forms. The indecomposables then take
    the form $\rho(x)$, which is a matrix with $\lambda$ on the diagonals,
    and $1$'s on the superdiagonals.
    \begin{exercise}
        Show this.
    \end{exercise}

    \begin{proof}
        The representation mapping $x$ to the $n\times n$ matrix with 
        $\lambda$ on 
        the diagonals, and $1$ on the superdiagonals is indecomposable,
        but not irreducible, because all irreps of $k[x]$ should be 
        one-dimensional. Call this representation
        $\rho : k[x] \to \on{End}_k(V_{\lambda,n})$.
        It follows from the representation that we defined that 
        $V_{\lambda,n}\cong k^n$.
    \end{proof}
\end{example}

\subsection{Ideals}

\begin{definition}
    A \emph{left (resp. right) ideal} of a $k$-algebra $A$ is a subspace 
    $I\subseteq A$ such that $aI\subseteq I$ for all $a\in I$.
    A \emph{two-sided ideal} is a subspace that is both a left and right
    ideal.
\end{definition}

\begin{example}
    \leavevmode
    \begin{enumerate}
        \item Given a subset $S\subset A$, we can define a two-sided 
            ideal generated by $S$ given by $$\langle S\rangle = \on{Span}_k\lbrace asb : a\in A, s\in S, b\in A\rbrace.$$
        \item For any homomorphism $\phi :A \to B$ of $k$-algebras,
            the kernel $\ker\phi$ form a two-sided ideal of $A$.
    \end{enumerate}
\end{example}

\begin{definition}
    An algebra $A\neq 0$ is \emph{simple} if the only ideals are 
    $0$ and $A$ --- that is, its ideals are the trivial ideals.
\end{definition}

\begin{exercise}
    Show that $\on{Mat}_n(k)$ is simple.
\end{exercise}

\begin{proof}
    Choose a basis of matrices $\lbrace E_{ij}\rbrace_{i,j=1}^n$, where 
    $E_{ij}$ is the $n\times n$ matrix with $1$ in the $(i,j)$ position, and
    $0$ elsewhere.
    Let $I$ be a non-zero ideal of $\on{Mat}_n(k)$, and let $A \in I$.
    Then, there exists indicies $1\leq s,t\leq n$ such that 
    $A_{st} \neq 0$. Then, $E_{ss}A$ produces a matrix that kills all
    the rows except for the $s$-th row. Multiplying by $E_{tt}$ on the right
    kills everything except for the $t$.
    Thus, $E_{ss}\cdot A\cdot E_{tt} = A_{st} \cdot E_{st}$, which is in $I$.
    It follows then that $I$ must necessarily contain all the matrices in 
    $\on{Mat}_n(k)$, since the matrices $E_{ij}$ form a basis for 
    $\on{Mat}_n(k)$.
\end{proof}

\subsection{Quotients}

Let $A$ be a $k$-algebra, and $I$ a two-sided ideal. Then, show that 
$A/I$ is a $k$-algebra, with elements given by $a+I$.
Given an $A$-module $V$, and a subrepresentation $W\subset V$.
Then, we may form an $A$-module given by $V/W$.

\begin{example}
    There is a bijection
    $$\lbrace \text{subrepresentations of regular $A$-modules} \longleftrightarrow \lbrace \text{left ideals of $A$}\rbrace.$$
\end{example}

\begin{exercise}
    Show that if $A=k[x_1,\cdots,x_n]$, and let $I$ be the ideal containing
    all homogeneous polynomials of degree $\geq N$. Show that $A/I$ is 
    an indecomposable $A$-module. (If you find this difficult, start with
    one variable first).
\end{exercise}

\begin{proof}
    We can induct on the number of variables in $A$. Let us first consider 
    $A = k[x]$. Let $I$ be the ideal containing all homogeneous polynomials
    of degree $\geq N$. Then, we can decompose $k[x]/I$ by degree in the 
    following way:
    $$A/I = \bigoplus_{n\geq N} k[x]/\langle x^n\rangle.$$
    The result follows from showing that each $k[x]/\langle x^n\rangle$
    is indecomposable. 
    We will show that each ideal of $k[x]/\langle x^n\rangle$ intersects 
    non-trivially. Suppose that $N$ is a non-trivial $k[x]$-module.
    Then, there exists some $$n:= \alpha_j x^j + \cdots + \alpha_{n-1} x^{n-1} +\langle x^n\rangle \in N.$$
    But $$n\alpha_j^{-1}x^{n-1-j} = x^{n-1} + \langle x^n\rangle,$$
    which implies that every ideal $N$ of $k[x]/\langle x^n\rangle$
    contains $x^{n-1} + \langle x^n\rangle$. Thus, each $k[x]/\langle x^n\rangle$
    is indecomposable.\\\\
    Now, let us induct on the number of variables, let $I_n := k[x_1,\cdots,x_n]_{\deg = n}$
    be the $k[x]$-submodule of all polynomials of degree $n$.
    Then, $$A/I = \bigoplus_{\ell\geq N} k[x_1,\cdots,x_n]/I_\ell.$$
    The same argument as before shows that $A/I$ is indecomposable, 
    by replacing $x^\ell$ with $(x_1^{i_1}\cdots x_n^{i_n})^\ell$, 
    with the condition that $i_1 + \cdots + i_n = \ell$ in the previous 
    argument.
\end{proof}

\subsection{Algebras defined by generators and relations}

These are $k$-algebras $A$ of the form
$$A = \frac{k\langle x_1,\cdots,x_n\rangle}{\langle f_1,\cdots,f_m\rangle}.$$

\begin{example}
    The Weyl algebra is the algebra given by 
    $$A = \frac{k\langle x,y\rangle}{\langle yx-xy-1\rangle}.$$
\end{example}

\begin{proposition}
    A basis for the Weyl algebra is given by
    $$\lbrace x^iy^j :  i,j\geq 0 \rbrace.$$
\end{proposition} 

\begin{proof}
    It is clear that elements of this form span the Weyl algebra. It thus 
    suffices to show linear independence. Let 
    $$E := t^ak[a][t,t^{-1}],$$
    where $a$ is a variable.
    Then, for any $f\in E$, we set the relation 
    $$xf = tf, \quad yf = \frac{\partial f}{\partial t}.$$
    One checks that $(xy-yx)f = f$, and thus 
    $E$ defines a module over the Weyl algebra. Suppose that 
    $$\sum_{i,j} G_j x^iy^j  = 0.$$
    Then, $$L = \sum_{i,j}C_{ij}t^i \left(\frac{\partial}{\partial t}\right)^j,$$
    acts by $0$ on $E$.
    Write 
    $$L = \sum_{j=0}^r Q_j(t) \left(\frac{\partial}{\partial t}\right)^j,$$
    where $Q_r(t) \neq 0$. 
    It follows then that $Lt^a = \sum_{j=0}^r Q_j(t) a(a-1)\cdots (a-j+1)t^{a-j}=0$,
    which implies then that 
    $a^rQ_r(t)t^{a-r} + \cdots = 0$, and so
    $Q_r(t) = 0$, which is a contradiction.
\end{proof}

\begin{definition}
    A representation $\rho : A \to \on{End}_k(A)$ of $A$ is \emph{faithful}
    if $\rho$ is injective.
\end{definition}

\begin{example}
    Let $A = k[t]$.
    Let $\on{char}k  =0$. Then, $E$ is faithful.
    But if $\on{char}k = p$ for $p$ a prime, then $E$ is not faithful.
    \begin{exercise}
        Show this.
    \end{exercise}
    \begin{proof}
        In characteristic $p$, if $0$ is in the kernel, then any integral
        multiple of $p$ is also in the kernel, and thus $E$ 
        is not faithful.
    \end{proof}
\end{example}

\section{Lecture 2, 05/10/2023}

\subsection{Semisimple $A$-modules}

As before, $A$ is a $k$-algebra, where $k$ is an algebraically closed field.

\begin{definition}
    A \emph{semisimple} (or \emph{completely reducible}) $A$-module is an
    $A$-module that admits a direct sum decomposition into irreducible
    $A$-modules: $$A = \bigoplus_i A_i,$$
    where each $A_i$ is an irreducible $A$-module.
\end{definition}

\begin{example}
    Let $V$ be an irreducible $A$-module of dimension $n$ over $k$.
    Then, there is a $k$-algebra homomorphism $\rho : A \to \on{End}_k(V)$.
    Moreover, there is an $A$-module isomorphism:
    $$\on{End}_k(V) \cong V^{\oplus n},\quad x\longmapsto (xv_1,\cdots,xv_n),$$
    where $\lbrace v_1,\cdots,v_n\rbrace$ is a basis of $V$.
    It follows then that $\on{End}_k(V)$ is a semisimple $A$-module.
    \begin{exercise}
        Check that this is an $A$-module isomorphism.
    \end{exercise}
    \begin{proof}
        Injectivity is clear, unless $k$ has prime characteristic, in which case
        it is not injective. We can write $V$ as a direct sum
        $$\on{End}_k(V) = \on{Hom}_k(V,V) = \on{Hom}_k\left(\bigoplus_i\C v_i,\bigoplus_j \C v_j\right) = \bigoplus_{i,j}\on{Hom}_k(\C v_i,\C v_j),$$
        and thus $\on{End}_k(V)$ takes basis elements of $V$ to basis elements
        of $V$. It follows then that the map is surjective. Thus, 
        $\on{End}_k(V) \to V^{\oplus n}$ is an isomorphism.
    \end{proof}
\end{example}

\begin{exercise}
    If $V$ is a semisimple, finite-dimensional $A$-module, 
    show that there is an isomorphism of $A$-modules:
    $$V \cong \bigoplus_{W_i \in \on{Irr}(A)} \on{Hom}_A(W_i,V)\otimes_k W_i.$$
\end{exercise}

\begin{proof}[Sketch Outlined in Class]
    Define a map $$\sum_i f_i\otimes w_i \longmapsto \sum_i f_i(w_i),$$
    and apply Schur's lemma. 
    Observe that $\bigoplus_{W_i \in \on{Irr}(A)} \on{Hom}_A(W_i,V)$ is the 
    multiplicity space --- that is,
    $$\bigoplus_{W_i\in\on{Irr}(A)} \on{Hom}_A(W_i,V)\otimes_kW_i \cong \bigoplus_{W_i\in\on{Irr}(A)} W_i^{\oplus \dim\on{Hom}_A(W_i,V)}.$$
\end{proof}

\begin{proof}
    Previously, we proved that $$\on{Hom}_k(V,V') \cong V^\ast \otimes_kV,$$
    in an exercise. Using this fact, we have:
    $$\on{Hom}_A(W_i,V) \otimes_k W_i \cong W_i^\ast \otimes_A V \otimes_k W_i \cong \on{Hom}_k(V^\ast \otimes_A W_i, W_i),$$
    and then by hom-tensor adjunction, we have:
    $$\on{Hom}_k(V^\ast\otimes_A W_i,W_i) \cong \on{Hom}_k(V, \on{Hom}_A(W_i,W_i)) \cong \on{Hom}_k(V, k) = V^\ast \cong V,$$
    where the second isomorphism follows by Schur's lemma, and the 
    last isomorphism follows by choosing a basis $v^i$ for $V^\ast$
    acting on $V$ by $v^i(v_j) = \delta_{ij}$ --- it is a non-canonical 
    isomorphism. Since $V^\ast$ can be equipped with the structure of an 
    $A$-module, it follows that $V^\ast \cong V$ is also an $A$-module
    isomorphism.
\end{proof}

Let $$V = \bigoplus_{W_i \in \on{Irr}(A)} V_{W_i} \otimes_kW_i,$$
where $V_{W_i}$ is the multiplicity space as in the above 
exercise,
and $U = \bigoplus_{W_i\in \on{Irr}(A)} U_{W_i} \otimes_kW_i,$
then $$\on{Hom}_A(V,U) \cong \bigoplus_{W_i \in \on{Irr}(A)} \on{Hom}_k(V_{W_i},U_{W_i}), \quad f \longmapsto (f_i).$$
Then, $f$ is injective, surjective, or an isomorphism if and only if 
all $f_i$ are also injective, surjective, or an isomorphism, respectively.

\begin{lemma}\label{lem14}
    Suppose that $V = \bigoplus_{i\in I}V_i$, where the $V_i$'s are irreducible
    $A$-modules, and assume that we have a surjective $A$-module 
    homomorphism $f : V \to U$. Then, there exists a subset $J\subset I$ 
    such that 
    $V_J := \bigoplus_{j \in J} V_j$ is mapped isomorphically 
    onto $U$ by $f$.
\end{lemma}

\begin{proof}
    Let $J\subseteq I$ be a maximal subset such that $f\vert_{V_J}$ 
    is injective. If $f(V_J) \neq U$, then there exists 
    $i\not\in J$ such that $f(V_i) \not\subseteq f(V_J)$. Then, 
    $$\bar{f} : V_i \longrightarrow U/f(V_J),$$
    is such that $\bar{f} \neq 0$. Then, by Schur's lemma, 
    it follows that $\bar{f}$ is injective, and thus 
    $f$ is injective on $\bigoplus_{j \in J\cup \lbrace i \rbrace} V_j,$
    which is a contradiction.
\end{proof}

\begin{proposition}\label{prop22}
    Let $V_i$, for $1\leq i \leq n$ be irreducible, finite-dimensional
    $A$-modules that are pairwise non-isomorphic, and let 
    $W\subset V = \bigoplus_{i=1}^m V_i^{\oplus n_i}$, be a 
    $A$-submodule. Then, $W$ is semisimple and 
    $W \cong \bigoplus_{i=1}^m V_i^{\oplus r_i}$.
    Moreover, the embedding $\phi : W \hookrightarrow V$ is given by 
    $$\phi_i : V^{\oplus r_i} \longrightarrow V_i^{\oplus n_i}, \quad (v_1,\cdots,v_{r_i}) \longmapsto (v_1,\cdots,v_{r_i})X_i,$$
    where $X_i$ is a $r_i \times n_i$ matrix with linearly independent rows.
    That is, $\phi$ restricts isotypic components.
\end{proposition}

\begin{proof}
    Let $W$ be an $A$-submodule of $V \cong \bigoplus_{W_i\in \on{Irr}(A)}V_{W_i}\otimes_k W_i$. Consider 
    $$f : V \longrightarrow V/W.$$
    Then, by Lemma \ref{lem14}, 
    $$V/W \cong \bigoplus_{W_i} X_{W_i}\otimes_k W_i,$$
    which implies that $$W = \ker f \cong \bigoplus_{W_i} \ker(V_{W_i}\to X_{W_i}) \otimes_k W_i,$$
    by the above discussion about the addivity of $\on{Hom}_k(-,-)$.
\end{proof}

\begin{corollary}\label{cor21}
    Let $V$ be a finite-dimensional, irreducible $A$-module, and let 
    $v_1,\cdots,v_n \in V$ be linearly independent. Then, for all
    $w_1,\cdots,w_n \in V$ that are also linearly independent, there exists 
    $a\in A$ such that $av_i = w_i$ for each $i$.
\end{corollary}

\begin{proof}
    Suppose otherwise. Consider an $A$-module homomorphism 
    $$f : A \longrightarrow V^{\oplus n}, \quad a\longmapsto (av_1,\cdots,av_n).$$
    It follows then that $\on{Im}(f) \cong V^{\oplus r}$, where
    $r < n$. By Proposition \ref{prop22}, there exists some 
    $u_1,\cdots,u_r \in V$ such that 
    $(u_1,\cdots,u_r) X = (v_1,\cdots,v_n)$, and 
    $X$ is a $r\times n$ matrix. Since $r < n$, it follows that 
    there is some $(q_1,\cdots,q_n)$ such that 
    $X(q_1,\cdots,q_n)^t = 0$, since the rank of $X$ is at most $r$, and so 
    $(v_1,\cdots,v_n) (q_1,\cdots,q_n)^t = 0$, which is a contradiction.
\end{proof}

\begin{theorem}[Density Theorem]
    \leavevmode
    \begin{itemize}
        \item[(i)] Let $V$ be a finite-dimensional irreducible $A$-module.
            Then, $$\rho : A \longrightarrow \on{End}_k(V),$$
            is surjective.
        \item[(ii)] Let $V_1,\cdots, V_r$ be pairwise non-isomorphic
            finite-dimensional irreducible $A$-modules. Then,
            $$\rho = (\rho_1,\cdots,\rho_r) : A \longrightarrow \prod_{i=1}^r \on{End}_k(V_i),$$
            is surjective.
    \end{itemize}
\end{theorem}

\begin{proof}
    \leavevmode
    \begin{itemize}
        \item[(i)] Let $x \in \on{End}_k(V)$, and $v_1,\cdots,v_n$ a basis of 
            $V$.
            Then, by Corollary \ref{cor21}, there exists $a\in A$ such that 
            $\rho(a)v_i = w_i$. It follows then that $\rho(a) = x$,
            and thus $\rho$ is surjective.
        \item[(ii)] Let $B_i := \on{Im}(\rho_i) = \on{End}_k(V_i)$
            by part (i). Let $B = \on{Im}\rho$.
            Then, it follows that 
            $B \subset \prod_{i=1}^rB_i$. But as an $A$-module,
            $$\prod_{i=1}^r \on{End}_k(V_i) \cong \bigoplus_{i=1}^rV_i^{\oplus d_i}.$$
            Thus, we may compose $\rho$ with a projection,
            $$A \longrightarrow \prod_{i=1}^r\on{End}_k(V_i) \longrightarrow \on{End}_k(V_i),$$
            so that $B = \prod_{i=1}^rB_i$, and thus $\rho$ is surjective.
    \end{itemize}
\end{proof}

\section{Lecture 3, 06/10/2023}

Recall from last time, we defined representations of algebras.

\begin{definition}[Dual Representations]
    Given a representation $V$ of a unital $k$-algebra $A$, 
    the \emph{dual representation} is given by 
    $V^\ast := \on{Hom}_k(V,k)$ has the structure of a right
    $A$-module structure given by 
    $$(f\cdot a)(v) = f(a\cdot v),\quad f\in V^\ast,\,\,a\in A, \,\,v\in V.$$
    Alternatively, one may think of $V^\ast$ as a representation of 
    the opposite algebra $A^{\on{op}}$, which has opposite multiplication
    $a\ast b  = ba$.
\end{definition}

\begin{theorem}\label{thm24}
    Let $A = \prod_{i=1}^r\on{Mat}_{d_i}(k)$. Then, the irreducible 
    representations of $A$ are 
    $$V_1 = k^{d_1},\cdots, V_r = k^{d_r},$$
    each of which obtains an $A$-module structure by multiplication.
    Moreover, any finite-dimensional representation of $A$ is isomorphic 
    to $\bigoplus V_i^{\oplus n_i}$.
\end{theorem}

\begin{proof}
    First, we show that the $V_i$'s are irreducible. It suffices to show that 
    for all $0\neq v\in V_i$, we have $Av = V_i$. 
    \begin{exercise}
        Show that for all $w\in V_i$, there exists $a\in A$ 
        such that $av = w$.
    \end{exercise}
    Now, let us suppose that $V$ is an $n$-dimensional representation of 
    $A$. Then, $V^\ast$ is an $n$-dimensional representation of 
    $A^{\on{op}}$. Note that there is a $k$-algebra isomorphism
    $$\on{Mat}_{d_i}(k)^{\on{op}} \stackrel{\simeq}{\longrightarrow} \on{Mat}_{d_i}(k),\quad X \longmapsto X^T.$$
    It follows then that $A^{\on{op}}\cong A$. Thus, we may think of 
    $V^\ast$ as an $A$-representation.
    Define a surjection
    $$\phi : \underbrace{A\oplus \cdots \oplus A}_{\text{$n$ times}} \longrightarrow V^\ast,\quad (a_1,\cdots,a_n) \longmapsto a_1e_1 + \cdots + a_ne_n,$$
    where $\lbrace e_1,\cdots,e_n\rbrace$ is a basis of $V^\ast$.
    Indeed, $\phi$ is a surjection since $k\subset A$. 
    Moreover, the map $\phi$ induces an injective morphism of 
    $A$-representations:
    $$\phi^\ast : V \longrightarrow (A\oplus \cdots \oplus A)^\ast.$$
    \begin{exercise}
        Show that $A^n \cong (A^n)^\ast$ as $A$-representations.
    \end{exercise}
    It follows then that $\phi^\ast(V) \cong V \cong \bigoplus_{i=1}^r V_i^{\oplus m_i}$,
    since $A^n \cong \bigoplus_{i=1}^r V_i^{\oplus nd_i}$.

\end{proof}

\subsection{Filtrations}

We have seen in our study of algebra representations that it is not 
necessarily the case that every algebra representation is completely
reducible. So, another way to study their representations is to consider
filtrations of representations.

\begin{definition}
    Let $A$ be a $k$-algebra. A \emph{(finite) filtration} of an 
    $A$-representation $V$ is a sequence of subrepresentations
    $$0 = V_0 \subset V_1 \subset \cdots \subset V_n = V.$$
\end{definition}

\begin{lemma}\label{lem14}
    Every finite-dimensional representation $V$ of a $k$-algebra $A$ 
    admits a finite filtration
    $$0 = V_0 \subset V_1 \subset \cdots \subset V_n = V,$$
    such that $V_i/V_{i-1}$ is irreducible for any $i$.
\end{lemma}

\begin{proof}
    We induct on $\dim_kV$. Let $V_1\subset V$ be an irreducible 
    $A$-representation. Consider $U = V/V_1$, which defines an
    $A$-representation. Then, by induction, we obtain a filtration
    for $U$ given by
    $$0\subset U_1 \subset \cdots\subset U_{m}\subset U,$$
    such that $U_i/U_{i-1}$ are irreducible.
    There is a projection $\pi : V \to V/V_1 = U$.
    From this, one may produce a filtration for $V$ 
    by setting $V_i := \pi^{-1}(U_i)$.
    Note here that $V_i/V_{i-1} \cong U_i/U_{i-1}$.
    \begin{exercise} 
        Fill in the details.
    \end{exercise}
    \begin{proof}
        We induct on $U$. If $U$ is simple, then we are done. Suppose that 
        $U$ is non-simple. Then, we have $0\subset U_1 \subset U$. 
        If $U/U_1$ is not irreducible, then put another 
        submodule $0 \subset U_1 \subset U_2 \subset U$ until we have a 
        filtration
        $$0 \subset U_1 \subset \cdots \subset U_m \subset U,$$
        such that $U_i/U_{i-1}$ is simple.
        Now, let us set $V_i = \pi^{-1}(U_i)$, where
        $\pi : V \to V/V_1 = U$ is the projection map. Then, we have 
        that $$V_i/V_{i-1} = \pi^{-1}(U_i)/\pi^{-1}(U_{i-1}) \cong U_i/U_{i-1},$$
        and thus we have a filtration 
        $$0 \subset V_1 \subset \cdots \subset V_m \subset V,$$
        such that the factors $V_i/V_{i-1}$ are irreducible.
    \end{proof}
\end{proof}

\subsection{Finite-Dimensional Algebras}

Let $A$ be a finite-dimensional $k$-algebra.

\begin{definition}
    The \emph{radical} of a finite-dimensional $k$-algebra $A$ is the 
    set of all $a\in A$ such that $aV_i=0$, for any irreducible 
    $A$-representation $V_i$. Denote this by $\on{Rad}(A)$. 
\end{definition}

\begin{proposition}
    $\on{Rad}(A)$ is a two-sided ideal of $A$.
\end{proposition}

\begin{proposition}
    \leavevmode
    \begin{itemize}
        \item[(i)] Let $I$ be a nilpotent two-sided ideal in $A$ --- 
            i.e. $I^n = 0$ for some $n$. Then, $I\subset \on{Rad}(A)$.
        \item[(ii)] $\on{Rad}(A)$ is the maximal two-sided nilpotent ideal
            in $A$.
    \end{itemize}
\end{proposition}

\begin{proof}
    \leavevmode
    \begin{itemize}
        \item[(i)] Let $V$ be an irreducible $A$-representation. 
            Let $0\neq v\in V$. Then, $Iv$ us a subrepresentation of 
            $V$. But by irreducibility of $V$, we have that 
            $Iv = V$, since $v\neq 0$.
            Thus, there exists some $x\in I$ such that $xv=v$, which implies
            that $x^nv = v = 0$, which is a contradiction.
        \item[(ii)] Let 
            $$0 = A_0 \subset A_1 \subset \cdots \subset A_n = A,$$
            be a filtration of the regular representation of $A$, 
            such that $A_i/A_{i-1}$ is irreducible for $1\leq i \leq n$.
            Such a filtration exists by Lemma \ref{lem14}.
            Then, by definition of the radical of $A$, we have:
            $$\on{Rad}(A)A_i/A_{i-1} = 0.$$
            It therefore follows that $\left(\on{Rad}(A)\right)^nA = 0$.
            But since $A$ has an identity, this then implies that 
            $\left(\on{Rad}(A)\right)^n = 0$.
    \end{itemize}
\end{proof}

\begin{example}
    Let $A := k[x]/(x^n)$, which is a finite-dimensional commutative 
    $k$-algebra. Then, $A$ has a unique irreducible representation
    given by $xv = 0$, for all $v\in k$.
    Here, we have that 
    $$\on{Rad}(A) = (x).$$
    Thus, $$A/\on{Rad}(A) \cong k.$$
\end{example}

\chapter{Week Eleven}

\section{Lecture 1, 09/10/2023}

Today, we shall prove the following theorem:

\begin{theorem}\label{thm25}
    A finite-dimensional $k$-algebra $A$ has only finitely many 
    irreducible representations $V_i$ (up to isomorphism).
    Moreover, these irreducible representations are finite-dimensional,
    and $$A/\on{Rad}(A) \cong \prod_i \on{End}_k(V_i) \cong \prod_i \on{Mat}_{\dim_k V_i}(k).$$
\end{theorem}

\begin{proof}
    We begin by first showing that every irreducible $A$-module is 
    finite-dimensional. Let $V$ be an irreducible $A$-module. 
    Let $0\neq v \in V$. Then, $Av$ is a subrepresentation of 
    $V$. But by the irreducibility of $V$, then it follows necessarily
    that $Av=V$, since $Av\neq 0$ since $v\neq 0$. Since $A$ is 
    finite-dimensional, it follows that $V$ is finite-dimensional. \\\\
    We now wish to show that there are only finitely many irreducible 
    $A$-modules. Suppose that $V_1,\cdots,V_r$ are 
    non-isomorphic irreducible $A$-modules. Then, by the density theorem, 
    the map $A \to \prod_{i=1}^r \on{End}(V_i)$ is surjective, 
    and it thus follows that $$r \leq \sum_{i=1}^r \dim_k \on{End}(V_i)\leq \dim_kA,$$
    where the inequaltiy follows by the surjectivity of the map. It thus 
    follows that there are at most $\dim_kA$ many irreducible $A$-modules.\\\\
    Let $V_1,\cdots,V_k$ be all non-isomorphic representations of $A$.
    Then, again by the density theorem there is a surjective $k$-algebra 
    homomorphism
    $$A \longrightarrow \prod_{i=1}^k \on{End}(V_i),$$
    whose kernel is $\on{Rad}(A)$. It follows by the first isomorphism
    theorem that 
    $$A/\on{Rad}(A) \cong \prod_{i=1}^k\on{End}_k(V_i).$$
\end{proof}

\begin{corollary}
    Let $V_1,\cdots,V_k$ be all the irreducible $A$-modules up to isomorphism.
    Then,
    $$\sum_{k=1}^k (\dim_kV_i)^2 \leq \dim_kA.$$
\end{corollary}

\begin{example}
    Let $A$ be the subalgebra of $\on{Mat}_n(k)$ consisting of the upper 
    triangular $n\times n$ matrices. 
    Then, we claim that $\on{Rad}(A)$ is the subalgebra of strictly upper 
    triangular matrices with $0$'s on the diagonal.
    An irreducible $A$-module is given by 
    $$\rho : A \longrightarrow \on{End}_k(k) \cong k,$$
    sending any element of $A$ to any entry on its diagonal.
    That is $\rho : (x_{ij})_{i\geq j} \mapsto x_{kk}$, $1\leq k \leq n$.
    These give all the irreducible representations.
    It follows then that 
    $$A/\on{Rad}(A) \cong k^n.$$
\end{example}

\begin{definition}
    A finite-dimensional $k$-algebra $A$ is \emph{semisimple} if
    $\on{Rad}(A) = 0$.
\end{definition}

\begin{proposition}
    For a finite-dimensional $k$-algebra $A$, the following are equivalent:
    \begin{itemize}
        \item[(i)] $A$ is semisimple,
        \item[(ii)] Let $V_1,\cdots,V_k$ be all the irreducible 
            $A$-modules up to isomorphism. Then,
            $$\sum_{i=1}^k (\dim_kV_i)^2 = \dim_kA,$$
        \item[(iii)] Let $d_i := \dim_kV_i$. Then,
            $$A \cong \prod_{i=1}^k\on{Mat}_{d_i}(k),$$
        \item[(iv)] Every finite-dimensional $A$-module is completely
            reducible --- that is, isomorphic to a direct sum of 
            irreducible representations,
        \item[(v)] Every $A$-submodule $U$ of a finite-dimensional
            $A$-module $V$ has a complementary $A$-submodule $W$ --- i.e.
            $V\cong U \oplus W$,
        \item[(vi)] $A$ is a completely reducible $A$-module.
    \end{itemize}
\end{proposition}

\begin{proof}
    The (i) $\Longleftrightarrow$ (ii) direction follows from Theorem 
    \ref{thm25}.
    Theorem \ref{thm25}, \ref{thm24} also implies (ii) $\Longleftrightarrow$ 
    (iii).
    (iii) $\implies$ (iv) follows also from Theorem \ref{thm24}.\\\\
    (iv) $\implies$ (v).  Let $V$ be a finite-dimensional representation of 
    $A$. Let $U\subset V$ be a strict $A$-submodule. Let 
    $$M := \lbrace \text{subrepresentations $X\subseteq V$ such that $X\cap U =\lbrace0\rbrace$}\rbrace.$$
    Since $\lbrace 0 \rbrace \in M$, it follows that $M\neq\emptyset$.
    Let $W\in M$ be of largest possible dimension.\\\\
    We now claim that $V=U \oplus W$.
    Indeed, $U\oplus W \subseteq V$ by construction. Suppose that 
    $U\oplus W \neq V$. Then, by the complete reducibility of $V$,
    there exists an element a simple $A$-submodule
    $S\subseteq V$ such that $S \not\subseteq U\oplus W$.
    It follows then that $S \cap (U\oplus W) = \lbrace 0 \rbrace$, and 
    thus $S \oplus W \in M$, which is a contradiction, since $W$ is 
    of maximal dimension by assumption.
    \begin{exercise}
        Show that (v) $\implies$ (iv).
    \end{exercise}
    \begin{proof}
        Suppose that every $A$-submodule $U$ of a finite-dimensional $A$-module 
        $V$ has a complementary $A$-submodule $W$ such that $V = U\oplus W$.
        Then, if $U$ and $W$ are simple we are done. \\\\
        Thus, suppose that 
        $U$ is not simple. Then, every submodule of $U$ has a complementary
        submodule as well. That is $U = U_1 \oplus U_2$. Repeat this
        process for each of the factors in the decomposition of $U$ 
        until we can no longer find non-trivial submodules --- that is,
        we have only irreducible factors. Repeat this for $W$ as well, and 
        we find that $V$ decomposes as a direct sum of irreducible
        $A$-submodules.
    \end{proof}
    (vi) $\implies$ (v) is trivial.
    It thus remains to show (vi) $\implies$ (iii). 
    Suppose that $$A \cong \bigoplus_{i=1}^m V_i^{\oplus n_i},$$
    where the $V_i$'s are irreducible $A$-modules. Note that a priori, we 
    do not know if the factors appearing in the decomposition of $A$ 
    is a complete list of irreducible $A$-representations.\\\\
    Since $A$ is itself an $A$-module, we may consider
    $$\on{End}_A(A) \cong \on{Hom}_A(A,A) \cong \prod_{i=1}^m\on{Mat}_{n_i}(k),$$
    which follows from Schur's lemma, which says that we can only map isotypic 
    components to isotypic components.
    We note that there is an isomorphism 
    $$\on{End}_A(A) \cong A^{\on{op}} \cong A,$$
    and thus $A \cong \prod_{i=1}^m\on{Mat}_{n_i}(k)$, which thus proves the 
    result.
    \begin{exercise}
        Show that $\on{End}_A(A) \cong A^{\on{op}}$.
    \end{exercise}
    \begin{proof}
        Every $A$-module endomorphism of $A$ is given by right-multiplication
        of elements of $A$. Moreover, for any map $\varphi_a \in \on{End}_A(A)$
        mapping $\varphi_a : x\mapsto xa$, we have 
        $$(\varphi_a \circ \varphi_b)(x) = \varphi_a(xb) = xba = \varphi_{ba}(x).$$
        Defining a product $$a \ast_{\on{End}_A(A)} b := \varphi_a\circ \varphi_b,$$
        we thus have the relation:
        $$a \ast_{\on{End}_A(A)} b = ba,$$
        which is precisely the way that the opposite algebra is defined, 
        and thus $\on{End}_A(A) = A^{\on{op}}$.
    \end{proof}
\end{proof}

\subsection{Characters of $A$-modules}

Let $A$ be a $k$-algebra, and $V$ a finite-dimensional representation of 
$A$ with $\rho  :A \to \on{End}_k(V)$.
The \emph{character} of $V$ is
$$\chi_V : A \longrightarrow k,$$ defined by 
$$\chi_V(a) = \on{tr}(\rho(a)).$$
Let $$[A,A] := \on{Span}_k\lbrace [x,y] = xy-yx\rbrace,$$
be the \emph{commutator} of $A$. This induces a map
$$\chi_V : A/[A,A] \longrightarrow k.$$
That is, $\chi_V$ factors through $A/[A,A]$.

\begin{exercise}
    Show that if $W\subset V$ are finite-dimensional representations of $A$,
    then $$\chi_V = \chi_W + \chi_{V/W}.$$
\end{exercise}

\begin{proof}
    Choose a basis for $V$ such that the first $n = \dim_kW$ vectors 
    form a basis for $W$. In this basis, every element of $\rho(a)$
    has the form 
    $$\begin{pmatrix} 
        A&B\\
        0&C
    \end{pmatrix},$$
    where $A$ and $C$ are restrictions of $\rho(a)$ to $W$ and $V/W$, 
    respectively, and $B$ is a projection of $W$. 
    Indeed, we have:
    $$\begin{pmatrix}
        A & B\\
        0 & C
        \end{pmatrix} \begin{pmatrix}
        w\\
        0
        \end{pmatrix} = \begin{pmatrix}
        Aw\\
        0
    \end{pmatrix}.$$
    For some element $u \in V/W$, we may write $u = v+w$, from which
    we obtain:
    $$\begin{pmatrix}
        A & B\\
        0 & C
    \end{pmatrix} \begin{pmatrix}
    w\\
    v
        \end{pmatrix} = \begin{pmatrix}
        Aw + Bv\\
        Cv
    \end{pmatrix},$$
    which implies that $Aw + Bv \in W$. Thus, we have that 
    $$\on{tr}(\rho(a)) = \on{tr}(\rho(a)\vert_W) + \on{tr}(\rho(a)\vert_{V/W}).$$
    That is, $\chi_V = \chi_W + \chi_{V/W}$, since the trace is independent 
    of our choice of basis.
\end{proof}

\section{Lecture 2, 12/10/2023}

\section{Lecture 3, 13/10/2023}

\begin{lemma}\label{lem15}
    Let $W$ be a finite-dimensional $A$-module. Then:
    \begin{itemize}
        \item[(i)] Any homomorphism $\theta : W\to W$ is either an isomorphism
            or nilpotent.
        \item[(ii)] If $\theta_s : W\to W$ for $s=1,\cdots,n$ is a nilpotent
            homomorphism, then so is $\theta=\theta_1+\cdots+\theta_n$.
    \end{itemize}
\end{lemma}

\begin{proof}
    \leavevmode
    \begin{itemize}
        \item[(i)]
        \item[(ii)] By inducting on $n$, if $n=1$, then there is nothing
            to prove. Suppose that $\theta$ is not nilpotent.
            Then, by part (i), $\theta$ is an isomorphism.
            It follows then that 
            $1 = \theta^{-1}\theta_1 + \cdots + \theta^{-1}\theta_n$,
            and thus $\theta^{-1}\theta_i$ is nilpotent, because 
            it is not an isomorphism for each $i=1,\cdots,n$.
            Then, 
            $$1 - \theta^{-1}\theta_1 = \theta^{-1}\theta_2 + \cdots \theta^{-1}\theta_n,$$
            and on the left hand side we have an isomorphism, but on the right
            hand side we have a nilpotent element by the inductive step,
            which is a contradiction.
    \end{itemize}
\end{proof}

\begin{proof}[Back to Proof of Theorem]
    Recall that $\theta_s:V_1\to V_1$, for $s=1,\cdots,n$, and
    $\sum_{s=1}^n\theta_s = 1$. Then, by Lemma \ref{lem15}, it follows
    that there exists some $s$ such that $\theta_s$ is an isomorphism.
    We can assume that $s=1$.
    Recall that $\theta_s$ is defined by the composition
    $$V_1 \stackrel{i_1}{\longrightarrow} V \stackrel{p_1'}{\longrightarrow} V_1' \stackrel{i_1'}{\longrightarrow} V \stackrel{p_1}\longrightarrow V_1,$$
    where $i_1,i_1'$ are injections, and $p_1,p_1'$ are projection maps.
    Sinec $\theta_1$ is an isomorphism, then this implies that 
    $p_1\circ i_1'$ is surjective, and further $p_1'\circ i_1$ is injective.
    From this, we may conclude that $V_1' \cong  \ker(p_1\circ i_1') \oplus \on{Im}(p_1'\circ i_1).$
    But since $V_1'$ is indecomposable, it follows then that 
    $V_1 \cong \on{Im}(p_1'\circ i_1) \cong V_1$.
    It thus follows that both of the maps 
    $p_1'\circ i_1 : V_1 \to V_1'$ and $p_1\circ i_1' : V_1' \to V_1$ 
    are isomorphisms.
    Now, let 
    $$W = \bigoplus_{j>1}V_j,\quad W' = \bigoplus_{j>1}V_j'.$$
    Then, $V = V_1 \oplus W = V_1'\oplus W'$.
    Define a map
    $$h : W \stackrel{i}{\longrightarrow} V = V_1'\oplus W' \stackrel{p}{\longrightarrow} W',$$
    where $i$ is an injection, and $p$ is a projection map. 
    \paragraph{Claim:} $h$ is an isomorphism.\\\\
    Suppose that $h(w) = 0$, for some $w\in W$. Then, $p(w) = 0$, and 
    so $w\in V_1'$. 
    Consider the projection $p_1:  V_1 \oplus W \to V_1$.
    Then, if $p(w) = 0$, then $p_1\circ i_1'(w) = 0$, and since 
    $p_1\circ i_1'$ is an isomorphism, it follows that $w=0$.
    Thus, $\ker h = \lbrace 0 \rbrace$, and thus $h$ is injective.
    Since $\dim W = \dim W'$, $h$ is an isomorphism. Then, by induction, we 
    are done.
\end{proof}

\begin{remark}
    In general, Krull-Schmidt might fail for infinite-dimensional 
    representations. See exercises in (Etingof et al.)
\end{remark}

\subsection{Representations of Tensor Products of Algebras}

Let $A$, $B$, be two $k$-algebras. Then, their tensor product over $k$
is given by:
$$A\otimes_kB,$$
obtains a $k$-algebra structure by 
$$(a_1\otimes_kb_1)\cdot (a_2\otimes_kv_2) = a_1a_2\otimes_kb_1b_2,$$
for $a_1,a_2 \in A$, and $b_1,b_2\in B$. It is a unital 
$k$-algebra, with unit given by $1\otimes_k1$.

\begin{exercise}
    Show that there is a $k$-algebra isomorphism:
    $$\on{Mat}_m(k) \otimes_k\on{Mat}_n(k) \cong \on{Mat}_{mn}(k).$$
\end{exercise}

\begin{proof}
    For $M \in \on{Mat}_m(k)$, and $N \in \on{Mat}_n(k)$, define
    $$M\otimes N := \begin{pmatrix}
        M_{11} N & \cdots & M_{1m}N\\
        \vdots & \ddots & \vdots\\
        M_{m1}N & \cdots & M_{mm}N
    \end{pmatrix},$$
    and the isomorphism follows.
\end{proof}

Let $V$ be an $A$-module, and $W$ a $B$-module.
Then, $V\otimes_kW$ has the structure of an $A\otimes_kB$-module.
The module structure is given by:
$$(a\otimes b)(v\otimes w) = av\otimes bw,$$
for $v\in V$, $w\in W$, $a\in A$, $b\in B$.

\begin{theorem}\label{thm28}
    \leavevmode
    \begin{itemize}
        \item[(i)] Let $V$ be an irreducible finite-dimensional $A$-module,
            and $W$ an irreducible finite-dimensional $B$-module.
            Then, $V\otimes_kW$ is an irreducible finite-dimensional
            representation of $A\otimes_kB$.
        \item[(ii)] Any irreducible finite-dimensional $A\otimes_kB$-module
            is of the form $V\otimes_kW$, for a unique irreducible $A$-module
            $V$, and irreducible $B$-module $W$.
    \end{itemize}
\end{theorem}

\begin{proof}
    \leavevmode
    \begin{itemize}
        \item[(i)] By the density theorem, the maps $A \to \on{End}_k(V)$
            and $B \to \on{End}_k(W)$ are surjective.
            It follows therefore that $$A\otimes_kB \to\on{End}_k(V)\otimes_k\on{End}_k(W)\cong \on{End}_k(V\otimes_kW),$$
            is surjective. Thus, $V\otimes_kW$ is irreducible.
        \item[(ii)] Let $M$ be a finite-dimensional irreducible of 
            $A\otimes_kB$. Then, we have maps 
            $\varphi_A: : A \to \on{End}_k(M)$,
            and $\varphi_B : B \to \on{End}_k(M)$.
            Let $A' := \on{Im}(\varphi_A)$, and $B' := \on{Im}(\varphi_B)$,
            both of which are finite-dimensional $k$-algebras.
            It follows then that $M$ is a representation of $A'\otimes B'$.
            So, we can assume that $A$ and $B$ are both finite-dimensional.
            Thus, $A/\on{Rad}(A)$ and $B/\on{Rad}(B)$ are both 
            isomorphic to matrix algebras. 
            \paragraph{Claim:} $\on{Rad}(A\otimes_kB) = \on{Rad}(A)\otimes_kB + A\otimes_k \on{Rad}(B) =: J$.\\\\
            First $J$ is a nilpotent two-sided ideal in $A\otimes_kB$ (Exercise).
            Thus, $J \subset \on{Rad}(A\otimes_kB)$. 
            It is left as an exercise to show that there is an isomorphism:
            $$(A\otimes_k B)/J \cong A/\on{Rad}(A) \otimes_k B/\on{Rad}(B).$$
            \begin{proof}[Proof of Exercise]
            \end{proof}
            Both factors in the right hand side of the isomorphism are 
            semisimple.
            Thus, its tensor product is semisimple, and $(A\otimes_kB)/J$ 
            is semisimple, and thus 
            $\on{Rad}(A\otimes_kB)\subseteq J$.
            Since $M$ is a finite-dimensional, irreducible representation of 
            the semisimple algebra
            $(A\otimes_kB)/\on{Rad}(A\otimes_kB)$, it follows then that 
            $M \cong V\otimes_kW$, where $V$ is an irreducible 
            $A/\on{Rad}(A)$-module, and $W$ an irreducible 
            $B/\on{Rad}(B)$-module.
    \end{itemize}
\end{proof}

\begin{remark}
    The above theorem may fail for infinite-dimensional representations.
    For instance, consider $A = B = \C(x)$, which is an 
    infinite-dimensional$\C$-algebra.
    Then, $V=W= \C(x)$ is an irreducible module over $A$, and $B$, 
    respectively.
    Then, $V\otimes_kW = \C(x) \otimes_\C \C(y)$ 
    is not irreducible. See (Etingof et al.)
\end{remark}


\chapter{Week Twelve}

\section{Lecture 1, 16/10/2023}

\subsection{Double Centraliser Theorem}

\begin{theorem}[Double Centraliser Theorem]
    Let $A$ and $B$ be two $k$-subalgebras of the algebra $\on{End}_k(E)$,
    were $E$ is a finite-dimensional $k$-vector space, such that 
    $A$ is semisimple, and $B = \on{End}_A(E)$.
    Then,
    \begin{itemize}
        \item[(i)] $A = \on{End}_E(B)$. That is, the centraliser of $A$ in $A$ is $A$.
        \item[(ii)] $B$ is also semisimple.
        \item[(iii)] As an $A\otimes_kB$-module, $E$ decomposes 
            as $$E \cong \bigoplus_{i\in I} V_i\otimes W_i,$$
            where the $V_i$'s are all irreducible $A$-modules,
            and $W_i$'s are all irreducible $B$-modules.
            In particular, there is a natural bijection
            between irreducible $A$-modules and irreducible $B$-modules.
    \end{itemize}
\end{theorem}

\begin{proof}
    Suppose that $A$ is semisimple. Then, 
    $$E \cong \bigoplus_{i\in I} V_i \otimes W_i,$$ 
    for $V_i$ an irrep of $A$, and $W_i = \on{Hom}_A(V_i,E)$, which 
    is the multiplcity space of $V_i$ in $E$
    Since $A \subset \on{End}_k(E)$, we have that 
    $W_i \neq 0$. It follows then that $I$ indexes the set of all 
    irreps of $A$. This implies that 
    $$A \cong \prod_{i \in I} \on{End}_k(V_i).$$
    Now, $$B = \on{Hom}_A(E,E) \cong \prod_{i\in I} \on{End}_k(W_i),$$
    by Schur's lemma. So, $B$ is semisimple, and
    $$\on{End}_B(E) = \prod_{i\in I}\on{End}_k(V_i).$$ 
\end{proof}

\subsection{Schur-Weyl Duality}

Suppose that $E = V^{\otimes n}$, where $V$ is a finite-dimensional 
$\C$-vector space. We wish to apply the double centraliser theorem
to this. Let 
$$A := \on{Im}\left(\C[\mathfrak{S}_n] \to \on{End}_\C(E)\right).$$
Then, $A$ is semisimple. Let $B := \on{End}_A(E)$. 
We have the following theorem:

\begin{theorem}
    $B$ is the image of $\mcal{U}(\mathfrak{gl}(V))$ under its natural action
    on $E = V^{\otimes n}$, where 
    $$\mcal{U}(\mathfrak{gl}(V)) := T(\mathfrak{gl}(V))/\langle X\otimes Y -Y\otimes X - XY + YX\rangle,$$
    is the \emph{universal enveloping algebra} of the Lie algebra
    $\mathfrak{gl}(V)$. 
    That is, $B$ is generated by elements of the form 
    $$\Delta_n(b) := b\otimes 1 \otimes \cdots \otimes 1 + 1\otimes b \otimes \cdots \otimes 1 + \cdots + 1\otimes \cdots \otimes 1\otimes b,$$
    where each of the summands in $\Delta_n(b)$ have $n$ factors.
\end{theorem}

\begin{lemma}\label{lem17}
    \leavevmode
    \begin{itemize}
        \item[(i)] For all finite-dimensional vector spaces $U$ over $\C$,
            $\on{Sym}^nU$ is spanned by elements of the form
            $u \otimes \cdots \otimes u$, for some $u\in U$.
        \item[(ii)] For any $\C$-algebra $A$, its symmetric powers $\on{Sym}^nA$
            is generated by elements of the form $\Delta_n(a)$, for $a\in A$.
    \end{itemize}
\end{lemma}

\begin{proof}
    \leavevmode
    \begin{itemize}
        \item[(i)] This follows from the fact that $\on{Sym}^n(U)$ is an 
            irreducible $\on{GL}(U)$-representation (we will not prove this).
            It is left as an exercise to show this, if you are willing 
            (see Etingof et al., Problem 4.12.3). It follows then that 
            $\on{Span}_\C(\lbrace u\otimes \cdots \otimes u : u \in U \rbrace)$
            is a non-zero subrepresentation.
        \item[(ii)] By the fundamental theorem on symmetric functions, there 
            exists a polynomial $p$ with rational coefficients such that 
            $p(H_1(x),\cdots,H_n(x)) = x_1\cdots x_n$, where 
            $x = (x_1,\cdots,x_n)$, and $H_m(x) := \sum_i x_i^m$. This tells us
            then that 
            $$p(\Delta_n(a), \Delta_n(a^2), \cdots, \Delta_n(a^n)) = a\otimes \cdots \otimes a,\quad a\in A.$$
            Then, using (i), we are done.
    \end{itemize}
\end{proof}

The above lemma then implies the theorem. We have:
$$B = \on{End}_A(V^{\otimes n}) = \on{Sym}^n\on{End}_k(V).$$
We may now apply the double centraliser theorem, which implies the following:

\begin{theorem}[Schur-Weyl Duality]
    \leavevmode
    \begin{itemize}
        \item[(i)] The image $A$ of $\C[\mathfrak{S}_n]$ in 
            $\on{End}(V^{\otimes n}$, and the image of $B$ of 
            $\mcal{U}(\mathfrak{gl}(V))$ in $\on{End}(V^{\otimes n})$ 
            are centralisers of each other.
        \item[(ii)] Both $A$ and $B$ are semisimple. In particular, 
            $V^{\otimes n}$ is a semisimple $\mcal{U}(\mathfrak{gl}(V))$-module.
        \item[(iii)] We have a decomposition of $A\otimes B$-representations
            by: 
            $$V^{\otimes n} = \bigoplus_{\text{$\lambda$ a partition of $n$}}V_\lambda \otimes L_\lambda,$$
            where $V_\lambda$ are the irreducible representations of 
            $\mathfrak{S}_n$ that we have studied before, and 
            $L_\lambda$ is either $0$ or distinct (i.e. non-isomorphic)
            irreps of $\mcal{U}(\mathfrak{gl}(V))$.
    \end{itemize}
\end{theorem}

\begin{remark}
    We may replace $\mcal{U}(\mathfrak{gl}(V))$ by $\on{GL}(V)$.
    Then, Schur-Weyl duality as stated still holds. 
    We can do this because of the following proposition below.
\end{remark}

\begin{proposition}
    The image of $\on{GL}(V)$ in $\on{End}_\C(V^{\otimes n})$ spans $B$.
\end{proposition}

\begin{proof}
    Let $B' := \on{Span} (\lbrace g^{\otimes n} : g\in \on{GL}(V)\rbrace)$.
    Let $b \in \on{End}_\C(V)$, then we claim that $b^{\otimes n} \in B'$.
    If we can show this then we are done by Lemma \ref{lem17}.
    Observe that $t\on{Id} + b \in \on{GL}(V)$ for all but finitely many
    $t$. Consider a linear functional $f : \on{End}_\C(V^{\otimes n}) \to \C$ 
    such that $f\vert_{B'} = 0$. This implies that 
    $$f( (t\on{Id} + b)^{\otimes n}) = 0,$$
    for all but finitely many $t$, where $f$ is a polynomial in $t$. 
    Therefore, $f = 0$, since a polynomial can only have finitely many zeroes.
\end{proof}

\subsection{The Characters of $L_\lambda$}

Recall that $L_\lambda$ are the irreducible representations of $\on{GL}(V)$.
Let $\dim V = n$, and $g \in \on{GL}(V)$, and that $x_1,\cdots,x_n$ are
eigenvalues of $g$ on $V$. 

\begin{theorem}[Weyl Character Formula]
    The representations $L_\lambda = 0$ if and only if $N < k$, where 
    $k$ is the number of parts of $\lambda = L(\lambda)$. If $N \geq k$, then  
    the character of $L_\lambda$ is the Schur polynomial 
    $$s_\lambda(x) := \frac{\det (x_i^{\lambda_j + N - j})}{\Delta(x)},\quad \Delta(x) = \prod_{1\leq i<j\leq N}(x_i-x_j).$$
    In particular,
    $$\dim_\C L_\lambda = \prod_{1\leq i < j \leq n} \frac{\lambda_i - \lambda_j + (j-1)}{j-i}.$$
    \begin{exercise}
        For those who have learned the Weyl character formula before, think
        about why this is equivalent to the Weyl character formula from the 
        Lie algebra course.
    \end{exercise}
\end{theorem}

\begin{proposition}
    $$\prod_m (x_1^m + \cdots + x_N^m)^{i_m} = \sum_{\text{$\lambda$ partitions $n \leq N$}} \chi_\lambda(C_{\widehat{i}})s_\lambda(x),$$
    where $\chi_\lambda(C_{\widehat{i}})$ are the characters of 
    the irreducible $\C[\mathfrak{S}_n]$-modules $V_\lambda$.
\end{proposition}

\begin{proof}[Idea]
    Given $s \in C_i$, 
    $$\on{tr}_{V^{\otimes n}}(g^{\otimes n}s) = \prod_m (x_1^m + \cdots + x_N^m)^{i_m}.$$
\end{proof}

Next time, we will talk more about finite-dimensional algebras. For now, we 
give some examples.

Consider the universal enveloping algebra of $\mathfrak{sl}_2$,
given by: $$\mcal{U}(\mathfrak{sl}_2) = \frac{\C[e,h,f]}{\langle he-eh-2e, hf-fh+2f, ef-fe-h\rangle}.$$
Then:

\begin{theorem}
    $\mcal{U}(\mathfrak{sl}_2)$ has one irreducible representation $V_d$ 
    of dimension $d$ (up to isomorphism), where $V_d$ is the space of 
    homogeneous polynomials of two variables $x$, $y$ of degree $d-1$.
    That is, 
    $$V_d = \lbrace x^{d-1}, x^{d-2}y,\cdots, y^{d-1}\rbrace.$$
    The generators act by differential operators:
    $$\rho(h) = x\frac{\partial}{\partial x} - y\frac{\partial}{\partial y},\quad \rho(e) = x\frac{\partial}{\partial y},\quad \rho(f) = y\frac{\partial}{\partial x}.$$
\end{theorem}

\section{Lecture 2, 19/10/2023}

We wish to study the structure of finite-dimensional algebras.

\subsection{Lifting of Idempotents}

Let $k$ be an algebraically closed field. Recall the notion of an idempotent.
Let $A$ be a $k$-algebra and $I \subset A$ a nilpotent two-sided ideal.

\begin{proposition}\label{prop28}
    Let $e_0 \in A/I$ be an idempotent --- i.e. $e_0^2 = e_0$.
    Then, there exists an idempotent $e\in A$ such that the image of 
    $e$ in $A/I$ is $e_0$. This idempotent $e$ is called a \emph{lift} of $e_0$
    to $A$, and it unique up to conjugacy by an element of $1 + I$.
\end{proposition}

\begin{remark}
    The last sentence makes sense, since $I$ is nilpoten, and thus 
    $1+I$ is invertible, which allows us to conjugate the idempotent.
\end{remark}

\begin{proof}
    Suppose first that $I^2 = 0$. Let $$\pi : A \longrightarrow A/I,$$
    and let $e_1 \in \pi^{-1}(e_0)$, which a priori is not necessarily 
    an idempotent. By definition $e_1 - e_0 \in I$. Now, we can define a 
    new element $$a = e_1^2 - e_1.$$ There is some $x\in I$ 
    such that $e_1 = e_0 + x$. Then,
    $e_1^2 - e_1 = e_0^2 + e_0x + xe_0 + x^2-e_0-x \in I$.
    Now, we have
    \begin{align*}
        e_0a - ae_0 &= e_0(e_1^2 - e_1) - (e_1^2-e_1)e_0\\
                    &= e_0(e_0^2 + e_0x + xe_0 + x^2-e_0-x) - (e_0^2 + e_0x + xe_0 + x^2-e_0-x)e_0\\
                    &=e_0^2x + e_0xe_0 - e_0^2 - e_0x - e-0xe_0 - xe_0^2 + e_0^2 +xe_0\\
                    &= (e_0^2-e_0)x - x(e_0^2-e_0)\\
                    &=(\underbrace{e_0-e_0}_{\in I})\underbrace{x}_{\in I} - \underbrace{x}_{\in I}(\underbrace{e_0-e_0}_{\in I})\\
                    &= 0.
    \end{align*}
    Now, let $e := e_1 + b$, for some $b\in I$. We want to show that $e^2 = e$,
    if and only if $a = b-be_0-e_0b$. Let us set $b = (1-2e_0)a$. Then, 
    one checks that $a = b-be_0-e_0b$.
    \begin{exercise}
        Verify this in your own time. Just remember that our $k$-algebra
        is not necessarily commutative.
    \end{exercise}
    So, we have $e = e_1 + (1-2e_0)\cdot (e_1^2-e_1)$, which is an idempotent.
    Now, we wish to show that this idempotent is unique up to conjugacy.
    Suppose that $e' = e + c$, for some $c \in I$ is another idempotent.
    Then,
    $$(e')^2 - e' = e^2 + ec + ce - e -c  = 0,$$
    if and only if $ec + ce = c$. This is equivalent to the condition that 
    $ece = 0$, and $(1-e)c(1-e) = 0$. That is, given another idempotent $e'$,
    we get these two relations. So, we have that:
    $$c = ec(1-e) + (1-e)ce = e[e,c] - [e,c]e,$$
    where $[-,-]$ is the commutator. It follows thus that 
    $$e + c = (1-[e,c])e(1+[e,c]) = (1-[e,c])e(1-[e,c])^{-1},$$
    which proves the proposition when $I^2 = 0$. Now, we will show by 
    induction on $k$ that there exists a lift $e_k$ of $e_{k-1} \in A/I^k$ 
    to $A/I^{k+1}$, which is unique up to conjugation by an element in 
    $1 + I^k$ (because $I^N = 0$ for some $N$). 
    Suppose that this is true for $k = m-1$. Then, for $k = m$, consider a map 
    $A/I^{m+1} \to A/I^m$, and observe that $(I^m)^2 = 0$ in $A/I^{m+1}$.
    This completes the proof.
\end{proof}

\begin{definition}
    A \emph{complete system of orthogonal idempotents} in a unital $k$-algebra 
    $B$ is a collection of idempotents $e_1,\cdots,e_n \in B$ such that 
    $e_i\cdot e_j = 0$ if $i\neq j$, and $e_1 + \cdots + e_n = 1$.
\end{definition}

\begin{corollary}\label{cor23}
    Let $e_{01},\cdots, e_{0m}$ be a complete system of orthogonal idempotents
    of $A/I$. Then, there exists a lift of a complete system of orthogonal 
    idempotents $e_1,\cdots,e_m$ such that each $e_i$ is equal to $e_{0i}$
    in the image of $A/I$.
\end{corollary}

\begin{proof}
    We induct on $m$. Suppose $m=2$. Then, by Proposition \ref{prop28}, 
    we can find $e_1 \in A$ such that $e_1^2 = e_1$
    such that the image of $e_1$ in $A/I$ is $e_{01}$. Let 
    $e_2 = 1-e_1$. Then, $e_2^2 = e_2$, and $e_1e_2 = 0$, and $e_1+e_2=1$.
    Suppose now that $m > 2$. Let $e_1 \in A$ be such that $e_1^2 = e_1$, and
    the image of $e_1$ in $A/I$ is $e_{01}$.
    Let us consider $$A' := (1-e_1)A(1-e_1) \subset A.$$
    This is a subalgebra, with unit given by $1-e_1$ in $A'$.
    We are given that $e_{02},\cdots,e_{0m}$ is a complete system of 
    orthogonal idempotents in for $A'/(1-e_1)I(1-e_1)$. By induction, 
    there exists idempotents $\widetilde{e_2},\cdots,\widetilde{e_m}$ in $A'$
    such that $\widetilde{e_2} + \cdots + \widetilde{e_m} = 1-e_1$,
    and $\widetilde{e_i}\widetilde{e_j} = 0$ for $i\neq j$. 
    So, it follows then that $e_1,\widetilde{e_2},\cdots,\widetilde{e_m}$
    is a complete system of orthogonal idempotents for $A$.
\end{proof}

\subsection{Projective Covers}


\begin{theorem}
    Let $A$ be a $k$-algebra and $P$ a left $A$-module. The following properties
    of $P$ are equivalent:
    \begin{itemize}
        \item[(i)] Consider the morphisms of $A$-modules: 
            $$\begin{tikzcd}
                        & P \arrow[d, "\nu"] \\
  M \arrow[r, "\alpha"] & N                 
  \end{tikzcd}$$ 
            Then, there exists a map $u : P \to M$ such that 
            $\alpha \circ u = \nu$.
        \item[(ii)]
            Any surjective homomorphism $\alpha : M \to P$ splits. 
            That is, there exists $u : P \to M$ such that 
            $\alpha \circ u = \on{id}$.
        \item[(iii)] There exists another $A$-module $Q$ such that 
            $P\oplus Q$ is a free $A$-module.
        \item[(iv)] The functor $\on{Hom}_A(P,-)$ is exact.
    \end{itemize}
\end{theorem}

\begin{definition}
    A module satisfying the above conditions is called a \emph{projective 
    module}.
\end{definition}

\begin{theorem}\label{thm34}
    Let $A$ be a finite-dimensional $k$-algebra with simple $A$-modules 
    $M_1,\cdots, M_n$. Then,
    \begin{itemize}
        \item[(i)] For each $i=1,\cdots,n$, there exists a unique indecomposable
            finitely generated projective $A$-module $P_i$ such that 
            $$\dim \on{Hom}_A(P_i,M_j) = \delta_{ij}.$$
        \item[(ii)] There is an $A$-module isomorphism:
            $$A \cong \bigoplus_{i=1}^n P_i^{\oplus \dim M_i}.$$
        \item[(iii)] Any indecomposable finitely generated 
            projective $A$-module is isomorphic to $P_i$ for some $i$.
    \end{itemize}
\end{theorem}

\begin{definition}
    These projective modules $P_i$, as seen in the above theorem, is called a 
    \emph{projective cover} of $M_i$. 
\end{definition}

\subsection{Lecture 3, 20/10/2023}

\begin{proof}[Proof of Theorem \ref{thm34}]
    Recall that 
    $$A/\on{Rad}(A) \cong \prod_{i=1}^n \on{End}_k(M_i),$$
    and $\on{Rad}(A)$ is a nilpotent two-sided ideal. Let 
    $e_{ij}^0 := E_{jj}^i$, for $1\leq i \leq n$, and $1\leq j \leq \dim M_i$
    be a complete system of orthogonal idempotents of the matrix 
    algebra $A/\on{Rad}(A)$. Then, by Corollary \ref{cor23}, we can left
    $\lbrace e_{ij}^0\rbrace$ to a complete set of orthogonal idempotents
    $e_{ij}$ to $A$. Now, let $P_{ij} := Ae_{ij}$, which gives subalgebras
    of $A$. 
    \begin{exercise}
        Show that this implies that 
        $$A = \bigoplus_{i,j} Ae_{ij}.$$
    \end{exercise}
    Thus, we have that 
    $$A = \bigoplus_{i,j} Ae_{ij} = \bigoplus_{i=1}^n \bigoplus_{j=1}^{\dim M_i}P_{ij},$$
    as $A$-modules. It follows then that the $P_{ij}$ are projective.
    So, we have:
    $$\on{Hom}_A(P_{ij},M_k) = \on{Hom}_A(Ae_{ij},M_k) \cong e_{ij}M_k.$$
    The last isomorphism is left as an exercise. In particular, defining 
    a homomorphism $\on{Hom}_A(Ae_{ij},M_k) \to e_{ij}M_k$ is the same
    as finding an element of $e_{ij}M_k$. 
    This then implies that 
    $$\dim \on{Hom}_A(P_{ij},M_k) = \delta_{ik}.$$
    Now, $P_{ij} \cong P_{ij}'$, because $e_{ij}$ is conjugate to 
    $e_{ij}'$ by an element in $A^\times$ --- the invertible elements of $A$,
    by Proposition \ref{prop28}.\\\\
    We write $P_i := P_{ij}$. It now remains to show that the $P_i$'s
    are indecomposable. Suppose otherwise. Then, if $P_i = Q_1 \oplus Q_2$,
    then either $\on{Hom}_A(Q_1,M_k) = 0$ for all $k$, or 
    $\on{Hom}_A(Q_2,M_k) = 0$ for all $k$. Otherwise, the dimension of the 
    hom space will be at least $2$. It follows then that $Q_1 = 0$, or $Q_2=0$.
    So, $P_i$ is actually indecomposable.\\\\
    Now, any indecomposable projective module occurs in the decomposition
    of $A$.
\end{proof}

\begin{proposition}
    Let $N$ be any finite-dimensional representation of $A$. Then,
    $$\dim \on{Hom}_A(P_i,N) = [N:M_i],$$
    where the right hand side denotes the multiplicity of occurrence of $M_i$ 
    in the Jordan-Holder series of $N$. That is, for a series
    $$0 \subset N_1 \subset \cdots \subset N_m\subset N,$$
    such that $N_i/N_{i+1}$ are irreducible, then 
    $$[N:M_i] := \vert \lbrace j : N_j/N_{j+1} \cong M_i\rbrace\vert.$$
\end{proposition}

\begin{proof}
    If $N = M_j$, then $[M_j : M_i] = \delta_{ij} = \dim\on{Hom}_A(P_i,M_j)$.
    If the sequence $$0 \longrightarrow N_1 \longrightarrow N_2 \longrightarrow N_3 \longrightarrow 0,$$
    is exact, then $$\on{Hom}_A(P_i,N_1) \longrightarrow \on{Hom}_A(P_i,N_2)\longrightarrow \on{Hom}_A(P_i,N_3) \longrightarrow 0,$$
    is also exact, by definition of projectivity of $P_i$.
\end{proof}

\subsection{The Cartan Matrix of a Finite-Dimensional Algebra}

Let $A$ be a finite-dimensional $k$-algebra, and $M_i$ and $P_i$ a collection
of irreducible $A$-modules, and projective $A$-modules, respectively, with 
$i=1,\cdots,n$. Let 
$$C_{ij} := \dim\on{Hom}_A(P_i,P_j) = [P_j:M_i].$$

\begin{definition}
    The matix $C = (C_{ij})$ is the \emph{Cartan matrix} of $A$.
\end{definition}

\begin{example}
    Let $A$ be the $k$-algebra of $2\times 2$ upper triangular matrices.
    Observe that $$\begin{pmatrix}
        a & b\\
        0 & b
        \end{pmatrix}\begin{pmatrix}
        0 & b'\\
        0& 0
        \end{pmatrix} = \begin{pmatrix}
        0 & ab'\\
        0 & 0
    \end{pmatrix}.$$
    Similarly,
    $$\begin{pmatrix}
        \ast & \ast\\
        0 & \ast
        \end{pmatrix} \begin{pmatrix}
        0&\ast \\
        0 & \ast
        \end{pmatrix} = \begin{pmatrix}
        0 &\ast\\
        0& \ast
    \end{pmatrix}.$$
    So, we have two composition series of $A$ given by:
    $$0 \subset \left\lbrace \begin{pmatrix}
            0 & b\\
            0 & 0
            \end{pmatrix}\right\rbrace \subset \left\lbrace \begin{pmatrix}
            0 & b\\
            0 & c
        \end{pmatrix}\right\rbrace \subset A,$$
        and
    $$0 \subset \left\lbrace \begin{pmatrix}
            a & 0\\
            0 & 0
            \end{pmatrix}\right\rbrace \subset \left\lbrace \begin{pmatrix}
            b & c\\
            0 & 0
        \end{pmatrix}\right\rbrace \subset A.$$
        The idempotents are then given by 
        $$e_1 = \begin{pmatrix}
            1 & 0\\
            0 & 0
            \end{pmatrix},\quad e_2 = \begin{pmatrix}
            0 & 0\\
            0 & 1
        \end{pmatrix},$$
        which forms a system of orthogonal idempotents.
        Further, we have $$A = Ae_1 \oplus Ae_2 = \left\lbrace \begin{pmatrix}
                \ast & 0 \\
                0 & 0
                \end{pmatrix}\right\rbrace \oplus \left\lbrace \begin{pmatrix}
                0 & b\\
                0 & c
            \end{pmatrix}\right\rbrace.$$ 
        In this case, we have $M_1 = \C$, given by 
        $$A \longrightarrow \C, \quad \begin{pmatrix}
            a & b\\
            0 & c
        \end{pmatrix} \longmapsto a,$$
        and $M_2 = \C$, with 
        $$A\longrightarrow \C,\quad \begin{pmatrix}
            a&b\\
            0&c
        \end{pmatrix} \longmapsto c.$$
        So, here we have:
        $$P_1 := Ae_1 = M_1, \quad Ae_2 = P_2.$$
        Let us now look at the composition series of $P_2$. Then, we have:
        $$0 \subset \left\lbrace\begin{pmatrix}
                0 & b\\
                0 & 1
            \end{pmatrix}\right\rbrace \subset P_2.$$
        Thus, we obtain the entries of the Cartan matrix:
        $$[P_1:M_1] = 1,\quad [P_1:M_2] = 0,\quad [P_2:M_1] = 1,\quad [P_2:M_2] = 1.$$
        So,
        $$C = \begin{pmatrix}
            1 & 1\\
            0 & 1
        \end{pmatrix}.$$
        Thus, we may conclude that 
        $$\dim \on{Hom}_A(P_1,P_2) = [P_2:M_1] = 1.$$
        This is left as an exercise for us to think about.
\end{example}

\chapter{Textbook exercises I went and did}

\section{Serre, Chapter 2}

\begin{enumerate}
    \item[(2.1)] \begin{proof}
            We recall the character formula for $\on{Sym}^2(V)$:
            $$\chi_{\on{Sym}^2(V)}(g) = \frac{1}{2}(\chi(g)^2 + \chi(g^2)).$$
            Then, computing directly,
            \begin{align*}
                (\chi+\chi')_{\on{Sym}^2(V)}(g) &= \frac{1}{2}( (\chi + \chi')^2 + (\chi + \chi')(g^2))\\
                                                &= \frac{1}{2}(\chi^2 + \chi(g^2)) + \frac{1}{2}(\chi'^2 + \chi'(g^2)) + \chi\chi'\\ 
                                                &= \chi^2_{\on{Sym}^2(V)} + \chi'^{2}_{\on{Sym}^2(V)} + \chi \chi'.
            \end{align*}
            The calculation for $(\chi + \chi')_{\Lambda^2(V)}$ is analogous,
            using 
            $$\chi_{\Lambda^2(V)} = \frac{1}{2}(\chi(g)^2 - \chi(g^2)).$$
        \end{proof}
    \item[(2.2)]
        Recall that the permutation representation is given by 
        $$V = \bigoplus_{x\in X}\C e_x,$$
        with 
        $$\rho : G \longrightarrow \on{GL}(V), g\longmapsto (e_x\longmapsto e_{gx}).$$
        The matrix representation of the map $e_x\mapsto e_{gx}$ going to 
        be a $\vert X \vert \times \vert X \vert$ matrix, with an entry 
        of $1$ at the $(x,gx)$ position. Thus, the diagonal entries 
        correspond to elements for which $gx = x$ --- that is, they 
        are given by the points which are fixed by $g$.
        Then, by definition, 
        $$\chi_\rho(g) = \on{tr}(e_x \mapsto e_{gx}),$$
        and thus the $\chi_\rho(g)$ counts the number of elements of $X$
        fixed by $g$.
    \item[(2.3)] Define a $G$-representation on $V^\ast$ by:
        $$\rho : G \longrightarrow \on{GL}(V^\ast),\quad g\longmapsto(f\longmapsto g\cdot f),$$
        where $g\cdot f(v) = f(g^{-1}v)$, for $v\in V$, $f\in V^\ast$.
        Then, we have that:
        $$\langle g^{-1}v, f\rangle = \langle v,g\cdot f\rangle,$$
        and thus it satisfies the property that
        $$\langle gv,gf\rangle = \langle v,f\rangle.$$
    \item[(2.4)] This was one of the exercises above. We have proven this.
    \item[(2.5)] From the lectures, we know that if $\chi$ is an irreducible
        character, and $\psi$ is any character of $G$, then the inner product
        $\langle \chi,\psi\rangle$ counts the number of times that $\chi$
        appears in the decomposition of $\psi$.
        Thus, the amount of times that the trivial representation $1$ 
        appears in $\chi$ is given by
        $$\langle \chi,1\rangle = \frac{1}{\vert G\vert}\sum_{g\in G}\chi(g).$$
    \item[(2.6)]
        \begin{itemize}
            \item[(a)] 
            \item[(b)] The character of the corresponding representation
                is the $\vert X\vert^2 \times \vert X\vert^2$ matrix
                with $1$'s at the $((x,y), (gx,gy))$ entry.
                It follows then that the diagonal entries are given by 
                $g$ such that $gx=x$, and $gy=y$, and so 
                its corresponding character is given by $\chi_\rho^2$.
        \end{itemize}
    \item[(2.7)] Let $r_G$ denote the character of the regular 
        representation of $G$. Then, $r_G(1) = \vert G\vert$, and 
        $r_G(g) = 0$ for $g\neq 1$.
        Now, let $\rho : G \to \on{GL}(V)$ be a linear $G$-representation such 
        that $\chi_\rho(g) = \on{tr}(\rho(g)) = 0$ for $g\neq 1$.\\\\
        We have that $\chi_\rho(1) = \dim_\C V$. If $\rho$ is irreducible,
        then we know from the lectures that its dimension must divide 
        $\vert G\vert$. Otherwise, $V$ decomposes into a direct sum of 
        irreducible representations, each of which has dimensions that 
        divide $\vert G\vert$. Thus, $\dim_\C V$ divides $\vert G\vert$.
    \item[(2.8)] We recall that $W_1,\cdots,W_s$ are the irreducible
        representations of $G$ (up to isomorphism), and let 
        $V = U_1\oplus \cdots \oplus U_m$ be the decomposition of $V$
        into a direct sum of irreps. 
        Define $$V_i = \bigoplus_{\text{$j$ such that $U_j\cong_G W_i$}}U_j.$$
        Then, $V = V_1 \oplus \cdots \oplus V_s$ is the canonical decomposition.
        \begin{itemize}
            \item[(a)] Let $H_i := \on{Hom}_\C(W_i,V)$, where 
                each $h \in H_i$ maps $W_i$ into $V_i$.
                We proceed inductively. 
                Let $V = V_i =  W_i$. Then, 
                $\dim_\C H_i = 1 = \dim_\C V_i/\dim_\C W_i$ by Schur's lemma.
                Inducting on the number of copies of isomorphic copies of 
                $W_i$ in $V_i$ then gives the result.
            \item[(b)]
                Define a map 
                $$F : H_i\otimes W_i \longrightarrow V_i,\quad \sum_\alpha h_\alpha \otimes w_\alpha \longmapsto \sum_\alpha h_\alpha(w_\alpha).$$
                Reduce to the case where $V_i = W_i$.
                Then, the map is given by $h\otimes w \mapsto h(w)$.
                Since $h$ is a scalar multiple of the identity map by Schur's
                lemma, it follows that $F$ defines an isomorphism.
                Inducting on the number of isomorphic copies of $W_i$ 
                in $V_i$ gives the result.
            \item[(c)] This is basically an immediate consequence of (b).
        \end{itemize}
    \item[(3.1)] Let $\rho : A \to \on{GL}(V)$ be any irreducible 
        representation of an abelian group $A$. 
        Then, since $A$ is abelian, for any $g,h\in A$, we have 
        $\rho(h)^{-1}\rho(g)\rho(h) = \rho(g)$, and thus by 
        Schur's lemma $\rho(g) = \lambda\on{id}_V$ for all $g\in G$,
        where $\lambda \in \C$.
    \item[(3.2)] 
        \begin{itemize}
            \item[(a)] Consider the restriction representation
                of the irreducible representation $\rho$, given by
                $\rho\vert_{Z(G)} : Z(G) \to \on{GL}(V)$. 
            \item[(b)]
        \end{itemize}
    \item[(3.3)]
    \item[(3.4)] Let $V = \bigoplus_{g\in G} \C e_g$ be the regular 
        representation of $G$. Then, $\on{Res}_H^GV = \bigoplus_{h\in H}\C e_h$,
        which is stable since clearly for any $h' \in H$, 
        $$h'\cdot \on{Res}_H^GV  h'\cdot \bigoplus_{h\in H}\C e_h = \bigoplus_{h\in H} \C e_{h'h} = \on{Res}_H^GV,$$ and thus the restriction is 
        $H$-stable. Then, 
        $$\on{Ind}_H^G \on{Res}_H^GV = \on{Res}_H^GV \otimes_{\C[H]} \C[G] = \bigoplus_{h\in H}\C e_h \otimes_{\C[H]}\C[G] = V.$$
        Since every irreducible $G$-representation is contained in the regular
        representation, the result follows.
    \item[(3.5)] We have shown this before in an exercise given in class.
    \item[(3.6)] Let $G = H\times K$, and $\theta$ a $H$-representation, and 
        $\rho = \on{Ind}_H^G\theta$. Let $r_K$ be the regular representation of 
        $K$.
\end{enumerate}

\end{document}
